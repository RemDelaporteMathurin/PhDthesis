\newglossaryentry{monoblock}{
	name=monoblock,
	description={A unit brick made of a tungsten armour with a CuCrZr cooling pipe and a copper interlayer},
}
\newglossaryentry{plasma}{
	name=plasma,
	description={Ionised gas. Sometimes refered as the fourth state of matter}
}
\newglossaryentry{divertor}{
	name=divertor,
	description={Component of a tokamak diverting the magnetic field lines creating one or several X-points. The divertor is responsible for exhausting heat and particle from the plasma. The targets of the divertor is where the plasma is directed to absorb heat and particle fluxes},
}
\newglossaryentry{strike point}{
	name=strike point,
	description={Location on the divertor where the serparatrix crosses the solid target. It is where the highest heat and particle fluxes are observed},
}
\newglossaryentry{x-point}{
	name=X-point,
	description={Point in space where the poloidal field has zero magnitude},
}
\newglossaryentry{separatrix}{
	name=separatrix,
	description={Last closed flux surface, intersects the X-point},
}
\newglossaryentry{private flux region}{
	name=private flux region,
	description={WEST divertor region located between the strike points}
}
\newglossaryentry{fuel recycling}{
	name=fuel recycling,
	description={Phenomenon occuring when particles from the plasma are neutralised at the walls and ionised again}
}
\newglossaryentry{breeding blanket}{
	name=breeding blanket,
	description={Components of a fusion reactor responsible for producing tritium from the neutrons of the fusion reactions},
}
\newglossaryentry{first wall}{
	name=first wall,
	description={Cover of the inner surface of the vacuum vessel in front of the plasma},
}

\newglossaryentry{isotope}{
	name=isotope,
	description={Member of a family of elements sharing the same number of protons but different numbers of neutrons}
}
\newglossaryentry{transmutation}{
	name=transmutation,
	description={Conversion of a chemical element into another chemical element by changing the number of neutrons or protons in the nucleus}
}
\newglossaryentry{transmutation gas}{
	name=transmutation gas,
	description={Gas produced by transmutation (typically helium or hydrogen)}
}

\newglossaryentry{diffusion}{
	name=diffusion,
	description={For particles: process resulting from the random motion of particles from high concentration regions to regions of low concentrations. For heat: transport of thermal energy from high temperature regions to low temperature regions}
}
\newglossaryentry{permeation}{
	name=permeation,
	description={Transport of particles through a material membrane}
}
\newglossaryentry{advection}{
	name=advection,
	description={Transport of particles (or heat) via motion of the medium (as opposed to diffusion where the medium can be immobile)}
}
\newglossaryentry{trapping}{
	name=trapping,
	description={Process involving a particle being retained in a potential energy well}
}
\newglossaryentry{detrapping}{
	name=detrapping,
	description={Opposite to trapping}
}
\newglossaryentry{Soret effect}{
	name=Soret effect,
	description={Diffusion assisted by thermal gradients}
}
\newglossaryentry{thermophoresis}{
	name=thermophoresis,
	description={See Soret effect}
}


\newglossaryentry{dislocation loop}{
	name=dislocation loop,
	description={Defects formed by agglomerations of point-defects into planes}
}
\newglossaryentry{self-interstitial}{
	name=self-interstitial,
	description={An atom initially in the lattice located at interstitial sites}
}
\newglossaryentry{vacancy}{
	name=vacancy,
	description={Defect in the lattice where a metal atom is missing},
	plural=vacancies
}
\newglossaryentry{Frenkel pair}{
	name=Frenkel pair,
	description={A self-interstitial atom and a vacancy}
}
\newglossaryentry{trap mutation}{
	name=trap mutation,
	description={Process where over-pressurised helium clusters eject a metal lattice atom creating a Frenkel pair}
}
\newglossaryentry{self-trapping}{
	name=self-trapping,
	description={See trap mutation entry}
}

\newglossaryentry{loop punching}{
	name=loop punching,
	description={Helium bubble growth mechanism leading to the formation of a dislocation loop}
}
\newglossaryentry{fuzz}{
	name=fuzz,
	description={Porous, nanostructure sometimes observed on tungsten under helium irradiation}
}
\newglossaryentry{tendril}{
	name=tendril,
	description={Nanometre-thick columns forming the fuzz}
}
\newglossaryentry{blistering}{
	name=blistering,
	description={Formation of blisters: plastic deformation (swelling) of the metal near the surface due to high pressure in bubbles}
}
\newglossaryentry{bursting}{
	name=bursting,
	description={Explosion of a gas bubble near the metal surface}
}


\newglossaryentry{fluence}{
	name=fluence,
	description={Total quantitiy of particles a sample has been exposed to over a given period of time. Product of the particle flux by the exposure time}
}
\newglossaryentry{retention}{
	name=retention,
	description={Local concentration of soluted species (hydrogen or helium) including the mobile and trapped species}
}
\newglossaryentry{inventory}{
	name=inventory,
	description={Total quantity of particles in a domain (sample, component, reactor...). It can be obtained by integrating the retention over the domain}
}
\newglossaryentry{startup inventory}{
	name=startup inventory,
	description={Tritium inventory required by new reactors to account for the initial tritium losses (processing time, trapping in components...)}
}

% Glossary entries (used in text with e.g. \acrfull{fpsLabel} or \acrshort{fpsLabel})
\newacronym{H}{H}{hydrogen}
\newacronym{He}{He}{helium}
\newacronym{D}{D}{deuterium}
\newacronym{T}{T}{tritium}
\newacronym{Li}{Li}{lithium}
\newacronym{W}{W}{tungsten}
\newacronym{Be}{Be}{beryllium}


\newacronym[longplural={Plasma Facing Units}, shortplural={PFUs}]{pfuLabel}{PFU}{Plasma Facing Unit}
\newacronym[longplural={Plasma Facing Materials}, shortplural={PFMs}]{pfm}{PFM}{Plasma Facing Material}
\newacronym{cfc}{CFC}{Carbon fiber composite}
\newacronym{ivt}{IVT}{Inner Vertical Target}
\newacronym{ovt}{OVT}{Outer Vertical Target}
\newacronym{isp}{ISP}{Inner Strike Point}
\newacronym{osp}{OSP}{Outer Strike Point}

\newacronym{hcpb}{HCPB}{Helium-Cooled-Pebble-Bed}
\newacronym{wcll}{WCLL}{Water-Cooled-Lithium-Lead}
\newacronym{hcll}{HCLL}{Helium-Cooled-Liquid-Lead}
\newacronym{dcll}{DCLL}{Dual-Coolant-Lithium-Lead}
\newacronym{tbr}{TBR}{tritium breeding ratio}
\newacronym{fpy}{FPY}{full power year}


\newglossaryentry{tokamak}{
	name=tokamak,
	description={Toroidal fusion reactor using magnetic fields to confine the plasma. Acronym for "TOroidalnaïa KAmera s MAgnitnymi Katouchkami", which means "Toroidal chamber with magnetic coils"}
}
\newglossaryentry{stellarator}{
	name=stellarator,
	description={Toroidal fusion reactor using magnetic fields to confine the plasma. The difference with the tokamak is the twisted magnetic coils and the absence of a central solenoid inducing a current in the plasma}
}
\newacronym{candu}{CANDU}{CANada Deuterium Uranium reactor}
\newacronym{west}{WEST}{Tungsten Environment Steady state Tokamak}
\newacronym{jet}{JET}{Joint European Torus}
\newacronym{iter}{ITER}{International Thermonuclear Experimental Reactor}
\newacronym{demo}{DEMO}{DEMOnstration power plant}
\newacronym{sparc}{SPARC}{Soonest Private-funded Affordable Robust Compact}
\newacronym{arc}{ARC}{Affordable Robust Compact}
\newacronym{nif}{NIF}{National Ignition Facility}
\newacronym{mast-u}{MAST-U}{Mega Amp Spherical Tokamak Ugrade}
\newacronym{asdex}{ASDEX}{Axially Symmetric Divertor Experiment}
\newacronym{step}{STEP}{Spherical Tokamak for Electricity Production}
\newacronym{ukaea}{UKAEA}{United-Kingdom Atomic Energy Authority}

\newacronym{md}{MD}{Molecular Dynamics}
\newacronym{dft}{DFT}{Density Functional Theory}
\newacronym{eels}{EELS}{Electron Energy Loss Spectroscopy}
\newacronym{fib}{FIB}{Focused Ion Beam}
\newacronym{tds}{TDS}{Themo-Desorption Spectroscopy}
\newacronym{tem}{TEM}{Transmission Electron Microscopy}
\newacronym{mms}{MMS}{Method of Manufactured Solutions}
\newacronym{mes}{MES}{Method of Exact Solutions}
\newacronym{mre}{MRE}{Macroscopic Rate Equations}
\newacronym{vv}{V\&V}{Verification and Validation}
\newglossaryentry{p1}{
	name=P1,
	description={Lagrange finite element of order 1, also called piecewise linear elements}
}
\newglossaryentry{p2}{
	name=P2,
	description={Lagrange finite element of order 2}
}
\newglossaryentry{p3}{
	name=P3,
	description={Lagrange finite element of order 3}
}
\newacronym[description={type of metallic lattice}]{bcc}{bcc}{body-centered cubic}
\newglossaryentry{lattice}{
	name=lattice,
	description={Three-dimensional crystalline structure of metals. The lattice is how the atoms are ordered within a metal}
}

\newacronym{fdm}{FDM}{Finite Difference Method}
\newacronym{fem}{FEM}{Finite Element Method}
\newacronym{festim}{FESTIM}{Finite Element Simulation of Tritium in Materials}
\newacronym{tmap7}{TMAP7}{Tritium Migration Analysis Program, Version 7}
\newacronym{hiipc}{HIIPC}{Hydrogen Isotope Inventory Processes Code}
\newacronym{crds}{CRDS}{Coupled Reaction Diffusion System}
\newacronym{mhims}{MHIMS}{Migration of Hydrogen Isotopes in MaterialS}
\newacronym{tessim}{TESSIM}{Tritium Extraction System SIMulator}
\newacronym{hit}{HIT}{Hydrogen Isotope Transport code}
\newglossaryentry{achlys}{
	name=ACHLYS,
	description={Hydrogen transport code developed by UKAEA}
}
\newglossaryentry{paramak}{
	name=Paramak,
	description={Code for generating parametric tokamak geometries}
}
\newglossaryentry{openmc}{
	name=OpenMC,
	description={Open-source neutronics code}
}
\newglossaryentry{fenics}{
	name=FEniCS,
	description={Open-source finite element solving library}
}
\newacronym{imas}{IMAS}{Integrated Modelling Analysis Suite}
\newacronym{srim}{SRIM}{Stopping Range of Ions in Matter}
\newglossaryentry{solps}{
	name=SOLPS-ITER,
	description={Primary plasma boundary code used at ITER}
}
\newglossaryentry{soledge}{
	name=SOLEDGE3X,
	description={Plasma boundary code}
}

\newacronym{cg}{CG}{Conjugate Gradient}
\newacronym{tnc}{TNC}{Truncated Newton method}

\newacronym{icrf}{ICRF}{Ion Cyclotron Range of Frequencies}
\newacronym{endf}{ENDF}{Evaluated Nuclear Data File}
