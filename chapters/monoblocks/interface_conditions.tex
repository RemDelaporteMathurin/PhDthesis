According to \sidecite{krom_hydrogen_2000} since the solubility of hydrogen atoms in solids is low, the chemical potential of solute hydrogen $\mu$ is expressed by:
\begin{equation}
    \mu = \mu_0 + RT \ln\left( \frac{c_\mathrm{m}}{N_L}\right)
\end{equation}
where $\mu_0$ is the chemical potential in a reference state in \si{J.mol^{-1}}, $R$ the ideal gas constant, $T$ the temperature in \si{K}, $c_\mathrm{m}$ the mobile hydrogen concentration in \si{m^{-3}} and $N_L$ the lattice site concentration in \si{m^{-3}}.

Assuming that only free hydrogen atoms contribute to the overall flux in the material, the particle flux $J$ in \si{m^{-2}.s^{-1}} can be expressed by Fick's law:
\begin{equation}
    J = - D \nabla c_\mathrm{m}
\end{equation}
where $D$ is the diffusion coefficient of hydrogen in a non-stress lattice expressed in \si{m^{2}.s^{-1}}. 


%The temporal evolution of hydrogen concentration in materials is governed by Fick's law of diffusion (see Equation \ref{eq:diffusion equation}).

%\begin{equation}
%    \frac{\partial c}{\partial t}=\vec{\nabla} \cdot\left(D \vec{\nabla}c\right)+ f
%    \label{eq:diffusion equation}
%\end{equation}
%where $c$ is the hydrogen concentration in \si{m^{-3}}, D is the diffusion coefficient of the mobile hydrogen in the material in \si{m^2.s^{-1}} and $f$ is the source term in \si{m^{-3}.s^{-1}}.

%In order to take into account the conservation of chemical potential across an interface, one must not ensure concentration continuity but rather the continuity of the chemical potential of the species in the material.

%The chemical potential is expressed  expressed in \si{J.mol^{-1}} as follow: \textcolor{red}{YC-> pas sûr qu'il faille garder cette equation ma foi...}
%\begin{equation}
%    \mu = \mu_0 + RT\ln{\left(\frac{c}{S} \frac{1}{p_0}\right)}
%\end{equation}
%where $\mu_0$ is the species standard chemical potential, $R$ is the gas constant expressed in \si{J.K^{-1}.mol^{-1}}, $T$ is the temperature in \si{K}, $S$ is the solubility of the species in the material expressed in \si{m^{-3}.Pa^{-0.5}} and $p_0$ is the pressure in \si{Pa} at some reference state.

%It can be shown that the continuity of $\mu$ is equivalent to that of $c/S$ \sidecite{}.
The local equilibrium at the interface between two materials must ensure  the continuity of both the chemical potential $\mu$ (see Equation \ref{eq: muconservation}) and the particle flux (see Equation \ref{eq: flux conservation}).
\begin{equation}
            \mu^- = \mu^+  \label{eq: muconservation}  
\end{equation}
    
\begin{equation}
        D^- \nabla c_\mathrm{m}^- = D^+ \nabla c_\mathrm{m}^+ \label{eq: flux conservation} 
\end{equation}
The chemical potential continuity can also be ensured by the continuity of the quantity $c_\mathrm{m}/S$ (see Equation \ref{eq: c/s conservation}) for mechanical stress-free materials in thermodynamic equilibrium and the Soret effect being neglected:
\begin{equation}
                \left(\frac{c_\mathrm{m}}{S}\right)^- = \left(\frac{c_\mathrm{m}}{S}\right)^+  \label{eq: c/s conservation}  
\end{equation}
Here, the quantity $c_\mathrm{m}/S$, with $S$ the solubility of hydrogen expressed in \si{m^{-3}.Pa^{-0.5}}, is equivalent to the root square of the pressure of an imaginary gas in thermodynamic equilibrium between the two solids and for which Sievert's law is applied.  
% Ref annex B
This assumption is correct as long as the time needed to reach the equilibrium is low compared to the time of the simulation.
\ref{appendix transient model} described the characteristic time calculated by a transient interface model.
For long exposure time (as in Section \ref{iter case}) as well as for high temperatures, it is shown that the characteristic time is small enough for the equilibrium model to be valid.

From Equation \ref{eq: c/s conservation}, one can deduce that a solubility discontinuity across an interface induces a discontinuity of mobile hydrogen concentration $c_\mathrm{m}$.
This can also be interpreted as the chemical potentials at a reference state being different in different materials \sidecite{kirchheim_25_2014}, as the lattice site concentration.

% To ensure a correct treatment of the material interface in hydrogen transport codes, two approaches can be used.
% The most natural is to solve the hydrogen mobile concentration transport in both materials (see Equation \ref{eq: diffusion equation}) and impose an internal condition at the interface between the two materials to verify Equation \ref{eq: c/s conservation} \sidecite{longhurst_tmap7_2008}.
% The other way is to apply a change of variable in the hydrogen transport equation with $\phi = c_\mathrm{m}/S$ (see Equation \ref{eq: diffusion equation changed})  \sidecite{smith_abaqusstandard_2009} when internal boundary can not be defined. Because $\phi$ is solved, the ratio $c_\mathrm{m}/S$ is continuous by default at the material interfaces.

To ensure a correct treatment of the material interface in hydrogen transport codes, two approaches can be employed.
The most straightforward approach is to solve the hydrogen mobile concentration transport in both materials (see Equation \ref{eq: diffusion equation}) and enforce the concentration jump at the interface between the two materials with an internal condition verifying  Equation \ref{eq: c/s conservation} \sidecite{longhurst_tmap7_2008}.

\begin{equation}
    \frac{\partial c_\mathrm{m}}{\partial t}=\nabla \cdot\left(D \nabla c_\mathrm{m}\right) + f
   \label{eq: diffusion equation}
\end{equation}
where $f$ is the source term in \si{m^{-3}.s^{-1}}.

Another method is to perform a change of variable in Fick's second law of diffusion with $\phi = c_\mathrm{m}/S$ \sidecite{smith_abaqusstandard_2009} when internal conditions cannot be set.
Equation \ref{eq: diffusion equation} therefore reads:

\begin{align}
    \frac{\partial \phi S}{\partial t} &= \nabla \cdot\left(D \nabla \phi S\right) + f \nonumber \\
    &=\nabla \cdot\left( D S \nabla \phi + D \phi \nabla S\right) + f \label{eq: diffusion equation changed}
\end{align}

Because $\phi$ is computed, the ratio $c_\mathrm{m}/S$ is continuous by default at the material interfaces.

This second approach is used for instance in the \textit{mass-diffusion} procedure of the Abaqus code.
However, this procedure cannot be directly used to simulate hydrogen transport with thermomechanical loading \textit{(coupled temperature displacement} procedure) \sidecite{smith_abaqusstandard_2009}.
Indeed, the \textit{mass-diffusion} procedure can solve mechanically-assisted diffusion problems, knowing the spatio-temporal evolution of the mechanical pressure.
The \textit{temperature-displacement} procedure is used to simultaneously solve diffusion and mechanical problems.
To account for both mechanical fields, trapping influence on diffusion and transient heat transfer, User Subroutines are employed.
\ref{appendix abaqus} describes the implementation based on user Subroutines.
This interface model has also been implemented into the current hydrogen transport code FESTIM \sidecite{delaporte-mathurin_finite_2019} using FEniCS \sidecite{alnaes_fenics_2015}.

All plots in this work were generated with Matplotlib \sidecite{hunter_matplotlib_2007}.

%To do so, a variable change is performed in Equation \ref{eq:diffusion equation} so that the quantity $\theta = c/S$ is directly computed in FESTIM.
%Equation 1 can therefore be written as:

%\begin{equation}
%        \frac{\partial \theta S}{\partial t}=\vec{\nabla} \cdot\left(D \vec{\nabla}(\theta S)\right)+ f
%    \label{eq:diffusion equation theta}
%\end{equation}

%The quantity $\theta$ obtained by solving Equation \ref{eq:diffusion equation theta}.
%By multiplying $\theta$ by the local value of the solubility $S$, the particle concentration $c$ can be obtained and a discontinuity ensuring the local equilibrium at interfaces (see Equation \ref{eq: interface equations}).


To ensure the implementation of the interface model in FESTIM is error free, analytical verification is required since physical trend tests are not rigorous enough. The verification process aims at checking the code is correctly solving the governing Equations \ref{eq: flux conservation}, \ref{eq: c/s conservation}, \ref{eq: diffusion equation} and \ref{eq: diffusion equation changed}.
To this end, two methods were used: the Method of Exact Solutions (MES) and the Method of Manufactured Solutions (MMS).
It is worth noting that these two methods do not require to be physically realistic since this is a purely mathematical exercise.
In practice, not having real life properties or realistic domain sizes can even facilitate the construction of a general test case and therefore ease the results reproduction by others.
A complete H transport problem would include coupling with trapping/detrapping (as in Section \ref{iter case}).
The implementation of the trapping/detrapping coupling in FESTIM has already been analytically verified in \sidecite{delaporte-mathurin_finite_2019} therefore this Section will only focus on verifying the solving of Equations \ref{eq: flux conservation}, \ref{eq: c/s conservation}, \ref{eq: diffusion equation} and \ref{eq: diffusion equation changed}.

\subsection{Method of Exact Solutions (MES)}

\begin{figure} [h]
    \centering
    \begin{subfigure}{0.5\linewidth}
        \centering
        \includegraphics[width=\linewidth]{Figures/Chapter3/monoblocks/interface_condition/out_MES_case1.pdf}
        \caption{Case 1: $\alpha = 2$, $\beta = 1.5$, $\gamma=0.6$, $\tilde{c}_L = 2$, $\tilde{f}=1$}
    \end{subfigure}%
    % \hfill
    \begin{subfigure}{0.5\linewidth}
        \centering
        \includegraphics[width=\linewidth]{Figures/Chapter3/monoblocks/interface_condition/out_MES_case2.pdf}
        \caption{Case 2:  $\alpha = 1.5$, $\beta = 0.5$, $\gamma=0.4$, $\tilde{c}_L = 0.25$, $\tilde{f}=2$}
    \end{subfigure}
    \caption{Concentration profiles simulated by FESTIM against analytical solutions.}
    \label{fig:comparison MES}
\end{figure}

The uni-dimensional test case considered in this Section was made of two subdomains $\Omega_1$ and $\Omega_2$ and is described as follow:
\begin{subequations}
\begin{align}
    \Omega &= [0, L] = \Omega_1 \cup \Omega_2 \\
    \Omega_1 &= [0, x_\mathrm{int}] \\
    \Omega_2 &= [x_\mathrm{int}, L] \\
    D &= \begin{cases}
        D_1,& \text{ in } \Omega_1\\
        D_2,& \text{ in } \Omega_2
    \end{cases} \\
    S &= \begin{cases}
        S_1,& \text{ in } \Omega_1\\
        S_2,& \text{ in } \Omega_2
    \end{cases}
\end{align}
\end{subequations}

The following dimensionless quantities are introduced:
\begin{subequations}
    \begin{align}
        \tilde{c}_\mathrm{m} &= c_\mathrm{m} / c_0\\
        \tilde{x} &= x / L \\
        \tilde{f} &= f \frac{L^2}{D_\mathrm{eq} c_0} \\
        \alpha &= D_2/D_1 \\
        \beta &= S_2/S_1 \\
        \gamma &= x_\mathrm{int}/L\\
    \end{align} 
\end{subequations}
where $D_\mathrm{eq} = (D_1 D_2)^{1/2}$.

By integrating Equation \ref{eq: diffusion equation} and assuming steady-state (\textit{i.e.} $\partial c/\partial t=0$), one can obtain the following dimensionless form:

\begin{equation}
        \tilde{c}_\mathrm{m}= 
\begin{cases}
    -\frac{1}{2}\alpha^{1/2}\tilde{f} \tilde{x}^2 + a_1 \tilde{x} + b_1,& \text{ in } \Omega_1\\
    -\frac{1}{2}\alpha^{-1/2}\tilde{f} \tilde{x}^2 + a_2 \tilde{x} + b_2,& \text{ in } \Omega_2
\end{cases}
\label{eq:MES c}
\end{equation}

% \begin{equation}
%         c_\mathrm{m}= 
% \begin{cases}
%     \frac{-f}{2D_1} x^2 + a_1 x + b_1,& \text{ in } \Omega_1\\
%     \frac{-f}{2D_2} x^2 + a_2 x + b_2,& \text{ in } \Omega_2
% \end{cases}
% \tilde
% \label{eq:MES c}
% \end{equation}
where $a_1$, $b_1$, $a_2$, $b_2$ are the unknowns of the problem to be determined.
The boundary conditions and the equilibrium law at the interface are defined as:
\begin{subequations} \label{eq: bcs MES}
\begin{align} 
        \tilde{c}_\mathrm{m}(\tilde{x}=0) & = 1 \\
        \tilde{c}_\mathrm{m}(\tilde{x}=1) & =  \tilde{c}_L \\
        \tilde{c}_\mathrm{m}^-(\tilde{x}=\gamma) & =  \beta \; \tilde{c}_\mathrm{m}^+(\tilde{x}=\gamma)\\
        \nabla \tilde{c}_\mathrm{m}^-(\tilde{x}=\gamma) & =  \alpha \nabla \tilde{c}_\mathrm{m}^+(\tilde{x}=\gamma)
\end{align}
\end{subequations}


Equation \ref{eq:MES c} can be solved with these constraints and coefficients describing $c_\mathrm{m}$ therefore read:

% \begin{align}
%     \begin{split}
%         a_1 &= \frac{- 2 D_{1} D_{2} S_{2} c_{0} + D_{1} S_{1} \left(2 D_{2} c_L - f L^{2}\right) + f x_\mathrm{int}^{2} \left(D_{1} S_{1} - D_{2} S_{2}\right)}{2 D_{1} \left(D_{1} S_{1} L - D_{1} S_{1} x_\mathrm{int} + D_{2} S_{2} x_\mathrm{int}\right)}  \\
%         b_1 &= c_0 \\
%         a_2 &= \frac{- 2 D_{1} D_{2} S_{2} c_{0} + D_{1} S_{1} \left(2 D_{2} c_{L} - f L ^{2}\right) + f x_\mathrm{int}^{2} \left(D_{1} S_{1} - D_{2} S_{2}\right)}{2 D_{2} \left(D_{1} S_{1} L  - D_{1} S_{1} x_\mathrm{int} + D_{2} S_{2} x_\mathrm{int}\right)}\\
%         b_2 &= \frac{2 D_{1} D_{2} L S_{2} c_{0} - x_\mathrm{int} \left(D_{1} S_{1} - D_{2} S_{2}\right) \left((L f x_\mathrm{int} +  \left(2 D_{2} c_{L} - L^{2} f\right)\right)}{2 D_{2} \left(D_{1} S_{1} L - D_{1} S_{1} x_\mathrm{int} + D_{2} S_{2} x_\mathrm{int}\right)} 
%     \end{split}
% \end{align}


\begin{align}
    \begin{split}
        a_1 &= a_0 \; \alpha^{1/2}  \\
        b_1 &= 1 \\
        a_2 &= a_0 \; \alpha^{-1/2}\\
        b_2 &= \tilde{c}_L + \frac{1}{2} \alpha^{-1/2} \tilde{f} - a_2 \\
        %a_0 &= \frac{- 2 D_{1} D_{2} S_{2} c_{0} + D_{1} S_{1} \left(2 D_{2} c_L - f L^{2}\right) + f x_\mathrm{int}^{2} \left(D_{1} S_{1} - D_{2} S_{2}\right)}{2 \left(D_{1} S_{1} L - D_{1} S_{1} x_\mathrm{int} + D_{2} S_{2} x_\mathrm{int}\right)}
        a_0 &= \frac{2 \alpha^{1/2}( \tilde{c}_L - \beta) + \tilde{f}  (\gamma^{2} \left( 1 - \alpha \beta\right) - 1)}{2 \left( 1  + \alpha \beta - \gamma \right)} \\
        % a_0 &= \frac{- 2 D_{1} D_{2} S_{2} c_{0} + D_{1} S_{1} \left(2 D_{2} c_L - f L^{2}\right)}{2 \left(D_{1} S_{1} L - D_{1} S_{1} x_\mathrm{int} + D_{2} S_{2} x_\mathrm{int}\right)} \\
        % &~~~~~~~~~+ \frac{f x_\mathrm{int}^{2} \left(D_{1} S_{1} - D_{2} S_{2}\right)}{2 \left(D_{1} S_{1} L - D_{1} S_{1} x_\mathrm{int} + D_{2} S_{2} x_\mathrm{int}\right)}
    \end{split}
    \label{eq: MES c coefficients}
\end{align}

It is worth noting that when $\beta=1$ (\textit{i.e.} $S_1 = S_2 = S$) the solution becomes independent of $S$ and {$c_\mathrm{m}^{-}(x_\mathrm{int}) = c_\mathrm{m}^{+}(x_\mathrm{int})$}.
Moreover, when $\alpha = 1$ (\textit{i.e.} $D_1 = D_2 = D$), then $a_1 = a_2 = a_0$ which is the solution for steady-state diffusion in a mono-material.

The solution computed by FESTIM was found to be in very good agreement with the analytical solution for several test cases (see Figure \ref{fig:comparison MES}).

 
However, this method does not exercise all the terms in the governing Equation \ref{eq: diffusion equation changed}.
For instance, this analytical solution is only uni-dimensional, steady state is assumed and material properties are constant within the materials.
Having an exact solution from an analytical resolution for a general problem (multidimensional, transient, heterogeneous material properties, etc...) is often complex.
In order to exercise all these terms, the Method of Manufactured Solutions (MMS) will therefore be employed for it offers a good alternative to unravel these complexities.

\subsection{Method of Manufactured Solutions (MMS)}

The goal of Method of Manufactured Solutions (MMS) is to have an exact solution which is general enough to exercise all the terms of the governing equations.
This method also allows to test the implementation of the FESTIM code on a multidimensional problem.

The domain $\Omega$ for this test problem is a unit square composed of two subdomains $\Omega_1$ and $\Omega_2$ (see Equation \ref{eq: MMS domain}).
\begin{subequations} \label{eq: MMS domain}
\begin{align}
    \Omega &= [0, 1] \times [0, 1] \\
    \Omega_1 &= [0, x_\mathrm{int}] \times [0, 1] \\
    \Omega_2 &= [x_\mathrm{int}, 1] \times [0, 1] \\
\end{align}
\end{subequations}
In order to unravel the complexity of an analytical resolution of the direct problem, a manufactured solution $c_M$ was constructed (see Equation \ref{eq: manufactured solution}) and the problem was solved backwards.

\begin{equation}
        c_M= 
\begin{cases}
    c_{M1},& \text{ on } \Omega_1\\
    \frac{S_2}{S_1} \cdot c_{M1},& \text{ on } \Omega_2
\end{cases}
\label{eq: manufactured solution}
\end{equation}
where $c_{M1} = 2 + \cos(2\pi x) \cdot \cos(2\pi y) + t$

It is worth noting that, when choosing a manufactured solution, one must ensure it satisfies all the governing equations (especially Equations \ref{eq: flux conservation} and \ref{eq: c/s conservation}).
In our case, $c_M$ ensures the flux conservation at the interface and the continuity of the quantity $c_\mathrm{m}/S$.

Properties are assumed time and space dependent in order to test every portion of the code (see Equation \ref{eq: MMS properties}).
\begin{subequations}
    \begin{align}
        D_1(x, y, t) & =  D_{1_0} \exp(-E_{D_1}/(k_B \cdot T(x,y, t) )) \\
        D_2(x, y, t) & =  D_{2_0} \exp(-E_{D_2}/(k_B \cdot T(x,y, t) )) \\
        S_1(x, y, t) & =  S_{1_0} \exp(-E_{S_1}/(k_B \cdot T(x,y, t) )) \\
        S_2(x, y, t) & =  S_{2_0} \exp(-E_{S_2}/(k_B \cdot T(x,y, t) )) \\
        T(x, y, t) & = 500 + 30 \cos(2\pi x) \cos(2 \pi y) \cos(2 \pi t)
\end{align} \label{eq: MMS properties}
\end{subequations}
with ${k_B = \SI{8.617e-5}{eV.K^{-1}}}$ the Boltzmann constant, $D_{1_0} = 1$, $E_{D_1} = 0.1$, $D_{2_0} = 2$, $E_{D_2} = 0.2$, $S_{1_0} = 1$, $E_{S_1} = 0.1$, $S_{2_0} = 2$ and $E_{S_2} = 0.2$.
The temperature $T$ varies around \SI{500}{K} so that, given the activation energies, properties do not approach zero.

By injecting the manufactured solution $c_M$ into the governing Equation \ref{eq: diffusion equation}, the source term can be expressed as:
\begin{align}
    f(x, y, t) &= \frac{\partial c_M}{\partial t} - \vec{\nabla} \cdot\left(D(x, y)
    \vec{\nabla}c_M\right) \nonumber \\
    &= \begin{cases}
        \frac{\partial c_{M_1}}{\partial t} - \vec{\nabla} \cdot\left(D_1(x, y)
    \vec{\nabla}c_{M_1}\right),& \text{ on } \Omega_1\\
    \frac{\partial c_{M_2}}{\partial t} - \vec{\nabla} \cdot\left(D_2(x, y)
    \vec{\nabla}c_{M_2}\right),& \text{ on } \Omega_2\\
    \end{cases}
\end{align}

The source term $f$ was then fed into FESTIM alongside with the initial and boundary conditions described below:
\begin{subequations}
    \begin{align}
        c_\mathrm{m}(x, y, t) &= c_M(x, y, t), \text{ on } \partial \Omega \\
        c_\mathrm{m}(x, y, t=0) &= c_M(x, y, t=0), \text{ on } \Omega
    \end{align}
\end{subequations}

The computed solution $c_\mathrm{comp}$ can then be compared with the manufactured solution $c_M$ in order to quantitatively measure the numerical error.
After running the MMS process, the computed solution and the manufactured solution were in very good agreement at several arbitrarily chosen times of simulation (see Figure \ref{fig: results MMS}).
% The stepsize was $\Delta t=0.01$.
The absolute difference between the manufactured solution and the computed one was found to be zero on the boundary and maximum at the interface between the two materials.
This is explained by the Dirichlet boundary conditions enforcing the computed solution on the boundary.
This difference decreases by increasing the mesh refinement and decreasing the stepsize.
Nonetheless, the error was found to remain orders of magnitude lower than the actual solution.

\begin{figure*}
    \centering
    \begin{subfigure}{0.33\linewidth}
        \centering
        \includegraphics[width=\linewidth]{Figures/Chapter3/monoblocks/interface_condition/u_computed_t0.01.pdf}
        \caption{Computed solution $c_\mathrm{comp}(t=0.01)$}
    \end{subfigure}%
    \begin{subfigure}{0.33\linewidth}
        \centering
        \includegraphics[width=\linewidth]{Figures/Chapter3/monoblocks/interface_condition/u_exact_t0.01.pdf}
        \caption{Exact solution $c_M(t=0.01)$}
    \end{subfigure}%
    \begin{subfigure}{0.33\linewidth}
        \centering
        \includegraphics[width=\linewidth]{Figures/Chapter3/monoblocks/interface_condition/diff_t0.01.pdf}
        \caption{Absolute difference}
    \end{subfigure}
    \begin{subfigure}{0.33\linewidth}
        \centering
        \includegraphics[width=\linewidth]{Figures/Chapter3/monoblocks/interface_condition/u_computed_t0.06.pdf}
        \caption{Computed solution $c_\mathrm{comp}(t=0.06)$}
    \end{subfigure}%
    \begin{subfigure}{0.33\linewidth}
        \centering
        \includegraphics[width=\linewidth]{Figures/Chapter3/monoblocks/interface_condition/u_exact_t0.06.pdf}
        \caption{Exact solution $c_M(t=0.06)$}
    \end{subfigure}%
    \begin{subfigure}{0.33\linewidth}
        \centering
        \includegraphics[width=\linewidth]{Figures/Chapter3/monoblocks/interface_condition/diff_t0.06.pdf}
        \caption{Absolute difference}
    \end{subfigure}
    \caption{Comparison of concentration fields simulated by FESTIM with manufactured solutions}
    \label{fig: results MMS}
\end{figure*}


% The chemical potential continuity at interface is not directly equivalent to a concentration continuity (see Equation \ref{eq: cconservation})  since the chemical potential in a reference state is not the same in different materials \sidecite{kirchheim_25_2014}, as the lattice site concentration.
% \begin{equation}
%             c^- = c^+  \label{eq: cconservation} 
% \end{equation}

In order to assess the influence of interface conditions on the outgassing flux, simulations are performed with chemical potential continuity (Equation \ref{eq: c/s conservation}) or mobile concentration continuity assuming in both cases flux conservation (Equation \ref{eq: flux conservation}).
For the sake of simplicity and to emphasis on the influence of interface conditions, no trapping was assumed and idea Dirichlet boundary conditions were set.
Simulations were performed on two test cases: W/Cu and Cu/EUROFER.
The materials properties used for the simulations can be found in Table \ref{tab:materials properties_1}.
In both cases the solute concentration $c_\mathrm{m}$ was set to \SI{1e20}{m^{-3}} at $x=0$
 and zero on the other boundary.


\subsection{W/Cu case}
A \SI{4}{mm}-thick slab made of \SI{2}{mm} of W (referred as $\Omega_1$) and \SI{2}{mm} of Cu (referred as $\Omega_2$)  at $T=\SI{500}{K}$ was first simulated.



\begin{figure*}
    \centering
    \subfloat[Solute profiles at steady state (W/Cu)]{%
        \label{fig: solute profiles w cu}
        \begin{overpic}[width=0.5\linewidth]{Figures/Chapter3/monoblocks/interface_condition/w_cu/solute_profiles.pdf}
            \put(35, 55){W}
            \put(65, 35){Cu}
        \end{overpic}
    }
    \subfloat[Outgassing flux (W/Cu)]{%
        \label{fig: outgassing flux w cu}
        \includegraphics[width=0.5\linewidth]{Figures/Chapter3/monoblocks/interface_condition/w_cu/comparison_fluxes.pdf}
    } \\
    \subfloat[Solute profiles at steady state (Cu/EUROFER)]{%
        \label{fig: solute profiles cu eurofer}
        \begin{overpic}[width=0.5\linewidth]{Figures/Chapter3/monoblocks/interface_condition/cu_eurofer/solute_profiles_cu_eurofer.pdf}
            \put(35, 55){Cu}
            \put(65, 35){EUROFER}
        \end{overpic}
     }
    \subfloat[Outgassing flux (Cu/EUROFER)]{%
        \label{fig: outgassing flux cu eurofer}
        \includegraphics[width=0.5\linewidth]{Figures/Chapter3/monoblocks/interface_condition/cu_eurofer/comparison_fluxes_cu_eurofer.pdf}
     }
    \caption{Influence of chemical potential conservation}
    \label{fig: influence of mu bi material}
\end{figure*}

In this case, the conservation of chemical potential resulted in higher steady-state concentration gradients (see Figure \ref{fig: solute profiles w cu}) due to the higher solubility of Cu.
For a given temperature (and therefore given diffusion coefficients), this implied an increase of the outgassing flux (see Figure \ref{fig: outgassing flux w cu}) compared to the case with concentration continuity.

\subsection{Cu/EUROFER case}

A \SI{4}{mm}-thick slab made of \SI{2}{mm} of Cu (referred as $\Omega_1$) and \SI{2}{mm} of EUROFER (referred as $\Omega_2$) at $T=\SI{600}{K}$ was simulated.
Contrarily to the previous case, since the solubility of EUROFER is lower than that of Cu, the steady-state concentration gradients (see Figure \ref{fig: solute profiles cu eurofer}) were lower compared to the case with concentration continuity.
The outgassing flux was therefore lower in the case of chemical potential conservation (see Figure \ref{fig: outgassing flux cu eurofer}).



\subsection{Identification technique}
Assuming the diffusion coefficients are known, the solubility coefficients in both materials can be identified by determining an effective diffusion coefficient.

The steady-state flux $\varphi_\infty$ can be expressed from Equations \ref{eq:MES c} and \ref{eq: MES c coefficients} as follow:
\begin{subequations}
    \begin{align}
        \varphi_\infty &= -D_2 \nabla c_2 =  -D_1 \nabla c_1 \\
        &= -D_2 a_2 \frac{c_\mathrm{m}(x=0)}{L} = -D_1 a_1 \frac{c_\mathrm{m}(x=0)}{L}\\
        &= -a_0 D_\mathrm{eq} \frac{c_\mathrm{m}(x=0)}{L}
    \end{align}
    \label{eq: steady state flux 1}
\end{subequations}

If this method allows to measure solubilities, it is however not always convenient considering the time required to reach steady-state.
One way to overcome this difficulty is to compute an equivalent effective diffusion coefficient noted $D_\mathrm{eff}$.
$D_\mathrm{eff}$ is computed by assuming an homogeneous material and a linear steady state profile.
The steady-state flux can therefore be written as:
\begin{subequations}
    \begin{align}
        \varphi_\infty &= -D_\mathrm{eff} \nabla c \\
        &= -D_\mathrm{eff} \frac{c_\mathrm{m}(x=L) - c_\mathrm{m}(x=0)}{L} \\
    \end{align}
    \label{eq: steady state flux 2}
\end{subequations}

By combining Equations \ref{eq: steady state flux 1} and \ref{eq: steady state flux 2}, $D_\mathrm{eff}$ reads:
\begin{equation}
    D_\mathrm{eff} = \frac{a_0 \; D_\mathrm{eq} \; c_\mathrm{m}(x=0) }{c_\mathrm{m}(x=L) - c_\mathrm{m}(x=0)}
\end{equation}

By fitting measurements of the outgassing flux with either an analytical transient solution or a simulation code (see Figures \ref{fig: outgassing flux w cu} and \ref{fig: outgassing flux cu eurofer}), one can estimate $D_\mathrm{eff}$ and therefore the coefficient $a_0$ which can finally be correlated to material properties (see Equation \ref{eq: MES c coefficients}).

However, some discrepancies were found between this method and the actual outgassing curve during the transient phase.
One way of getting rid of these is to fit the curve with an analytical solution of transient mass transfer in a 1D composite slab with conservation of chemical potential.
This can be utterly complex and is well beyond the scope of this study.

Moreover, surface effects and the presence of traps often complicate the analysis of the experimental data.
Therefore, a more thorough identification technique would be to use embedded hydrogen transport codes such as FESTIM in a parametric optimisation algorithm as described in previous work \sidecite{delaporte-mathurin_parametric_nodate}.
Such a process could be able to determine materials properties such as diffusion coefficients, solubilities and trap densities.

This implementation of chemical potential continuity was applied to a test case in order to simulate hydrogen transport an ITER-like monoblock.
For this application, trapping of hydrogen in defects was taken into account.
The governing equations for hydrogen transport therefore read:

\begin{subequations}
    \begin{align}
        \frac{\partial c_\mathrm{m}}{\partial t} &=\nabla \cdot\left(D(T) \nabla c_\mathrm{m}\right) + \varphi_\mathrm{imp} \cdot U(\textbf{x}) -\sum \frac{\partial c_{\mathrm{t}, i}}{\partial t} \\
        \frac{\partial c_{\mathrm{t}, i}}{\partial t} &=k(T) \cdot c_\mathrm{m} \cdot\left(n_{i}-c_{\mathrm{t}, i}\right)-p(T) \cdot c_{\mathrm{t}, i}
    \end{align}
\end{subequations}

where ${D(T)=D_0 \cdot \exp\big(-E_\mathrm{diff}/ (k_B \cdot T )\big)}$ is the diffusion coefficient in \si{m^2.s^{-1}}, $T$ the temperature in $\si{K}$ and ${k_B = \SI{8.617e-5}{eV.K^{-1}}}$ the Boltzmann constant, $c_{\mathrm{t}, i}$ is the concentration in \si{m^{-3}} of particles trapped in the i-th trap,
\begin{equation}
    k(T)=k_0\exp{\big(-E_{k} / (k_B \cdot T ) \big)}
\end{equation} and 
\begin{equation}
    p(T)=p_0\exp{\big(-E_{p}/ (k_B \cdot T )\big)}
\end{equation} 
are the trapping and detrapping rates expressed in \si{m^3.s^{-1}} and \si{s^{-1}} respectively, $n_i$ is the trap density in \si{m^{-3}}, $\varphi_\mathrm{imp}$ is the implanted particle flux in \si{m^{-2}.s^{-1}} and $U(\textbf{x})$ is the spatial distribution of the implanted particle flux.

The thorough description of this model as well as the verification of its implementation in FESTIM is given in \sidecite{delaporte-mathurin_finite_2019}.

The source term $\varphi_\mathrm{imp}$ is equal to \SI{5e23}{m^{-2}.s^{-1}} and its spatial distribution is:

\begin{equation}
    U(\textbf{x}) = \begin{cases}
    R_p^{-1},& \text{ if } x < R_p\\
    0,& \text{ else }
    \end{cases}
\end{equation}
where $R_p = \SI{2.5}{nm}$ is the implantation range.
Since the implantation range is very small compared to the monoblock dimensions, this source is equivalent to applying a Dirichlet boundary condition on the exposed surface \sidecite{delaporte-mathurin_parametric_2020}.
The value of this boundary condition therefore depends on $\varphi_\mathrm{imp}$, the implantation range $R_p$ and the diffusion coefficient $D$.


The chemical potential is conserved across interfaces by ensuring Equations \ref{eq: flux conservation} and \ref{eq: c/s conservation} where 
\begin{equation}
    {S = S_0 \cdot \exp\big(-E_S/ (k_B \cdot T )\big)}
\end{equation} is the solubility coefficient in \si{m^{-3}.Pa^{-0.5}}.

A comparison test was first performed and the FESTIM code was compared to TMAP7 \sidecite{longhurst_tmap7_2008} and Abaqus on a 1D case.
2D simulations of ITER-like monoblocks were then performed.
The influence of interface conditions on hydrogen inventory was studied in both cases.

The materials properties that have been set are detailed in Table \ref{tab:materials properties_1} and their thermal dependency is shown on Figure \ref{fig:properties_1}.
The trap properties in each material are detailed in Table \ref{tab:traps monoblock_1}.
All traps are homogeneously distributed in the materials, except for Trap 2 which is only located in the first micrometre behind the plasma facing surface $\Gamma_\mathrm{top}$ (see Figure \ref{fig: monoblock geometry}) to account for damage creation.


\begin{figure*}
     \subfloat[1D geometry \label{fig: monoblock 1D geometry}]{%
        \begin{overpic}[width=0.5\linewidth]{Figures/Chapter3/monoblocks/interface_condition/iter case/Monoblock 1D.pdf}
            \put(40, 50){\SI{6}{mm}}
            \put(40, 8){W}
            \put(62, 50){\SI{1}{mm}}
            \put(65, 8){Cu}
            \put(72, 50){\SI{1.5}{mm}}
            \put(72, 8){CuCrZr}
            \put(6, 25){\large$\Gamma_\mathrm{top}$}
            \put(85, 25){\large$\Gamma_\mathrm{coolant}$}
        \end{overpic}
     }
     \subfloat[2D geometry  \label{fig: monoblock 2d geometry}]{%
        \begin{overpic}[width=0.5\linewidth]{Figures/Chapter3/monoblocks/interface_condition/iter case/monoblock_sketch.pdf}
            \put(42, 5){\SI{28}{mm}}
            \put(97, 50){\SI{28}{mm}}
            \put(10, 32){\SI{13.5}{mm}}
            \put(42, 62){ \diameter \SI{12}{mm}}
            \put(42, 71){ \diameter \SI{15}{mm}}
            \put(42, 76){ \diameter \SI{17}{mm}}
            \put(20, 80){\large$\Gamma_\mathrm{top}$}
            \put(4, 60){\large$\Gamma_\mathrm{lateral}$}
            \put(78, 60){\large$\Gamma_\mathrm{lateral}$}
            \put(40, 41){\large$\Gamma_\mathrm{coolant}$}
        \end{overpic}
     }
     \caption{Monoblock geometry showing W armour \cruleme[grey]{0.3cm}{0.3cm}, Cu interlayer \cruleme[orange]{0.3cm}{0.3cm}, CuCrZr alloy cooling pipe  \cruleme[yellow]{0.3cm}{0.3cm}}\label{fig: monoblock geometry}
\end{figure*}

\begin{table*}
    \centering
    \begin{tabular}{p{1.7cm}  R{3cm}  R{3cm}  R{1.8cm}  R{1cm} R{1.8cm}  R{1cm}}
         & \multicolumn{2}{c}{Thermal properties} & \multicolumn{4}{c}{Hydrogen transport properties}\\
        \hline
        Material & $\rho \cdot C_p \newline(\si{J.K^{-1}.m^{-3}})$ & $\lambda \newline(\si{W.m^{-1}.K^{-1}})$ & $D_0 \newline(\si{m^2.s^{-1}})$ & $E_\mathrm{diff} \newline(\si{eV})$ & $S_0 \newline(\si{m^{-3}.Pa^{-0.5}})$ & $E_\mathrm{S} \newline(\si{eV})$\\
        \hline
        \\
        W \cite{frauenfelder_solution_1969}& %
        $5.1\times 10^{-6} \cdot T^3 \newline - 8.3\times 10^{-2}\cdot T^2 \newline + 6.0 \times 10^{2}\cdot T \newline +2.4\times 10^6$ &%
        $-7.8\times 10^{-9}\cdot T^3 \newline %
        +5.0\times 10^{-5}\cdot T^2 \newline%
        -1.1\times 10^{-1} \cdot T \newline%
        +1.8\times 10^{2}$ &%
        $2.4\times 10^{-7}$ & 0.39 &%
        $1.87\times 10^{24}$ & 1.04\\
        \\
        Cu \cite{reiter_compilation_1996}&%
        $1.7\times 10^{-4}\cdot T^3\newline %
        +6.1\times 10^{-2}\cdot T^2\newline %
        +4.7\times 10^2\cdot T\newline %
        +3.5\times 10^6$ &%

        $-3.9\times 10^{-8}\cdot T^3\newline %
        +3.8\times 10^{-5}\cdot T^2\newline %
        -7.9\times 10^{-2}\cdot T\newline %
        +4.0\times 10^2 $&%

        $6.6\times 10^{-7}$ &%
        0.39&%
        $3.14\times 10^{24}$ & 0.57\\
        \\
        CuCrZr \cite{serra_hydrogen_1998}& %
        $-1.8\times 10^{-4}\cdot T^3 \newline %
        +1.5\times 10^{-1}\cdot T^2\newline %
        +6.2\times 10^2\cdot T\newline %
        +3.5\times 10^6$ &%

        $5.3\times 10^{-7}\cdot T^3\newline %
        -6.5\times 10^{-4}\cdot T^2\newline %
        +2.6\times 10^{-1}\cdot T\newline %
        +3.1\times 10^2$ & %

        $3.9\times 10^{-7}$ & %
        0.42&%
        $4.28\times 10^{23}$ & 0.39\\
        \\
        EUROFER \cite{aiello_hydrogen_2002} & %
        - & - &
        $1.5\times 10^{-7}$ & %
        0.15 & %
        $6.14\times 10^{20}$ & 0.25
        \\
        \\
    \end{tabular}
    \caption{Materials properties used in the simulations. Thermal properties are fitted from ANSYS.}
    \label{tab:materials properties_1}
\end{table*}

\begin{figure}
    \centering
    \begin{subfigure}{0.75\linewidth}
        \centering
        \includegraphics[width=\linewidth]{Figures/Chapter3/monoblocks/interface_condition/iter case/thermal_prop.pdf}
        \caption{Thermal properties}
    \end{subfigure}
    \begin{subfigure}{0.75\linewidth}
        \centering
        \includegraphics[width=\linewidth]{Figures/Chapter3/monoblocks/interface_condition/H_properties.pdf}
        \caption{H transport properties}
    \end{subfigure}
    \caption{Material properties used in the simulations \cite{frauenfelder_solution_1969, reiter_compilation_1996, serra_hydrogen_1998, aiello_hydrogen_2002}}
    \label{fig:properties_1}
\end{figure}

\begin{table*}
    \centering
    \begin{tabular}{L{1.5cm} L{1.5cm} R{1.6cm} R{1.1cm} R{1.6cm} R{1.1cm} R{2cm}}
         & Material & $k_0 (\si{m^3.s^{-1}})$ &  $E_k (\si{eV})$ & $p_0 (\si{s^{-1}})$ & $E_p (\si{eV})$ & $n_i (\si{at.fr.})$ \\
        \hline
        \\
        Trap 1 & W & $3.8 \times 10^{-17}$ & 0.39 & $8.4 \times 10^{12}$& 1.20 & $5.0 \times 10^{-4}$ \\
        \\
       Trap 2 & W & $3.8 \times 10^{-17}$ & 0.39 & $8.4 \times 10^{12}$& 1.40 & $5.0 \times 10^{-3}$ \\
        \\
        Trap 3 & Cu & $6.0 \times 10^{-17}$ & 0.39 & $8.0 \times 10^{13}$ & 0.50 &$5.0 \times 10^{-5}$\\
        \\
        Trap 4 & CuCrZr & $1.2\times 10^{-16}$ & 0.42 & $8.0 \times 10^{13}$ & 0.50 &$5.0 \times 10^{-5}$\\
        \\
        Trap 5 & CuCrZr & $1.2\times 10^{-16}$ & 0.42 & $8.0 \times 10^{13}$ & 0.83 &$4.0 \times 10^{-2}$\\
        \\
    \end{tabular}
    \caption{Traps properties used in the simulations \cite{hodille_macroscopic_2015, dolan_assessment_1994}}
    \label{tab:traps monoblock_1}
\end{table*}


\subsection{1D case and comparison with TMAP7 and Abaqus}

\begin{figure}
    \centering
    \begin{subfigure}{0.5\linewidth}                              
        \includegraphics[width=\linewidth]{Figures/Chapter3/monoblocks/interface_condition/iter case/temperature_1D.pdf}
        \caption{1D case}
        \label{fig: 1D temperature}
    \end{subfigure}%
    \begin{subfigure}{0.5\linewidth}                          
        \includegraphics[width=\linewidth]{Figures/Chapter3/monoblocks/interface_condition/iter case/temperature_field_2d.pdf}
        \caption{2D case}
        \label{fig: 2D temperature}
    \end{subfigure}%
    \caption{Monoblock temperature simulated by FESTIM}
    \label{fig: temperature}
\end{figure}

\begin{figure}
    \centering
    \begin{subfigure}{1\linewidth}
        \includegraphics[width=\linewidth]{Figures/Chapter3/monoblocks/interface_condition/iter case/comparison_inventory_1d.pdf}
        \caption{1D case}
        \label{fig: 1D inventories}
    \end{subfigure}
    \begin{subfigure}{1\linewidth} 
        \includegraphics[width=\linewidth]{Figures/Chapter3/monoblocks/interface_condition/iter case/comparison_inventory_2d.pdf}
        \caption{2D case}
        \label{fig: 2D inventories}
    \end{subfigure}
    \caption{Influence of chemical potential conservation on hydrogen inventory}
\end{figure}

\begin{figure}
    \centering
    \includegraphics[width=\linewidth]{Figures/Chapter3/monoblocks/interface_condition/iter case/comparison_profiles.pdf}
    \caption{Comparison of concentrations profiles at several times}
    \label{fig: concentrations profiles 1D}
\end{figure}

\begin{figure}
    \centering
    \includegraphics[width=\linewidth]{Figures/Chapter3/monoblocks/interface_condition/iter case/comparison_codes.pdf}
    \caption{Comparison of results provided by FESTIM, TMAP7 and ABAQUS}
    \label{fig: code comparison}
\end{figure}

The 1D simulation case is a \SI{8.5}{mm}-thick composite slab made of W, Cu and CuCrZr (see Figure \ref{fig: monoblock 1D geometry}).
The plasma facing surface $\Gamma_\mathrm{top}$ is located at $x=\SI{0}{mm}$ and the surface cooled by water $\Gamma_\mathrm{coolant}$ is located at $x=\SI{8.5}{mm}$.


The boundary conditions are detailed in Equation \ref{eq: 1D BCs}.

\begin{subequations}
    \begin{align}
    T &= \SI{1200}{K}\quad \text { on } \Gamma_\mathrm{top}\\
    c_\mathrm{m} &=  \frac{\varphi_\mathrm{imp} \cdot R_p}{D} \quad \text { on } \Gamma_\mathrm{top}\\
    T &= \SI{373}{K} \quad \text { on } \Gamma_\mathrm{coolant}\\
    -D \nabla c_\mathrm{m} \cdot \vec{n} &= K_\mathrm{CuCrZr} \cdot c_\mathrm{m}^{2} \quad \text { on } \Gamma_\mathrm{coolant}  
    \end{align}
    \label{eq: 1D BCs}
\end{subequations}
with $\varphi_\mathrm{imp} = \SI{5e23}{m^{-2}.s^{-1}}$ the implanted particle flux, $R_p = \SI{1.25}{nm}$ the implantation depth, $\vec{n}$ the normal vector and $K_\mathrm{CuCrZr} = 2.9 \times 10^{-14}\cdot \exp{(-1.92/(k_B\cdot T))}$ the recombination coefficient of the CuCrZr (in vacuum) expressed in \si{m^4.s^{-1}} \sidecite{anderl_deuterium_1999}.

The Dirichlet boundary condition on $\Gamma_\mathrm{top}$ for the hydrogen transport corresponds to a flux balance between the implanted flux and the flux that is retro-desorbed at the surface.
The details can be found in \sidecite{delaporte-mathurin_parametric_2020}.

A comparison test was made with the codes TMAP7 and Abaqus with this set of parameters and very good agreement was found between the two codes (see Figure \ref{fig: code comparison}).



Two simulations were run, one ensuring mobile concentration $c_\mathrm{m}$ continuity at interfaces and the other ensuring the continuity of chemical potential $\mu$.

Up to \SI{5e6}{s}, the total hydrogen inventory was found to be insensitive to the conservation of chemical potential (see Figure \ref{fig: 1D inventories}).
It is only after this implantation that the inventory of the simulation with continuity of chemical potential started to diverge.
At $t=\SI{2.4e7}{s}$, the inventory with continuity of $\mu$ was more than two times higher than that of the one with continuity of $c_\mathrm{m}$.
This is explained by the high solubility ratio between Cu and CuCrZr leading to a higher concentration of mobile particles in CuCrZr and therefore a higher trapping rate.
Before reaching the W/Cu interface, the $c_\mathrm{m}$ and retention profiles are identical regardless of the interface condition (see Figure \ref{fig: concentrations profiles 1D}).
Once this interface is reached, the $c_\mathrm{m}$ profiles are affected by the interface condition.
However, even then, the trap density in Cu being low compared to other materials, the global inventory is not affected much.
For these two reasons, the inventories are identical before \SI{5e6}{s}.

\subsection{2D case}
The boundary conditions for the 2D case are similar with an additional Dirichlet boundary condition on $\Gamma_\mathrm{lateral}$.
This boundary condition accounts for recombination on the lateral sides of the monoblock.
\begin{subequations}
    \begin{align}
        T &=  \SI{1200}{K}\quad \text { on } \Gamma_\mathrm{top}\\
        c_\mathrm{m} &=  \frac{\varphi_\mathrm{imp} \cdot R_p}{D}\quad \text { on } \Gamma_\mathrm{top}\\
        T &= \SI{373}{K}\quad \text { on } \Gamma_\mathrm{coolant}\\
        -D \nabla c_\mathrm{m} \cdot \vec{n} &= K_\mathrm{CuCrZr} \cdot c_\mathrm{m}^{2} \quad \text { on } \Gamma_\mathrm{coolant} \\
        c_\mathrm{m} &= 0 \quad \text { on } \Gamma_\mathrm{lateral}
    \end{align}
\end{subequations}

with $\varphi_\mathrm{imp} = \SI{5e23}{m^{-2}.s^{-1}}$ the implanted particle flux, $R_p = \SI{1.25}{nm}$ the implantation depth, $\vec{n}$ the normal vector and $K_\mathrm{CuCrZr} = 2.9 \times 10^{-14}\cdot \exp{(-1.92/(k_B\cdot T))}$ the recombination coefficient of the copper alloy (in vacuum) expressed in \si{m^4.s^{-1}} \sidecite{anderl_deuterium_1999}.


\begin{figure}s
    \centering
    \begin{subfigure}{0.5\linewidth}
        \centering
        \includegraphics[width=\linewidth]{Figures/Chapter3/monoblocks/interface_condition/iter case/solute_c.pdf}
        \caption{$c_\mathrm{m}$ (continuity of $c_\mathrm{m}$)}
    \end{subfigure}%
    \begin{subfigure}{0.5\linewidth}
        \centering
        \includegraphics[width=\linewidth]{Figures/Chapter3/monoblocks/interface_condition/iter case/solute_mu.pdf}
        \caption{$c_\mathrm{m}$ (continuity of $\mu$)}
    \end{subfigure}
    \begin{subfigure}{0.5\linewidth}
        \centering
        \includegraphics[width=\linewidth]{Figures/Chapter3/monoblocks/interface_condition/iter case/retention_c.pdf}
        \caption{Retention (continuity of $c_\mathrm{m}$)}
    \end{subfigure}%
    \begin{subfigure}{0.5\linewidth}
        \centering
        \includegraphics[width=\linewidth]{Figures/Chapter3/monoblocks/interface_condition/iter case/retention_mu.pdf}
        \caption{Retention (continuity of $\mu$)}
    \end{subfigure}
    \caption{2D concentration fields at $t=\SI{2.4e7}{s}$}
    \label{fig: concentrations fields 2d}
\end{figure}

The concentrations fields (see Figure \ref{fig: concentrations fields 2d}) showed results similar to those obtained in the 1D case (see Figure \ref{fig: concentrations profiles 1D}).
The interface condition had no influence whatsoever on the mobile particle concentration $c_\mathrm{m}$ in the W.
However, $c_\mathrm{m}$ was higher in Cu and CuCrZr in the case with chemical potential conservation (up to \SI{1.5e24}{m^{-3}} in CuCrZr at $t=\SI{2.4e7}{s}$).
As in the 1D case, this increase of $c_\mathrm{m}$ lead to an increase of the trap occupancy and therefore an increase of the local retention.

The hydrogen inventories in the monoblock were found to be identical for most of the implantation time (see Figure \ref{fig: 2D inventories}).
It was only after \SI{5e6}{s} that the inventory with chemical potential conservation was significantly higher than the one with $c_\mathrm{m}$ continuity at interfaces.

In both the 1D and 2D case, the flux which is retro-desorbed from the monoblock to the plasma does not depend on the interface conditions since interface are far from the exposed surface.
Moreover, outgassing flux through the cooling pipe greatly depends on the boundary condition imposed at the cooling surface.
Therefore, in order to assess the impact of interface conditions on the outgassing flux through the cooling pipe, uncertainties must first be lift regarding the recombination process occurring on surfaces in contact with water.

