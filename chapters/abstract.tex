\chapter*{Abstract}
\addcontentsline{toc}{chapter}{Abstract} % Add the preface to the table of contents as a chapter

\textbf{English} \\
Future fusion reactors will use a mixed fuel of deuterium and tritium.
As a radioactive isotope of hydrogen, tritium can represent a nuclear safety hazard and its inventory in the reactors materials must be controlled.
In ITER, the tritium in-vessel safety limit is \SI{700}{g}.

The tritium inventory of the ITER divertor was numerically estimated with the FESTIM code, which was developed to simulate hydrogen transport in tungsten monoblocks.
A parametric study was performed varying the exposure conditions (surface temperature and surface hydrogen concentration) and a behaviour law was extracted.
This behaviour law provided a rapid way of estimating a monoblock inventory for a given exposure time and for given surface concentration and temperature.
This behaviour law was then used and interfaced with output data from the edge-plasma code SOLPS-ITER in order to estimate the hydrogen inventory of the whole ITER divertor.
Under conservative assumptions, the total hydrogen inventory (deuterium and tritium) was found to be well below the ITER tritium safety limit, reaching $\approx \SI{14}{g}$ after 25 000 pulses of \SI{400}{s}.

To investigate the influence of helium exposure on these results, a helium bubble growth model was developed.
The results of this helium growth model were in good agreement with published numerical results and experimental observations.
A parametric study was performed to investigate the influence of exposure conditions on the bubbles density and size.
To investigate the influence of helium bubbles on hydrogen transport, deuterium TDS experiments of tungsten pre-damaged with helium were then reproduced.
The distribution of bubbles density and size was computed using this helium bubble growth model and the results were used in FESTIM simulations.
It was found that exposing tungsten to helium could potentially reduce the hydrogen inventory by saturating defects, making it impossible for hydrogen to get trapped.
Moreover, the effect of helium bubbles (creation of additional traps for hydrogen) is limited to the near surface region (small compared to the monoblock's scale)

\textbf{French} \\
Les futurs réacteurs à fusion nucléaire utiliseront un combustible composé de deutérium et de tritium.
En tant qu'isotope radioactif de l'hydrogène, le tritium peut représenter un risque en terme de sûreté nucléaire et son inventaire dans les matériaux des réacteurs doit être maîtrisé.
Dans ITER, la limite de sécurité du tritium dans la chambre à vide est de \SI{700}{g}.

L'inventaire de tritium du divertor ITER a été estimé numériquement avec le code FESTIM, qui a été développé pour simuler le transport d'hydrogène dans des monoblocs de tungstène.
Une étude paramétrique a été réalisée en faisant varier les conditions d'exposition (température de surface et concentration en hydrogène de surface) et une loi de comportement a été extraite.
Cette loi de comportement a permis d'estimer rapidement les inventaires monobloc pour un temps d'exposition donné et pour une concentration et une température de surface données.
Elle a ensuite été utilisée et interfacée avec les données de sortie du code edge-plasma SOLPS-ITER afin d'estimer l'inventaire d'hydrogène de l'ensemble du divertor d'ITER.
Avec des hypothèses conservatrices, l'inventaire total d'hydrogène (deutérium et tritium) s'est avéré bien en dessous de la limite de sécurité du tritium d'ITER, atteignant $\approx \SI{14}{g}$ après 25
000 impulsions de \SI{400}{s}.

Pour étudier l'influence de l'exposition à l'hélium sur ces résultats, un modèle de croissance de bulles d'hélium a été développé.
Les résultats de ce modèle de croissance de l'hélium étaient en accord avec les résultats numériques et observations expérimentales déjà publiés.
Une étude paramétrique a été réalisée pour étudier l'influence des conditions d'exposition sur la densité et la taille des bulles.
Afin d'étudier l'influence des bulles d'hélium sur le transport de l'hydrogène, des expériences TDS en deutérium sur du tungstène pré-endommagé à l'hélium ont ensuite été reproduites.
Les distributions de la densité et de la taille des bulles ont été calculées à l'aide de ce modèle de croissance des bulles d'hélium et les résultats ont été utilisés dans les simulations FESTIM.
Il a été constaté que l'exposition du tungstène à l'hélium pouvait potentiellement réduire l'inventaire d'hydrogène en saturant les défauts, rendant impossible le piégeage de l'hydrogène.
De plus, l'effet des bulles d'hélium (création de pièges supplémentaires pour l'hydrogène) est limité au proche région de surface (petite par rapport à l'échelle du monobloc)
