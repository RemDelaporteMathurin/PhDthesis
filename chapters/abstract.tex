\chapter*{Abstract}
\addcontentsline{toc}{chapter}{Abstract} % Add the preface to the table of contents as a chapter

Future fusion reactors will use a mixed fuel of deuterium and tritium.
As a radioactive isotope of hydrogen, tritium can represent a nuclear safety hazard and its inventory in the reactors materials must be controlled.
In ITER, the tritium in-vessel safety limit is \SI{700}{g}.

The tritium inventory of the ITER divertor was estimated using numerical modelling.
The FESTIM code was developed to simulate hydrogen transport in tungsten monoblocks.
A parametric study was performed varying the exposure conditions (surface temperature and surface hydrogen concentration) and a behaviour law was extracted.
This behaviour law provided a rapid way of estimating a monoblock inventory for a given exposure time and for given surface concentration and temperature.
This behaviour law was then used and interfaced with output data from the edge-plasma code SOLPS-ITER.
Under conservative assumptions, the total hydrogen inventory (deuterium and tritium) was found to be well below the ITER tritium safety limit, reaching $\approx \SI{14}{g}$ after 25 000 pulses of \SI{400}{s}.

To investigate the influence of helium exposure on these results, a helium bubble growth model was developed.
The results of this helium growth model were in good aggreement to published numerical results and experimental observations.
A parametric study was performed to investigate the influence of exposure conditions on the bubbles density and size.
To investigate the influence of helium bubbles on hydrogen transport, deuterium TDS experiments of tungten pre-damaged with helium were then reproduced.
The distribution of bubbles density and size was computed using this helium bubble growth model and the results were used in FESTIM simulations.
It was found that exposing tungsten to helium could potentially reduce the hydrogen inventory by saturating the defects.
Moreover, the effect of helium bubbles (creation of additional traps for hydrogen) is limited to the near surface region (small compared to the monoblock's scale)
