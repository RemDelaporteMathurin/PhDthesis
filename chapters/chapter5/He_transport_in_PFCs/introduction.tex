\refch{Chapter4} focussed on the estimation of the tritium inventory in the \gls{iter} \gls{divertor}, taking into account only hydrogen implantation.
However, the \gls{divertor} of a \gls{tokamak} will not only be exposed to hydrogen: it will also be bombarded by high energy helium ions.

This Chapter will thererfore focus on determining the effect of helium on hydrogen transport and its impact on the conclusions made in \refch{Chapter4}.

It will first assess the different sources of helium in a tungsten \gls{divertor}, which are the direct implantation of helium ions, the production of helium from tritium decay, and the production of helium from \gls{transmutation}.

Then, a helium bubble growth model will be presented and applied to different exposure conditions.
This model will be compared to published numerical results and experimental data.

Finally, based on the results of this new model, experiments investigating the effect of helium transport on hydrogen trapping will be reproduced.
The final conclusion will determine if the results obtained in previous chapters are jeopardised.
