In fusion devices, extreme fluxes of helium (He) and hydrogen (H) are expected.
These fluxes will be mostly located on the tungsten (W) divertor which will also exhaust the plasma He ashes.

Due to a strong W-He repulsion, interstitial He in W tend to form clusters \sidecite{becquart_density_2009, becquart_migration_2006}.
Eventually, when mobile clusters reach a critical size, trap-mutation (also called self-trapping) will occur.
Frenkel pairs (self-interstitial W atom and vacancy) will be produced and the mobile clusters will sit in the created vacancies \sidecite{boisse_modeling_2014}.
At this point, the clusters are immobile and will continue to grow by absorbing small helium clusters and serve as nuclei for bubble formations.
When the cluster is over-pressurised, additional Frenkel pairs will be created.
He bubbles can then form in W and their morphology depends on the exposure conditions \sidecite{taylor_investigating_2019, qin_helium_2019, lemahieu_h/he_2016}.
Such bubbles have been observed using Molecular Dynamics (MD)  \sidecite{hammond_helium_2019, hamid_molecular_2019, bergstrom_molecular_2017, maroudas_helium_2016, sefta_surface_2013} and Object Kinetic Monte Carlo \sidecite{valles_temperature_2017, de_backer_modeling_2015}.
He bubbles can alter the mechanical properties of W \sidecite{das_hardening_2019, nguyen_modeling_2019} and reduce the thermal properties of components \sidecite{qu_degradation_2018, cui_thermal_2017}.
Eventually, when over-pressurised bubbles are close to the surface, bursting can occur as shown by Sefta \textit{et al} in MD simulations \sidecite{sefta_helium_2013}.
Bursting greatly modifies surfaces morphology by increasing roughness and producing craters \sidecite{ialovega_hydrogen_2020} but also W-fuzz \sidecite{ de_temmerman_effect_2015, mccarthy_enhanced_2020, khan_helium_2020, bernard_temperature_2017}.
Moreover, He exposure can alter H retention in W \cite{ialovega_hydrogen_2020, markelj_hydrogen_2017, ogorodnikova_deuterium_2011, baldwin_effect_2011, miyamoto_microscopic_2011, ueda_simultaneous_2009}.

An effort has been made to assess He transport in W using atomistically informed macroscopic models called cluster dynamics models implemented using finite differences.
Some examples of such implementations are the work of Faney \textit{et al} \sidecite{faney_spatially_2014, faney_spatially_2015} and the Xolotl code \sidecite{blondel_continuum-scale_2018}.
The simulated results are promising but these codes require substantial computational resources considering the thousands of equations that need to be solved.
The current study therefore proposes an approach to further simplify these models so that they can be easily implemented in finite element codes and later coupled to H transport modelling codes such as FESTIM.
This could serve as a base for H-He coupled simulations to better assess H transport in plasma facing materials and couple with additional physics like heat transfer.

The simplified model presented in this work is applied on a simple case and compared with existing results of the literature to ensure the model is not over-simplified.
Then, the influence of exposure conditions is investigated by running a parametric study varying temperature and implanted particle flux.
The results of this parametric study are analysed using a regression method previously employed\sidecite{delaporte-mathurin_parametric_2020}.
Experiments are then conducted to quantitatively assess the He bubble density and size in He irradiated W.
The current model is finally compared to these experimental results.
