\setchapterimage[3cm]{seaside}
\chapter{Model description}
\label{Chapter2} % For referencing the chapter elsewhere, use \ref{Chapter2} 
\section{H transport} \label{description_H_transport_model}
\subsection{Bulk model}
As in most of MRE models, HIs are split in several populations which are the mobile and the trapped ones in the i-th trap, described using their concentration (respectively $c_\mathrm{m}$ and $c_{\mathrm{t},i}$).
The spatio temporal evolution of these concentrations are commonly described by the following reaction-diffusion system:

\begin{equation}
    \frac{\partial c_\mathrm{m}}{\partial t}=\vec{\nabla} \cdot\left(D(T) \vec{\nabla}c_\mathrm{m}\right)+\Gamma-\sum \frac{\partial c_{\mathrm{t}, i}}{\partial t}
    \label{eq:mobile}
\end{equation}

\begin{equation}
    \frac{\partial c_{\mathrm{t}, i}}{\partial t}=k(T) \cdot c_\mathrm{m} \cdot\left(n_{i}-c_{\mathrm{t}, i}\right)-p(T) \cdot c_{\mathrm{t}, i}
    \label{eq:trapped}
\end{equation}

In Equation \ref{eq:mobile}, ${D(T)=D_0 \exp\big(\frac{-E_\mathrm{D}}{k_B \cdot T}\big)}$ is the diffusion coefficient in \si{m^2.s^{-1}}, $T$ being the temperature in $\si{K}$ and ${k_B = 8.617 \times 10^{-5} \si{eV.K^{-1}}}$ the Boltzmann constant, $\Gamma$ is the volumetric source term of particles in \si{m^{-3}.s^{-1}}, $k(T)=k_0\exp{\big(\frac{-E_{k, i}}{k_B \cdot T}\big)}$ and $p(T)=p_0\exp{\big(\frac{-E_{p, i}}{k_B \cdot T}\big)}$ are the trapping and detrapping rates expressed in \si{m^3.s^{-1}} and \si{s^{-1}} respectively and $n_i$ is the trap density in \si{m^{-3}}.
The unit of the different concentration, $c_\mathrm{m}$ and $c_{\mathrm{t},i}$ are in $ \si{m^{-3}}$ but can be expressed in atomic fraction (at.fr.) by normalising them to the atomic density of the material.
\subsection{Boundary conditions}

Several boundary conditions will be employed in order to constrain either the actual solution (Dirichlet) or the solution gradient (Neumann, Robin) at the domain's boundaries.

\subsubsection{Dirichlet boundary conditions}

Concentration can be fixed at the surface by applying Dirichlet boundary conditions as described in Equation \ref{eq:dirichlet bc}.
\begin{equation}
    c_\mathrm{m} = c_0(x, t) \quad \text { on } \partial \Omega
    \label{eq:dirichlet bc}
\end{equation}
where $\partial \Omega$ is the domain boundary.

$c_0$ can be calculated from Sievert's law of equilibrium and Equation \ref{eq:dirichlet bc} reads:
\begin{equation}
    c_\mathrm{m} = S(T) \sqrt{P}\quad \text { on } \partial \Omega
\end{equation}
where $P$ is the partial pressure of hydrogen at the boundary in \si{Pa}, $S(T)=S_0 \exp(\frac{-E_S}{k_B T})$ is the material solubility in \si{m^{-3}.Pa^{-1/2}} and $T$ is the local temperature in \si{K}.
This law of equilibrium is a steady-state approximation of a more complex model which takes into account flux exchanges between adsorbed and mobile concentrations at the boundary.
It is therefore valid when applied to cases where the kinetics is slow enough for the system to remain at equilibrium.

\subsubsection{Neumann \& Robin boundary conditions}

Concentration gradient can also be constrained on the boundaries as described in Equation \ref{eq: neuman robin bc}.

\begin{equation}
    D(T)\nabla c_\mathrm{m} = f(x, t) \quad \text { on } \partial \Omega
    \label{eq: neuman robin bc}
\end{equation}
where $D(T) = D_0 \exp(\frac{-E_D}{k_B T}) $ is the diffusion coefficient in \si{m^2.s^{-1}}, $T$ is the local temperature in \si{K} and $\partial \Omega$ is the domain boundary.

$f$ can be expressed as a molecular recombination flux and Equation \ref{eq: neuman robin bc} reads:
\begin{equation}
    D(T)\nabla c_\mathrm{m} = K_r(T) c_\mathrm{m}^n \quad \text { on } \partial \Omega
\end{equation}
where $K_r(T) = K_{r_0} \exp(\frac{-E_{K_r}}{k_B T}) $ is the recombination coefficient expressed in \si{m^{3n-2}.s^{-1}} and $n \in \{1, 2\}$ is the order of the recombination.


\subsubsection{Analytical simplification}
triangle without recombination
triangle with recombination

\subsection{Interface conditions}


\section{Heat transfer}
\subsection{Bulk model}

The heat equation is described as follow:
\begin{equation}
    \rho \cdot C_p \frac{\partial T}{\partial t}=\nabla \cdot (\lambda \nabla T)
    \label{eq:heat equation}
\end{equation}
where $\rho$ is the density of the material in \si{kg.m^{-3}}, $C_p$ its specific heat capacity expressed in \si{J.kg^{-1}.K^{-1}} and $\lambda$ the thermal conductivity expressed in \si{W.K^{-1}}.

\subsection{Boundary conditions}
\section{Summary}