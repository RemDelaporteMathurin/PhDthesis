A novel technique for hydrogen transport properties has been demonstrated based on TPD experiments.
By embedding FESTIM in a parametric optimisation routine, one can rapidly (from a few dozens of minutes to a few hours for complex cases) reproduce experimental data and therefore identify parameters such as trapping sites binding energies and densities.

The performances of several optimisation algorithms have been studied and the Nelder-Mead algorithm seems to be the most efficient.
This hydrogen transport properties identification technique has been applied to Tungsten, Aluminium, EUROFER and Beryllium with successful outcome.
Moreover, this technique provides a tool for experimenters to rapidly analyse their data and extract information with ease.
Limitations of this technique such as the choice of the number of trapping sites or the existence of several local minima have been shown.
Including other experiments that would provide profilometry data (such as NRA or SIMS) in order to constrain the optimisation algorithm is a way of improving the accuracy of the results.

This tool could be employed in the future to reproduce permeation experiments or identify surface properties such as recombination coefficients or solubilities.