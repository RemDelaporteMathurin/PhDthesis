\setchapterimage{west_div_roux_cropped}
\setchapterpreamble[u]{\margintoc}
\chapter{Divertor inventory estimation}\label{Chapter4}\labch{Chapter4}
\labch{Divertor inventory estimation}

% \section{Introduction}

This Chapter focusses on the estimation of the \gls{H} \gls{inventory} in the \glspl{divertor} of \acrshort{west} and \gls{iter}.
This estimation relies on the \gls{monoblock} behaviour law computed in \refch{Chapter3}.
This behaviour law allows rapid evaluations of the \glspl{monoblock} \gls{H} \gls{inventory} for any exposure condition.
Inputs are taken from \gls{soledge}-EIRENE \cite{bufferand_three-dimensional_2019} and \gls{solps} \cite{kaveeva_solps-iter_2020} plasma simulations.
The influence of several control parameters is investigated: the input power (i.e.\ how much heating power is injected in the plasma), the gas puffing rate, and the \gls{divertor} pressure of neutral particles in \gls{iter}.
Gas puffing is used in most \glspl{tokamak} to locally increase the \gls{plasma} density \cite{zweben_effect_2014}.
One of the advantages of gas puffing is a better coupling of the \gls{icrf} heating with the plasma \cite{zhang_scrape-off_2019}.

\section{Methodology}
To make use of the \gls{monoblock} \gls{inventory} behaviour law, distribution of surface concentrations and surface temperatures along the \glspl{divertor} will be required.
They will be converted from plasma simulations outputs.

\subsection{Plasma simulations}

\begin{figure}[h!]
    \centering
    \includegraphics[width=0.95\linewidth]{Figures/Chapter4/coordinates.pdf}
    \caption{Poloidal cross section of \gls{west} and \gls{iter} showing the \glspl{divertor} in red.}
    \labfig{reactors}
\end{figure}
This Section describes the parameters of the plasma simulations.
These simulations were run with \gls{soledge}-EIRENE for \gls{west} and \gls{solps} for \gls{iter}.
In a nutshell, these codes solve, for each species (ions and electrons), the particle density, velocity and temperature.
The equations at stake are comparable to the Navier-Stokes equations coupled to the heat equation and interactions with the electromagnetic fields in the \gls{plasma} \sidecite{bufferand_numerical_2015}.
Subsequently, the incident particle fluxes, heat fluxes and particle energy can be calculated along the \gls{tokamak} wall.
For the \gls{soledge}-EIRENE runs, the puffing rate and the input power were used as control parameters.
For \gls{solps} calculation, the \gls{divertor} neutral pressure is the control parameter.

\subsubsection{\gls{soledge}-EIRENE runs}
The Lower-Single-Null magnetic configuration (i.e.\ a signle \gls{x-point} in the lower part of the vacuum vessel) used for the 2D simulations in \gls{soledge}-EIRENE transport code (v588.165) are based on the experimental \gls{west} \gls{plasma} discharge \#54903 at $T_\mathrm{flat-top} = \SI{8}{s}$ (see \reffig{reactors}).
In order to get as many \gls{divertor} conditions as possible, the puffing rate was varied from \SI{4.5e20}{molecule.s^{-1}} to \SI{4.72e21}{molecule.s^{-1}} and the input power from \SI{0.449}{MW} to \SI{2.5}{MW}.
The setup parameters of the simulation are listed in \reftab{soledge parameters}.
$R_\mathrm{wall}$ is the recycling coefficient of main chamber wall, $R_\mathrm{pump}$ is the recycling coefficient of the pump, $D$ is the cross-field mass diffusivity perpendicular to the flux surface, $\nu$ is the momentum diffusivity, $\chi_e$ and $\chi_i$ are the energy diffusivity for electrons and ions, respectively.
While the value of these coefficients is required for the sake of reproducibility, their detailed description \sidecite{ciraolo_first_2019} is outside the scope of this research.
The gas puff position is set inside the \gls{private flux region} (i.e.\ the region between the two \glspl{strike point}) and the pump position is set under the baffle.

\begin{table}[!ht]
    \centering
    \caption{Setup parameters used in the \gls{soledge} simulations.}
    \begin{tabular}{L{0.4\linewidth}  R{0.4\linewidth}}
    \hline \\
    Plasma composition & Deuterium, no impurity \\
    \\
    Recycling coefficients &  $R_\mathrm{wall} = 0.99$ \\
     & $R_\mathrm{pump} = 0.95$ \\
    \\
    SOL input power & from \SI{0.449}{MW} to \SI{2.5}{MW} \\
    \\
    Gas puffing rate & from \SI{4.5e20}{molecule.s^{-1}} to \SI{4.72e21}{molecule.s^{-1}} \\
    \\
    Drifts & - \\
    \\
    Transport coefficients & $D = \SI{0.3}{m^2.s^{-1}}$ \\
     & $\nu = \SI{0.3}{m^2.s^{-1}}$ \\
     & $\chi_e = \chi_i = \SI{1.0}{m^2.s^{-1}}$ \\
    \end{tabular}
    \labtab{soledge parameters}
\end{table}


\subsubsection{\gls{solps} runs}
Several \gls{iter} cases were taken from \sidecite{pitts_physics_2019} with \gls{divertor} neutral pressures varying from \SI{1.8}{Pa} to \SI{11.2}{Pa}.
These \gls{solps} \sidecite{kaveeva_solps-iter_2020} scenarios can be found in the \gls{iter} \gls{imas} database \sidecite{imbeaux_design_2015, park_assessment_2020}.
The nine simulations used in this work are labelled 122396, 122397, 122398, 122399, 122400, 122401, 122402, 122403 and 122404.
These have been run in baseline burning plasma conditions ($Q=10$ with \SI{50}{MW} of input power).


\begin{figure}[h!]
    \centering
    \begin{overpic}[width=\linewidth]{Figures/Chapter4/example.pdf}
        % \linethickness{2pt}
        \thicklines
        \put(55,36){\color{black}\vector(-1, 0){30}}
        \put(25,75){\color{black}\vector(0, -1){30}}
        \put(35,45){\color{black}\vector(1, 1){30}}
    \end{overpic}

    \caption{Method of WEST \gls{divertor} \gls{H} \gls{inventory} estimation based on the surface concentration, the surface temperature and the behaviour law obtained in \refch{Chapter3}.}
    \labfig{behaviour law example}
\end{figure}

\subsection{Estimation of exposure conditions}

\begin{figure*}[h]
    \centering
    \begin{subfigure}{0.5\linewidth}
        \includegraphics[width=\linewidth]{Figures/Chapter4/implantation_range.pdf}
        \caption{Implantation range $R_p$.}
        \labfig{implantation range vs energy}
    \end{subfigure}%
    \begin{subfigure}{0.5\linewidth}                          
        \includegraphics[width=\linewidth]{Figures/Chapter4/reflection_coeff.pdf}
        \caption{Reflection coefficient $r$.}
        \labfig{reflection coeff vs energy}
    \end{subfigure}
    \caption{Evolution of the implantation range and the reflection coefficient as a function of incident energy $E$ and angle of incidence.}
\end{figure*}

According to the behaviour law obtained in \refch{Chapter3}, the temporal evolution of the \gls{H} \gls{inventory} along the \glspl{divertor} can be estimated from the surface concentration of mobile hydrogen and surface temperature (see \reffig{behaviour law example}).
% This inventory distribution can then be projected onto the whole divertor geometry for better visualisation (see \reffig{top view}).

The distribution of the exposure conditions (angles of incidence, particles energies, particles fluxes and heat flux) are produced by \gls{soledge}/\gls{solps} along the \glspl{divertor} of \gls{west} and \gls{iter} (see \reffig{reactors} and \reffig{behaviour law example}).
These exposure conditions are converted into distributions of surface temperature $T_\mathrm{surface}$ and surface hydrogen concentration $c_\mathrm{surface}$ by \refeq{thermal behaviour law} and \refeq{c_surface}.

Note: see \refsec{monoblock thermal behaviour} for more details on the \gls{monoblock} thermal behaviour.
\begin{equation}
    c_\mathrm{surface} = c_\mathrm{surface, \, ions} + c_\mathrm{surface, \, atoms}
    \labeq{c_surface}
\end{equation}
$c_\mathrm{surface, \, ions}$ and $c_\mathrm{surface, \, atoms}$ are the contributions of the ions and atoms to the surface hydrogen concentration.
They can be expressed as:
\begin{equation}
    c_\mathrm{surface, \, i} = \frac{R_{p, \mathrm{i}} \ \varphi_\mathrm{imp, \,i}}{D(T_\mathrm{surface})}
\end{equation}
where $R_{p, i}$ is the implantation depth in \si{m}, $\varphi_{\mathrm{imp}, \,i}$ is the implanted particles flux in \si{m^{-2}.s^{-1}} and $D$ is the \gls{H} diffusion coefficient in \si{m^{2}.s^{-1}} (see \refsec{triangle model}).

Finally, the implanted flux can be expressed as:
\begin{equation}
    \varphi_{\mathrm{imp}, \,i} = (1 - r_\mathrm{i}) \, \varphi_{\mathrm{incident} \, i}
\end{equation}
where $r_i$ is the reflection coefficient and, $\varphi_{\mathrm{incident} \, i}$ is the incident particle flux expressed in \si{m^{-2}.s^{-1}}.

The implantation range $R_p$ and the reflection coefficient $r$ depend on the incident energy and angle of incidence of particles.
These relations can be obtained from \gls{srim} \sidecite{ziegler_srim_2010} simulations (see \reffig{implantation range vs energy}).
It was found that the angle of incidence had low influence on the implantation range.
$R_p$ can therefore be expressed as a function of the incident energy only:
\begin{equation}
    R_p = 1.9\times 10^{-10} E ^{0.59}
    \labeq{implantation range}
\end{equation}
where $E$ is the incident energy in \si{eV}.

The evolution of the reflection coefficient $r$ can also be estimated with \gls{srim}.
The reflection coefficient varies from around 0.5 at \SI{0}{^\circ} to 0.8 at \SI{80}{^\circ} (see \reffig{reflection coeff vs energy}).
According to \sidecite{park_assessment_2020}, the incident angles for ions and atoms were assumed to be \SI{60}{^\circ} and \SI{45}{^\circ}, respectively.
It should be noted that since \gls{srim} is based on the binary collision approximation, values around \SI{10}{eV} might not be fully valid.

All of these steps have been automated and packaged into a tool called divHretention.
divHretention can directly interpret \gls{solps}/\gls{soledge} data and produce a distribution of \gls{monoblock} \gls{inventory} as in \reffig{behaviour law example}.
\marginnote{
The source-code of the tool is under version control and openly available via GitHub under a MIT licence \cite{delaporte-mathurin_irfmdivhretention_2021}.
The divHretention python package is distributed via PyPi \cite{delaporte-mathurin_divhretention_nodate}.
Moreover, all the results obtained in this Chapter can be reproduced with the scripts available at \url{https://github.com/RemDelaporteMathurin/divHretention-Nucl.Fusion-2021}.
}

\section{ITER results}

\begin{figure*}[h!]
    \captionsetup[subfigure]{format=plain,singlelinecheck=true}  % needed to center the subcaptions
    \centering
    \begin{subfigure}{0.42\linewidth}
        \includegraphics[width=\linewidth]{Figures/Chapter4/ITER/inventory_along_inner_divertor.pdf}
        \caption{Inner Vertical Target.}
    \end{subfigure}%
    \begin{subfigure}{0.58\linewidth}
        \includegraphics[width=\linewidth]{Figures/Chapter4/ITER/inventory_along_outer_divertor.pdf}
        \caption{Outer Vertical Target.}
        \labfig{distrib outer target}
    \end{subfigure}
    \caption{Surface temperature, surface concentration and \gls{inventory} per unit thickness along the \gls{iter} \gls{divertor} with neutral pressures varying from \SI{2}{Pa} to \SI{11}{Pa}. The area corresponds to the 95\% confidence interval.}
\end{figure*}


\begin{figure}[h]
    \centering
    \includegraphics[width=\linewidth]{Figures/Chapter4/ITER/inventory_vs_divertor_pressure.pdf}
    \caption{Hydrogen \gls{inventory} in the \gls{iter} \gls{divertor} as a function of neutral pressure after \SI{e7}{s} of exposure (approximately 25 000 discharges).}
    \labfig{inventory vs neutral pressure}
\end{figure}


\begin{figure}[h!]
    \centering
    \begin{subfigure}{\linewidth}
        \includegraphics[width=\linewidth]{Figures/Chapter4/ITER/inventory_at_strike_points.pdf}
        \caption{\Gls{inventory} per unit thickness after \SI{e7}{s} of exposure (approximately 25 000 discharges). Area corresponds to the 95\% confidence interval.}
        \labfig{local inventory neutral pressure}
    \end{subfigure}
    \begin{subfigure}{\linewidth}
        \includegraphics[width=\linewidth]{Figures/Chapter4/ITER/ratio_ions_atoms.pdf}
        \caption{Contribution of ions to the surface concentration of H.}
        \labfig{ion contribution neutral pressure}
    \end{subfigure}%
    \caption{H retention at the \glspl{strike point} (defined as maximum temperature) as a function of the \gls{divertor} neutral pressure.}
\end{figure}

\begin{figure}[h!]
    \centering
    \includegraphics[width=\linewidth]{Figures/Chapter4/ITER/inventory_vs_time.pdf}
    \caption{Evolution of the \gls{H} \gls{inventory} of the \gls{iter} \gls{divertor} with the number of \SI{400}{s} discharges.}
    \labfig{iter vs time}
\end{figure}


% peak temperature
Peak temperatures at \glspl{strike point} increased when decreasing the \gls{divertor} neutral pressure (see \reffig{distrib outer target}).
The peak temperature at the outer strike point reached \SI{2000}{K} at \SI{2}{Pa} and more than \SI{1000}{K} at the inner strike point, which is in accordance with the results obtained by Pitts et al.\ \sidecite{pitts_physics_2019}.

% global inventory
The \gls{inventory} in the whole \gls{divertor} is computed as follows:
\begin{equation}
    \mathrm{inv_{divertor}} = N_\mathrm{cassettes} \cdot (\mathrm{inv_{IVT}} + \mathrm{inv_{OVT}})
\end{equation}
with $N_\mathrm{cassettes}=54$ the number of cassettes, $\mathrm{inv_{IVT}}$ and $\mathrm{inv_{OVT}}$ the total \gls{inventory} in the \gls{ivt} and \gls{ovt} respectively (in one cassette).

\begin{align}
    \mathrm{inv_{IVT}} &= N_\mathrm{PFU-IVT} \cdot \int_\mathrm{IVT} \mathrm{inv_{MB}}(x)\: dx \\
    \mathrm{inv_{OVT}} &= N_\mathrm{PFU-OVT} \cdot \int_\mathrm{OVT} \mathrm{inv_{MB}}(x)\: dx
\end{align}
$N_\mathrm{PFU-IVT}=16$ and $N_\mathrm{PFU-OVT}=22$ the number of plasma facing units per cassette in the inner and outer targets respectively (see \refsec{divertor section}), $\mathrm{inv_{MB}}$ is the \gls{monoblock} \gls{inventory} per unit thickness and $x$ the distance along the targets.
Here, $\int_\mathrm{OVT} \mathrm{inv_{MB}}(x) dx$ corresponds to the area of the profile shown on \reffig{distrib outer target}.

The \gls{inventory} in the outer target was found to be nearly twice that of the inner target.
This is largely explained by the larger number of plasma facing units in the outer target and therefore a greater exposed surface.
The global \gls{inventory} increased with the \gls{divertor} neutral pressure and a roll-over is observed above \SI{7}{Pa} (see \reffig{inventory vs neutral pressure}).
This roll-over is consistent with the results obtained in \sidecite{pitts_physics_2019}.
The \gls{inventory} increase was found to be more important in the outer vertical target.
This was explained by the fact that the plasma is more detached at the inner target.
Therefore the surface temperature reduction is more significant in the outer vertical target and the surface concentration is increased (see \reffig{distrib outer target}).

The maximum \gls{inventory} was found at around \SI{7}{Pa} and was approximately \SI{14}{g} of H, which is well below the \gls{iter} in-vessel safety limit of tritium (\SI{1}{kg}), especially considering only half of this quantity will be tritium.
This is especially true considering that this was for a very long exposure time of \SI{e7}{s}, which corresponds to 25 000 pulses of \SI{400}{s}.


% local inventories
The inventory at the inner \gls{strike point} is constant from \SI{4}{Pa} whereas the inventory at the outer \gls{strike point} globally increases with the \gls{divertor} neutral pressure (see \reffig{local inventory neutral pressure}).
The contribution of ions to the surface concentration at the inner strike point is around 50 \% and tends to decrease with increasing neutral pressure (see \reffig{ion contribution neutral pressure}).
At low \gls{divertor} neutral pressure, the contribution of ions at the outer strike point is around 90 \% and tends to decrease with increasing neutral pressure.
This can be explained by the fact that in both inner and outer targets, the integrated flux of ions decreases with increasing neutral pressure whereas the integrated flux of atoms increases, leading to a greater proportion of neutral particles.

% temporal evolution
For all \gls{divertor} neutral pressures, the temporal evolution of the \gls{divertor} \gls{inventory} is approximately the same (see \reffig{iter vs time}).
The \gls{inventory} is plotted as a function of the number of \gls{iter} discharges (see \reffig{plasma cycle}).
The additional \gls{inventory} per \SI{400}{s} discharge was found to decrease with time.
Past 300 discharges, the additional \gls{inventory} per discharge decreases with the number of discharges.
The maximum is around \SI{5}{mg/discharge} between 30 and 100 discharges.

\section{WEST results}

All the computations have been made for very long exposure times (\SI{e7}{s}) in order to better visualise trends.
Even though cycling can have an effect on \gls{H} outgassing at the \gls{monoblock} plasma facing surface, it was shown in \refsec{influence of cycling} that the evolution of the \gls{monoblock} \gls{inventory} with the fluence was not affected.
Moreover, it can be shown that the \glspl{divertor} inventories evolve with a power law dependence of time.

\subsection{Influence of the input power}

The input power was varied between \SI{0.49}{MW} and \SI{2.0}{MW}.
Two puffing rate values were used: \SI{2.5e21}{molecule.s^{-1}} and \SI{4.4e21}{molecule.s^{-1}}.

\begin{figure}[h]
    \centering
    \includegraphics[width=\linewidth]{Figures/Chapter4/WEST/inventory_along_divertor_input_power.pdf}
    \caption{Distribution of surface temperature $T_\mathrm{surface}$, surface concentration $c_\mathrm{surface}$ and \gls{inventory} per unit thickness along the \gls{west} \gls{divertor} with input powers varying from \SI{0.49}{MW} to \SI{2.0}{MW} with a puffing rate of \SI{2.5e21}{molecule.s^{-1}}.}
    \labfig{divertor distr power scan}
\end{figure}

\begin{figure}[h]
    \centering
    \begin{subfigure}{\linewidth}
        \includegraphics[width=\linewidth]{Figures/Chapter4/WEST/inventory_at_sps_and_private_zone_vs_input_power.pdf}
        \caption{\Gls{inventory} per unit thickness after \SI{e7}{s} of exposure. The area corresponds to the 95\% confidence interval.}
        \labfig{local retention vs input power}
    \end{subfigure}
    \begin{subfigure}{\linewidth}                          
        \includegraphics[width=\linewidth]{Figures/Chapter4/WEST/ions_ratio_vs_input_power.pdf}
        \caption{Contribution of ions to the surface concentration of H.}
        \labfig{ion ration vs input power}
    \end{subfigure}%
    \caption{H \gls{inventory} at the inner and outer \glspl{strike point} (\gls{isp} and \gls{osp}) and in the \gls{private flux region} as a function of the input power with a puffing rate of \SI{2.5e21}{molecule.s^{-1}}.}
\end{figure}

\begin{figure}
    \centering
    \includegraphics[width=\linewidth]{Figures/Chapter4/WEST/inventory_vs_input_power.pdf}
    \caption{Evolution of the \gls{west} \gls{divertor} \gls{inventory} as a function of input power for several puffing rates.}
    \labfig{inventory vs input power}
\end{figure}

\begin{figure}[h]
    \centering
    \includegraphics[width=\linewidth]{Figures/Chapter4/WEST/inventory_vs_time_west.pdf}
    \caption{Temporal evolution of \gls{pfuLabel} inventories for different values of puffing rate (left) and input power (right).}
    \labfig{temporal evolution west}
\end{figure}

% local inventories
The maximum \gls{retention} was found to be located at the \glspl{strike point} (see \reffig{divertor distr power scan}).
The \gls{retention} at the outer strike point was higher than at the inner strike point.
The \gls{retention} at the \glspl{strike point} was found to increase with the input power whereas it slightly decreased in the \gls{private flux region} (see \reffig{local retention vs input power}).
This was explained by an attachment of the \gls{plasma} decreasing the particle flux in the \gls{private flux region}.
Since the surface temperature is constant, this leads to a decrease in the surface concentration of hydrogen as seen on \reffig{divertor distr power scan}.
On the other hand, the increasing temperature at the \glspl{strike point} only enhanced the \gls{diffusion} process while remaining low enough so that hydrogen could get trapped.

The total \gls{inventory} in the \gls{west} \gls{divertor} is computed as follows:
\begin{equation}
    \mathrm{inv}_\mathrm{divertor} = N_\mathrm{PFU} \cdot \int \mathrm{inv}_\mathrm{MB}(x)\: dx
    \labeq{inventory WEST}
\end{equation}
where $N_\mathrm{PFU} = 480$ is the number of \gls{pfuLabel} in \gls{west}, $\mathrm{inv}_\mathrm{MB}$ is the \gls{inventory} per unit thickness in \si{H.m^{-1}} (see \reffig{divertor distr power scan}) and $x$ the distance along the target in \si{m}.

The \gls{divertor} \gls{inventory} increased with the input power (see \reffig{inventory vs input power}) and evolved as the power 0.3 of the input power.
The maximum \gls{divertor} \gls{inventory} was \SI{8.8e23}{H} at \SI{2.0}{MW} of input power.
This value of input power is still relatively low.
Increasing the puffing rate lead to an increase in the \gls{inventory}.
This will be explained more thoroughly in \refsec{density scan}.

At the \glspl{strike point}, the \gls{retention} is dominated by the ion flux whereas neutrals are dominant in the \gls{private flux region} (see \reffig{ion ration vs input power}).
The contribution of ions at the \glspl{strike point} increased with the input power but remained approximately constant in the \gls{private flux region}.

The \gls{divertor} \gls{inventory} was found to increase as a power law of time (see \reffig{temporal evolution west}).


\subsection{Influence of the puffing rate} \labsec{density scan}

A parametric study on the puffing rate was performed.
The puffing rate was varied between \SI{4.4e20}{molecule.s^{-1}} and \SI{4.7e21}{molecule.s^{-1}}.
The input power was fixed to \SI{0.45}{MW}.

\begin{figure}[h]
    \centering
    \includegraphics[width=\linewidth]{Figures/Chapter4/WEST/inventory_along_divertor.pdf}
    \caption{Distribution of surface temperature $T_\mathrm{surface}$, surface concentration $c_\mathrm{surface}$ and \gls{inventory} per unit thickness along the \gls{west} \gls{divertor} with a puffing rate varying from \SI{4.4e20}{s^{-1}} to \SI{4.7e21}{s^{-1}} with \SI{0.45}{MW} of input power.}
    \labfig{divertor distr density scan}
\end{figure}

\begin{figure}[h]
    \centering
    \includegraphics[width=\linewidth]{Figures/Chapter4/WEST/inventory_vs_puffing_rate.pdf}
    \caption{Evolution of the \gls{west} \gls{divertor} \gls{inventory} as a function of puffing rate.}
    \labfig{inventory vs puffing rate}
\end{figure}

\begin{figure}[h!]
    \centering
    \begin{subfigure}{\linewidth}
        \includegraphics[width=\linewidth]{Figures/Chapter4/WEST/inventory_at_sp_and_private_zone.pdf}
        \caption{\Gls{inventory} per unit thickness after \SI{e7}{s} of exposure. The area corresponds to the 95\% confidence interval.}
        \labfig{local retention vs puffing rate}
    \end{subfigure}
    \begin{subfigure}{\linewidth}
        \includegraphics[width=\linewidth]{Figures/Chapter4/WEST/ion_ratio_at_sp_and_private_zone.pdf}
        \caption{Contribution of ions to the surface concentration of H. \gls{isp} and \gls{osp} stand for Inner Strike Point and Outer Strike Point respectively.}
        \labfig{ion contribution vs puffing rate}
    \end{subfigure}%
    \caption{H \gls{retention} at the \glspl{strike point} and in the \gls{private flux region} as a function of puffing rate with \SI{0.45}{MW} of input power.}
\end{figure}

The maximum \gls{retention} was again located at the \glspl{strike point} for all puffing rates values (see \reffig{divertor distr density scan}).
The \gls{inventory} at the outer strike point was higher than at the inner strike point.
The \gls{inventory} in the \gls{private flux region} was found to increase with the puffing rate whereas it was almost constant at the \glspl{strike point} (see \reffig{local retention vs puffing rate}).
As for the power scan, the ions' contribution to the \gls{inventory} is rather low in the \gls{private flux region} (see \reffig{ion contribution vs puffing rate}).
Moreover, the contribution of ions decreases rapidly at the \glspl{strike point} and represents only half of the surface concentration at \SI{4e21}{molecule.s^{-1}}.

The \gls{inventory} in the whole \gls{west} \gls{divertor} is computed from \refeq{inventory WEST}.
As for the power scan, the \gls{divertor} \gls{inventory} increased as the power 0.2 of the puffing rate (see \reffig{inventory vs puffing rate}).
The maximum \gls{inventory} was found to be \SI{5e23}{H} at \SI{4.7e21}{molecule.s^{-1}}.

The \gls{divertor} \gls{inventory} was found to increase as a power law of time.

\section{Summary}


The \gls{monoblock} behaviour law proposed in \refch{Chapter3} was used to estimate fuel \gls{retention} in the \glspl{divertor} of \gls{west} and ITER.
The impact of key control parameters on the \gls{divertor} \gls{inventory} was studied (the input power, the puffing rate and the \gls{divertor} neutral pressure).

It was shown that the \gls{inventory} in \gls{west} increases as the power $0.3$ of the input power and as the power $0.2$ of the puffing rate.
The \gls{inventory} in the \gls{iter} \gls{divertor} was found to first increase with the neutral pressure up to \SI{7}{Pa} then decrease, though the variation was smoother.
The \gls{inventory} in the outer vertical target of the \gls{iter} \gls{divertor} is twice that of the inner vertical target.
These results were in good agreement with the observations made in \sidecite{pitts_physics_2019}.

However, it should be noted that both machines do not operate in the same regime.
While \gls{west} operates at low input power, \gls{iter} operates at high input power with a high recycling \gls{divertor} .
These differences in the operation regime can explain different trends.

The maximum hydrogen \gls{inventory} in the \gls{iter} \gls{divertor} was approximately \SI{14}{g} after \SI{e7}{s} of continuous plasma exposure (25 000 \gls{iter} discharges), which is well below the in-vessel safety limit (\SI{1}{kg} or \SI{700}{g} excluding the cryopumps).
Note that the total number of discharges in \gls{iter} will be approximately 23 300 \cite{pitts_physics_2019}.
Moreover, since the behaviour law is based on 2D \gls{monoblock} simulations, this value is an upper estimate (see \refsec{influence of dimensionality}).
2D simulations are indeed conservative in terms of \gls{inventory} (see \refsec{influence of dimensionality}).

The underlying \gls{monoblock} model has also a few limitations, as detailed in \refch{Chapter3}.
First, the set of trapping parameters that was used may not be relevant for every region of the \gls{divertor} .
These properties can however be experimentally estimated.
The accuracy of the results could therefore be improved by running a new batch of \gls{festim} \gls{monoblock} simulations with different trapping parameters like neutron-induced traps.

Then, this model does not take into account \gls{retention} in \gls{Be} co-deposited layers (i.e.\ \gls{Be} particles eroded from the \gls{first wall} redeposited in other locations of the vessel, trapping hydrogen).
These are expected to be the main driver for \gls{H} \gls{retention} in \gls{iter} \sidecite{de_temmerman_data_2021, schmid_walldyn_2015}.
However, this work is still relevant for full-tungsten environments like \gls{west} or \gls{demo}.
