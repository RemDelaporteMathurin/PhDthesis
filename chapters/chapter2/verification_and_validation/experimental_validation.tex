Now that the code has been verified (\textit{ie} it solves the governing equations correctly), experimental validation is still required to check that these equations actually represent experimentally observed processes.
A very good example of experiments that can be reproduced are Thermo-Desorption Spectroscopy (TDS) experiments also called Thermally Programmed Desorption (TPD) experiments.
The principle of such experiments is to load samples with hydrogen isotopes either via gas infusion, plasma implantation or electrochemical charging.
The samples are then heated up to different temperatures to desorb the trapped hydrogen.
By measuring the outgassing flux of particles throughout the time of the experiment, desorption spectra are obtained.
These spectra often exhibit several peaks and each peak correspond to a kind of trap (in most cases).

This technique is therefore employed to characterise materials and their defects. 
It is also a very good application case for experimental validation of the hydrogen transport model.

This section describes the technique that was employed to easily reproduce these TDS experiments.

\subsubsection{Methodology} \label{methodology}
Fitting experimental data by manually tweaking parameters as in \sidecite{yu_deuterium_2019, hodille_macroscopic_2015} can be really time-consuming, sometimes days in some cases.
Moreover, some possible solutions in the parameter space might be missed by the user.
This process has been automated by embedding \gls{festim} in a minimisation algorithm.

As in manual fitting, the parametric optimisation problem is solved by minimising a function representing the residual between simulated results and some reference data.
This function $f$ is called \textit{cost function}.
Considering fitting one or several \gls{tds} spectra (in order to identify for instance trapping parameters or diffusion coefficients), $f$ can simply be the mean absolute error described in Equation \ref{eq:cost function} representing the residual between the simulated spectrum and the experimental reference: 

\begin{equation}
    f(\textbf{x})=\frac{\sum_{i=0}^{N}  \alpha_i(T_i)\left| d_{i}-d_{\mathrm{sim}}\right|}{\sum_{i=0}^{N}  \alpha_i(T_i)}
    \label{eq:cost function}
\end{equation}

where \textbf{x} is the set of parameters used for the simulation, $d_\mathrm{sim}$ are the values of the simulated spectrum, $N$ is the number of experimental points $(T_i, d_i)$.
In Equation \ref{eq:cost function}, $f(\textbf{x})$ can be weighted by coefficients $\alpha_i$ in order to have a better fit on specific regions of the spectrum.
%Note that this cost function could as well be a root mean square error or any type of residual.

The parametric optimisation problem can now be solved by finding the minimum of the cost function $f$.
The global optimisation routine is illustrated on Figure \ref{fig:diagramm}.
\begin{figure}
    \centering
    \includegraphics[width=\linewidth]{Figures/Chapter3/Parametric_optimisation/algorithm diagram.pdf}
    \caption{Diagram of the embedding of FESTIM within a parametric optimisation routine based on SciPy \cite{virtanen_scipy_2020}.}
    \label{fig:diagramm}
\end{figure}

A comparative study of the several optimisation algorithms which can be employed has been made.
These algorithms require the user to give an initial set of parameters called \textit{initial guess} and evaluate the cost function with several parameters sets until the convergence criterion is reached.
As in \sidecite{drexler_model-based_2019}, the Python package SciPy \sidecite{virtanen_scipy_2020} will be employed.

Four minimisation algorithms have been benchmarked against a test case.
In the following example an experimental \gls{tds} spectrum from Ogorodnikova et al.\ \sidecite{ogorodnikova_deuterium_2003} will be fitted and materials properties such as trap density and detrapping energy will be identified.
For this example case, two intrinsic traps and one extrinsic trap are set.
The only free parameters are $E_1$ and $n_1$, respectively the detrapping energy and density of trap 1.
The other parameters are constrained and described in \sidecite{delaporte-mathurin_finite_2019}.
The cost function $f$ has been plotted on Figure \ref{fig:cost function} as function of $E_1$ and $n_1$.

\begin{figure*} [h!]
    \centering
        \begin{subfigure}[t]{0.7\linewidth}
            \centering
            \includegraphics[width=\linewidth]{Figures/Chapter3/Parametric_optimisation/cost_function_2D.pdf}
            \caption{Normalised cost function.}
        \end{subfigure}
        \begin{subfigure}[t]{0.7\linewidth}
            \centering
            \includegraphics[width=\linewidth]{Figures/Chapter3/Parametric_optimisation/points_on_cost_function.pdf}
            \caption{Corresponding simulated \gls{tds} spectra.}
        \end{subfigure}%
    \caption{Normalised cost function $\hat{f} = (f - \min{f})/(\max{f}-\min{f})$ as function of $E_1$ (\si{eV}) and $n_1$ (\si{at.fr.}) with global minimum located at $(\SI{0.86}{eV}, \SI{1.2e-3}{at.fr.})$.}
    \label{fig:cost function}
\end{figure*}

In this case, when only two free parameters are set the cost function has only one minimum (it is not necessarily the case for higher dimension optimisation problems).
However, if one fixes the trap density $n_1$ above $\approx \SI{2e-3}{at.fr.}$, the cost function has two local minima which can lead the optimisation routine to converged to a non-global minimum.
Moreover, $f$ is smooth and quadratic around its minimum located at $(E_1, n_1) = (\SI{0.86}{eV}, \SI{1.2e-3}{at.fr.})$.
For detrapping energies below \SI{0.6}{eV} and/or densities below \SI{0.5e-3}{at.fr}, the cost function is constant.
This is because for these values, the contribution of this trapping site to the \gls{tds} spectrum is zero either because the density is close to zero, or because the energy is too low for these traps to be filled at the implantation temperature of \SI{300}{K}.
Variations in these regions do not modify the simulated spectrum and thus do not modify the cost function value.

\begin{figure} [ht]
    \centering
    \includegraphics[width=\linewidth]{Figures/Chapter3/Parametric_optimisation/algorithms_perfs.pdf}
    \caption{Number of cost function evaluations required to converge towards the global minimum with 100 different initial guesses sorted by distance to the global minimum for several minimisation algorithms. Each cost function evaluation takes \SI{20}{s} to compute. White stripes correspond to initial guesses for which the algorithm did not converge to the global minimum.}
    \label{fig:algos perfs}
\end{figure}


Four different optimisation algorithms are being compared: 
Nelder-Mead (also called the simplex method), Powell, \gls{tnc} and \gls{cg}.
Thorough descriptions of these algorithms would be beyond the scope of this research but can be found in \sidecite{nocedal_numerical_2006}.
The performances of these algorithms have been compared with 100 different initial guesses randomly distributed on the $(E_1,n_1)$ plane and are shown on Figure \ref{fig:algos perfs}.
It appears that the \gls{cg} algorithm is less robust since for some cases it didn't converge towards the global minimum (see white bands on Figure \ref{fig:algos perfs}).
Nelder-Mead algorithm appears to be the most efficient with initial guesses both close and far from the global minimum since the number of cost function evaluations ranges from 50 to 100 whereas other algorithms require more than 100.
This can be explained by the fact that Nelder-Mead is a derivative-free algorithm whereas \gls{tnc} and \gls{cg} algorithms need on the other hand to compute first order derivatives thus increasing the number of function evaluations.
This will be even more true when increasing the number of free parameters since the derivative will become more costly to compute.

It is worth noting that the Nelder-Mead algorithm is an unconstrained method.
If constraints or bounds are needed, \gls{tnc} might be a more suitable choice.

Though in the following, the Nelder-Mead algorithm will be employed in the following cases.


% \subsubsection{Results} \label{results}

% \subsection{Applications}

The fitting procedure has been employed to reproduce thermo-desorption experiments performed on Tungsten, EUROFER, Aluminium and Beryllium.


\subsubsection{Tungsten}

The TPD spectrum measured by Ogorodnikova \textit{et al} presented in Section \ref{optimisation algorithms} has been reproduced by setting all traps parameters as free parameters.
The fitting procedure has been run for several numbers of traps as shown on Figure \ref{fig:number of traps comparison}.
It is clear that setting only one trap is not sufficient to reproduce the experimental data.
The two traps case shows better results but also has a discrepancy near \SI{600}{K}.
This discrepancy is removed when setting a third extrinsic trap to the simulation.

For this last case, the five free parameters are the detrapping energies $E_{p, 1}$, $E_{p, 2}$, $E_{p, 3}$ and densities $n_1$, $n_2$ (the third trapping site being created during the implantation, for which the creation parameters are not part of the free parameters and taken from \sidecite{ogorodnikova_deuterium_2003} or \sidecite{hodille_macroscopic_2015}).
This optimisation case is therefore a 5D optimisation problem.
Every other parameters are taken from \sidecite{hodille_macroscopic_2015}.
The resulting fit is shown on Figure \ref{fig:5D TPD} alongside with the contribution of each trap to the total spectrum.
% (1 trap [ 0.88368005  1.44347635] 2 traps [ 0.83582125  1.23768655  0.98708714  6.85457283])
\begin{figure*} [ht]
    \centering
        \begin{subfigure}[t]{0.5\linewidth}
            \centering
            \captionsetup{width=.9\linewidth}
            \includegraphics[width=\linewidth]{Figures/Chapter3/Parametric_optimisation/number_of_traps.pdf}
            \caption{Comparison of the resulting fit with several numbers of traps in the simulation.}
            \label{fig:number of traps comparison}
        \end{subfigure}%
        \begin{subfigure}[t]{0.5\linewidth}
            \centering
            \captionsetup{width=.9\linewidth}
            \includegraphics[width=\linewidth]{Figures/Chapter3/Parametric_optimisation/Ogorodnikova_5D.pdf}
            \caption{Identified by the fitting procedure with $E_1 = \SI{0.83}{eV}$, $E_2 = \SI{0.97}{eV}$, $E_3 = \SI{1.51}{eV}$, $n_1 = \SI{1.18e-3}{at.fr.}$ and \newline $n_2 = \SI{7.22e-4}{at.fr.}$.  Dashed lines correspond to the temporal evolution of each trapping population's inventory.}
            \label{fig:5D TPD}
        \end{subfigure}%
    \caption{Fitting TPD spectrum performed on Tungsten by Ogorodnikova \textit{et  al} \cite{ogorodnikova_deuterium_2003}. Dots correspond to experimental data.}
    \label{fig:TPD ogorodnikova}
\end{figure*}
The identified parameters are similar to the ones found by Hodille \textit{et al.} in \sidecite{hodille_macroscopic_2015}.
The total fitting procedure took a few hundred of cost function evaluations.
One single cost function evaluation "costing" less than \SI{20}{s} to compute (for that specific case), the total procedure lasted less than \SI{3}{h}.

\subsubsection{EUROFER}

Hollingsworth \textit{et al} performed thermo-desorption on pre-damaged EUROFER at several damage levels \sidecite{hollingsworth_comparative_2019}.
Three spectra with similar exposure conditions have been fitted with one trapping site (since only one peak appears on the spectra) as shown on Figure \ref{fig:TPD EUROFER}.

\begin{figure} [ht]
    \centering
    \includegraphics[width=\linewidth]{Figures/Chapter3/Parametric_optimisation/EUROFER_hollingsworth.pdf}
    \caption{TPD spectra of damaged EUROFER \cite{hollingsworth_comparative_2019}. Fitted with one trapping site (solid line) $E=\SI{1.06}{eV}$ and densities of \SI{8.9e-3}{at.fr}, \SI{2.8e-2}{at.fr} and \SI{5.0e-2}{at.fr} for \SI{0}{dpa}, \SI{0.01}{dpa} and \SI{0.1}{dpa}, respectively. Dashed lines correspond to optimisations with an unweighted cost function. Dots correspond to experimental data.}
    \label{fig:TPD EUROFER}
\end{figure}

To put the emphasis on peaks, a weighting factor of 10 has been applied for $T \in [\SI{445}{K}, \SI{492}{K}]$.
Not applying this factor near the peak region results in a closer fit in other regions but a higher peak value.
The identified trap energy is $E_p$ \SI{1.06}{eV} for all spectra whereas the trap density $n$ is \SI{8.9e-3}{at.fr} for the undamaged sample, \SI{2.8e-2}{at.fr} for \SI{0.01}{dpa} and \newline \SI{5.0e-2}{at.fr} for \SI{0.1}{dpa}.
For all simulations the attempt frequency $p_0$ is \SI{1e13}{s^{-1}} and the diffusion coefficient is taken from \sidecite{esteban_hydrogen_2007}.

The total fitting procedure took less than two hours for fitting the three spectra.
A more thorough study of these experiments could involve constraining the algorithm with profilometry data obtained by Hollingsworth \textit{et al} \sidecite{hollingsworth_comparative_2019}.
Indeed, having a non-homogeneous trapping site distribution could help having a better fit of both the profilometry data and the TPD spectra. 

\subsubsection{Aluminium}

The experiment performed on Aluminium by Quiros \textit{et al} \sidecite{quiros_blistering_2019, quiros_blister_2017} has also been reproduced with FESTIM.

Only one trap has been set in the simulation and its energy $E_p$ and density $n$ are set as free parameters.
Every other parameters are fixed and taken from \sidecite{quiros_blister_2017, quiros_blistering_2019}.
The resulting simulated TPD spectrum is shown on Figure \ref{fig:TPD alu}.
\begin{figure} [ht]
    \centering
    \includegraphics[width=\linewidth]{Figures/Chapter3/Parametric_optimisation/alu_quiros.pdf}
    \caption{TPD spectrum of aluminium exposed to \SI{3e23}{H.m^{-2}} at \SI{618}{K} \cite{quiros_blister_2017, quiros_blistering_2019}. Fitted with one trapping site $n = \SI{1.8e-2}{at.fr.}$ and $E =\SI{1.1}{eV}$. Dots correspond to experimental data. Dashed lines correspond to the temporal evolution of each trapping population's inventory.}
    \label{fig:TPD alu}
\end{figure}


The identified parameters are $n = \SI{1.8e-2}{at.fr.}$ and $E =\SI{1.1}{eV}$.
The trapping sites density is significantly higher than the one described in \sidecite{quiros_blister_2017}.
However, the TPD spectrum obtained with this procedure better fits the experimental data since the one obtained by Quiros \textit{et al} requires a 10-fold increase.
The fitting procedure took less than a hundred cost function evaluations, which corresponds in total to a few dozens of minutes.

\subsubsection{Beryllium}
Be co-deposition experiments performed by Baldwin \textit{et al} \sidecite{baldwin_experimental_2014} were reproduced with FESTIM using this optimisation technique.
In this experiment, a \SI{1}{\micro m} thick Be-D layer is grown on Tungsten at \SI{330}{K}. 
Following the strategy proposed by Baldwin \textit{et al}, only the thermo-desorption phase has been simulated with two trapping sites with homogeneously distributed densities and with initial occupancies $f_i$.
There are therefore three free parameters per trap (energy, density and initial occupancy) which makes this optimisation problem 6D.
It is assumed that the surface flux is the net balance between incoming flux from the chamber (very low since the pressure is $\SI{1}{\micro Pa}$) and the molecular recombination flux.
All the other parameters are described in \sidecite{baldwin_experimental_2014}.

The resulting optimised TPD spectrum is shown on Figure \ref{fig:tpd baldwin}.
The optimised parameters are $E_{p, 1} = \SI{0.75}{eV}$, $n_1 = \SI{1.09e-1}{at.fr.}$, $f_1=0.73$, $E_{p, 2} = \SI{0.93}{eV}$, \newline ${n_2 = \SI{3.40e-2}{at.fr.}}$, $f_2=0.28$.
These values are in agreement with the ones found by Baldwin \textit{et al} \sidecite{baldwin_experimental_2014} and took only a few of minutes to compute since the implantation phase was not simulated.

\begin{figure}
    \centering
    \includegraphics[width=\linewidth]{Figures/Chapter3/Parametric_optimisation/baldwin_be.pdf}
    \caption{TPD spectrum of co-deposited Be-D \cite{baldwin_experimental_2014} simulated with two trapping sites. Dots correspond to experimental data.}
    \label{fig:tpd baldwin}
\end{figure}

\subsubsection{Discussion}

Even though an automated technique is proposed, the user still has some choices to make in order to ensure the credibility of the fitted spectrum.
As shown on Figure \ref{fig:TPD EUROFER}, weighting the cost function near regions of interest will result in a better fit in these regions.
Users should also be aware of the number of traps the data is being fitted with.
As shown on Figure \ref{fig:number of traps comparison} too few traps in the simulation will not result in a satisfactory fit (even though the optimisation routine will converge to an optimised solution).
Moreover, as shown on Figure \ref{fig:hurley_comparison}, one single TPD spectrum can be reproduced with several traps of different energies and densities.
This means that the cost function with several traps as free parameters can have several local minima of very similar values.
Adding traps to an optimisation problem can also help having a better fit of the experimental data in some cases.
But artificially adding more and more traps is not necessarily realistic and could lead to misinterpretation of the results.

\begin{figure}[ht]
    \centering
    \includegraphics[width=\linewidth]{Figures/Chapter3/Parametric_optimisation/hurley_comparison.pdf}
    \caption{TPD spectrum reproduced with several sets of parameters showing the existence of several solutions to a single optimisation problem.}
    \label{fig:hurley_comparison}
\end{figure}

In the first case with only one trapping site, as described by Hurley \textit{et al} in \sidecite{hurley_numerical_2015}, the binding energy is \SI{0.55}{eV} and the trap density is \SI{2.08e24}{m^{-3}}.
The appearance of two peaks is due to the desorption on different sides of the sample as explained in \sidecite{hurley_numerical_2015}.
In the second case, the curve as been reproduced with two trapping sites which energies and densities are respectively \SI{0.51}{eV} and \SI{0.57}{eV} and \SI{2.02e24}{m^{-3}} and \SI{2.12e24}{m^{-3}}.
In the third case, it has been reproduced with three trapping sites which energies and densities are respectively \SI{0.55}{eV}, \SI{0.38}{eV} and \SI{0.51}{eV} and \SI{2.12e24}{m^{-3}}, \SI{2.26e24}{m^{-3}} and \newline \SI{2.13e24}{m^{-3}}.

This example illustrates how a single spectrum can be simulated with several sets of parameters by varying the number of traps in the simulation.
One way to avoid this from happening is to have a set of experiments with varying parameters such as the implantation temperature, the heating ramp, the fluence, dwelling time before TPD, etc.
