\setchapterpreamble[u]{\margintoc}
\chapter{Conclusion}\labch{conclusion}

This Chapter will summarise the key findings of this research as well as the main contributions thereof.
It will also review the limitations of the study and propose recommendations for future work.

\section*{Key findings}

This study aimed to estimate the tritium inventory in the \gls{iter} tungsten \gls{divertor}.
One of the objectives was to evaluate the potential influence of helium exposure on this tritium inventory.
The results indicate that, over the operation period of \gls{iter}, the \gls{divertor} tritium inventory should remain well below the safety limit.
Indeed, this component would not be the limiting factor and, if the limit is hit, it would likely be due to other causes (retention in co-deposited layers, other in-vessel components...).
Moreover, the results suggest that the presence of helium could potentially reduce the tritium inventory by saturating the pre-existing defects in tungsten.

\section*{Contributions to the field}
One of the main contribution of this research is the development of the \gls{festim} code: a hydrogen transport code that has been developed to answer the main questions of the study.
The first article introducing \gls{festim} was published during the master project preceding this PhD research \cite{delaporte-mathurin_finite_2019}.
At the time of writing, \gls{festim} is used by a handful of researchers, engineers and students and applied on other cases.
Making \gls{festim} available to other researchers (by making it open-source) would greatly benefit the broader community.

The parametric optimisation method used in \refch{Chapter2} now provides an efficient way of automatically fitting experimental data without manually tweaking parameters, saving precious time in the process.
This method was published in a proceedings article \cite{delaporte-mathurin_parametric_2021}.

The initial monoblock results shown in \refch{Chapter3} were published in several articles \cite{delaporte-mathurin_finite_2019, delaporte-mathurin_parametric_2021, delaporte-mathurin_influence_2021-1}.
The method of running parametric subcomponent simulations and then extract a behaviour law was nothing new.
However, applying it to the estimation of the \gls{divertor} inventory saved a significant amount of time compared to using the ``brute force'' approach of modelling the whole \gls{divertor} at a time.
This method was published in 2020 \cite{delaporte-mathurin_parametric_2020} and can be now be applied to other components like the \gls{breeding blanket}, where the tritium inventory is also an issue.

The model developed to simulate the growth of helium bubbles in tungsten gives a rapid way of implementing and testing new physical models.
This work was published in 2021 \cite{delaporte-mathurin_influence_2021}.

Finally, during this PhD research, several contributions were made to the open-source \gls{paramak} code and related neutronics tools \cite{shimwell_paramak_2021}.

\section*{Limitations}

Many of the technical limitations of this research lie with the development issues of \gls{festim}.
Since the code was built from scratch, features were added gradually.
For instance, at the time the simulations in \refsec{influence of exposure conditions} were run, the surface concentration could not be inhomogeneous \emph{and} directly dependent on the inhomogeneous surface temperature due to the heat flux.
Therefore, the choice that was made was to impose a homogeneous surface temperature instead (and a homogeneous surface concentration).
Even though the required feature was added a few months later, re-running all the \gls{festim} simulations was too much time-consuming given the time constraints.
Many of these development drawbacks could have been alleviated if \gls{festim} had been open-sourced, as external community experts such as \gls{fenics} developers could have more easily contributed to its development or bug fixes.

This study also has physical limitations inherent to the assumptions that have been made.
While some of these assumptions are conservative (i.e.\ represent a worst-case scenario) and do not jeopardise the key findings, others were made as a response to uncertainties.
For instance, no desorption was assumed on the monoblocks gaps.
The value of the recombination coefficient at the interface between the coolant and the monoblock also has a high uncertainty as it was measured in vacuum.


\section*{Recommendations for future work}

A more accurate behaviour law for the monoblock inventory can be obtained by redoing the parametric study.
Instead of varying the surface concentration and temperature, one should vary the incident heat flux $\varphi_\mathrm{heat}$ as well as the product of the implantation range and implanted particle flux $\varphi_\mathrm{imp} \ R_p$.
Assuming the uncertainty regarding the recombination coefficient of tungsten is lifted, and recombination on gaps cannot be neglected, 3D simulations could also be run for the most exposed monoblocks.
Hydrogen recombination from CuCrZr in contact with water should also be studied for this can reduce even more the tritium inventory while increasing the \gls{permeation} flux to the coolant.
Moreover, it would be extremely important to try and validate these results experimentally on a monoblock to scale.
Monoblocks could for instance be exposed to hydrogen and then their hydrogen content could be measured from thermo-desorption techniques.
These results could then be compared to FESTIM simulations.

To estimate the tritium inventory in the \gls{iter} vacuum vessel, studies should now focus on other components like the \gls{iter} \gls{first wall}, the tritium breeding system, etc.

Regarding the influence of helium on hydrogen transport, future work could involve experimental studies investigating the simultaneous exposure of hydrogen and helium in tungsten to confirm, or infirm the suggested results of this research.
For example, re-running the experiments presented in \refsec{helium hydrogen tds experiments} but running the TDS only up to \SI{750}{K} could help deconvolute the phenomena at stake.
If the results interpretations made in \refch{Chapter5} were to be confirmed, running this experiment with different helium fluences/fluxes would produce different bubble densities/size distribution and could help identifying a more general expression for the trap density per unit surface around bubbles.
