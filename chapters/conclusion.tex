\setchapterpreamble[u]{\margintoc}
\chapter{Conclusion}\labch{conclusion}

This Chapter will summarise the key findings of this research as well as the main contributions thereof.
It will also review the limitations of the study and propose recommandations for future work.

\section*{Key findings}

This study aimed to estimate the tritium inventory in the ITER tungsten \gls{divertor}.
One of the objectives was to evaluate the potential influence of helium exposure on the tritium retention.
The results indicate that, over the operation time of ITER, the \gls{divertor} tritium inventory should remains well below the safety limit.
Indeed, this component will not be the limiting factor and if the limit is hit, it would likely be due to other causes (retention in co-deposited layers, other in-vessel components...).
Moreover, the results suggest that the presence of helium could reduce the tritium inventory.

\section*{Contributions to the field}
The findings of this PhD research were the subject of several articles as well as talks in international conferences.
The main contribution is the \gls{festim} code, a hydrogen transport code that has been developed to answer the main questions of the study.
At the time of writing, \gls{festim} is used by a handful of researchers, engineers and students and applied on other cases.
Making \gls{festim} available to other researchers (by making it open-source) would greatly benefit the broader community.

The method of running parametric sub-component simulations and then extract a behaviour law was nothing new.
However, applying it to the estimation of the \gls{divertor} inventory saved a significant amount of time compared to using the ``brute force'' approach of modelling the whole \gls{divertor} at a time.
This method can be applied to other components like the \gls{breeding blanket}, where the tritium inventory is also an issue.

Finally, the model developed to simulate the growth of helium bubbles in tungsten gives a rapid way of implementing and testing new physical models.

\section*{Limitations}
\begin{itemize}
    \item since the code was not open-sourced, it was sometimes hard to receive help from external experts
    \item However, using exclusively open-source tools (\gls{fenics}, SALOME, Paraview...), working remotely has not been a technical issue
    \item Time constraints
\end{itemize}

\section*{Recommandations for future work}
\begin{itemize}
    \item Behaviour law should be remade with varying $\varphi_\mathrm{imp} \ R_p$ and $\varphi_\mathrm{heat}$ instead of $c_\mathrm{surface}$ and $T_\mathrm{surface}$
    \item Inventory in other in-vessel components (like breeding blankets) should be investigated
    \item Simultaneous H/He exposure to investigate if the He exposure still inhibits the hydrogen trapping
\end{itemize}

