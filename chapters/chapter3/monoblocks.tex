\setchapterimage{monoblocks2}
\setchapterpreamble[u]{\margintoc}
\chapter{Monoblocks}\label{Chapter3}\labch{Chapter3}

\section{Introduction}
ITER and DEMO divertors will be composed of small unit bricks called \textit{monoblocks} (see Figure \ref{fig: inner target photo}).
Monoblocks are typically made of a tungsten substrate with a cooling pipe running through.
This cooling channel is necessary to keep the component's temperature below its operating limit.

Several monoblock designs are currently studied for DEMO with varying dimensions, different materials for the cooling pipe or the interlayer, etc \sidecite{vizvary_european_2020, huang_tungsten_2016, hirai_use_2016, domptail_design_2020}.
The main candidate is the ITER-like design, which is the type of monoblock that will be used in ITER \cite{hirai_use_2016}.
This design has a tunsgten substrate with a CuCrZr cooling pipe and a Cu interlayer for compliance (see Figure \ref{fig: monoblocks with pipe}).
In ITER, monoblocks will be \SI{12}{mm}-thick whereas they will be thiner in DEMO (\SI{4}{mm}) \sidecite{you_european_2018}.

\begin{figure}
    \centering
    \includegraphics[width=\linewidth]{Figures/Chapter3/inner_target_iter.jpg}
    \caption{Prototype of the inner vertical target (source: ITER Organization).}
    \label{fig: inner target photo}
\end{figure}

\begin{figure}
    \centering
    \includegraphics[width=0.7\linewidth]{Figures/Chapter3/monoblocks_with_pipe.png}
    \caption{ITER-like monoblocks.}
    \label{fig: monoblocks with pipe}
\end{figure}

In order to assess the behaviour of H in the divertor, component-level simulations of monoblocks are required.

This chapter will focus on simulating H transport in ITER-like monoblocks.
First, the influence of interface conditions at the W/Cu and Cu/CuCrZr interfaces will be evaluated.
% Then, the impact of load cycling (heat and particles fluxes) will be assessed.
A parametric study will then be performed to link the monoblock H inventory to the exposure conditions.
Finally, the influence of edge effects (desorption on poloidal and toroidal gaps) will be highlighted.

\section{Interface conditions}
In order to assess the influence of interface conditions on the outgassing flux, simulations are performed with chemical potential continuity (Equation \ref{eq: c/s conservation}) or mobile concentration continuity assuming in both cases flux conservation.
For the sake of simplicity and to emphasis on the influence of interface conditions, no trapping was assumed and ideal Dirichlet boundary conditions were set.
Simulations were performed on two test cases: W/Cu and Cu/EUROFER.
The materials properties used for the simulations can be found in Table \ref{tab:materials properties_1}.
In both cases the solute concentration $c_\mathrm{m}$ was set to \SI{1e20}{m^{-3}} at $x=0$ and zero on the other boundary.

% The steady-state temperature field exhibits high temperature gradients between the plasma-facing surface and the cooling surface (see Figure \ref{fig:2D temperature monoblock}).

\begin{figure}
    \centering
    \includegraphics[width=0.8\linewidth]{Figures/Chapter3/monoblocks/interface_condition/iter case/temperature_field_2d.pdf}
    \caption{Steady-state temperature field of an ITER-like monoblock simulated with FESTIM.}
    \label{fig:2D temperature monoblock}
\end{figure}


% TODO replace this figure!!!
\begin{figure}
    \centering
    \includegraphics[width=\linewidth]{Figures/Chapter3/monoblocks/interface_condition/iter case/comparison_inventory_2d.pdf}
    \caption{Influence of chemical potential conservation on hydrogen inventory.}
    \label{fig: 2D inventories}
\end{figure}


\begin{figure}
    \centering
    \begin{subfigure}{0.5\linewidth}
        \centering
        \includegraphics[width=\linewidth]{Figures/Chapter3/monoblocks/interface_condition/iter case/solute_c.pdf}
        \caption{$c_\mathrm{m}$ (continuity of $c_\mathrm{m}$)}
    \end{subfigure}%
    \begin{subfigure}{0.5\linewidth}
        \centering
        \includegraphics[width=\linewidth]{Figures/Chapter3/monoblocks/interface_condition/iter case/solute_mu.pdf}
        \caption{$c_\mathrm{m}$ (continuity of $\mu$)}
    \end{subfigure}
    \begin{subfigure}{0.5\linewidth}
        \centering
        \includegraphics[width=\linewidth]{Figures/Chapter3/monoblocks/interface_condition/iter case/retention_c.pdf}
        \caption{Retention (continuity of $c_\mathrm{m}$)}
    \end{subfigure}%
    \begin{subfigure}{0.5\linewidth}
        \centering
        \includegraphics[width=\linewidth]{Figures/Chapter3/monoblocks/interface_condition/iter case/retention_mu.pdf}
        \caption{Retention (continuity of $\mu$)}
    \end{subfigure}
    \caption{2D concentration fields at $t=\SI{2.4e7}{s}$}
    \label{fig: concentrations fields 2d}
\end{figure}
Two cases were examined, one with mobile concentration $c_\mathrm{m}$ continuity at interfaces and the other with continuity of chemical potential (\textit{ie.} continuity of $c_\mathrm{m}/S$).

Up to \SI{5e6}{s}, there was no difference in the total hydrogen inventory between the two cases (see Figure \ref{fig: 2D inventories}).
It is only after this implantation that the inventory of the two cases started to diverge.
At $t=\SI{2.4e7}{s}$, the inventory of the continuity of chemical potential case was higher than with continuity of $c_\mathrm{m}$.
This is explained by the high solubility ratio between Cu and CuCrZr leading to a higher concentration of mobile particles in CuCrZr and therefore a higher trapping rate.
However, even then, the trap density in Cu being low compared to other materials, the global inventory is not affected much.
For these two reasons, the inventories are unaffected before \SI{5e6}{s}.

Similarily, before reaching the W/Cu interface, the $c_\mathrm{m}$ and retention profiles are identical regardless of the interface condition (see Figure \ref{fig: concentrations fields 2d}).
Once the W/Cu interface is reached, the $c_\mathrm{m}$ profiles are affected by the interface condition.
The interface condition had no influence whatsoever on the mobile particle concentration $c_\mathrm{m}$ in the W.
However, $c_\mathrm{m}$ was higher in Cu and CuCrZr in the case with chemical potential conservation (up to \SI{1.5e24}{m^{-3}} in CuCrZr at $t=\SI{2.4e7}{s}$).
This increase of $c_\mathrm{m}$ leads to an increase of the trap occupancy and therefore an increase of the local retention.

The retro-desorbed flux (from the monoblock to the plasma) does not depend on the interface conditions since interfaces are far from the exposed surface.
Moreover, outgassing flux through the cooling pipe greatly depends on the boundary condition imposed at the cooling surface.
Therefore, in order to assess the impact of interface conditions on the outgassing flux through the cooling pipe, uncertainties must first be lift regarding the recombination process occurring on surfaces in contact with water.

% \subsection{Summary}
% should we keep this
% The influence of interface conditions between materials has been studied with FESTIM.
% A novel approach has been implemented in FESTIM in order to ensure equilibrium at the interfaces.
% The implementation has been verified using the Method of Exact Solutions and the Method of Manufactured Solutions.
% % A comparison test has been performed with TMAP7 and Abaqus and the three codes show very good agreement.

% H transport through Cu/EUROFER and W/Cu composite slabs has been studied.
% It is shown that the interface condition can have an impact on the outgassing flux.
% This modelling work will help design future permeation barriers in DEMO.
% A method for identifying material properties with either an analytical solution or with the FESTIM code is also described.

% The influence of interface conditions is also studied on the ITER monoblock test case in both 1D and 2D.
% It is shown that this has very low influence up to \SI{5e6}{s} and that discrepancies only start to appear after a very long exposure time.
% This is because interfaces are far from the exposed surface and hydrogen atoms only reach these interface after a long exposure time.
% The continuity of mobile concentration can therefore be employed safely for monoblocks H transport simulations in order to assess monoblocks inventory.
% This is especially true when desorption is assumed on the edges of the monoblock where less H particles will reach the interfaces.
% In other words, the effect of the interfaces conditions is negligible compared to the edge effects (this will be shown in more details in Section \ref{3D edge effects}).


% \section{Influence of cycling}

% \begin{itemize}
%     \item Redo one or two sims in Etienne paper ?
%     \item continuous first to load the monoblock then cycling
% \end{itemize}

\section{Exposure conditions} \label{influence of exposure conditions}
Monoblocks in a fusion reactor will be exposed to a wide range of exposure conditions (heat and particle fluxes) and their behaviour (both in terms of heat transfer and hydrogen transport) will change based on these conditions.
In ITER, these fluxes range from $\approx$ \SI{10}{MW.m^{-2}} and $\approx$ \SI{e23}{H.m^{-2}.s^{-1}} to $\approx$ \SI{10}{MW.m^{-2}} and $\approx$ \SI{e23}{H.m^{-2}.s^{-1}} at the inner strike point.
The distribution of these fluxes depend on many operation parameters.

One way of simulating a whole divertor would be to simulate each and every monoblock for a given scenario along one Plasma-Facing Unit.
However, this method would be computationally expensive as it requires to redo the simulations for every scenario.

Another, more efficient method, is to perform a parametric study on a monoblock.
The exposure parameters are varied and for each set of parameters, the quantity of interest (here the hydrogen inventory) is computed.
A relationship is then produced between the exposure parameters and the quantity of interest.
This method is more robust in the sense that it does not require to run additional simulations once this relationship is obtained but simply use this relationship to obtain the quantity of interest.

In this work, we employed the second method.
The goal of this Section is to establish the relationship between the exposure conditions of the monoblock and its hydrogen content at a given time.

\subsection{Simulation description}

The monoblock model has one trap per material for the sake of simplicity and computational time, but the method could be applied to any set of trapping parameters.


\subsubsection{Boundary conditions}

The concentration of mobile particles $c_\mathrm{m}$ is imposed on $\Gamma_\mathrm{top}$.
Molecular recombination is assumed on $\Gamma_\mathrm{coolant}$.
Even though it could be assumed on the toroidal gaps between monoblocks, it can be shown that its influence on the macroscopic behaviour remains low.
This is because the maximum retention zone is observed near the cooling pipe and not near the gaps.
Desorption from the other surfaces is therefore assumed to be zero for simplification purposes.
Uniform heat loads $\varphi_H$ are first applied on the surface $\Gamma_\mathrm{top}$ with a Neumman boundary condition.
The temperature will then be constrained on $\Gamma_\mathrm{top}$ with a Dirichlet boundary condition in Section \ref{relationship_T_surf_c_surf}.
A convective exchange condition is set on surface $\Gamma_\mathrm{coolant}$.
All the other surfaces are assumed thermally insulated.
The set of boundary conditions can finally be described as follow:

\begin{subequations}
    \begin{align}
    -\lambda \vec{\nabla} T \cdot \vec{n} &=\varphi_{H} \quad \text{or} \quad T = T_\mathrm{surface}\quad &\text { on } \Gamma_\mathrm{top}\\
    c_\mathrm{m} &=  c_\mathrm{surface}\quad &\text { on } \Gamma_\mathrm{top}\\
    -\lambda \vec{\nabla} T\cdot \vec{n} &= -h \cdot \left(T_\mathrm{coolant} - T\right)\quad &\text { on } \Gamma_\mathrm{coolant}\\
    -D \vec{\nabla} c_\mathrm{m} \cdot \vec{n} &= K_\mathrm{CuCrZr} \cdot c_\mathrm{m}^{2} \quad &\text { on } \Gamma_\mathrm{coolant}
    \end{align}
\end{subequations}
with $h=\SI{70000}{W.m^{-2}.K^{-1}}$ being the heat exchange coefficient calculated from the Sieder-Tate correlation for the forced convection regime, $T_\mathrm{coolant}= \SI{323}{K}$ and $\vec{n}$ the normal vector and $K_\mathrm{CuCrZr} = 2.9 \times 10^{-14}\cdot \exp{(-1.92/(k_B\cdot T))}$ the recombination coefficient of the copper alloy (in vacuum) expressed in \si{m^4.s^{-1}} \sidecite{anderl_deuterium_1999}.

\subsection{Influence of \texorpdfstring{$T_\mathrm{surface}$}{Tsurface} and \texorpdfstring{$c_\mathrm{surface}$}{csurface} on hydrogen inventory} \label{relationship_T_surf_c_surf}

In this section, the total inventory of hydrogen in monoblocks has been calculated as a function of $T_\mathrm{surface}$ and $c_\mathrm{surface}$.
Temperature and mobile concentration of hydrogen were imposed with Dirichlet boundary conditions on $\Gamma_\mathrm{top}$ with $T_\mathrm{surface}$ varying from $T_\mathrm{coolant}$ to \SI{1200}{K} and $c_\mathrm{surface}$ varying arbitrarily from \SI{e20}{m^{-3}} to \SI{6e22}{m^{-3}}.
For surface temperatures below \SI{500}{K}, 1D simulations were performed for the penetration depth of hydrogen remained very low (a few microns) and 1D approximation was sufficient \sidecite{benannoune_numerical_2019}.
For temperatures above \SI{500}{K} for which edge effects become dominant, 2D simulations have been performed.

After $ \SI{e7}{s}$ a high retention zone appeared far from the exposed surface $\Gamma_\mathrm{top}$ (see Figure \ref{fig:retention fields}).
This high retention zone is due to thermal effects.
As seen in Figures \ref{fig:T field 1 MW} and \ref{fig:T field 10 MW}, the temperature was found to decrease in regions close to the cooling pipe $\Gamma_\mathrm{coolant}$ leading to an increase in trap occupancy, creating this high retention zone.
This is however not true for monoblocks where $T_\mathrm{surface} \approx T_\mathrm{coolant}$ since the temperature gradient in the domain is very low.
Instead, trap occupancy is close to one and the retention is high in the whole region where hydrogen has penetrated and not only far from the top surface.

\begin{figure*}
    \centering
    \begin{subfigure}{0.5\linewidth}
        \centering
        \includegraphics[height=\linewidth]{Figures/Chapter3/monoblocks/parametric_study/retention_T=7.000e+02;c=1.00e+20.pdf}
        \caption{$T_\mathrm{surface} = \SI{700}{K}$ and $c_\mathrm{surface} = \SI{e20}{m^{-3}}$}
    \end{subfigure}%
    \begin{subfigure}{0.5\linewidth}
        \centering
        \includegraphics[height=\linewidth]{Figures/Chapter3/monoblocks/parametric_study/retention_T=1.000e+03;c=1.00e+21.pdf}
        \caption{$T_\mathrm{surface} = \SI{1000}{K}$ and $c_\mathrm{surface} = \SI{e21}{m^{-3}}$}
    \end{subfigure}
    \caption{Example retention fields in \si{m^{-3}} after a \SI{e7}{s} exposure}
    \label{fig:retention fields}
\end{figure*}


In order to obtain this continuous field (see Figure \ref{fig:inventory T c}), more than 600 simulations randomly distributed on the parameter plane were run and analysed using a Gaussian process machine learning algorithm \sidecite{rasmussen_gaussian_2006} as in \sidecite{shimwell_multiphysics_2019} based on the python package inference-tools \sidecite{chris_bowman_c-bowmaninference-tools_2020}.
The inventory obtained by the Gaussian regression process is also given for a constant value of $c_\mathrm{surf}=\SI{2e21}{m^{-3}}$ (top inset) and a constant temperature $T=\SI{850}{K}$ (left inset).
The Gaussian regression process was particularly appropriate as it calculates a local standard deviation $\sigma$ based on the localisation of the datapoints and the deviation of the computed inventories.
The lower the density of simulation points, the higher was the value of $\sigma$ (for example around \SI{850}{K} on the top inset of Figure \ref{fig:inventory T c}).
However, despite the lack of simulation in this region, the value of $\sigma$ was still acceptable (only a few percents of the inventory) ensuring the quality of the resulting interpolation.

As expected, inventory was found to globally increase with $c_\mathrm{surface}$.
For $T_\mathrm{surface} > \SI{550}{K}$, the inventory tended to decrease with surface temperature.
However, for $T_\mathrm{surface} < \SI{550}{K}$, inventory increased with surface temperature.
This phenomenon is due to a trade-off between an increase of the detrapping rate and an increase of the diffusion coefficient making the hydrogen particles penetrate deeper into the bulk.
Above $\SI{550}{K}$, detrapping becomes dominant and inventory decreases.
This mapping of inventory as a function of $T_\mathrm{surface}$ and $c_\mathrm{surface}$ provides an easy way of estimating the inventory in monoblocks for several exposure conditions without having to run many simulations.
Indeed, to estimate the inventory with different exposure conditions, one only needs to associate these conditions $(\varphi_\mathrm{inc}, E)$ to a couple $(c_\mathrm{surf}, T_\mathrm{surf})$.

\begin{figure*} [h]
    \centering
    \includegraphics[width=\linewidth]{Figures/Chapter3/monoblocks/parametric_study/inventory_T_c_profiles.pdf}
    \caption{Evolution of the inventory after a \SI{e7}{s} exposure as a function of $T_\mathrm{surface}$ and $c_\mathrm{surface}$ alongside with simulation points (grey crosses). The simulations points were fitted with a Gaussian regression process \cite{chris_bowman_c-bowmaninference-tools_2020} providing the standard deviation $\sigma$.}
    \label{fig:inventory T c}
\end{figure*}

\subsection{Discussion}
Even though this methodology provides a rapid way of estimating hydrogen content in the whole divertor, several assumptions have however been made.


% Influence of cycling
First, a steady state exposure was considered for simplification purposes.
This result is however conservative.
As seen in \sidecite{delaporte-mathurin_finite_2019, hodille_estimation_2017}, cycling effects could have an influence in regions where $T_\mathrm{surface}$ varies a lot, for example within \SI{10}{cm} on both sides of the strike points.
Though, since a large majority of monoblocks stay at room temperature, even during operations the thermal effect should remain low and discrepancies would rather be due to particle flux evolution along the target.

% Be deposits
This study presents the hydrogen trapping in W monoblocks.
It shows that the latter remains low but, as already pointed out by JET studies, the trapping on Be co-deposited layers is expected to be the main mechanism for tritium retention in ITER \sidecite{brezinsek_beryllium_2015, heinola_fuel_2015}.
Such layers could be found in the cold regions of the divertor but as soon as the strike points hit these layers, they should be sputtered away (as sputtering of Be is possible even at low energy \cite{bjorkas_variables_2013, brezinsek_beryllium_2015}).
The retention where the deposited layers are not present (either sputtered or not formed anyway) would then be given by the model presented here.

% Coolant recombination
The molecular recombination coefficient at the surface of the cooling pipe was taken from \sidecite{anderl_deuterium_1999} and was measured in vacuum.
One could argue that recombination in presence of water will be facilitated.
This parameter has a very low influence on the inventory since it is dominated by retention in tungsten.
This parameter will however have an influence on the permeation flux and should be studied in future work.

% Gap recombination
Similarly, the influence on molecular recombination on the sides of the monoblock was found to have a low impact on the results.
By assuming an instantaneous recombination coefficient, the relative error on the monoblock inventory was found to be significant only in hot regions (\textit{ie} within \SI{10}{cm} on both sides of the strike points).
The influence on the total divertor inventory is therefore low (less than \SI{5}{\%} after a \SI{e7}{s} exposure) since it is dominated by regions where $T_\mathrm{surface} \approx T_\mathrm{coolant}$.

% ELMs
It should be noted that specific scenarios like edge localised modes (ELMs) were also not taken into account in this work since their time scale is very short.
ELMs are transient plasma events releasing thermal energy and locally increasing the heat flux at the surface of the monoblock.
MRE simulations by Hu and Hassanein \sidecite{hu_predicting_2015} suggest that a \SI{400}{s} discharge with \SI{1}{Hz} or \SI{10}{Hz} ELMs significantly reduces (77 \%) the inventory in W materials.
However, the modelling of the ELM is simulated by increasing the temperature for a very short time without changing the incident flux of particles that can also be much higher thus balancing the fuel retention reduction.
Another study by Schmid \textit{et al} \sidecite{schmid_diffusion-trapping_2016} also simulated the effect of \SI{1}{Hz} ELMs on fuel retention in W.
The outcome is that \SI{6}{s} of \SI{1}{Hz}-ELMs does not affect significantly the fuel retention, though the temperature excursion in those simulations are smaller than for the one of Hu and Hassanein.
Thus, the effect of ELMs, especially the balance between increase of heat flux, incident energy and particle flux, could either favour or disfavour trapping, diffusion and migration and therefore the overall retention.

% Surface process
In this study the model to link the concentration of mobile particles at the surface (implantation zone) with the exposure condition considers that the particles are implanted in the bulk and that the recombination coefficient is very high since many uncertainties concerning the recombination coefficient are yet to be lifted.
However, if an exothermic process is considered as in \sidecite{ogorodnikova_recombination_2019}, this should have low influence since recombination is very quick at a temperature close to that of the coolant.

On the other hand, experimental results \sidecite{t_hoen_strongly_2013} suggest that for ion energy below \SI{5}{eV/H}, typical of detached plasma as the one treated in the previous section, the surface process can be important and limits the uptake of hydrogen, i.e. the adsorption on the surface and the further absorption from surface to bulk could be the limiting process for the growth of $c_\mathrm{surface}$ during such exposure.
The evolution of $c_\mathrm{surface}$ to the exposure condition for that range of energy (and therefore the inventory) would then be different.
The advantage of the presented method is that taking into account such process is realtively easy as no expensive simulations are needed.
One would only need to modify the model giving $c_\mathrm{surface}$
as a function of $(E_\mathrm{inc},\varphi_\mathrm{inc})$ to include the different surface processes.
To this end, one can use kinetic surface models \sidecite{hodille_retention_2017, zaloznik_deuterium_2017, pecovnik_influence_2019, guterl_effects_2019}.

% traps
Trap properties have a great impact on the inventory.
In this study, a homogeneous trap distribution is assumed for simplification purposes.
A more thorough study could investigate the influence on trap distribution, energy and density.
Trap properties might also vary along the divertor based on exposure conditions.
Moreover the impact of neutrons must be assessed as neutron-induced traps have a high detrapping energy.


% Helium

Finally, helium implantation in the materials and bubble formation could modify the hydrogen transport in monoblocks.

% Yann doesn't like Summaries, we'll see what the others say
% \subsection{Summary}
% ITER-like monoblocks have been studied using a novel method in order to estimate the hydrogen content as a function of exposure conditions such as the implanted particle flux, the ion energy, the heat load and the monoblock surface temperature.
% Several hundred data points have been simulated with FESTIM and analysed to estimate the hydrogen inventory in monoblocks for any input conditions using a Gaussian regression process, a machine learning algorithm which calculates the confidence interval for each point.
% Thanks to this relation, one can easily estimate hydrogen content in the whole divertor without having to run all the simulations.
% An application has been made based on the output from a SOLPS calculation of exposure conditions distribution on the ITER divertor and shows that for these conditions the inventory could reach \SI{e20}{H} per monoblock near strike points after a \SI{e7}{s} exposure.
% The total hydrogen content in ITER divertor is estimated to be \SI{8}{g} which is well below the inner-vessel safety limit of \SI{1}{kg}.

% This behaviour law will be used in \refch{Divertor inventory estimation} to estimate the hydrogen inventory of WEST and ITER divertors.

\section{Edge effects} \label{3D edge effects}
So far, only 2D monoblocks simulations were run, assuming an infinite thickness (or assuming no desorption from the poloidal gaps).
The goal of this section is to assess the influence of 3D edge effects on the monoblocks simulation results.
It will be shown that the error induced by 2D assumption decreases for thick monoblocks.

\subsection{Methodology}

The DEMO monoblock geometry differs slightly from the ITER geometry (see Figure \ref{fig: geometry DEMO monoblock}) but the general concept is the same, meaning the observations made in this Section are valid for the ITER geometry.


\begin{figure}
    \centering
        \begin{overpic}[width=\linewidth]{Figures/Chapter3/monoblocks/3D_monoblocks/sketch.pdf}
            \put(21, 15){\SI{23}{mm}}
            \put(51, 50){\SI{25}{mm}}
            % \put(21, 15){\SI{13.5}{mm}}
            \put(21, 56){ \diameter \SI{12}{mm}}
            \put(21, 63){ \diameter \SI{15}{mm}}
            \put(21, 68){ \diameter \SI{17}{mm}}
            \put(10, 70){\large$\Gamma_\mathrm{top}$}
            \put(0, 60){\large$\Gamma_\mathrm{lateral}$}
            \put(43, 60){\large$\Gamma_\mathrm{lateral}$}
            \put(21, 45){\large$\Gamma_\mathrm{coolant}$}
            \put(66, 71){\SI{4}{mm}}
            \put(66, 50){\SI{5}{mm}}
        \end{overpic}
    \caption{3D geometry of the DEMO monoblock used for the simulations showing W armour \cruleme[grey]{0.3cm}{0.3cm}, Cu interlayer \cruleme[orange]{0.3cm}{0.3cm}, CuCrZr alloy cooling pipe  \cruleme[yellow]{0.3cm}{0.3cm}.}
    \label{fig: geometry DEMO monoblock}
\end{figure}

The boundary conditions for the steady state heat transfer problem are the same as for the 2D case (see Equation \ref{eq: bc thermal DEMO monoblock}).
The boundary conditions for the transient H transport problem are similar (see Equations \ref{eq: bc H transport DEMO monoblock}).
A non-homogeneous mobile concentration is assumed at the plasma exposed surface to simulate an implanted source of particles (see Section \ref{triangle model}).
Depending on the simulation case (with or without desorption on the gaps), the other external surfaces (except the cooling surfaces) will either be insulated or an instantaneous recombination will be assumed.


\begin{subequations}
    \begin{align}
    -\lambda \vec{\nabla} T \cdot \vec{n} &=\varphi_\mathrm{heat} \quad  &\text { on } \Gamma_\mathrm{top}\\
    -\lambda \vec{\nabla} T\cdot \vec{n} &= -h \cdot \left(T_\mathrm{coolant} - T\right)\quad &\text { on } \Gamma_\mathrm{coolant}
    \end{align}
    \label{eq: bc thermal DEMO monoblock}
\end{subequations}
where $\varphi_\mathrm{heat}=\SI{10}{MW}$, $h=\SI{7e4}{W.m^{-2}.K^{-1}}$ is the heat exchange coefficient and $T_\mathrm{coolant} = \SI{323}{K}$ is the coolant temperature.

\begin{subequations}
    \begin{align}
    c_\mathrm{m} &=  \frac{\varphi_\mathrm{imp} R_p}{D} \quad &\text { on } \Gamma_\mathrm{top}\\
    -D \vec{\nabla} c_\mathrm{m} \cdot \vec{n} &= K_\mathrm{CuCrZr} \cdot c_\mathrm{m}^{2} \quad &\text { on } \Gamma_\mathrm{coolant} \\
    c_\mathrm{m} &=  0 \quad \text{or} \quad -D \vec{\nabla} c_\mathrm{m} \cdot \vec{n} = 0 &\text { on } \Gamma_\mathrm{lateral} \text{  and  } \Gamma_\mathrm{pipe}
    \end{align}
    \label{eq: bc H transport DEMO monoblock}
\end{subequations}
where $\varphi_\mathrm{imp} = \SI{1.6e22}{H.m^{-2}.s^{-1}}$ is the implanted particle flux, $R_p = \SI{1e-9}{m}$ is the particle implantation depth, $K_\mathrm{CuCrZr}=2.9\times 10^{-14}\exp{-1.92/(k_B T)}$ is the H recombination coefficient in CuCrZr expressed in \si{m^4.s^{-1}}.

Two intrinsic traps were set in W, one trap in the Cu interlayer and two traps in the CuCrZr cooling pipe (see Table \ref{tab:traps monoblock DEMO}).

\begin{table*}
    \centering
    \begin{tabular}{L{1.5cm} L{1.5cm} R{1.6cm} R{1.1cm} R{1.6cm} R{1.1cm} R{2cm}}
         & Material & $k_0 (\si{m^3.s^{-1}})$ &  $E_k (\si{eV})$ & $p_0 (\si{s^{-1}})$ & $E_p (\si{eV})$ & $n_i (\si{at.fr.})$ \\
        \hline
        \\
        Trap 1 & W & $9.0 \times 10^{-17}$ & 0.39 & $1 \times 10^{13}$& 0.78 & $1.0 \times 10^{-3}$ \\
        \\
       Trap 2 & W & $9.0 \times 10^{-17}$ & 0.39 & $1 \times 10^{13}$& 1.00 & $4.0 \times 10^{-4}$ \\
        \\
        Trap 3 & Cu & $6.0 \times 10^{-17}$ & 0.39 & $8.0 \times 10^{13}$ & 0.50 &$5.0 \times 10^{-5}$\\
        \\
        Trap 4 & CuCrZr & $1.2\times 10^{-16}$ & 0.42 & $8.0 \times 10^{13}$ & 0.50 &$5.0 \times 10^{-5}$\\
        \\
        Trap 5 & CuCrZr & $1.2\times 10^{-16}$ & 0.42 & $8.0 \times 10^{13}$ & 0.83 &$4.0 \times 10^{-2}$\\
        \\
    \end{tabular}
    \caption{Traps properties used in the 3D DEMO monoblocks simulations.}
    \label{tab:traps monoblock DEMO}
\end{table*}

Transient simulations up to \SI{e7}{s} were run.

\subsection{Standard case}

FESTIM simulations were run with and without desorption on the gaps.

% Temperature field

The temperature field obtained was very similar to the 2D case (see Figure \ref{fig: T field 3D monoblock}) with a top surface temperature of approximately \SI{1200}{K}.

\begin{figure} [h]
    \centering
    \includegraphics[width=\linewidth]{Figures/Chapter3/monoblocks/3D_monoblocks/temperature_3D_monoblock.png}
    \caption{Temperature field of the 3D DEMO monoblock.}
    \label{fig: T field 3D monoblock}
\end{figure}


% Retention fields
As expected, a higher retention was observed in the case without desorption (see Figure \ref{fig:retention fields 3D monoblocks}).
This is explained by the surface losses.
The maximum retention for the case with desorption is three orders of magnitude lower than that of the case without desorption.
In both cases, the higher retention was found to be in the CuCrZr cooling pipe.
This is consistent with the obsverations made previously.

\begin{figure} [h]
    \centering
    \begin{subfigure}{\linewidth}
        \centering
        \includegraphics[width=\linewidth]{Figures/Chapter3/monoblocks/3D_monoblocks/MB 3D desorption.png}
        \caption{Instantaneous recombination on the gaps.}
    \end{subfigure}
    \begin{subfigure}{\linewidth}
        \centering
        \includegraphics[width=\linewidth]{Figures/Chapter3/monoblocks/3D_monoblocks/MB 3D no desorption.png}
        \caption{No desorption on the gaps.}
    \end{subfigure}
    \caption{Retention fields of the DEMO monoblock after \SI{e7}{s} of continuous exposure with or without recombination on the gaps showing the isometric view (left), a central slice (top right) and a central clip showing the cooling pipe (bottom right). Note that the colour bars are different.}
    \label{fig:retention fields 3D monoblocks}
\end{figure}

% Inventory
The total H inventory in the monoblock was also between one and three orders of magnitude lower in the case with desorption (see Figure \ref{fig: inventory vs time DEMO monoblock}).
This difference increased with the exposure time.
Moreover, the steady state was reached way earlier for the case with desorption whereas the inventory kept increasing after \SI{e7}{s} for the insulated case.
This means that not taking desorption from the gaps into account in 2D simulations is a conservative assumption in terms of H inventory.
The simulations performed in Section \ref{influence of exposure conditions} then overestimate the monoblock H inventory.

\begin{figure} [h]
    \centering
    \includegraphics[width=\linewidth]{Figures/Chapter3/monoblocks/3D_monoblocks/inventory.pdf}
    \caption{Temporal evolution of the monoblock inventory.}
    \label{fig: inventory vs time DEMO monoblock}
\end{figure}

% fluxes
However, the outgassing fluxes from the monoblock gaps cannot be estimated with 2D (or 1D) simulations.
These fluxes were found to be five orders of magnitude higher than the permeation flux to the coolant: the particle flux towards the vacuum vessel was approximately \SI{e12}{H.s^{-1}} whereas the flux towards the cooling channel was below \SI{e8}{H.s^{-1}} (see Figure \ref{fig: fluxes DEMO monoblock}).
These fluxes can be compared to the retrodesorbed flux (\textit{ie} the flux of implanted particles that diffuse back to the exposed surface).
According to Equation \ref{eq:flux balance}, the value of this flux is equal to $\varphi_\mathrm{imp} \times \SI{23}{mm} \times \SI{4}{mm} = \SI{1.5e18}{H.s^{-1}}$.
The values of the outgassing fluxes from both the gaps and the cooling surface are therefore orders of magnitude lower than that of the retrodesorbed flux.
This means 3D edge effects will not affect previous results regarding the outgassing to the vessel.
They will however impact the value of the contamination flux towards the coolant as assuming an instantaneous recombination on the gaps will lead to way less particles reaching the cooling surface and therefore a lower flux.

\begin{figure} [h]
    \centering
    \includegraphics[width=\linewidth]{Figures/Chapter3/monoblocks/3D_monoblocks/fluxes.pdf}
    \caption{Temporal evolution of outgassing fluxes for the case with desorption from the gaps.}
    \label{fig: fluxes DEMO monoblock}
\end{figure}

\subsection{Influence of the monoblock thickness}

The same simulations were run with different monoblock thicknesses (3, 4 and \SI{14}{mm}).
In order to investigate the influence of the recombination on the gaps, the ratio $\mathrm{inv}_\mathrm{desorption} / \mathrm{inv}_\mathrm{no desorption}$ (monoblock inventories with or without desorption) was computed for each simulation case.
The closer the ratio is to unity, the less significant the desorption.

Increasing the monoblock thickness lead to an increase of the ratio (see Figure \ref{fig: ratio 3D thickness monoblock}).
For the smallest thickness (\SI{3}{mm}), the ratio $\mathrm{inv}_\mathrm{desorption} / \mathrm{inv}_\mathrm{no desorption}$ was approximately \SI{10}{\%} at \SI{10000}{s}.
It decreased down to $3\times 10^{-4}$ after \SI{1e7}{s} of exposure.
The same behaviour was observed for all thicknesses though the decrease was less significant for higher thicknesses.

\begin{figure} [h]
    \centering
    \includegraphics[width=\linewidth]{Figures/Chapter3/monoblocks/3D_monoblocks/influence_of_thickness.pdf}
    \caption{Temporal evolution of the ratio $\mathrm{inv}_\mathrm{desorption} / \mathrm{inv}_\mathrm{no desorption}$ for several thicknesses.}
    \label{fig: ratio 3D thickness monoblock}
\end{figure}

\subsection{Summary}
The influence of H desorption on poloidal and toroidal gaps between monoblocks was investigated.
The DEMO baseline geometry (based on an ITER-like concept) was simulated with FESTIM.
It was shown that taking into account recombination on these gaps could drastically decrease the monoblock H inventory (by several orders of magnitude).
Said otherwise, this makes the 2D assumption for monoblocks less valid for 1) thin monoblocks 2) long exposure times.
Indeed, a 2D assumption implies assuming an infinite monoblock thickness (which is totally unrealistic for the ITER and DEMO designs).
However, this assumption is conservative since it will only overestimate the H inventory.
Moreover, performing 3D simulations is more computationaly expensive than 2D simulations.

Regarding the outgassing flux assessment, taking into acount


\section{Summary}
H transport in ITER-like monoblocks was simulated with FESTIM.
Several aspects of the simulations were studied.

It was shown that the choice of interface conditions (continuity of chemical potential or continuity of mobile concentration) had low impact on the monoblock inventory.
Depending on the boundary condition used at the cooling surface of the monoblock, it could however have an impact on the permeation flux.

The effect of loading cycles was also investigated.
Between plasma pulses, a zone with higher retention appears near the plasma exposed surface due to the temperature variation.
These modifications of the retention fields vanish as soon as the next cycle starts again and this zone is heated up again.
This means that cycling has no effect on the global retention field and that cycles can safely be concatenated (continuous exposure) to simulate H transport.

A parametric study was then performed in order to assess the influence of exposure conditions (surface concentration and surface temperature).
A 2D behaviour law was obtained correlating exposure conditions to the monoblock inventory.
This law will be extremely useful to estimate H retention in divertors since not all monoblocks will be exposed to the same exposure.

Finally, 3D simulations of DEMO ITER-like monoblocks have been run.
The desorption on the poloidal and toroidal gaps drastically reduced the inventory while increasing the outgassing flux to the vacuum chamber.
The permeation flux to the coolant decreased.
This means that 2D simulations are conservative in terms of H inventory.
This however undersestimates outgassing fluxes.
Future work should involve refining the behaviour law obtained in Section \ref{influence of exposure conditions} by running new 3D simulations and compute not only the H inventory but also the total outgassing.
