%%%%%%%%%%%%%%%%%%%%%%%%%%%%%%%%%%%%%%%%%
% kaobook
% LaTeX Template
% Version 1.2 (4/1/2020)
%
% This template originates from:
% https://www.LaTeXTemplates.com
%
% For the latest template development version and to make contributions:
% https://github.com/fmarotta/kaobook
%
% Authors:
% Federico Marotta (federicomarotta@mail.com)
% Based on the doctoral thesis of Ken Arroyo Ohori (https://3d.bk.tudelft.nl/ken/en)
% and on the Tufte-LaTeX class.
% Modified for LaTeX Templates by Vel (vel@latextemplates.com)
%
% License:
% CC0 1.0 Universal (see included MANIFEST.md file)
%
%%%%%%%%%%%%%%%%%%%%%%%%%%%%%%%%%%%%%%%%%

%----------------------------------------------------------------------------------------
%	PACKAGES AND OTHER DOCUMENT CONFIGURATIONS
%----------------------------------------------------------------------------------------

\documentclass[
    a4paper, % Page size
    fontsize=10pt, % Base font size
    twoside=true, % Use different layouts for even and odd pages (in particular, if twoside=true, the margin column will be always on the outside)
	%open=any, % If twoside=true, uncomment this to force new chapters to start on any page, not only on right (odd) pages
	%chapterentrydots=true, % Uncomment to output dots from the chapter name to the page number in the table of contents
	numbers=noenddot, % Comment to output dots after chapter numbers; the most common values for this option are: enddot, noenddot and auto (see the KOMAScript documentation for an in-depth explanation)
]{kaobook}

% Set the language
\usepackage[english]{babel} % Load characters and hyphenation
\usepackage[english=british]{csquotes} % English quotes

\setcounter{margintocdepth}{\sectiontocdepth}


% Load packages for testing
\usepackage{blindtext}
%\usepackage{showframe} % Uncomment to show boxes around the text area, margin, header and footer
%\usepackage{showlabels} % Uncomment to output the content of \label commands to the document where they are used
\usepackage{graphicx}
\usepackage[percent]{overpic}
\usepackage[utf8]{inputenc}
% \usepackage{natbib}
\usepackage{amsmath}
\usepackage{siunitx}
\sisetup{group-separator = {\,}}
\usepackage{subcaption}
\usepackage{todonotes}
\usepackage[version=4]{mhchem}
\usepackage{wasysym}
\usepackage{overpic}
\usepackage{lipsum}
\usepackage[dvipsnames]{xcolor}
\usepackage{calrsfs}

% package for code blocks
\usepackage{listings}

\DeclareMathAlphabet{\pazocal}{OMS}{zplm}{m}{n}
\newcommand{\Aa}{\mathcal{A}}
\newcommand{\Ab}{\pazocal{A}}

\definecolor{codegreen}{rgb}{0,0.6,0}
\definecolor{codegray}{rgb}{0.5,0.5,0.5}
\definecolor{codepurple}{rgb}{0.58,0,0.82}
\definecolor{backcolour}{rgb}{0.95,0.95,0.92}
\definecolor{plasmapink}{RGB}{239, 0, 255}
\definecolor{lithiumgreen}{RGB}{0, 110, 50}
\definecolor{steelgray}{RGB}{138,138,138}
\definecolor{magnetred}{RGB}{237,41,57}

\lstdefinestyle{mystyle}{
    backgroundcolor=\color{backcolour},   
    commentstyle=\color{codegreen},
    keywordstyle=\color{magenta},
    numberstyle=\tiny\color{codegray},
    stringstyle=\color{codepurple},
    basicstyle=\ttfamily\footnotesize,
    breakatwhitespace=false,         
    breaklines=true,                 
    captionpos=b,                    
    keepspaces=true,                 
    numbers=left,                    
    numbersep=5pt,                  
    showspaces=false,                
    showstringspaces=false,
    showtabs=false,                  
    tabsize=2
}

\lstset{style=mystyle}

% Load the bibliography package
\usepackage{kaobiblio}
\addbibresource{bibfile.bib} % Bibliography file

\AtEveryBibitem{
    \clearfield{urlyear}
    \clearfield{urldate}
    \clearfield{urlmonth}
}

\RenewDocumentCommand{\formatmargincitation}{m}{%
	\parencite{#1} \citeauthor*{#1} (\citeyear{#1})\\
}

% Load mathematical packages for theorems and related environments
\usepackage[framed=true]{kaotheorems}

% Load the package for hyperreferences
\usepackage{kaorefs}

\graphicspath{{examples/documentation/images/}{images/}} % Paths in which to look for images

\makeindex[columns=3, title=Alphabetical Index, intoc] % Make LaTeX produce the files required to compile the index
\usepackage[automake, acronym]{glossaries-extra}

\setabbreviationstyle[acronym]{long-short}
\glssetcategoryattribute{acronym}{glossdescfont}{emph}

\renewcommand*{\glsfirstlongdefaultfont}[1]{\emph{#1}}

\makeglossaries % Make LaTeX produce the files required to compile the glossary
\newglossaryentry{monoblock}{
	name=monoblock,
	description={A unit brick made of a tungsten armour with a CuCrZr cooling pipe and a copper interlayer},
}
\newglossaryentry{plasma}{
	name=plasma,
	description={}
}
\newglossaryentry{divertor}{
	name=divertor,
	description={},
}
\newglossaryentry{breeding blanket}{
	name=breeding blanket,
	description={},
}
\newglossaryentry{first wall}{
	name=first wall,
	description={},
}

\newglossaryentry{isotope}{
	name=isotope,
	description={Member of a family of elements sharing the same number of protons but different numbers of neutrons.}
}
\newglossaryentry{transmutation}{
	name=transmutation,
	description={Conversion of a chemical element into another chemical element by changing the number of neutrons or protons in the nucleus.}
}

\newglossaryentry{diffusion}{
	name=diffusion,
	description={}
}
\newglossaryentry{advection}{
	name=advection,
	description={}
}
\newglossaryentry{trapping}{
	name=trapping,
	description={}
}
\newglossaryentry{Soret effect}{
	name=Soret effect,
	description={}
}
\newglossaryentry{thermophoresis}{
	name=thermophoresis,
	description={}
}


\newglossaryentry{dislocation loop}{
	name=dislocation loop,
	description={}
}
\newglossaryentry{self-interstitial}{
	name=self-interstitial,
	description={}
}
\newglossaryentry{vacancy}{
	name=vacancy,
	description={},
	plural=vacancies
}
\newglossaryentry{Frenkel pair}{
	name=Frenkel pair,
	description={}
}
\newglossaryentry{trap mutation}{
	name=trap mutation,
	description={}
}
\newglossaryentry{self-trapping}{
	name=self-trapping,
	description={See trap mutation entry.}
}

\newglossaryentry{loop punching}{
	name=loop punching,
	description={}
}
\newglossaryentry{fuzz}{
	name=fuzz,
	description={}
}
\newglossaryentry{tendril}{
	name=tendril,
	description={}
}
\newglossaryentry{blistering}{
	name=blistering,
	description={}
}
\newglossaryentry{bursting}{
	name=bursting,
	description={}
}


\newglossaryentry{flux}{
	name=flux,
	description={}
}
\newglossaryentry{fluence}{
	name=fluence,
	description={}
}

% Glossary entries (used in text with e.g. \acrfull{fpsLabel} or \acrshort{fpsLabel})
\newacronym{H}{H}{Hydrogen}
\newacronym{He}{He}{Helium}
\newacronym{D}{D}{Deuterium}
\newacronym{T}{T}{Tritium}
\newacronym{Li}{Li}{Lithium}
\newacronym{W}{W}{Tungsten}


\newacronym[longplural={Plasma Facing Units}, shortplural={PFUs}]{pfuLabel}{PFU}{Plasma Facing Unit}
\newacronym[longplural={Plasma Facing Materials}, shortplural={PFMs}]{pfm}{PFM}{Plasma Facing Material}
\newacronym{cfc}{CFC}{carbon fiber composite}
\newacronym{ivt}{IVT}{Inner Vertical Target}
\newacronym{ovt}{OVT}{Outer Vertical Target}
\newacronym{isp}{ISP}{Inner Strike Point}
\newacronym{osp}{OSP}{Outer Strike Point}

\newacronym{hcpb}{HCPB}{Helium-Cooled-Pebble-Bed}
\newacronym{wcll}{WCLL}{Water-Cooled-Lithium-Lead}
\newacronym{hcll}{HCLL}{Helium-Cooled-Liquid-Lead}
\newacronym{dcll}{DCLL}{Dual-Coolant-Lithium-Lead}
\newacronym{tbr}{TBR}{tritium breeding ratio}
\newacronym{fpy}{FPY}{full power year}


\newacronym{candu}{CANDU}{CANada Deuterium Uranium reactor}
\newacronym{west}{WEST}{Tungsten Environment Steady state Tokamak}
\newacronym{jet}{JET}{Joint European Torus}
\newacronym{iter}{ITER}{International Thermonuclear Experimental Reactor}
\newacronym{sparc}{SPARC}{Soonest Private-funded Affordable Robust Compact}
\newacronym{arc}{ARC}{Affordable Robust Compact}
\newacronym{nif}{NIF}{National Ignition Facility}
\newacronym{mast-u}{MAST-U}{Mega Amp Spherical Tokamak Ugrade}
\newacronym{step}{STEP}{Spherical Tokamak for Electricity Production}
\newacronym{ukaea}{UKAEA}{United-Kingdom Atomic Energy Authority}

\newacronym{md}{MD}{Molecular Dynamics}
\newacronym{dft}{DFT}{Density Functional Theory}
\newacronym{eels}{EELS}{Electron Energy Loss Spectroscopy}

\newacronym{bcc}{bcc}{body-centered cubic}


\newacronym{festim}{FESTIM}{Finite Element Simulation of Tritium in Materials}

 % Include the glossary definitions

\makenomenclature % Make LaTeX produce the files required to compile the nomenclature

% Reset sidenote counter at chapters
%\counterwithin*{sidenote}{chapter}

%----------------------------------------------------------------------------------------

% Changing the width between text and margin
\renewcommand{\marginlayout}{%
    \newgeometry{
        top=27.4mm,             % height of the top margin
        bottom=27.4mm,          % height of the bottom margin
        inner=24.8mm,           % width of the inner margin
        textwidth=127mm,        % width of the text
        marginparsep=8.2mm,     % width between text and margin
        marginparwidth=29.4mm,  % width of the margin
    }%
}


\begin{document}

% Define colours
\definecolor{yellow}{RGB}{180, 95, 6}
\definecolor{grey}{RGB}{183, 183, 183}
\definecolor{orange}{RGB}{228, 146, 64}

% Define column types
\newcolumntype{L}[1]{>{\raggedright\arraybackslash}p{#1}}
\newcolumntype{C}[1]{>{\centering\arraybackslash}p{#1}}
\newcolumntype{R}[1]{>{\raggedleft\arraybackslash}p{#1}}
\newcommand\cruleme[3][black]{\textcolor{#1}{\rule{#2}{#3}}}


%----------------------------------------------------------------------------------------
%	BOOK INFORMATION
%----------------------------------------------------------------------------------------
\titlehead{PhD thesis title}
\subject{My awesome subject}

\title[Hydrogen transport in tokamak plasma facing components]{Hydrogen transport in tokamak plasma facing components}
\subtitle{a subtitle}

\author[Rémi Delaporte-Mathurin]{Rémi Delaporte-Mathurin}

\date{\today}

\publishers{An Awesome Publisher}

%----------------------------------------------------------------------------------------

\frontmatter % Denotes the start of the pre-document content, uses roman numerals

%----------------------------------------------------------------------------------------
%	OPENING PAGE
%----------------------------------------------------------------------------------------

%\makeatletter
%\extratitle{
%	% In the title page, the title is vspaced by 9.5\baselineskip
%	\vspace*{9\baselineskip}
%	\vspace*{\parskip}
%	\begin{center}
%		% In the title page, \huge is set after the komafont for title
%		\usekomafont{title}\huge\@title
%	\end{center}
%}
%\makeatother

%----------------------------------------------------------------------------------------
%	DEDICATION
%----------------------------------------------------------------------------------------

\dedication{
	The harmony of the world is made manifest in Form and Number, and the heart and soul and all the poetry of Natural Philosophy are embodied in the concept of mathematical beauty.\\
	\flushright -- D'Arcy Wentworth Thompson
}

%----------------------------------------------------------------------------------------
%	OUTPUT TITLE PAGE AND PREVIOUS
%----------------------------------------------------------------------------------------

% Note that \maketitle outputs the pages before here

% If twoside=false, \uppertitleback and \lowertitleback are not printed
% To overcome this issue, we set twoside=semi just before printing the title pages, and set it back to false just after the title pages
% \KOMAoptions{twoside=semi}
\maketitle
% \KOMAoptions{twoside=false}

%----------------------------------------------------------------------------------------
%	Abstract
%----------------------------------------------------------------------------------------

\chapter*{Abstract}
\addcontentsline{toc}{chapter}{Abstract} % Add the preface to the table of contents as a chapter

Future fusion reactors will use a mixed fuel of deuterium and tritium.
As a radioactive isotope of hydrogen, tritium can represent a nuclear safety hazard and its inventory in the reactors materials must be controlled.
In ITER, the tritium in-vessel safety limit is \SI{700}{g}.

The tritium inventory of the ITER divertor was numerically estimated with the FESTIM code, which was developed to simulate hydrogen transport in tungsten monoblocks.
A parametric study was performed varying the exposure conditions (surface temperature and surface hydrogen concentration) and a behaviour law was extracted.
This behaviour law provided a rapid way of estimating a monoblock inventory for a given exposure time and for given surface concentration and temperature.
This behaviour law was then used and interfaced with output data from the edge-plasma code SOLPS-ITER in order to estimate the hydrogen inventory of the whole ITER divertor.
Under conservative assumptions, the total hydrogen inventory (deuterium and tritium) was found to be well below the ITER tritium safety limit, reaching $\approx \SI{14}{g}$ after 25 000 pulses of \SI{400}{s}.

To investigate the influence of helium exposure on these results, a helium bubble growth model was developed.
The results of this helium growth model were in good agreement with published numerical results and experimental observations.
A parametric study was performed to investigate the influence of exposure conditions on the bubbles density and size.
To investigate the influence of helium bubbles on hydrogen transport, deuterium TDS experiments of tungsten pre-damaged with helium were then reproduced.
The distribution of bubbles density and size was computed using this helium bubble growth model and the results were used in FESTIM simulations.
It was found that exposing tungsten to helium could potentially reduce the hydrogen inventory by saturating defects, making it impossible for hydrogen to get trapped.
Moreover, the effect of helium bubbles (creation of additional traps for hydrogen) is limited to the near surface region (small compared to the monoblock's scale)


%----------------------------------------------------------------------------------------
%	TABLE OF CONTENTS & LIST OF FIGURES/TABLES
%----------------------------------------------------------------------------------------

\begingroup % Local scope for the following commands

% Define the style for the TOC, LOF, and LOT
%\setstretch{1} % Uncomment to modify line spacing in the ToC
%\hypersetup{linkcolor=blue} % Uncomment to set the colour of links in the ToC
\setlength{\textheight}{23cm} % Manually adjust the height of the ToC pages

% Turn on compatibility mode for the etoc package
\etocstandarddisplaystyle % "toc display" as if etoc was not loaded
\etocstandardlines % toc lines as if etoc was not loaded

\tableofcontents % Output the table of contents

\listoffigures % Output the list of figures

% Comment both of the following lines to have the LOF and the LOT on different pages
\let\cleardoublepage\bigskip
\let\clearpage\bigskip

\listoftables % Output the list of tables

\endgroup

%----------------------------------------------------------------------------------------
%	MAIN BODY
%----------------------------------------------------------------------------------------

\mainmatter % Denotes the start of the main document content, resets page numbering and uses arabic numbers
\setchapterstyle{kao} % Choose the default chapter heading style

% Chapter 1
\setchapterimage{jet}
\setchapterpreamble[u]{\margintoc}
\chapter{Fusion: general introduction}\label{Chapter1}
% For referencing the chapter elsewhere, use \ref{Chapter1} 

blablabla need for clean and abundant energy

\section{Thermonuclear fusion}
\begin{itemize}
    \item e=mc2
    \item DT reaction
    \item magnetic confinement --> Tokamak
    \item triple product
    \item Choice of a plasma facing material
\end{itemize}

The fundamental principle of nuclear fusion is to fuse two light nuclei into a heavier nucleus.
The mass difference between the products and the reactants is released in the form of energy (see Equation \ref{eq: emc2}).
\begin{equation}
    E = \Delta m c^2
    \label{eq: emc2}
\end{equation}
where $\Delta m$ is the mass difference, $c$ is the speed of light.
This process is the opposite process of nuclear fission, powering the current nuclear plants.

When looking at the different binding energies per nucleon of the elements (see Figure \ref{fig: binding energy per nucleon}), it becomes clear that light elements release energy from fusion and heavy elements release energy from fission.

\begin{figure} [h]
    \centering
    \includegraphics[width=\linewidth]{Figures/Chapter1/binding_energy_per_nucleon.pdf}
    \caption{Binding energy per nucleon \cite{kohn-seemann_alfkoehnfusion_plots_2021}}
    \label{fig: binding energy per nucleon}
\end{figure}

Nuclei are positively charged.
To be able to fuse, these must overcome the Coulomb barrier induced by the electromagnetic repulsion (see Figure \ref{fig: potential energy diagram fusion}).
This Coulomb barrier increases with the charge of the nuclei (\textit{ie} the number of protons).
This means that the nuclei must collide with a high enough velocity.
At the atomistic scale, the velocity $v_\mathrm{th}$ is a function of temperature (see Equation \ref{eq: thermal velocity}).
This is one of the reasons why the probability of a fusion reaction (called cross-section) is temperature dependent.

\begin{equation}
    v_\mathrm{th} = \sqrt{\frac{k_B T}{m}}
    \label{eq: thermal velocity}
\end{equation}
where $k_B = \SI{1.3806e-23}{m^2.s^{-2}.kg.K^{-1}}$ is the Boltzmann constant, $T$ is the nucleus temperature in \si{K} and $m$ is the nucleus mass in \si{kg}.


\begin{figure} [h]
    \centering
    \includegraphics[width=\linewidth]{Figures/Chapter1/potential_energy.pdf}
    \caption{Potential energy diagram}
    \label{fig: potential energy diagram fusion}
\end{figure}

Hydrogen, as the lightest element, has the lowest fusion temperature.
It is also the most abundant element on Earth (although bond to other elements).
Depending on which hydrogen isotope is used, different fusion reactions are available (see Equations \ref{eq: fusion reactions}).
Each of these reactions has a different cross-section.

\begin{subequations}
    \begin{equation}
         \ce{^2H + ^2H -> ^3H (\SI{1.01}{MeV})+ n (\SI{3.02}{MeV})}
    \end{equation}
    \begin{equation}
        \ce{^2H + ^2H -> ^3He (\SI{0.82}{MeV}) + n (\SI{2.45}{MeV})}
    \end{equation}
    \begin{equation}
        \ce{^2H + ^3H -> ^4He (\SI{3.5}{MeV}) + n (\SI{14.1}{MeV})}
    \end{equation}
    \begin{equation}
        \ce{^2H + ^3He -> ^4He (\SI{3.6}{MeV}) + p (\SI{14.7}{MeV})}
    \end{equation}
    \label{eq: fusion reactions}
\end{subequations}

\begin{figure} [h]
    \centering
    \includegraphics[width=\linewidth]{Figures/Chapter1/cross_sections_vs_temperature__Bosch.pdf}
    \caption{Fusion cross sections}
    \label{fig: fusion cross sections}
\end{figure}

The deuterium-tritium (DT) reaction is the one with the highest cross-section (probability) at "low" temperature (see Figure \ref{fig: fusion cross sections}).
This is the reason why this reaction has been the focus of nuclear fusion for decades.
More recently, private companies have started experimenting with more exotic reactions like proton-boron (TAE Technologies) or D-He3 (Helion Energy).

% \begin{figure} [h]
%     \centering
%     \includegraphics[width=\linewidth]{Figures/Chapter1/nuc_fus.pdf}
%     \caption{DT reaction}
% \end{figure}

\section{Tokamaks: how to bottle a star}

\subsection{Technology}
As explained above, for fusion to occur, the fuel must be heated up to millions of degrees.
A these temperatures, the DT gas becomes a plasma where electrons are teared out from the nuclei.
The principle of magnetic confinement reactors is to trap the electrically charged particles in a magnetic cage.
The electrons and ions then gyrate around the magnetic field lines and the Larmor radius of the gyration is given by:
\begin{equation}
    R =  \frac{\sqrt{2 m T}}{e B}
\end{equation}
where $m$ is the mass of the particle, $T$ its temperature, $e$ its charge and $B$ the magnetic field.
In a hot plasma (\SI{10}{keV}) with a strong magnetic field of \SI{3}{T}, the Larmor radius of ions is $\approx \SI{1}{mm}$, which is much smaller compared to the size of a reactor.
The Larmor radius of an electron is orders of magnitude smaller due to its lower mass.
Straight magnetic lines can therefore confine charged particles in the direction perpendicular to the field lines.
However, the particles are not confined in the parrallel direction.

This issue can be solved by closing the field lines, forming a torus-shaped magnetic field.
However, this configuration poses another problem: bending the field lines creates a magnetic field gradient in the radial direction.
This magnetic field gradient and the centrifugal force cause the particles to drift upwards (or downwards depending on their charge).
Due to this drift, the particles end up escaping the magnetic confinement until they touch the walls of the chamber and neutralise (making it impossible for them to fuse).

Two options exist to compensate this drift.
The first is to add a poloidal component to the magnetic field and twist the magnetic lines.
This is done by inducting a current in the plasma thanks to a central solenoïd (see Figure~\ref{fig:tokamak magnetic field}).
This configuration is called Tokamak which stands for "toroidalnaïa kamera s magnitnymi katouchkami" (toroidal chamber with magnetic coils).

\begin{figure}[h]
    \includegraphics[width=\linewidth]{Figures/Chapter1/tokamak_magnetic_fields.png}
    \caption{Magnetic field lines in a tokamak. Contribution of the toroidal and poloidal components.}
    \label{fig:tokamak magnetic field}
\end{figure}

The second option is to twist the magnetic field by twisting the toroidal coils themselves.
This configuration is called a stellarator and has the advantage of not having an induced current and is therefore inherently steady-state.
The tokamak, on the other hand, is a pulsed device.
This is because the central solenoid has a limited capacity.
The main drawback of the the stellarator is the complexity of the coils.
Since each coil has a unique shape, the cost of manufacturing such a reactor is higher than tokamaks for which coils can be manufactured in serial.

\subsection{Triple product}
The power balance in a fusion reactor is given by:
\begin{equation}
    \frac{\partial W}{\partial t} = P_\mathrm{fusion} + P_\mathrm{heating} - P_\mathrm{losses}
    \label{eq: plasma energy balance}
\end{equation}
$W$ is the thermal energy stored in the plasma and can be expressed in \si{J.m^{-3}} by:
\begin{equation}
    W = 3 n T
\end{equation}
where $n$ is the plasma density in \si{m^{-3}} and $T$ is the plasma temperature in \si{eV}.
$P_\mathrm{fusion}$, expressed in \si{W.m^{-3}}, is the power generated from fusion reactions themselves and can be expressed by:
\begin{equation}
    P_\mathrm{fusion} = n_D n_T \left\langle \sigma \right\rangle E
\end{equation}
where $n_D$ and $n_T$ are the densities in \si{m^{-3}} of deuterium and tritium respectively, $\left\langle \sigma \right\rangle$ is the DT cross-section in \si{m^3.s^{-1}} and $E$ is the energy of the fusion reaction in \si{J}.
Because the neutrons have little interaction with the plasma, $E \approx E_\alpha = \SI{3.56}{MeV}$.
Moreover, assuming a 50\%-50\% mixture of deuterieum and tritium, $n_D = n_T = \frac{1}{2} n$.
The fusion power can therefore be written as:
\begin{equation}
    P_\mathrm{fusion} = \frac{1}{4} n^2 \left\langle \sigma \right\rangle E_\alpha
\end{equation}

The amplification factor $Q$ defines the ratio of the fusion power by the heating power.
The heating power $P_\mathrm{heating}$ is therefore written as:
\begin{equation}
    P_\mathrm{heating} = \frac{P_\mathrm{fusion}}{Q}
\end{equation}

Finally, $P_\mathrm{losses}$ is the rate at which the plasma loses energy, either by losing mass (particles escaping the magnetic cage) or by radiation.
It is characterised by an energy confinement time $\tau_E$ and can be expressed as:
\begin{equation}
    P_\mathrm{losses} = \frac{W}{\tau_E} = \frac{3 n T}{\tau_E}
\end{equation}

Assuming energy equilibrium (\textit{ie} $\frac{\partial W}{\partial t} = 0$) Equation \ref{eq: plasma energy balance} can therefore be written as:
\begin{align}
    P_\mathrm{losses} &= P_\mathrm{fusion} + P_\mathrm{heating} \\
    \Leftrightarrow \frac{3 n T}{\tau_E} &= \frac{1}{4} n^2 \left\langle \sigma \right\rangle E_\alpha + \frac{P_\mathrm{fusion}}{Q}
\end{align}

Re-arranging the terms, one can obtain:
\begin{equation}
    n T \tau_E = \frac{12 T^2}{\left\langle \sigma \right\rangle E_\alpha} \cdot \frac{1}{1 + \frac{1}{Q}}
    \label{eq: triple product}
\end{equation}

$n T \tau_E$ is known as the \textit{triple product}, a figure of merit describing the performance of a fusion reactor.
Equation \ref{eq: triple product} gives the triple-product required to achieve a given amplification factor $Q$ at a given temperature $T$.

When $P_\mathrm{heating}$ approaches zero (\textit{ie} the auxilliary heating systems are shut down), the amplification factor $Q$ approaches $\infty$.
Therefore 
\begin{equation}
    n T \tau_E \rightarrow \frac{12 T^2}{\left\langle \sigma \right\rangle E_\alpha} \geq \SI{3e21}{keV.s.m^{-3}}
\end{equation}

This is known as the Lawson criterion, which needs to be satisfied in order to reach \textit{ignition} ($Q = \infty$).


Fusion devices can therefore be classified into three categories.
Stars like our Sun have very high confinement times and densities while remaining at relatively low temperatures (the sun Core is at \SI{1.2}{keV}).
Magnetic confinement devices (tokamaks, stellarators, etc.) exhibit temperatures orders of magnitude higher than stars but have confinement times of the order of $\sim \SI{1}{s}$.
A third way of achieving fusion is to heat and compress a target of fuel with either lasers (NIF, Laser Mega Joule), pistons (General Fusion) or by smashing it at high speed with a projectile (First Light Fusion).
These devices, known as \textit{inertial fusion devices}, exhibit extremely high densities ($\sim \SI{e31}{m^{-3}}$) but short confinement times ($\sim \SI{e-11}{s}$).
 
So far, no fusion device has been able to even reach \textit{break-even} ($Q = 1$) (see Figure \ref{fig: triple product vs T}).
The record of $Q = 0.68$ by the European tokamak JET (Joint European Torus) and was performed in 1997.
The objective of the ITER tokamak, currently under construction in France, is to demonstrate an amplification factor of $Q=10$ over \SI{400}{s}. 

\begin{figure}
    \centering
    \includegraphics[width=\linewidth]{Figures/Chapter1/triple_product_vs_T.pdf}
    \caption{Triple product. An interactive version of this plot is available at \href{https://remdelaportemathurin.github.io/fusion-world/}{https://remdelaportemathurin.github.io/fusion-world/}.}
    \label{fig: triple product vs T}
\end{figure}

\subsection{Divertor}
\begin{itemize}
    \item Particle and heat exhaust
    \item Different types of divertors (double null, super x, snowflake...)
    \item Actively cooled divertors with monoblocks
\end{itemize}

\subsection{Plasma-facing materials}
\begin{itemize}
    \item checklist of a plasma facing material
    \item candidates (tungsten, berylium)
\end{itemize}

\section{Plasma-surface interactions in tokamaks}

The wall of a fusion reactor (first wall and divertor) is bombarded with high energy hydrogen (deuterium and tritium) and helium ions.
Due to their small size, these ions can penetrate the metal lattice and be implanted in the material.

Many interactions occur between hydrogen, helium and tungsten atoms.

\subsection{H/W \& He/W interactions}
\begin{figure*}[h!]
    \begin{subfigure}{0.9\linewidth}
        \includegraphics[width=\linewidth]{Figures/Chapter1/HI transport sketch.pdf}
        \caption{H}
    \end{subfigure}
    \begin{subfigure}{0.9\linewidth}
        \includegraphics[width=\linewidth]{Figures/Chapter1/He transport sketch.pdf}
        \caption{He}
    \end{subfigure}
    \caption{Interactions of solute species in tungsten}
\end{figure*}

\begin{itemize}
    \item potential energy diagram
\end{itemize}

\subsubsection{Diffusion}
Diffusion of solute species is thermally activated and the diffusion coefficient $D$ expressed in \si{m^2.s^{-1}} usually follows an Arrhenius law:
\begin{equation}
    D = D_0 \exp{(-E_D/k_B T)}
\end{equation}

% MD simulations
Diffusion coefficients (also called diffusivities) can be computed by Molecular Dynamics (MD).
The principle of Molecular Dynamics is to calculate the trajectory of atoms in a simulation box.
The trajectory of a particle $i$ in a system of $N$ particles can be computed from Newton's second law of motion:

\begin{equation}
    m_i \vec{a_i} = \sum_{j=1 \, j \neq i}^N \vec{F}_{i,j}
\end{equation}
where $m_i$ is the mass of the particle, $\vec{a_i}$ is its acceleration, $\vec{F}_{i,j}$ is the force applied to the particle due to its interaction with particle $j$.
The forces between atoms is only a function of the interatomic potentials that can be estimated from \textit{ab initio} computations (Density Functional Theory) [insert refs] or from methods based on machine learning (pseudo-potentials) [insert refs].

By measuring the trajectory of a diffusing species from MD simulations for a long time, its diffusion coefficient $D$ can be estimated from Einstein equation \cite{einstein_uber_1905}:
\begin{equation}
    \lim_{x\to\infty} \frac{\langle R^2(t) \rangle}{6t} = D
\end{equation}
where $\langle R^2(t) \rangle$ is the mean squared displacement of the species.

Faney estimated the diffusion coefficient of He with this method \cite{faney_numerical_2013,faney_spatially_2014,faney_spatially_2015}.


% Experiments

% if you have time, it'd be nice to have a graph showing the diffusion coefficients of H and He according to different authors

\begin{itemize}
    \item how to calculate D with MD
    \item how to calculate D with permeation experiments (permeability)
\end{itemize}

\subsubsection{Surface processes}

When a surface is in contact with a gas, absorption of species can be described by an absorption flux:
\begin{equation}
    \varphi_\mathrm{abs} = n K_\mathrm{abs} P
\end{equation}
where $n$ is the absorption order, $K_\mathrm{abs}$ is the absorption coefficient expressed in \si{m^{-2}.s^{-1}.Pa^{-1}} and $P$ is the partial pressure of hydrogen in \si{Pa}.

Desorption of solute species at the surface is expressed by a desorption flux:
\begin{equation}
    \varphi_\mathrm{des} = K_\mathrm{des} c_\mathrm{surface}^n
\end{equation}
where $K_\mathrm{des}$ is the desorption coefficient expressed in \si{m^{-2+3n}.s^{-1}}, $c_\mathrm{surface}$ is the surface concentration in \si{m^{-3}}, and $n$ is the order of the desorption.

When the equilibrium between absorption and desorption is reached, $\varphi_\mathrm{abs} = \varphi_\mathrm{des}$, which gives:
\begin{equation}
    n K_\mathrm{abs} P = K_\mathrm{des} c_\mathrm{surface}^n
    \label{eq: equilibrium absorption desorption}
\end{equation}

By rearranging Equation \ref{eq: equilibrium absorption desorption}:
\begin{equation}
    c_\mathrm{surface} = \sqrt[n]{n \frac{K_\mathrm{abs}}{K_\mathrm{des}}} \sqrt[n]{P}
\end{equation}

When the absorption/desorption order is $n=1$ (monoatomic absorption):
\begin{equation}
    c_\mathrm{surface} = K_H P
\end{equation}
This relationship is known as Henry's law of solubility and $K_H = K_\mathrm{abs}/K_\mathrm{des}$ is the material solubility expressed in \si{m^{-3}.Pa^{-1}}.

When the absorption/desorption order is $n=2$ (diatomic absorption):
\begin{equation}
    c_\mathrm{surface} = K_S \sqrt{P}
\end{equation}
This equilbrium is known as Sievert's law of solubility and $K_S = \sqrt{2 K_\mathrm{abs}/K_\mathrm{des}}$ is the material solubility expressed in \si{m^{-3}.Pa^{-0.5}}.

\subsubsection{Trapping at defects}

\begin{itemize}
    \item can be measured by TDS
    \item can be computed from DFT
\end{itemize}


\subsubsection{Clustering}
Single He atoms implanted into the material will diffuse rapidly due to the high W-He repulsion.
This high repulsive W-He interaction is such that interstitial He atoms will preferably rearrange into groups of atoms in order to minimise the number of repulsive interactions \sidecite{hamid_molecular_2019, hammond_large-scale_2018}.
This phenomena is called "clustering".
Small clusters are themselves mobile as long as all the He atoms within the cluster are occupying interstitial position in the solid lattice.
The activation energy for interstitial He atoms and clusters ranges from 0.15 to 0.45 eV according to Perez \textit{et al} \sidecite{perez_mobility_2017}.
He clusters will eventually grow by interacting with either interstitial He atoms or other clusters.
If their size is big enough then their pressure will be sufficient to knock off a W atom from the lattice thus creating a W vacancy and an interstitial W atom (a Frenkel pair).
This process is called trap mutation or "self-trapping" and the trapped clusters act as nuclei for bubble formation.

\subsubsection{Bubble nucleation}

Trap mutation has been modelled in W using DFT \sidecite{boisse_modelling_2014} and Monte Carlo computations \sidecite{de_backer_modeling_2015}.
It has been shown that this phenomena depends not only on the number of He atoms in the cluster but also on temperature, position of the cluster to the free surface or even the crystal orientation \sidecite{blondel_modeling_2017, hu_interactions_2014, hu_dynamics_2014}.
At this point, the trapped cluster occupies the newly created W vacancy position.
It is considered immobile since it would requires either diffusion of another vacancy next to it, or recombination of the Frenkel pair in order to diffuse \sidecite{morishita_nucleation_2007}.

\subsubsection{Bubble growth}

Once a bubble nucleus is created via trap mutation, it can continue to grow via three main mechanisms: absorption of clusters, loop punching or blistering.
Two or more bubbles can also coalesce and form a bigger bubble.

Each of these mechanisms can become dominant over another depending on the implantation and the W conditions. 
\subsubsection{Vacancies clustering}

Bubbles can continue to grow by absorbing interstitial He atoms or mobile He clusters (\textit{ie} that haven't self trapped).
Considering that vacancies are mobile in the solid, the volume of a bubble could also increase if a vacancy or a vacancy cluster interact with a He bubble.
The same is true for He-vacancies clusters.

There is no experimental evidence of He clustering with interstitial W atoms \sidecite{faney_spatially_2014}.

This process is described by cluster dynamics equations in which interaction between the clusters is governed by pairs of association and dissociation rates.
\subsubsection{Dislocation loop}

During the growth of a He bubble by absorbing He atoms, if the pressure increases until reaching a critical value, dislocation loop punching can occur.
During the punching event, a whole facet of W atoms is pushed and the vacant lattice sites are absorbed by the bubble allowing the bubble to expand and reducing the pressure in it \sidecite{sefta_surface_2013}.
The produced self-interstitial W atoms will likely be attracted by \textit{image forces} at the surface and will contribute to the roughening of the surface and/or formation of surface structures.

Dislocation loops happen at very high pressure and if the number of vacancies in the lattice is low compared to the amount of He atoms.
This is the case when a high He flux is applied and the He ions energy is low so that no displacement damaged is produced \sidecite{sefta_surface_2013}.
If vacancies were created via He ions implantation, they could interact with existing He bubbles which would have the effect of increasing the volume and thus decreasing the pressure (assuming no change in temperature and no other implantation mechanism).

\subsubsection{Blistering}

When He is implanted at low temperature (<1000K) in W surfaces, appearance of blisters is observed \sidecite{baldwin_formation_2010}.
These blisters are plastic deformation (swelling) of the metal near the surface due to high pressure in He bubbles.
This phenomena is separated from the loop punching even though loop punching can be considered as a plastic deformation.
Blistering happens only at low temperatures because only then the growth rate of the bubble is greater than the dissolution in the bulk (which depends on the thermally activated diffusion coefficient and/or solubility).
Eventually if the rate of incoming He atoms is greater that the rate of re-dissolution in the bulk, blisters can rupture.
Similarly, if the rate of incoming He atoms is lesser than the rate of re-dissolution in the bulk, the blister will collapse.

\subsubsection{Bubble coalescence}
Coalescence of He bubbles has been observed in MD simulations \sidecite{hamid_molecular_2019, hammond_helium_2019, zhang_simulation_2019} and would tend to increase the bubble size decreasing the bubble density at the same time.
This may not have an impact on He concentration on the macroscopic scale but might influence bubble bursts.
\subsubsection{Bubble pressure}
The pressure inside the bubble and the bubble radius are two parameters of interest and are correlated.
Sefta and co-workers \sidecite{sefta_surface_2013} proposed to use the Wolfer equation of state in order to determine the number of He atoms contained in a He bubble based on its pressure, the latter being calculated from its radius and its surface tension.
Quir\'os \textit{et al} proposed a post-processing method to asses bubble radius and pressure from concentration profiles applied on H blistering \sidecite{quiros_blister_2017}.
One must be aware that if radii and pressure of bubbles computation is quite straightforward using MD \sidecite{zhang_simulation_2019} or cluster dynamics \sidecite{faney_spatially_2014-1} simulations it will be more complex to estimate these metrics considering a continuum model that does not keep track of every type of clusters but only a few of them.
The only information \textit{a priori} available in this case is indeed the local helium concentration and an equivalence could be found by either having a high density of small bubbles or a low density of big bubbles.

\subsubsection{Bursting}

When a bubble grows near the surface and is over-pressurised, bursting can occur.
After several increases of the bubble volume via punching loops making the W lattice to be deformed and the ligament thickness to decrease, the latter can rupture which would make all the He atoms contained in the bubble to be released to the vacuum.
This is why He bursting is characterised by sharp drops in the He inventory.

It has been observed by Sefta and co-workers that bursting is more likely to happen at high temperatures.
This phenomena contributes to surface roughening and could be the beginning of the formation of nano-fuzz \sidecite{sefta_helium_2013}.
Indeed, a bursting event could either form a crater on the W surface or an empty cavity due to self-healing.
In the last case, called a \textit{pinhole} bursting event, the cavity can be re-pressurised with He atoms.
Blondel \textit{et al} proposed to model bursting as a stochastic function of depth in the material rather than a calculation of the bubble pressure.
They have also shown that simulation parameters have an impact on the retention \sidecite{blondel_continuum-scale_2018}.
These differences are mainly due to 2D effects as more bursting events occur but with smaller bubbles.
They have shown that the size of the reaction network size (using cluster dynamics) does not seem to have an influence (between 250 and 200) as the first bursting events happen at clusters of size $\text{He}_{80}$.
Other simulation parameters (depth of the sample, pre-existing vacancies, bubble growth trajectory...) don't affect the simulations results as they converge for long time steps (100 s).

If bursting is not included in continuum simulations, the volume fraction of He present in the W could become very large and the dilute limit approximation could no longer be valid \sidecite{sefta_surface_2013}.
Care must though be taken considering what metric should be considered in continuum simulation in order to estimate bursting probability.


Bursting has been experimentally observed \sidecite{hamid_molecular_2019, woller_dynamic_2015} leading to craters formation.


\subsubsection{W tendrils or "nano-fuzz"}

When He is implanted in W surfaces at high temperature, modification of the surface can occur.
Nano structures have been observed at high temperature (>1000K), high flux (>\SI{1e21}{He^+.m^{-2}.s^{-1}}) and long exposure (t>\SI{1e2}{s}) \sidecite{baldwin_formation_2010}.
These structures are composed of tendrils (similar to nanometric cylinders) that are fragile \sidecite{nishijima_sputtering_2011}.
If these structures were to be removed during a plasma operation via erosion, W atoms could be fed into the plasma and therefore reduce the performance of the tokamak.
Moreover, this phenomena could increase the W dust formation in the reactor and lead to contamination and safety issues.

Fuzz formation could be due to bursting events and/or accumulation of self interstitial W atoms at the surface \sidecite{baldwin_effects_2009, baldwin_helium_2008, woller_dynamic_2015, hammond_helium_2017}.
Thermal properties of the media is also affected by the formation of W fuzz \sidecite{wirtz_influence_2016} which could have a severe impact during ELM-like events.
After 1h of plasma implantation, nanostructuring can be found deep in the bulk (up to several hundred of $\mu$m).
According to Baldwin and Doerner \sidecite{baldwin_formation_2010}, heavy alloying helps to reduce formation of He induced fuzz.

Bernard \textit{et al} shown that temperature has a strong influence on fuzz formation \sidecite{bernard_temperature_2017} and Takamura \textit{et al} shown fuzz could be grown under relevant tokamak conditions (high-flux He plasma irradiation and surface temperature greater than \SI{1250}{K}) \sidecite{takamura_formation_2006}.


Fuzz formation has been observed on the Alcator C-Mod divertor by Wright \textit{et al} \sidecite{wright_tungsten_2012}.

\subsubsection{Cracks}

It is not clear that He implantation has a role to play in cracks formation since cracks have also been observed during pure thermal shock on W PFC \sidecite{wirtz_influence_2016}.
Under some specific conditions, cracks can close due to thermal expansion which induce frictional loads on the structure.
The formation of W nano-fuzz could also bridge those cracks as observed by Lemahieu \textit{et al} \sidecite{lemahieu_h/he_2016}.

\subsubsection{Diminution of thermal performances}

Several of the above phenomena can have an impact on plasma facing materials thermal performances.
First, having a network of bubbles will lead to a reduction of local apparent conductivity as thermal constriction will occur between the bubbles.
Therefore, for a given heat load of the surface of a PFC, temperature will likely increase.
Then, development of surface structures will be accompanied by surface roughening therefore modifying the reflectivity and emissivity \sidecite{tokunaga_synergistic_2004}.
For a given incident flux, the net radiative flux to which the surface of the component is exposed will then increase.
This could lead to reduction of the PFC heat exhaust capacity and furthermore local melting \sidecite{wirtz_influence_2016}.

Having He transport affecting heat transfer significantly would lead to a strong coupling between the two (since He transport is strongly temperature dependent) which would imply to think of proper ways to deal with this coupling numerically.

NOTE: an interesting study would be to investigate thermal constriction due to the presence of inhomogeneities (He bubbles) in which thermal conductivity is low compared to the one of the W.


\subsection{He/H interactions}
Lee \textit{et al} studied the influence of He implantation on D retention.
They showed with ERD depth profiles (up to 40 nm) that D is trapped where He is trapped and proposed that He bubbles produce secondary defects around them which can trap D \sidecite{lee_hydrogen_2007}.
These defects can be interstitial loops produced by loop punching.
Lee \textit{et al} also suggested that no evidence had been found on trapping of D by chemisorption on the inner surface of a He bubble nor by molecular interaction with the He cluster.
The privileged mechanism is therefore the trapping of D in the defects made by the stress field induced by the He bubble to the crystalline structure of the W.

It has been shown by Ueda \textit{et al} that He implantation (even in small amounts) greatly affects H blistering in W \sidecite{ueda_simultaneous_2009}.
With only 0.1\% of He in the ion beam one can observe that H blistering is completely suppressed for temperature greater than \SI{653}{K}.
At lower temperature, H blistering occurs but is significantly reduced.
This phenomena is due to the fact that H migration to the bulk and accumulation at grain boundaries is avoided by He bubbles at the near surface which act as a diffusion plug for H.
The same phenomena has been observed by Miyamoto \textit{et al} \sidecite{miyamoto_microscopic_2011} which contributes to reducing H retention.
Markelj \textit{et al} \sidecite{markelj_hydrogen_2017} showed however that He implantation can increase D retention in the He clustering zone.
This implies that observed reduction or increase of D retention in mixed H-He plasma experiment are depending on the depth of implantation.

[ADD Mykola's experiment]

% \subsection{Experimental methods}
% \begin{itemize}
%     \item TDS
%     \item Permeation
%     \item Profilometry
%     \begin{itemize}
%         \item NRA
%         \item SIMS Secondary Ion Mass Spectrometry
%         \item ERDA Elastic Recoil Detection Analysis
%     \end{itemize}
%     \item 
% \end{itemize}
% \subsection{Modelling methods}
% \begin{itemize}
%     \item DFT, MD
% \end{itemize}

\section{Roadmap}

why we care about Hydrogen in materials: safety, embrittlement, recycling fluxes (plasma)


\setchapterimage{model-6}
\setchapterpreamble[u]{\margintoc}
\chapter{Model description}\labch{Chapter2}
\label{Chapter2} % For referencing the chapter elsewhere, use \ref{Chapter2} 
\section{Introduction}
This Chapter describes the mathematical models needed to simulate H transport in tokamaks plasma facing components.
The numerical scheme to solve these equations and an introduction to the finite element method alongside with its implementation in the FESTIM code are also presented.
Finally, the verification \& validation of FESTIM is performed to guaranty its reliability.

\section{H transport} \label{description_H_transport_model}

This Section describes the main model for simulating H transport in materials.

\subsection{Macroscopic Rate Equations model}

The Macroscopic Rate Equations (MRE) model will be employed.
The principle is to split the hydrogen isotopes into several populations: the mobile hydrogen particles and the hydrogen particles trapped in the i-th trap.
Their concentration in \si{H.m^{-3}} are respectively $c_\mathrm{m}$ and $c_{\mathrm{t},i}$.
The unit of the different concentrations are in $ \si{m^{-3}}$ but can be expressed in atomic fraction (at.fr.) by normalising them to the atomic density of the material.

Fick's first law of diffusion states that the particle flux is driven by the concentration gradient.
The particle flux $J$ is therefore expressed by:
\begin{equation}
    J = - D \nabla c_\mathrm{m}
\end{equation}
where $D$ is the diffusion coefficient in \si{m^2.s^{-1}}.
This expression neglects the Soret effect or the effect of hydrostatic pressure.

% This particle flux can be expressed differently if additionnal physics are accounted for.
% For instance, taking into account the Soret effect (also called thermodiffusion or thermophoresis), the particle flux is expressed as:
% \begin{equation}
%     J = - D \nabla c_\mathrm{m} - D \frac{c_\mathrm{m} Q}{R T^2} \nabla T
% \end{equation}
% where $Q$ is the Soret coefficient (or heat of transport) expressed in \si{J.mol^{-1}}, $T$ is the temperature in \si{K} and $R=\SI{8.314}{J.mol^{-1}.K^{-1}}$ is the gas constant.
% This effect will enhance the diffusion when particles diffuse from a hot region to a cold region.
% It should be noted however that the Soret coefficient is fairly difficult to measure.

The spatio temporal evolution of these concentrations are commonly described by the following reaction-diffusion system:

\begin{equation}
    \frac{\partial c_\mathrm{m}}{\partial t}=\nabla \cdot J+\Gamma-\sum \frac{\partial c_{\mathrm{t}, i}}{\partial t}
    \label{eq:mobile}
\end{equation}

\begin{equation}
    \frac{\partial c_{\mathrm{t}, i}}{\partial t}=k \cdot c_\mathrm{m} \cdot\left(n_{i}-c_{\mathrm{t}, i}\right)-p \cdot c_{\mathrm{t}, i}
    \label{eq:trapped}
\end{equation}

In Equation \ref{eq:mobile}, $\Gamma$ is the volumetric source term of particles in \si{m^{-3}.s^{-1}}.
The volumetric source term can be used to simulate any process producing H in the bulk.
This is the case for plasma implantation and nuclear reactions producing H.

In Equation \ref{eq:trapped}, $k$ and $p$ are the trapping and detrapping rates expressed in \si{m^3.s^{-1}} and \si{s^{-1}} respectively and $n_i$ is the trap density in \si{m^{-3}}.

At steady state (\textit{ie} $\frac{\partial c_\mathrm{m}}{\partial t} = 0$ and $\frac{\partial c_{\mathrm{t}, i}}{\partial t} = 0$), the mobile H concentration is independent of trapping effects.
Equation \ref{eq:trapped} can be rewritten as:
\begin{equation}
    c_{\mathrm{t}, i} = n_i \frac{1}{\frac{p}{k c_\mathrm{m}} + 1}
    \label{eq: steady state ct}
\end{equation}
The quantity $(p / (k c_\mathrm{m}) + 1)^{-1}$ determines the filling ratio of the trap. 
When it approaches one (high mobile concentration, low detrapping rate or high trapping rate), the trapped concentration approaches the trap density.
When it approahces zero (high detrapping rate, low mobile concentration or low trapping rate), the trapped concentration approaches zero.
Moreover, the mobile H concentration is independent of trapping effects.

\subsection{Boundary conditions}

Several boundary conditions will be employed in order to constrain either the actual solution (Dirichlet) or the solution gradient (Neumann, Robin) at the domain's boundaries.

\subsubsection{Dissociation and recombination fluxes}

The concentration gradient can also be constrained on the boundaries (see Equation \ref{eq: neuman robin bc}).

\begin{equation}
    - D(T)\nabla c_\mathrm{m} \cdot \mathbf{n} = f(x, t) \quad \text { on } \partial \Omega
    \label{eq: neuman robin bc}
\end{equation}
where $D(T) = D_0 \exp(\frac{-E_D}{k_B T}) $ is the diffusion coefficient in \si{m^2.s^{-1}}, $T$ is the local temperature in \si{K}, $\mathbf{n}$ is the boundary normal vector and $\partial \Omega$ is the domain boundary.

$f$ can also be expressed as a molecular recombination flux:
\begin{equation}
    - D(T)\nabla c_\mathrm{m} \cdot \mathbf{n} = K_r(T) c_\mathrm{m}^n \quad \text { on } \partial \Omega
    \label{eq: recombination flux}
\end{equation}
where $K_r(T) = K_{r_0} \exp(\frac{-E_{K_r}}{k_B T}) $ is the recombination coefficient expressed in \si{m^{3n-2}.s^{-1}}, $\mathbf{n}$ is the boundary normal vector and $n \in \{1, 2\}$ is the order of the recombination.
Recombination occurs when hydrogen particles located at the surface of the material combine with other particles (which can be other hydrogen particles) and are no longer bonded with the metal surface.
It can happen both in presence of a vacuum or when the metal is in contact with a fluid (gas or fluid).

Similarily, a dissociation flux can be applied when a surface is in contact with a gas atmosphere of H (see Equation \ref{eq: dissociation flux}).
\begin{equation}
    - D(T)\nabla c_\mathrm{m} \cdot \mathbf{n} = K_d(T) P \quad \text { on } \partial \Omega
    \label{eq: dissociation flux}
\end{equation}
where $K_d(T) = K_{d_0} \exp(\frac{-E_{K_d}}{k_B T}) $ is the dissociation coefficient expressed in \si{m^{-3}.Pa^{-1}} and $\mathbf{n}$ is the boundary normal vector.
Dissociation is the opposite process of recombination and occurs when particles in the surrounding atmosphere or fluid reach the metal surface and are adsorbed.
These particle can then reach the bulk and diffuse in the metal.

A steady-state approximation of the flux balance between recombination and dissociation fluxes is the Sievert's law (see Equation \ref{eq: Sievert's law}).

\begin{equation}
    c_\mathrm{m} = S(T) \sqrt{P}\quad \text { on } \partial \Omega
    \label{eq: Sievert's law}
\end{equation}
where $P$ is the partial pressure of hydrogen at the boundary in \si{Pa}, $S(T)=S_0 \exp(\frac{-E_S}{k_B T})$ is the material solubility in \si{m^{-3}.Pa^{-1/2}} and $T$ is the local temperature in \si{K}.
This law of equilibrium is a steady-state approximation of a more complex model which takes into account flux exchanges between adsorbed and mobile concentrations at the boundary.
It is therefore valid when applied to cases where the kinetics are fast enough for the system to remain at equilibrium.

\subsubsection{Analytical simplification for implanted sources of H} \label{triangle model}

Plasma implantation of hydrogen particles can be modelled with a volumetric source.
Typically, the depth of the implantation profile is a few nanometres (depending on the incident particles energy).
These profiles can be simulated by Monte Carlo codes like SRIM \sidecite{ziegler_srim_2010} and have a gaussian-like shape.
% In order to accurately model this source term, the size of the cells constituing the mesh (the spatial discretisation of the domain) must be less than a nanometre.
% This can be done easily in 1D but is very complicated in higher dimensions, especially when simulating centimetre-sized components.
This volumetric source term can be simplified into a Dirichlet boundary condition (\textit{ie} enforcing the mobile particle concentration at the exposed surface).

Let us consider a volumetric source term of hydrogen $\Gamma = \varphi_\mathrm{imp} \; f(x)$ where $f$ is a narrow Gaussian distribution.
The mobile particles concentration profile can be approximated by a triangular shape \sidecite{schmid_diffusion-trapping_2016} (see Figure \ref{fig:recomb sketch}).

\begin{figure*}[h!]
    \centering
    \includegraphics[width=0.75\linewidth]{Figures/Chapter2/recomb_sketch.pdf}
    \caption{Concentration profile with recombination flux and volumetric source term at $x=R_p$. Dashed lines correspond to the time evolution.}
    \label{fig:recomb sketch}
\end{figure*}

The concentration profile is therefore maximum at $x=R_p$.
The expression of $c_\mathrm{max}$ can be obtained by expressing the flux balance at equilibrium:

\begin{equation}
    \varphi_\mathrm{imp} = \varphi_\mathrm{recomb} + \varphi_\mathrm{bulk}
    \label{eq:flux balance}
\end{equation}
where $\varphi_\mathrm{recomb}$ is the recombination flux and $\varphi_\mathrm{bulk}$ is the migration flux.

$\varphi_\mathrm{bulk}$ can be expressed as:
\begin{equation}
    \varphi_\mathrm{bulk} = D \cdot \frac{c_\mathrm{max}}{R_d(t) - R_p}
\end{equation}

% When $t \rightarrow \infty$ or $R_d \gg R_p$ (a ratio of 10 or 100 is enough), $\varphi_\mathrm{bulk} \ll \varphi_\mathrm{recomb}$.
When $R_d \gg R_p$, $\varphi_\mathrm{bulk} \rightarrow 0$.
Equation \ref{eq:flux balance} can therefore be written as:
\begin{equation}
    \varphi_\mathrm{recomb} \approx \varphi_\mathrm{imp}
    \label{eq:flux balance 2}
\end{equation}

Moreover, according to Fick's law, $\varphi_\mathrm{recomb}$ can be expressed as:

\begin{eqnarray}
    \varphi_\mathrm{recomb} &= D \cdot \frac{c_\mathrm{max}-c_{0}}{R_{p}} = \varphi_\mathrm{imp}\\
    \Leftrightarrow c_\mathrm{max} &= \frac{\varphi_\mathrm{imp} R_{p}}{D}+ c_0
    \label{eq:c_max}
\end{eqnarray}

Assuming second order recombination, $\varphi_\mathrm{recomb}$ can also be expressed as a function of the recombination coefficient $K$:

\begin{eqnarray}
    \varphi_\mathrm{recomb} &= K c_{0}^{2} = \varphi_\mathrm{imp}\\
    \Leftrightarrow c_{0} &= \sqrt{\frac{\varphi_\mathrm{imp}}{K}}
    \label{eq:c_0}
\end{eqnarray}

By replacing Equation \ref{eq:c_0} in Equation \ref{eq:c_max} one can obtain:

\begin{equation}
    c_\mathrm{max} = \frac{\varphi_\mathrm{imp} R_{p}}{D}+\sqrt{\frac{\varphi_\mathrm{imp}}{K}}
\end{equation}

As the recombination process becomes fast (\textit{ie} $K \rightarrow \infty$), $c_0 \approx 0$ and $c_\mathrm{max} \approx \frac{\varphi_\mathrm{imp} R_{p}}{D}$.

Since the main driver for the diffusion is the value $c_\mathrm{max}$, when $R_p$ is negligeable compared to the dimension of the simulation domain, one can simply impose this value at the surface.
% This analytical simplification is especially useful to simulate implanted sources near the surface (\textit{eg} plasma implantation) without having to finely discretise the domain to fully represent the gaussian distribution.
% Such a discretisation can be easily done for 1D simulations but is very complex in 2D and 3D and often makes the mesh very large which increases drastically the computational cost.

A transient solution based on trap properties can be derived \sidecite{hodille_study_2016}:
\begin{equation}
    c_\mathrm{max}=(\frac{R_p \varphi_\mathrm{imp}}{D} + \sqrt{\frac{\varphi_\mathrm{imp}}{K}}) \cdot \frac{\tau}{t} \cdot\left(\sqrt{1+\frac{t}{\tau}}-1\right)^2
\end{equation}
where $\tau$ is a characteristic time expressed by:
\begin{equation}
    \tau = \frac{R_p \sum R_i \, n_i}{8 \varphi_\mathrm{imp}}
\end{equation}
In this expression, $R_i = (p / (k c_\mathrm{max}) + 1)^{-1}$ represents the maximum filling ratio of the trap $i$ and $n_i$ is the trap density.

\subsection{Interface condition: conservation of chemical potential}
% According to Krom \textit{et al} \sidecite{krom_hydrogen_2000}, since the solubility of hydrogen atoms in solids is low, the chemical potential of solute hydrogen $\mu$ is expressed by:
% \begin{equation}
%     \mu = \mu_0 + RT \ln\left( \frac{c_\mathrm{m}}{N_L}\right)
% \end{equation}
% where $\mu_0$ is the chemical potential in a reference state in \si{J.mol^{-1}}, $R$ the ideal gas constant, $T$ the temperature in \si{K}, $c_\mathrm{m}$ the mobile hydrogen concentration in \si{m^{-3}} and $N_L$ the lattice site concentration in \si{m^{-3}}.

% % Assuming that only free hydrogen atoms contribute to the overall flux in the material, the particle flux $J$ in \si{m^{-2}.s^{-1}} can be expressed by Fick's law:
% % \begin{equation}
% %     J = - D \nabla c_\mathrm{m}
% % \end{equation}
% % where $D$ is the diffusion coefficient of hydrogen expressed in \si{m^{2}.s^{-1}}. 


% The local equilibrium at the interface between two materials must ensure  the continuity of both the chemical potential $\mu$ (see Equation \ref{eq: muconservation}) and the particle flux (see Equation \ref{eq: flux conservation}).
% \begin{equation}
%     \mu^- = \mu^+  \label{eq: muconservation}  
% \end{equation}
    
% \begin{equation}
%     D^- \nabla c_\mathrm{m}^- = D^+ \nabla c_\mathrm{m}^+ \label{eq: flux conservation} 
% \end{equation}

The continuity of chemical potential is conveyed by the continuity of $P$, the local partial pressure of hydrogen at equilibrium.
In a metal, $P$ can be expressed from Sievert's law of solubility:
\begin{equation}
    P = (c_\mathrm{m}^-/S^-)^2
\end{equation}
with $S$ the solubility of H in the materials expressed in \si{m^{-3}.Pa^{-0.5}}.
% A way to picture the continuity of chemical potential is to imagine a gas layer at the interface between two materials at a partial pressure $P_\mathrm{eq}$.
% $P_\mathrm{eq}$ can be expressed by the solubility law at the surface of each material.
At the interface between two metallic surfaces, the chemical potential continuity is therefore conveyed by the continuity of the quantity $c_\mathrm{m}/S$.:
\begin{equation}
    (c_\mathrm{m}^-/S^-)^2 = (c_\mathrm{m}^+/S^+)^2
\end{equation}

In the case of a metal in contact with a non-metallic liquid behaving according to Henry's law (\textit{eg} a molten salt):
\begin{equation}
    (c_\mathrm{m}^-/S^-)^2 = c_\mathrm{m}^+/S^+
\end{equation}
with $S$ the solubility of H in the materials expressed in \si{m^{-3}.Pa^{-0.5}} or \si{m^{-3}.Pa^{-1}}.

% This assumption is correct as long as the time needed to reach the equilibrium is low compared to the time of the simulation.
% For long exposure time as well as for high temperatures, the characteristic time is small enough for the equilibrium model to be valid (see page \refpage{Interface transient model}).

% From Equation \ref{eq: c/s conservation}, one can deduce that a solubility discontinuity across an interface induces a discontinuity of mobile hydrogen concentration $c_\mathrm{m}$.
% This can also be interpreted as the chemical potentials at a reference state being different in different materials \sidecite{kirchheim_25_2014}, as the lattice site concentration.

FESTIM ensures the continuity of chemical potential by performing a change of variable in Fick's second law of diffusion with $\phi = c_\mathrm{m}/S$ (in the case of a metal) \sidecite{smith_abaqusstandard_2009} when internal conditions cannot be set.
Neglecting the trapping and generation terms, Equation \ref{eq:mobile} therefore reads:

\begin{align}
    \frac{\partial \phi S}{\partial t} &= \nabla \cdot\left(D \nabla \phi S\right) + f \nonumber \\
    &=\nabla \cdot\left( D S \nabla \phi + D \phi \nabla S\right) + f \label{eq: diffusion equation changed}
\end{align}

% Because $\phi$ is computed, the ratio $c_\mathrm{m}/S$ is continuous by default at the material interfaces.

% This second approach is used for instance in the \textit{mass-diffusion} procedure of the Abaqus code \sidecite{smith_abaqusstandard_2009}.
% This interface model has also been implemented into the current hydrogen transport code FESTIM \sidecite{delaporte-mathurin_finite_2019} using FEniCS \sidecite{alnaes_fenics_2015}.

After solving Equation \ref{eq: diffusion equation changed} for $\phi$, $c_m$ can be retrieved by multiplying the solution by $S$.

\section{Heat transfer}
Due to the numerous processes that are thermally activated, it is essential have an accurate temperature field.
Moreover, most tokamak plasma facing components are exposed to intense heat fluxes and are actively cooled, exhibiting high temperature gradients.
The temperature fields are even more complex when dealing with non-trivial geometries like monoblocks or breeding blankets.
For these reasons, heat transfers need to be modelled.

The equation describing heat conduction in solids is described as follows:
\begin{equation}
    \rho \cdot C_p \frac{\partial T}{\partial t}=\nabla \cdot (\lambda \nabla T)
    \label{eq:heat equation}
\end{equation}
where $\rho$ is the density of the material in \si{kg.m^{-3}}, $C_p$ its specific heat capacity expressed in \si{J.kg^{-1}.K^{-1}} and $\lambda$ the thermal conductivity expressed in \si{W.m^{-1}.K^{-1}}.

The thermal properties $C_p$, $\rho$ and $\lambda$ are usually temperature dependent.

For heat transfer problems, three types of boundary conditions can be imposed.

First, the temperature can be fixed on the boundary (see Equation \ref{eq:dirichlet bc T}).
\begin{equation}
    T = T(x, y, z, t) \quad \text { on } \partial \Omega
    \label{eq:dirichlet bc T}
\end{equation}
where $\partial \Omega$ is the domain boundary.

On the other hand, a heat flux can also be imposed by enforcing the temperature gradient (see Equation \ref{eq: neumann bc T}).
\begin{equation}
    -\lambda \nabla T \cdot \mathbf{n} = f(x, y, z, t) \quad \text { on } \partial \Omega
    \label{eq: neumann bc T}
\end{equation}
where $\lambda$ is the thermal conductivity in \si{W.m^{-1}.K^{-1}}, $\mathbf{n}$ is the boundary normal vector and $\partial \Omega$ is the domain boundary.

Finally, to model a convective heat flux when the surface is in contact with a fluid (\textit{eg} cooling pipes, natural convection...), a Robin boundary condition needs to be employed (see Equation \ref{eq: convective bc T}).
\begin{equation}
    -\lambda \nabla T \cdot \mathbf{n} = h (T - T_\mathrm{ext}) \quad \text { on } \partial \Omega
    \label{eq: convective bc T}
\end{equation}
where $\lambda$ is the thermal conductivity in \si{W.m^{-1}.K^{-1}}, $\mathbf{n}$ is the boundary normal vector, $h$ is the heat exchange coefficient in \si{W.m^{-2}.K^{-1}}, $T_\mathrm{ext}$ is the fluid temperature in \si{K} and $\partial \Omega$ is the domain boundary.
The heat exchange coefficient is usually dependent on the temperature.

\section{Implementation}


The models described in this Section can be hard to solve analytically for complex problems (complex geometries, transients, combined boundary conditions, etc).
The code FESTIM \sidecite{delaporte-mathurin_finite_2019} was therefore developped in order to solve these equations numerically.

\subsection{The finite element method: FEniCS}
FESTIM is based on the Finite Element Method to solve this set of differential equations and boundary conditions.
Several finite element libraries are openly available (deal.II \sidecite{arndt_dealii_2021}, MFEM \sidecite{kolev_tzanio_modular_2010}, MOOSE \sidecite{permann_moose_2020}, FreeFEM++ \sidecite{hecht_new_2012}, ...).
The open-source python package FEniCS \sidecite{alnaes_fenics_2015} was employed.
The finite element method is a versatile tool that can solve any partial differential equation on an arbitrary geometry in 1D, 2D or 3D.
The main advantage of this method compared to the finite difference method is the simplicity of its application to complex geometries and unstructured meshes.
Indeed, implementing a finite difference scheme for such a problem would be tedious and extra care must be taken for mistakes in the implementation could result in losses in efficiency and accuracy of the numerical solution.

This section will detail the finite element method applied to a simple diffusion equation (see Equation \ref{eq: example poisson}).

\begin{equation}
    -\nabla^2 u = f
    \label{eq: example poisson}
\end{equation}

The first step of the finite element method is to represent the solution $u$ as a combination of polynomial expressions (see Equation \ref{eq: FEM solution}).

\begin{equation}
    u = \sum^N_{i=0}u_i \phi_i(x, y, z)
    \label{eq: FEM solution}
\end{equation}
where $u_i$ are the coefficient to be determined (called degrees of freedom) and $\phi_i$ are polynomials (see Figure \ref{fig: example approximated solution}).

\begin{figure}
    \centering
    \includegraphics[width=\linewidth]{Figures/Chapter2/approximated_solution.pdf}
    \caption{Example of an approximated solution $u$ (exact in blue, approximated in orange) with basis functions $\phi_i$}
    \label{fig: example approximated solution}
\end{figure}


The second step is to build a variational formulation (also called weak form) of the governing equation \ref{eq: example poisson}.
To do so, the recipe is to multiply the PDE by a function $v$ (called the test function) and integrate over an arbitrary element $\Omega_e$.
The following expression is obtained:
\begin{equation}
    \int_{\Omega_e} -\nabla^2 u v dx = \int_{\Omega_e} f v dx \quad \forall v
    \label{eq: weak form 1}
\end{equation}

When using $N+1$ different test functions, Equation \ref{eq: weak form 1} then gives rise to a system of $N+1$ equations.
This form is called the weak form because it relaxes the requirement of Equation \ref{eq: example poisson} and instead requires to solve Equation \ref{eq: weak form 1} for all test functions.

Equation \ref{eq: weak form 1} needs now to be rewritten in order to only have first order derivatives.
To do so, Gauss-Green's lemma is employed:
\begin{equation}
    \int_{\Omega_e} -\nabla^2 u v dx = \int_{\Omega_e} \nabla u \cdot \nabla v dx - \int_{\partial \Omega_e} \frac{\partial u}{\partial n} v dx
    \label{eq: gauss-green}
\end{equation}

The variational form therefore reads:
\begin{equation}
    \int_{\Omega_e} \nabla u \cdot \nabla v dx = \int_{\Omega_e} f v dx + \int_{\partial \Omega_e} \frac{\partial u}{\partial n} v dx \quad \forall v
    \label{eq: weak form 2}
\end{equation}
where the last term of the equation either vanishes due to Dirichlet boundary conditions or is imposed.

From Equation \ref{eq: weak form 2}, a system of $N+1$ equations can be solved to determine the coefficients $u_i$ in Equation \ref{eq: FEM solution}.
Once the $u_i$ coefficients are known, an approximated solution can be computed.

\subsection{Main features of FESTIM}
FESTIM provides an even higher level of abstraction than FEniCS by providing a user-friendly interface dedicated to H transport and H transfer problems.
Users only have to provide inputs such as material properties, traps properties, geometry, solving parameters, without having to dive into the finite element implementation.

% user friendly
Multi-dimensional transient simulations coupled with heat transfer can therefore be run fairly easily without finite element knowledge.
Nevertheless, since FESTIM is object-oriented, advanced users will always be able to turn FESTIM inside-out to adapt the code to their specific needs (specific boundary conditions, slight changes in the governing equations...).
Since FESTIM is written in python - which is a fairly easy-to-learn programming langage - no advanced level of coding is required.

% physics
As mentioned above, FESTIM simulates the transport (diffusion and trapping) of H and additional physics can be incorporated, such as the Soret effect (also called heat of transport) and conservation of chemical potential at interfaces...
Various types of boundary conditions are available for both the H transport (imposed concentration, recombination flux, dissociation flux, implanted source approximation...) and the heat transfer problems (imposed temperature, imposed flux, convective flux...).
Traps densities in FESTIM can also be time-dependent allowing the users to simulate extrinsic traps (\textit{eg} irradiation induced traps, stress induced traps...).

% geometry
Thanks to the finite element method, geometries used in FESTIM can be complex (see Figure \ref{fig: example mesh}).
The meshing capability of FESTIM is limited to 1D meshes and it was decided to instead make FESTIM accept (with the XDMF format) complex meshes from third-party applications dedicated to meshing such as SALOME or GMSH.
These third-party applications can for instance be usefull to run CAD-based simulations.
Users can also decide to use the FEniCS built-in meshing tool.

\begin{figure}
    \centering
    \includegraphics[width=0.5\linewidth]{Figures/Chapter2/example_mesh.png}
    \caption{Example of a complex 3D geometry (here a breeding blanket section) mesh readable by FESTIM \cite{dark_influence_2021}.}
    \label{fig: example mesh}
\end{figure}

% visualisation
Similarly, FESTIM (FEniCS) visualisation functions are limited.
FESTIM is not a graphical application but the files generated by FESTIM (XDMF, CSV, TXT) can easily be read and post-processed by specialised tools such as Paraview \sidecite{ahrens_paraview_2005}, matplotlib \sidecite{hunter_matplotlib_2007}, NumPy \sidecite{harris_array_2020}, etc.

% what FE
Regarding the default finite elements used in FESTIM are Continuous Galerkin elements but it can be switched to Discontinuous Galerkin when needed.
This is usefull when the trapped concentration is discontinuous and help avoiding under- or over-shoots in the concentration field.

% Adaptive step size
When dealing with transient problems, FESTIM provides an adaptive stepsize allowing the stepsize to increase (by a user-defined factor) when the convergence criterion is easily reached by the solver.
This greatly improve the performance of the code since less timesteps are needed.



\section{Verification \& Validation}

Before using the FESTIM code for analysis, it has to be verified and validated.
The verification \& validation process (often called V\&V) has two goals: (1) to prove that the governing equations are correctly solved and that the code is error free and (2) to demonstrate that the governing equations actually reproduce processes observed experimentally.
In other words, verification is answering the question "Are we building the code right?" and validation is answering the question "Are we building the right code?".

This Section details the V\&V of the FESTIM code.

\subsection{Analytical verification}
Verification is the process of ensuring the governing equations are correctly solved in FESTIM.
This is an integral part of every simulation code for it guarantees the code is error free.
It is generally hard to simply substitute this process by code comparison (cross-checks between two different codes) because often the codes are implemented differently.
Moreover, if the code we are comparing with is not verified, then obtaining similar results does not give any guarantee on the code accuracy as two different codes can have the same bug.

Several methods can be used to verify a code but the \gls{mes} and the \gls{mms} are employed here.

Both methods consist in comparing a computed solution with an exact solution and measuring the error.
The exact solution in the \gls{mes} is obtained by solving the governing equations analytically.
When using the \gls{mms}, the problem is reversed: an arbitrary exact solution (called \textit{manufactured solution}) is chosen and injected in the governing equations.
It is then possible to determine source terms and boundary conditions.
These are then fed into the code and the computed solution is compared to the manufactured (exact) solution.

This \gls{mms} is often used to unravel the complexity of governing equations \sidecite{dudson_verification_2016, roache_code_2002}.
This is particularly useful when dealing with complex geometries or to exercise non-trivial material propoerties.

This section describes two verification cases of \gls{festim}.
The first one uses the \gls{mes} and the second one the \gls{mms}.
More complex and thorough verification cases are shown in Appendix \ref{appendix verification}.

\subsubsection{Case 1: H transport (\gls{mes})} \label{analytical}

\begin{figure}
    \centering
    \includegraphics[width=\linewidth]{Figures/Chapter3/mes_festim_effective_diffusion.pdf}
    \caption{Temporal evolution of the outward particle flux $\varphi$ at $x=l$ (Case 1).}
    \label{fig:FESTIM vs analytical}
\end{figure}

% Although validation against experiments could show that FESTIM is able reproduce the data with a given set of parameters, objective verification against analytical solutions is first required to ensure that the governing Equations \ref{eq:mobile} and \ref{eq:trapped} are solved correctly.

For this verification case, a 1D slab is considered with a thickness $l$.
The concentration of mobile particles was set to $c_0$ on one side of the slab and set to zero on the other side.
Only one trap is considered in this case and its density $n$ is homogeneously distributed.

The trapping parameter $\zeta$ is defined in \sidecite{longhurst_verification_2005} as follow:
\begin{equation}
    \zeta = \frac{p}{k \: n} + \frac{c_\mathrm{m}}{n}
\end{equation}

In our case, we choose the trapping and detrapping rates $k$ and $p$, the concentration $c_0$ and the temperature $T$ so that $\zeta \gg \frac{c_\mathrm{m}}{n}$.
This is known as the \textit{effective diffusivity regime} where the diffusion is almost identical to the case where there are no traps.
In this regime, the governing equations are identical as a pure diffusion regime and are therefore easy to solve analytically.

The coefficient $D$ is then replaced by an effective diffusion coefficient:
\begin{equation}
    D_\mathrm{eff} = \frac{D}{1+\frac{1}{\zeta}}
\end{equation}
The particle flux at the background surface ($x=l$) is expressed in $\si{H.m^{-2}.s^{-1}}$ and finally defined in \sidecite{longhurst_verification_2005} by:
\begin{equation}
    \varphi(t) = \frac{c_0 D}{l}\bigg[1+2\sum_{m=1}^{\infty}(-1)^m \exp\bigg(-m^2\frac{\pi^2 \:D_\mathrm{eff} \: t}{l^2}\bigg)\bigg]
\label{eq:flux analytical}
\end{equation}
Note: The infinite sum has been truncated at $m=10000$.

All the parameters are defined in Table \ref{tab:parameters analytical verification}.
These parameters have been chosen for the sake of verification and do not necessarily represent realistic conditions as verification is a mathematical exercise.
\begin{table}
    \centering
    \begin{tabular}{p{2.3cm} p{2cm} r}
        Parameter & Units & Value \\
        \hline
        \\
        $D_0$ & $\si{m^2.s^{-1}}$ & 2.0 \\
        $k_0$ & $\si{m^3.s^{-1}}$ & 0.01 \\
        $p_0$ & $\si{s^{-1}}$ & 1.0 \\
        \\
        $E_D$ & $\si{eV}$ & 0.2 \\
        $E_k$ & $\si{eV}$ & 0.1 \\
        $E_p$ & $\si{eV}$ & 0.1 \\
        \\
        $c_0$ & $\si{m^{-3}}$ & 2.0 \\
        $n$ & $\si{m^{-3}}$ & 2.0 \\
        $l$ & $\si{m}$ & 1.5\\
        \\
        $T$ & $\si{K}$ & 300 \\
        \\
        $t_f$ & $\si{s}$ & 2000 \\
        \\
    \end{tabular}
    \caption{Parameters used for the analytical verification (Case 1).}
    \label{tab:parameters analytical verification}
\end{table}
One can notice on Figure \ref{fig:FESTIM vs analytical} that the numerical results are in good agreement with the analytical solution.
The relative L2 error between analytical and numerical solutions was found to be $\approx 1 \%$ with 1000 piecewise linear elements (P1) and a stepsize of \SI{1}{s}.
This value decreases with the stepsize and with the element size (see \reffig{MES evolution of L2 error as function of dt and dx}).
\begin{figure}
    \centering
    \includegraphics[width=\linewidth]{Figures/Chapter2/error_vs_element_size_and_dt.pdf}
    \caption{Evolution of the L2 error on $\varphi$ as a function of the stepsize and element length (Case 1).}
    \labfig{\gls{mes} evolution of L2 error as function of dt and dx}
\end{figure}


Since this test case is very similar to a pure diffusion case, it does not exercise all terms of the governing equations.
To do so, the governing equations would have to be solved for a generic case which proves to be complex.
This is why the \gls{mms} will be used instead.

\subsubsection{Case 2: H transport (\gls{mms})} \label{mms}

\paragraph{Principle}
The \gls{mms} is often used to unravel the complexity of governing equations \sidecite{dudson_verification_2016, roache_code_2002}.
This is particularly useful when dealing with complex geometries or to exercise non-trivial material properties.

The principle of the \gls{mms} is to manufacture an exact solution.
Again, physical realism is not a concern here as verification is a mathematical exercise.
This manufactured solution needs to be non trivial in order to test the robustness of the implementation.
It is then passed through the governing equations (either the heat equation or the hydrogen transport equations) and source terms are obtained.

Let us take a simple example with the Poisson equation defined on a 1D domain $[x_1, x_2]$:
\begin{equation}
    \frac{\partial u}{\partial t} = \frac{\partial^2 u}{\partial x^2} + q
    \labeq{poisson eq demo mms}
\end{equation}
where $u$ is the unknown, $q$ is the source term.

The manufactured solution is arbitrarily defined as:
\begin{equation}
    U(t, x) = A + \sin{(x + B t)}
\end{equation}
where $A$ and $B$ are real numbers, $t$ is the time, and $x$ is the spatial coordinate.
By replacing $u$ by $U(t, x)$ in \refeq{poisson eq demo mms}, we can identify the source term $q$ that would produce the solution $U(t, x)$:

\begin{equation}
    Q(t, x) = B \cos{(x + B t)} + \sin{(x + B t)}
\end{equation}

Several boundary conditions can be used to produce $U(t, x)$.
We can for instance set a Dirichlet boundary condition on the boundaries $x=x_1$ and $x=x_2$:
\begin{align}
    u(t, x_1) &= U(t, x_1) \\
    u(t, x_2) &= U(t, x_2)
\end{align}

or Neumann boundary conditions:
\begin{align}
    \frac{\partial u(t, x)}{\partial x}\Big | _{ x=x_1} &= \frac{\partial U(t, x)}{\partial x} \Big | _{ x=x_1} \\
    \frac{\partial u(t, x)}{\partial x}\Big | _{ x=x_2} &= \frac{\partial U(t, x)}{\partial x} \Big | _{ x=x_2}
\end{align}

or even a combination of Dirichlet and Neumann boundary conditions.

By solving \refeq{poisson eq demo mms} with $q = Q(t, x)$ and initial condition $u = U(0, x)$, we can obtain the computed solution $u_\mathrm{computed}$.
The error between the computed solution $u_\mathrm{computed}$ and the exact solution $U(t, x)$ can be calculated to assess the code accuracy.

\paragraph{Case 2a: Application to 1D hydrogen transport}

Let us apply the \gls{mms} to the hydrogen transport model on a 1D domain $\Omega$.
In order to exercise all terms in Equations \ref{eq:mobile} and \ref{eq:trapped}, the following manufactured solutions are chosen:
\begin{equation}
    \begin{cases}
    c_{m_D} = 1 + x^2 + \sin(t) \\
    c_{{t,1}_D} = 1 + x^2 + \cos(t)
    \end{cases}
    \label{eq: manufactured solutions}
\end{equation}

By combining Equations \ref{eq:mobile}, \ref{eq:trapped} and \ref{eq: manufactured solutions}, one can obtain the following source terms:
\begin{equation}
    \begin{cases}
    f = \cos(t) - \sin(t) - 2D \\
    g_1 = \nu_1 c_{{t,1}_D} - \nu_m c_{m_D} ( n_1 - c_{{t,1}_D}) - \sin(t)
    \end{cases}
    \label{eq:sources}
\end{equation}
$f$ is the source term of the mobile concentration equation and $g_1$ is the source term of the trapped concentration equation.

where $g_1$ is an additional source term in Equation \ref{eq:trapped}.
The Dirichlet boundary conditions for $c_\mathrm{m}$ and $c_{t,1}$ are:

\begin{equation}
    \begin{cases}
    c_\mathrm{m} = 1 + x^2 + \sin(t) \quad \text{on } \partial \Omega \\
    c_{t,1} = 1 + x^2 + \cos(t) \quad \text{on } \partial \Omega 
    \end{cases}
\end{equation}
where $\partial\Omega$ is the boundary of the domain.
Finally, initial values for $c_\mathrm{m}$ and $c_{t,i}$ are:
\begin{equation}
    \begin{cases}
    c_\mathrm{m}(t=0) = 1 + x^2 \\
    c_{t,1}(t=0) = 2 + x^2
    \end{cases}
\end{equation}
Once all these parameters are fed into FESTIM, one can easily compare the computed solution with the exact solution in Equation \ref{eq: manufactured solutions}.
The L2-norm $E_{c_\mathrm{m}}$ can then be calculated as follow:
\begin{equation}
    E_{c_\mathrm{m}} = \sqrt{\int_\Omega(c_{m_D} - c_\mathrm{m})^2dx}
\end{equation}
The evolution of $E_{c_\mathrm{m}}$ as function of the element size $h$ is shown on Figure \ref{fig:error vs h}.
One can notice that $E_{c_\mathrm{m}}$ increases as $A\cdot h^k$.
This is known as the \textit{asymptotic regime} and the coefficient $k$ is called the convergence rate.
$k$ typically approaches $N+1$ as $h$ approaches zero, $N$ being the order of the finite elements.
In this case, $k \approx 2$ as expected since first order finite elements have been used.

\begin{figure}
    \centering
    \includegraphics[width=1\linewidth]{"Figures/Chapter3/L2 error on Cm vs h"}
    \caption{Evolution of the L2 norm of the error as function of element size h for the 1D H transport case (Case 2a).}
    \label{fig:error vs h}
\end{figure}

\paragraph{Case 2b: Application to 2D hydrogen transport}


The same method can be applied to a 2D case.
Let us choose the following steady state test problem on a domain $\Omega = [0, 1] \times [0, 1]$ with the manufactured solution $c_D(x, y) = \sin(\omega \pi x) \sin(\omega \pi y)$.

\begin{align}
    \nabla \cdot D \nabla c_\mathrm{m} &= -f_1 \\
    k c_\mathrm{m} (n - c_\mathrm{t}) - p c_\mathrm{t} &= -f_2 \\
    c_\mathrm{m} &= c_\mathrm{t} = c_D \text{  on  } \partial \Omega \\
    D &= 2 \\
    p &= 3 \\
    k &= 2 \\
\end{align}

The source terms $f_1$ and $f_2$ and the boundary conditions can be obtained in a similar fashion by replacing $c_\mathrm{m}$ and $c_\mathrm{t}$ in the governing equations.

It was shown that the computed solutions was similar to the exact solutions (see Figure \ref{fig: results MMS 2D H transport}).
Moreover, the convergence rates confirm the mesh dependency of the computed solutions accuracy (see Figure \ref{fig: convergence rates H}).
% A super-convergence is observed for the P2 elements.

\begin{figure*}
    \centering
    \begin{subfigure}{0.3\linewidth}
        \centering
        \includegraphics[width=\linewidth]{Figures/Chapter2/c_m.pdf}
        \caption{Computed $c_\mathrm{m}$ (64 elements).}
    \end{subfigure}%
    \begin{subfigure}{0.3\linewidth}
        \centering
        \includegraphics[width=\linewidth]{Figures/Chapter2/c_t.pdf}
        \caption{Computed $c_\mathrm{t}$ (64 elements).}
    \end{subfigure}%
    \begin{subfigure}{0.3\linewidth}
        \centering
        \includegraphics[width=\linewidth]{Figures/Chapter2/c_exact.pdf}
        \caption{Exact solution $c_D$.}
    \end{subfigure}
    \caption{Comparison of the computed concentrations with the exact solution (Case 2b).}
    \label{fig: results MMS 2D H transport}
\end{figure*}

\begin{figure}
    \centering
    \includegraphics[width=\linewidth]{Figures/Chapter2/convergence_rate_H.pdf}
    \caption{Evolution of the L2 error on $c_\mathrm{m}$ (left) and $c_\mathrm{t}$ (right) showing the convergence rates for the 2D H transport case (Case 2b).}
    \label{fig: convergence rates H}
\end{figure}



\subsection{Experimental validation}

Now that the code has been verified (\textit{ie} it solves the governing equations correctly), experimental validation is still required to check that these equations actually represent experimentally observed processes.
A very good example of experiments that can be reproduced are Thermo-Desorption Spectroscopy (TDS) experiments also called Thermally Programmed Desorption (TPD) experiments.
The principle of such experiments is to load samples with hydrogen isotopes either via gas infusion, plasma implantation or electrochemical charging.
The samples are then heated up to different temperatures to desorb the trapped hydrogen.
By measuring the outgassing flux of particles throughout the time of the experiment, desorption spectra are obtained.
These spectra often exhibit several peaks and each peak correspond to a kind of trap (in most cases).

This technique is therefore employed to characterise materials and their defects. 
It is also a very good application case for experimental validation of the hydrogen transport model.

This section describes the technique that was employed to easily reproduce these TDS experiments.

\subsubsection{Methodology} \label{methodology}
Fitting experimental data by manually tweaking parameters as in \sidecite{yu_deuterium_2019, hodille_macroscopic_2015} can be really time-consuming, sometimes days in some cases.
Moreover, some possible solutions in the parameter space might be missed by the user.
This process has been automated by embedding \gls{festim} in a minimisation algorithm.

As in manual fitting, the parametric optimisation problem is solved by minimising a function representing the residual between simulated results and some reference data.
This function $f$ is called \textit{cost function}.
Considering fitting one or several \gls{tds} spectra (in order to identify for instance trapping parameters or diffusion coefficients), $f$ can simply be the mean absolute error described in Equation \ref{eq:cost function} representing the residual between the simulated spectrum and the experimental reference: 

\begin{equation}
    f(\textbf{x})=\frac{\sum_{i=0}^{N}  \alpha_i(T_i)\left| d_{i}-d_{\mathrm{sim}}\right|}{\sum_{i=0}^{N}  \alpha_i(T_i)}
    \label{eq:cost function}
\end{equation}

where \textbf{x} is the set of parameters used for the simulation, $d_\mathrm{sim}$ are the values of the simulated spectrum, $N$ is the number of experimental points $(T_i, d_i)$.
In Equation \ref{eq:cost function}, $f(\textbf{x})$ can be weighted by coefficients $\alpha_i$ in order to have a better fit on specific regions of the spectrum.
%Note that this cost function could as well be a root mean square error or any type of residual.

The parametric optimisation problem can now be solved by finding the minimum of the cost function $f$.
The global optimisation routine is illustrated on Figure \ref{fig:diagramm}.
\begin{figure}
    \centering
    \includegraphics[width=\linewidth]{Figures/Chapter3/Parametric_optimisation/algorithm diagram.pdf}
    \caption{Diagram of the embedding of FESTIM within a parametric optimisation routine based on SciPy \cite{virtanen_scipy_2020}.}
    \label{fig:diagramm}
\end{figure}

A comparative study of the several optimisation algorithms which can be employed has been made.
These algorithms require the user to give an initial set of parameters called \textit{initial guess} and evaluate the cost function with several parameters sets until the convergence criterion is reached.
As in \sidecite{drexler_model-based_2019}, the Python package SciPy \sidecite{virtanen_scipy_2020} will be employed.

Four minimisation algorithms have been benchmarked against a test case.
In the following example an experimental \gls{tds} spectrum from Ogorodnikova et al.\ \sidecite{ogorodnikova_deuterium_2003} will be fitted and materials properties such as trap density and detrapping energy will be identified.
For this example case, two intrinsic traps and one extrinsic trap are set.
The only free parameters are $E_1$ and $n_1$, respectively the detrapping energy and density of trap 1.
The other parameters are constrained and described in \sidecite{delaporte-mathurin_finite_2019}.
The cost function $f$ has been plotted on Figure \ref{fig:cost function} as function of $E_1$ and $n_1$.

\begin{figure*} [h!]
    \centering
        \begin{subfigure}[t]{0.7\linewidth}
            \centering
            \includegraphics[width=\linewidth]{Figures/Chapter3/Parametric_optimisation/cost_function_2D.pdf}
            \caption{Normalised cost function.}
        \end{subfigure}
        \begin{subfigure}[t]{0.7\linewidth}
            \centering
            \includegraphics[width=\linewidth]{Figures/Chapter3/Parametric_optimisation/points_on_cost_function.pdf}
            \caption{Corresponding simulated \gls{tds} spectra.}
        \end{subfigure}%
    \caption{Normalised cost function $\hat{f} = (f - \min{f})/(\max{f}-\min{f})$ as function of $E_1$ (\si{eV}) and $n_1$ (\si{at.fr.}) with global minimum located at $(\SI{0.86}{eV}, \SI{1.2e-3}{at.fr.})$.}
    \label{fig:cost function}
\end{figure*}

In this case, when only two free parameters are set the cost function has only one minimum (it is not necessarily the case for higher dimension optimisation problems).
However, if one fixes the trap density $n_1$ above $\approx \SI{2e-3}{at.fr.}$, the cost function has two local minima which can lead the optimisation routine to converged to a non-global minimum.
Moreover, $f$ is smooth and quadratic around its minimum located at $(E_1, n_1) = (\SI{0.86}{eV}, \SI{1.2e-3}{at.fr.})$.
For detrapping energies below \SI{0.6}{eV} and/or densities below \SI{0.5e-3}{at.fr}, the cost function is constant.
This is because for these values, the contribution of this trapping site to the \gls{tds} spectrum is zero either because the density is close to zero, or because the energy is too low for these traps to be filled at the implantation temperature of \SI{300}{K}.
Variations in these regions do not modify the simulated spectrum and thus do not modify the cost function value.

\begin{figure} [ht]
    \centering
    \includegraphics[width=\linewidth]{Figures/Chapter3/Parametric_optimisation/algorithms_perfs.pdf}
    \caption{Number of cost function evaluations required to converge towards the global minimum with 100 different initial guesses sorted by distance to the global minimum for several minimisation algorithms. Each cost function evaluation takes \SI{20}{s} to compute. White stripes correspond to initial guesses for which the algorithm did not converge to the global minimum.}
    \label{fig:algos perfs}
\end{figure}


Four different optimisation algorithms are being compared: 
Nelder-Mead (also called the simplex method), Powell, \gls{tnc} and \gls{cg}.
Thorough descriptions of these algorithms would be beyond the scope of this research but can be found in \sidecite{nocedal_numerical_2006}.
The performances of these algorithms have been compared with 100 different initial guesses randomly distributed on the $(E_1,n_1)$ plane and are shown on Figure \ref{fig:algos perfs}.
It appears that the \gls{cg} algorithm is less robust since for some cases it didn't converge towards the global minimum (see white bands on Figure \ref{fig:algos perfs}).
Nelder-Mead algorithm appears to be the most efficient with initial guesses both close and far from the global minimum since the number of cost function evaluations ranges from 50 to 100 whereas other algorithms require more than 100.
This can be explained by the fact that Nelder-Mead is a derivative-free algorithm whereas \gls{tnc} and \gls{cg} algorithms need on the other hand to compute first order derivatives thus increasing the number of function evaluations.
This will be even more true when increasing the number of free parameters since the derivative will become more costly to compute.

It is worth noting that the Nelder-Mead algorithm is an unconstrained method.
If constraints or bounds are needed, \gls{tnc} might be a more suitable choice.

Though in the following, the Nelder-Mead algorithm will be employed in the following cases.


% \subsubsection{Results} \label{results}

% \subsection{Applications}

The fitting procedure has been employed to reproduce thermo-desorption experiments performed on Tungsten, EUROFER, Aluminium and Beryllium.


\subsubsection{Tungsten}

The TPD spectrum measured by Ogorodnikova \textit{et al} presented in Section \ref{optimisation algorithms} has been reproduced by setting all traps parameters as free parameters.
The fitting procedure has been run for several numbers of traps as shown on Figure \ref{fig:number of traps comparison}.
It is clear that setting only one trap is not sufficient to reproduce the experimental data.
The two traps case shows better results but also has a discrepancy near \SI{600}{K}.
This discrepancy is removed when setting a third extrinsic trap to the simulation.

For this last case, the five free parameters are the detrapping energies $E_{p, 1}$, $E_{p, 2}$, $E_{p, 3}$ and densities $n_1$, $n_2$ (the third trapping site being created during the implantation, for which the creation parameters are not part of the free parameters and taken from \sidecite{ogorodnikova_deuterium_2003} or \sidecite{hodille_macroscopic_2015}).
This optimisation case is therefore a 5D optimisation problem.
Every other parameters are taken from \sidecite{hodille_macroscopic_2015}.
The resulting fit is shown on Figure \ref{fig:5D TPD} alongside with the contribution of each trap to the total spectrum.
% (1 trap [ 0.88368005  1.44347635] 2 traps [ 0.83582125  1.23768655  0.98708714  6.85457283])
\begin{figure*} [ht]
    \centering
        \begin{subfigure}[t]{0.5\linewidth}
            \centering
            \captionsetup{width=.9\linewidth}
            \includegraphics[width=\linewidth]{Figures/Chapter3/Parametric_optimisation/number_of_traps.pdf}
            \caption{Comparison of the resulting fit with several numbers of traps in the simulation.}
            \label{fig:number of traps comparison}
        \end{subfigure}%
        \begin{subfigure}[t]{0.5\linewidth}
            \centering
            \captionsetup{width=.9\linewidth}
            \includegraphics[width=\linewidth]{Figures/Chapter3/Parametric_optimisation/Ogorodnikova_5D.pdf}
            \caption{Identified by the fitting procedure with $E_1 = \SI{0.83}{eV}$, $E_2 = \SI{0.97}{eV}$, $E_3 = \SI{1.51}{eV}$, $n_1 = \SI{1.18e-3}{at.fr.}$ and \newline $n_2 = \SI{7.22e-4}{at.fr.}$.  Dashed lines correspond to the temporal evolution of each trapping population's inventory.}
            \label{fig:5D TPD}
        \end{subfigure}%
    \caption{Fitting TPD spectrum performed on Tungsten by Ogorodnikova \textit{et  al} \cite{ogorodnikova_deuterium_2003}. Dots correspond to experimental data.}
    \label{fig:TPD ogorodnikova}
\end{figure*}
The identified parameters are similar to the ones found by Hodille \textit{et al.} in \sidecite{hodille_macroscopic_2015}.
The total fitting procedure took a few hundred of cost function evaluations.
One single cost function evaluation "costing" less than \SI{20}{s} to compute (for that specific case), the total procedure lasted less than \SI{3}{h}.

\subsubsection{EUROFER}

Hollingsworth \textit{et al} performed thermo-desorption on pre-damaged EUROFER at several damage levels \sidecite{hollingsworth_comparative_2019}.
Three spectra with similar exposure conditions have been fitted with one trapping site (since only one peak appears on the spectra) as shown on Figure \ref{fig:TPD EUROFER}.

\begin{figure} [ht]
    \centering
    \includegraphics[width=\linewidth]{Figures/Chapter3/Parametric_optimisation/EUROFER_hollingsworth.pdf}
    \caption{TPD spectra of damaged EUROFER \cite{hollingsworth_comparative_2019}. Fitted with one trapping site (solid line) $E=\SI{1.06}{eV}$ and densities of \SI{8.9e-3}{at.fr}, \SI{2.8e-2}{at.fr} and \SI{5.0e-2}{at.fr} for \SI{0}{dpa}, \SI{0.01}{dpa} and \SI{0.1}{dpa}, respectively. Dashed lines correspond to optimisations with an unweighted cost function. Dots correspond to experimental data.}
    \label{fig:TPD EUROFER}
\end{figure}

To put the emphasis on peaks, a weighting factor of 10 has been applied for $T \in [\SI{445}{K}, \SI{492}{K}]$.
Not applying this factor near the peak region results in a closer fit in other regions but a higher peak value.
The identified trap energy is $E_p$ \SI{1.06}{eV} for all spectra whereas the trap density $n$ is \SI{8.9e-3}{at.fr} for the undamaged sample, \SI{2.8e-2}{at.fr} for \SI{0.01}{dpa} and \newline \SI{5.0e-2}{at.fr} for \SI{0.1}{dpa}.
For all simulations the attempt frequency $p_0$ is \SI{1e13}{s^{-1}} and the diffusion coefficient is taken from \sidecite{esteban_hydrogen_2007}.

The total fitting procedure took less than two hours for fitting the three spectra.
A more thorough study of these experiments could involve constraining the algorithm with profilometry data obtained by Hollingsworth \textit{et al} \sidecite{hollingsworth_comparative_2019}.
Indeed, having a non-homogeneous trapping site distribution could help having a better fit of both the profilometry data and the TPD spectra. 

\subsubsection{Aluminium}

The experiment performed on Aluminium by Quiros \textit{et al} \sidecite{quiros_blistering_2019, quiros_blister_2017} has also been reproduced with FESTIM.

Only one trap has been set in the simulation and its energy $E_p$ and density $n$ are set as free parameters.
Every other parameters are fixed and taken from \sidecite{quiros_blister_2017, quiros_blistering_2019}.
The resulting simulated TPD spectrum is shown on Figure \ref{fig:TPD alu}.
\begin{figure} [ht]
    \centering
    \includegraphics[width=\linewidth]{Figures/Chapter3/Parametric_optimisation/alu_quiros.pdf}
    \caption{TPD spectrum of aluminium exposed to \SI{3e23}{H.m^{-2}} at \SI{618}{K} \cite{quiros_blister_2017, quiros_blistering_2019}. Fitted with one trapping site $n = \SI{1.8e-2}{at.fr.}$ and $E =\SI{1.1}{eV}$. Dots correspond to experimental data. Dashed lines correspond to the temporal evolution of each trapping population's inventory.}
    \label{fig:TPD alu}
\end{figure}


The identified parameters are $n = \SI{1.8e-2}{at.fr.}$ and $E =\SI{1.1}{eV}$.
The trapping sites density is significantly higher than the one described in \sidecite{quiros_blister_2017}.
However, the TPD spectrum obtained with this procedure better fits the experimental data since the one obtained by Quiros \textit{et al} requires a 10-fold increase.
The fitting procedure took less than a hundred cost function evaluations, which corresponds in total to a few dozens of minutes.

\subsubsection{Beryllium}
Be co-deposition experiments performed by Baldwin \textit{et al} \sidecite{baldwin_experimental_2014} were reproduced with FESTIM using this optimisation technique.
In this experiment, a \SI{1}{\micro m} thick Be-D layer is grown on Tungsten at \SI{330}{K}. 
Following the strategy proposed by Baldwin \textit{et al}, only the thermo-desorption phase has been simulated with two trapping sites with homogeneously distributed densities and with initial occupancies $f_i$.
There are therefore three free parameters per trap (energy, density and initial occupancy) which makes this optimisation problem 6D.
It is assumed that the surface flux is the net balance between incoming flux from the chamber (very low since the pressure is $\SI{1}{\micro Pa}$) and the molecular recombination flux.
All the other parameters are described in \sidecite{baldwin_experimental_2014}.

The resulting optimised TPD spectrum is shown on Figure \ref{fig:tpd baldwin}.
The optimised parameters are $E_{p, 1} = \SI{0.75}{eV}$, $n_1 = \SI{1.09e-1}{at.fr.}$, $f_1=0.73$, $E_{p, 2} = \SI{0.93}{eV}$, \newline ${n_2 = \SI{3.40e-2}{at.fr.}}$, $f_2=0.28$.
These values are in agreement with the ones found by Baldwin \textit{et al} \sidecite{baldwin_experimental_2014} and took only a few of minutes to compute since the implantation phase was not simulated.

\begin{figure}
    \centering
    \includegraphics[width=\linewidth]{Figures/Chapter3/Parametric_optimisation/baldwin_be.pdf}
    \caption{TPD spectrum of co-deposited Be-D \cite{baldwin_experimental_2014} simulated with two trapping sites. Dots correspond to experimental data.}
    \label{fig:tpd baldwin}
\end{figure}

\subsubsection{Discussion}

Even though an automated technique is proposed, the user still has some choices to make in order to ensure the credibility of the fitted spectrum.
As shown on Figure \ref{fig:TPD EUROFER}, weighting the cost function near regions of interest will result in a better fit in these regions.
Users should also be aware of the number of traps the data is being fitted with.
As shown on Figure \ref{fig:number of traps comparison} too few traps in the simulation will not result in a satisfactory fit (even though the optimisation routine will converge to an optimised solution).
Moreover, as shown on Figure \ref{fig:hurley_comparison}, one single TPD spectrum can be reproduced with several traps of different energies and densities.
This means that the cost function with several traps as free parameters can have several local minima of very similar values.
Adding traps to an optimisation problem can also help having a better fit of the experimental data in some cases.
But artificially adding more and more traps is not necessarily realistic and could lead to misinterpretation of the results.

\begin{figure}[ht]
    \centering
    \includegraphics[width=\linewidth]{Figures/Chapter3/Parametric_optimisation/hurley_comparison.pdf}
    \caption{TPD spectrum reproduced with several sets of parameters showing the existence of several solutions to a single optimisation problem.}
    \label{fig:hurley_comparison}
\end{figure}

In the first case with only one trapping site, as described by Hurley \textit{et al} in \sidecite{hurley_numerical_2015}, the binding energy is \SI{0.55}{eV} and the trap density is \SI{2.08e24}{m^{-3}}.
The appearance of two peaks is due to the desorption on different sides of the sample as explained in \sidecite{hurley_numerical_2015}.
In the second case, the curve as been reproduced with two trapping sites which energies and densities are respectively \SI{0.51}{eV} and \SI{0.57}{eV} and \SI{2.02e24}{m^{-3}} and \SI{2.12e24}{m^{-3}}.
In the third case, it has been reproduced with three trapping sites which energies and densities are respectively \SI{0.55}{eV}, \SI{0.38}{eV} and \SI{0.51}{eV} and \SI{2.12e24}{m^{-3}}, \SI{2.26e24}{m^{-3}} and \newline \SI{2.13e24}{m^{-3}}.

This example illustrates how a single spectrum can be simulated with several sets of parameters by varying the number of traps in the simulation.
One way to avoid this from happening is to have a set of experiments with varying parameters such as the implantation temperature, the heating ramp, the fluence, dwelling time before TPD, etc.


\subsection{Comparison with TMAP7}

The FESTIM code was compared to TMAP7 \sidecite{longhurst_tmap7_2008}.
Since TMAP7 is a 1D code, a 1D test case was created.
% maybe say a bit more on TMAP7 here


\subsubsection{Geometry}
The 1D simulation case is a \SI{8.5}{mm}-thick composite slab made of W, Cu and CuCrZr (see Figure \ref{fig: monoblock 1D geometry}).
The plasma facing surface $\Gamma_\mathrm{top}$ is located at $x=\SI{0}{mm}$ and the surface cooled by water $\Gamma_\mathrm{coolant}$ is located at $x=\SI{8.5}{mm}$.

\begin{figure}
     \subfloat[1D geometry ]{
        \begin{overpic}[width=0.75\linewidth]{Figures/Chapter3/monoblocks/interface_condition/iter case/Monoblock 1D.pdf}
            \put(40, 50){\SI{6}{mm}}
            \put(40, 8){W}
            \put(62, 50){\SI{1}{mm}}
            \put(65, 8){Cu}
            \put(72, 50){\SI{1.5}{mm}}
            \put(72, 8){CuCrZr}
            \put(6, 25){\large$\Gamma_\mathrm{top}$}
            \put(85, 25){\large$\Gamma_\mathrm{coolant}$}
        \end{overpic}
     }
     \caption{2D ITER monoblock geometry showing W armour \cruleme[grey]{0.3cm}{0.3cm}, Cu interlayer \cruleme[orange]{0.3cm}{0.3cm}, CuCrZr alloy cooling pipe  \cruleme[yellow]{0.3cm}{0.3cm}}
     \label{fig: monoblock 1D geometry}
\end{figure}

\subsubsection{Boundary Conditions}
The boundary conditions are detailed in Equation \ref{eq: code comparison BCs}.

\begin{subequations}
    \begin{align}
    T &= \SI{1200}{K}\quad \text { on } \Gamma_\mathrm{top}\\
    c_\mathrm{m} &=  \frac{\varphi_\mathrm{imp} \cdot R_p}{D} \quad \text { on } \Gamma_\mathrm{top}\\
    T &= \SI{373}{K} \quad \text { on } \Gamma_\mathrm{coolant}\\
    -D \nabla c_\mathrm{m} \cdot \vec{n} &= K_\mathrm{CuCrZr} \cdot c_\mathrm{m}^{2} \quad \text { on } \Gamma_\mathrm{coolant}  
    \end{align}
    \label{eq: code comparison BCs}
\end{subequations}
with $\varphi_\mathrm{imp} = \SI{5e23}{m^{-2}.s^{-1}}$ the implanted particle flux, $R_p = \SI{1.25}{nm}$ the implantation depth, $\vec{n}$ the normal vector and $K_\mathrm{CuCrZr} = 2.9 \times 10^{-14}\cdot \exp{(-1.92/(k_B\cdot T))}$ the recombination coefficient of the CuCrZr (in vacuum) expressed in \si{m^4.s^{-1}} \sidecite{anderl_deuterium_1999}.

The Dirichlet boundary condition on $\Gamma_\mathrm{top}$ for the hydrogen transport corresponds to a flux balance between the implanted flux and the flux that is retro-desorbed at the surface (see Section \ref{triangle model}).

\subsubsection{Temperature}
\begin{figure*}
    \centering
    \includegraphics[width=0.5\linewidth]{Figures/Chapter3/monoblocks/interface_condition/iter case/temperature_1D.pdf}
    \caption{ITER monoblock temperature simulated by FESTIM (1D and 2D).}
    \label{fig: temperature}
\end{figure*}


\subsubsection{Results}
\begin{figure}
    \centering
    \includegraphics[width=\linewidth]{Figures/Chapter3/monoblocks/interface_condition/iter case/comparison_codes.pdf}
    \caption{Comparison of results provided by FESTIM, TMAP7 and ABAQUS}
    \label{fig: code comparison}
\end{figure}

A comparison test was made with the code TMAP7 with this set of parameters and very good agreement was found between the two codes (see Figure \ref{fig: code comparison}).


\section{Summary}

The macroscopic rate equations model describing the transport (diffusion and trapping) of H in solids was presented alongside with additional models such as the conservation of chemical potential at interfaces.
Due to the presence of thermally activated processes (diffusion, trapping, detrapping, surface processes, ...), the heat transfer equation has to be solved numerically.
All these equations are solved with the finite element code FESTIM, which heavily relies of FEniCS.

FESTIM has been verified using methods such as the Method of Exact Solutions and the Method of Manufactured Solutions.
On the other hand, it was shown that FESTIM could be employed to reproduce real-life experiments (TDS experiments) performed on Tungsten, Aluminium, Beryllium and EUROFER.
This validation process could be extended by reproducing other types of experiments such as permeation experiments and profilometry.
However, this set of equation (shared amongst H transport codes) has already proven to be capable of reproducing these experiments.
This has been done, for instance, during the validation of TMAP7 \sidecite{longhurst_tmap7_2008}.

Thanks to this verification \& validation process, it was shown that (1) the governing equations were correctly solved and (2) these equations can represent real life processes.

The FESTIM code can then be safely employed to perform analysis on tokamak components.

\setchapterimage{monoblocks2}
\setchapterpreamble[u]{\margintoc}
\chapter{Monoblocks}\label{Chapter3}\labch{Chapter3}

\section{Introduction}
ITER and DEMO divertors will be composed of small unit bricks called \textit{monoblocks} (see Figure \ref{fig: inner target photo}).
Monoblocks are typically made of a tungsten substrate with a cooling pipe running through.
This cooling channel is necessary to keep the component's temperature below its operating limit.

Several monoblock designs are currently studied for DEMO with varying dimensions, different materials for the cooling pipe or the interlayer, etc \sidecite{vizvary_european_2020, huang_tungsten_2016, hirai_use_2016, domptail_design_2020}.
The main candidate is the ITER-like design, which is the type of monoblock that will be used in ITER \cite{hirai_use_2016}.
This design has a tunsgten substrate with a CuCrZr cooling pipe and a Cu interlayer for compliance (see Figure \ref{fig: monoblocks with pipe}).
In ITER, monoblocks will be \SI{12}{mm}-thick whereas they will be thiner in DEMO (\SI{4}{mm}) \sidecite{you_european_2018}.

\begin{figure}
    \centering
    \includegraphics[width=\linewidth]{Figures/Chapter3/inner_target_iter.jpg}
    \caption{Prototype of the inner vertical target (source: ITER Organization).}
    \label{fig: inner target photo}
\end{figure}

\begin{figure}
    \centering
    \includegraphics[width=0.7\linewidth]{Figures/Chapter3/monoblocks_with_pipe.png}
    \caption{ITER-like monoblocks.}
    \label{fig: monoblocks with pipe}
\end{figure}

In order to assess the behaviour of H in the divertor, component-level simulations of monoblocks are required.

This chapter will focus on simulating H transport in ITER-like monoblocks.
First, the influence of interface conditions at the W/Cu and Cu/CuCrZr interfaces will be evaluated.
% Then, the impact of load cycling (heat and particles fluxes) will be assessed.
A parametric study will then be performed to link the monoblock H inventory to the exposure conditions.
Finally, the influence of edge effects (desorption on poloidal and toroidal gaps) will be highlighted.

\section{Interface conditions}
In order to assess the influence of interface conditions on the outgassing flux, simulations are performed with chemical potential continuity (Equation \ref{eq: c/s conservation}) or mobile concentration continuity assuming in both cases flux conservation.
For the sake of simplicity and to emphasis on the influence of interface conditions, no trapping was assumed and ideal Dirichlet boundary conditions were set.
Simulations were performed on two test cases: W/Cu and Cu/EUROFER.
The materials properties used for the simulations can be found in Table \ref{tab:materials properties_1}.
In both cases the solute concentration $c_\mathrm{m}$ was set to \SI{1e20}{m^{-3}} at $x=0$ and zero on the other boundary.

% The steady-state temperature field exhibits high temperature gradients between the plasma-facing surface and the cooling surface (see Figure \ref{fig:2D temperature monoblock}).

\begin{figure}
    \centering
    \includegraphics[width=0.8\linewidth]{Figures/Chapter3/monoblocks/interface_condition/iter case/temperature_field_2d.pdf}
    \caption{Steady-state temperature field of an ITER-like monoblock simulated with FESTIM.}
    \label{fig:2D temperature monoblock}
\end{figure}


% TODO replace this figure!!!
\begin{figure}
    \centering
    \includegraphics[width=\linewidth]{Figures/Chapter3/monoblocks/interface_condition/iter case/comparison_inventory_2d.pdf}
    \caption{Influence of chemical potential conservation on hydrogen inventory.}
    \label{fig: 2D inventories}
\end{figure}


\begin{figure}
    \centering
    \begin{subfigure}{0.5\linewidth}
        \centering
        \includegraphics[width=\linewidth]{Figures/Chapter3/monoblocks/interface_condition/iter case/solute_c.pdf}
        \caption{$c_\mathrm{m}$ (continuity of $c_\mathrm{m}$)}
    \end{subfigure}%
    \begin{subfigure}{0.5\linewidth}
        \centering
        \includegraphics[width=\linewidth]{Figures/Chapter3/monoblocks/interface_condition/iter case/solute_mu.pdf}
        \caption{$c_\mathrm{m}$ (continuity of $\mu$)}
    \end{subfigure}
    \begin{subfigure}{0.5\linewidth}
        \centering
        \includegraphics[width=\linewidth]{Figures/Chapter3/monoblocks/interface_condition/iter case/retention_c.pdf}
        \caption{Retention (continuity of $c_\mathrm{m}$)}
    \end{subfigure}%
    \begin{subfigure}{0.5\linewidth}
        \centering
        \includegraphics[width=\linewidth]{Figures/Chapter3/monoblocks/interface_condition/iter case/retention_mu.pdf}
        \caption{Retention (continuity of $\mu$)}
    \end{subfigure}
    \caption{2D concentration fields at $t=\SI{2.4e7}{s}$}
    \label{fig: concentrations fields 2d}
\end{figure}
Two cases were examined, one with mobile concentration $c_\mathrm{m}$ continuity at interfaces and the other with continuity of chemical potential (\textit{ie.} continuity of $c_\mathrm{m}/S$).

Up to \SI{5e6}{s}, there was no difference in the total hydrogen inventory between the two cases (see Figure \ref{fig: 2D inventories}).
It is only after this implantation that the inventory of the two cases started to diverge.
At $t=\SI{2.4e7}{s}$, the inventory of the continuity of chemical potential case was higher than with continuity of $c_\mathrm{m}$.
This is explained by the high solubility ratio between Cu and CuCrZr leading to a higher concentration of mobile particles in CuCrZr and therefore a higher trapping rate.
However, even then, the trap density in Cu being low compared to other materials, the global inventory is not affected much.
For these two reasons, the inventories are unaffected before \SI{5e6}{s}.

Similarily, before reaching the W/Cu interface, the $c_\mathrm{m}$ and retention profiles are identical regardless of the interface condition (see Figure \ref{fig: concentrations fields 2d}).
Once the W/Cu interface is reached, the $c_\mathrm{m}$ profiles are affected by the interface condition.
The interface condition had no influence whatsoever on the mobile particle concentration $c_\mathrm{m}$ in the W.
However, $c_\mathrm{m}$ was higher in Cu and CuCrZr in the case with chemical potential conservation (up to \SI{1.5e24}{m^{-3}} in CuCrZr at $t=\SI{2.4e7}{s}$).
This increase of $c_\mathrm{m}$ leads to an increase of the trap occupancy and therefore an increase of the local retention.

The retro-desorbed flux (from the monoblock to the plasma) does not depend on the interface conditions since interfaces are far from the exposed surface.
Moreover, outgassing flux through the cooling pipe greatly depends on the boundary condition imposed at the cooling surface.
Therefore, in order to assess the impact of interface conditions on the outgassing flux through the cooling pipe, uncertainties must first be lift regarding the recombination process occurring on surfaces in contact with water.

% \subsection{Summary}
% should we keep this
% The influence of interface conditions between materials has been studied with FESTIM.
% A novel approach has been implemented in FESTIM in order to ensure equilibrium at the interfaces.
% The implementation has been verified using the Method of Exact Solutions and the Method of Manufactured Solutions.
% % A comparison test has been performed with TMAP7 and Abaqus and the three codes show very good agreement.

% H transport through Cu/EUROFER and W/Cu composite slabs has been studied.
% It is shown that the interface condition can have an impact on the outgassing flux.
% This modelling work will help design future permeation barriers in DEMO.
% A method for identifying material properties with either an analytical solution or with the FESTIM code is also described.

% The influence of interface conditions is also studied on the ITER monoblock test case in both 1D and 2D.
% It is shown that this has very low influence up to \SI{5e6}{s} and that discrepancies only start to appear after a very long exposure time.
% This is because interfaces are far from the exposed surface and hydrogen atoms only reach these interface after a long exposure time.
% The continuity of mobile concentration can therefore be employed safely for monoblocks H transport simulations in order to assess monoblocks inventory.
% This is especially true when desorption is assumed on the edges of the monoblock where less H particles will reach the interfaces.
% In other words, the effect of the interfaces conditions is negligible compared to the edge effects (this will be shown in more details in Section \ref{3D edge effects}).


% \section{Influence of cycling}

% \begin{itemize}
%     \item Redo one or two sims in Etienne paper ?
%     \item continuous first to load the monoblock then cycling
% \end{itemize}

\section{Exposure conditions} \label{influence of exposure conditions}
Monoblocks in a fusion reactor will be exposed to a wide range of exposure conditions (heat and particle fluxes) and their behaviour (both in terms of heat transfer and hydrogen transport) will change based on these conditions.
In ITER, these fluxes range from $\approx$ \SI{10}{MW.m^{-2}} and $\approx$ \SI{e23}{H.m^{-2}.s^{-1}} to $\approx$ \SI{10}{MW.m^{-2}} and $\approx$ \SI{e23}{H.m^{-2}.s^{-1}} at the inner strike point.
The distribution of these fluxes depend on many operation parameters.

One way of simulating a whole divertor would be to simulate each and every monoblock for a given scenario along one Plasma-Facing Unit.
However, this method would be computationally expensive as it requires to redo the simulations for every scenario.

Another, more efficient method, is to perform a parametric study on a monoblock.
The exposure parameters are varied and for each set of parameters, the quantity of interest (here the hydrogen inventory) is computed.
A relationship is then produced between the exposure parameters and the quantity of interest.
This method is more robust in the sense that it does not require to run additional simulations once this relationship is obtained but simply use this relationship to obtain the quantity of interest.

In this work, we employed the second method.
The goal of this Section is to establish the relationship between the exposure conditions of the monoblock and its hydrogen content at a given time.

\subsection{Simulation description}

The monoblock model has one trap per material for the sake of simplicity and computational time, but the method could be applied to any set of trapping parameters.


\subsubsection{Boundary conditions}

The concentration of mobile particles $c_\mathrm{m}$ is imposed on $\Gamma_\mathrm{top}$.
Molecular recombination is assumed on $\Gamma_\mathrm{coolant}$.
Even though it could be assumed on the toroidal gaps between monoblocks, it can be shown that its influence on the macroscopic behaviour remains low.
This is because the maximum retention zone is observed near the cooling pipe and not near the gaps.
Desorption from the other surfaces is therefore assumed to be zero for simplification purposes.
Uniform heat loads $\varphi_H$ are first applied on the surface $\Gamma_\mathrm{top}$ with a Neumman boundary condition.
The temperature will then be constrained on $\Gamma_\mathrm{top}$ with a Dirichlet boundary condition in Section \ref{relationship_T_surf_c_surf}.
A convective exchange condition is set on surface $\Gamma_\mathrm{coolant}$.
All the other surfaces are assumed thermally insulated.
The set of boundary conditions can finally be described as follow:

\begin{subequations}
    \begin{align}
    -\lambda \vec{\nabla} T \cdot \vec{n} &=\varphi_{H} \quad \text{or} \quad T = T_\mathrm{surface}\quad &\text { on } \Gamma_\mathrm{top}\\
    c_\mathrm{m} &=  c_\mathrm{surface}\quad &\text { on } \Gamma_\mathrm{top}\\
    -\lambda \vec{\nabla} T\cdot \vec{n} &= -h \cdot \left(T_\mathrm{coolant} - T\right)\quad &\text { on } \Gamma_\mathrm{coolant}\\
    -D \vec{\nabla} c_\mathrm{m} \cdot \vec{n} &= K_\mathrm{CuCrZr} \cdot c_\mathrm{m}^{2} \quad &\text { on } \Gamma_\mathrm{coolant}
    \end{align}
\end{subequations}
with $h=\SI{70000}{W.m^{-2}.K^{-1}}$ being the heat exchange coefficient calculated from the Sieder-Tate correlation for the forced convection regime, $T_\mathrm{coolant}= \SI{323}{K}$ and $\vec{n}$ the normal vector and $K_\mathrm{CuCrZr} = 2.9 \times 10^{-14}\cdot \exp{(-1.92/(k_B\cdot T))}$ the recombination coefficient of the copper alloy (in vacuum) expressed in \si{m^4.s^{-1}} \sidecite{anderl_deuterium_1999}.

\subsection{Influence of \texorpdfstring{$T_\mathrm{surface}$}{Tsurface} and \texorpdfstring{$c_\mathrm{surface}$}{csurface} on hydrogen inventory} \label{relationship_T_surf_c_surf}

In this section, the total inventory of hydrogen in monoblocks has been calculated as a function of $T_\mathrm{surface}$ and $c_\mathrm{surface}$.
Temperature and mobile concentration of hydrogen were imposed with Dirichlet boundary conditions on $\Gamma_\mathrm{top}$ with $T_\mathrm{surface}$ varying from $T_\mathrm{coolant}$ to \SI{1200}{K} and $c_\mathrm{surface}$ varying arbitrarily from \SI{e20}{m^{-3}} to \SI{6e22}{m^{-3}}.
For surface temperatures below \SI{500}{K}, 1D simulations were performed for the penetration depth of hydrogen remained very low (a few microns) and 1D approximation was sufficient \sidecite{benannoune_numerical_2019}.
For temperatures above \SI{500}{K} for which edge effects become dominant, 2D simulations have been performed.

After $ \SI{e7}{s}$ a high retention zone appeared far from the exposed surface $\Gamma_\mathrm{top}$ (see Figure \ref{fig:retention fields}).
This high retention zone is due to thermal effects.
As seen in Figures \ref{fig:T field 1 MW} and \ref{fig:T field 10 MW}, the temperature was found to decrease in regions close to the cooling pipe $\Gamma_\mathrm{coolant}$ leading to an increase in trap occupancy, creating this high retention zone.
This is however not true for monoblocks where $T_\mathrm{surface} \approx T_\mathrm{coolant}$ since the temperature gradient in the domain is very low.
Instead, trap occupancy is close to one and the retention is high in the whole region where hydrogen has penetrated and not only far from the top surface.

\begin{figure*}
    \centering
    \begin{subfigure}{0.5\linewidth}
        \centering
        \includegraphics[height=\linewidth]{Figures/Chapter3/monoblocks/parametric_study/retention_T=7.000e+02;c=1.00e+20.pdf}
        \caption{$T_\mathrm{surface} = \SI{700}{K}$ and $c_\mathrm{surface} = \SI{e20}{m^{-3}}$}
    \end{subfigure}%
    \begin{subfigure}{0.5\linewidth}
        \centering
        \includegraphics[height=\linewidth]{Figures/Chapter3/monoblocks/parametric_study/retention_T=1.000e+03;c=1.00e+21.pdf}
        \caption{$T_\mathrm{surface} = \SI{1000}{K}$ and $c_\mathrm{surface} = \SI{e21}{m^{-3}}$}
    \end{subfigure}
    \caption{Example retention fields in \si{m^{-3}} after a \SI{e7}{s} exposure}
    \label{fig:retention fields}
\end{figure*}


In order to obtain this continuous field (see Figure \ref{fig:inventory T c}), more than 600 simulations randomly distributed on the parameter plane were run and analysed using a Gaussian process machine learning algorithm \sidecite{rasmussen_gaussian_2006} as in \sidecite{shimwell_multiphysics_2019} based on the python package inference-tools \sidecite{chris_bowman_c-bowmaninference-tools_2020}.
The inventory obtained by the Gaussian regression process is also given for a constant value of $c_\mathrm{surf}=\SI{2e21}{m^{-3}}$ (top inset) and a constant temperature $T=\SI{850}{K}$ (left inset).
The Gaussian regression process was particularly appropriate as it calculates a local standard deviation $\sigma$ based on the localisation of the datapoints and the deviation of the computed inventories.
The lower the density of simulation points, the higher was the value of $\sigma$ (for example around \SI{850}{K} on the top inset of Figure \ref{fig:inventory T c}).
However, despite the lack of simulation in this region, the value of $\sigma$ was still acceptable (only a few percents of the inventory) ensuring the quality of the resulting interpolation.

As expected, inventory was found to globally increase with $c_\mathrm{surface}$.
For $T_\mathrm{surface} > \SI{550}{K}$, the inventory tended to decrease with surface temperature.
However, for $T_\mathrm{surface} < \SI{550}{K}$, inventory increased with surface temperature.
This phenomenon is due to a trade-off between an increase of the detrapping rate and an increase of the diffusion coefficient making the hydrogen particles penetrate deeper into the bulk.
Above $\SI{550}{K}$, detrapping becomes dominant and inventory decreases.
This mapping of inventory as a function of $T_\mathrm{surface}$ and $c_\mathrm{surface}$ provides an easy way of estimating the inventory in monoblocks for several exposure conditions without having to run many simulations.
Indeed, to estimate the inventory with different exposure conditions, one only needs to associate these conditions $(\varphi_\mathrm{inc}, E)$ to a couple $(c_\mathrm{surf}, T_\mathrm{surf})$.

\begin{figure*} [h]
    \centering
    \includegraphics[width=\linewidth]{Figures/Chapter3/monoblocks/parametric_study/inventory_T_c_profiles.pdf}
    \caption{Evolution of the inventory after a \SI{e7}{s} exposure as a function of $T_\mathrm{surface}$ and $c_\mathrm{surface}$ alongside with simulation points (grey crosses). The simulations points were fitted with a Gaussian regression process \cite{chris_bowman_c-bowmaninference-tools_2020} providing the standard deviation $\sigma$.}
    \label{fig:inventory T c}
\end{figure*}

\subsection{Discussion}
Even though this methodology provides a rapid way of estimating hydrogen content in the whole divertor, several assumptions have however been made.


% Influence of cycling
First, a steady state exposure was considered for simplification purposes.
This result is however conservative.
As seen in \sidecite{delaporte-mathurin_finite_2019, hodille_estimation_2017}, cycling effects could have an influence in regions where $T_\mathrm{surface}$ varies a lot, for example within \SI{10}{cm} on both sides of the strike points.
Though, since a large majority of monoblocks stay at room temperature, even during operations the thermal effect should remain low and discrepancies would rather be due to particle flux evolution along the target.

% Be deposits
This study presents the hydrogen trapping in W monoblocks.
It shows that the latter remains low but, as already pointed out by JET studies, the trapping on Be co-deposited layers is expected to be the main mechanism for tritium retention in ITER \sidecite{brezinsek_beryllium_2015, heinola_fuel_2015}.
Such layers could be found in the cold regions of the divertor but as soon as the strike points hit these layers, they should be sputtered away (as sputtering of Be is possible even at low energy \cite{bjorkas_variables_2013, brezinsek_beryllium_2015}).
The retention where the deposited layers are not present (either sputtered or not formed anyway) would then be given by the model presented here.

% Coolant recombination
The molecular recombination coefficient at the surface of the cooling pipe was taken from \sidecite{anderl_deuterium_1999} and was measured in vacuum.
One could argue that recombination in presence of water will be facilitated.
This parameter has a very low influence on the inventory since it is dominated by retention in tungsten.
This parameter will however have an influence on the permeation flux and should be studied in future work.

% Gap recombination
Similarly, the influence on molecular recombination on the sides of the monoblock was found to have a low impact on the results.
By assuming an instantaneous recombination coefficient, the relative error on the monoblock inventory was found to be significant only in hot regions (\textit{ie} within \SI{10}{cm} on both sides of the strike points).
The influence on the total divertor inventory is therefore low (less than \SI{5}{\%} after a \SI{e7}{s} exposure) since it is dominated by regions where $T_\mathrm{surface} \approx T_\mathrm{coolant}$.

% ELMs
It should be noted that specific scenarios like edge localised modes (ELMs) were also not taken into account in this work since their time scale is very short.
ELMs are transient plasma events releasing thermal energy and locally increasing the heat flux at the surface of the monoblock.
MRE simulations by Hu and Hassanein \sidecite{hu_predicting_2015} suggest that a \SI{400}{s} discharge with \SI{1}{Hz} or \SI{10}{Hz} ELMs significantly reduces (77 \%) the inventory in W materials.
However, the modelling of the ELM is simulated by increasing the temperature for a very short time without changing the incident flux of particles that can also be much higher thus balancing the fuel retention reduction.
Another study by Schmid \textit{et al} \sidecite{schmid_diffusion-trapping_2016} also simulated the effect of \SI{1}{Hz} ELMs on fuel retention in W.
The outcome is that \SI{6}{s} of \SI{1}{Hz}-ELMs does not affect significantly the fuel retention, though the temperature excursion in those simulations are smaller than for the one of Hu and Hassanein.
Thus, the effect of ELMs, especially the balance between increase of heat flux, incident energy and particle flux, could either favour or disfavour trapping, diffusion and migration and therefore the overall retention.

% Surface process
In this study the model to link the concentration of mobile particles at the surface (implantation zone) with the exposure condition considers that the particles are implanted in the bulk and that the recombination coefficient is very high since many uncertainties concerning the recombination coefficient are yet to be lifted.
However, if an exothermic process is considered as in \sidecite{ogorodnikova_recombination_2019}, this should have low influence since recombination is very quick at a temperature close to that of the coolant.

On the other hand, experimental results \sidecite{t_hoen_strongly_2013} suggest that for ion energy below \SI{5}{eV/H}, typical of detached plasma as the one treated in the previous section, the surface process can be important and limits the uptake of hydrogen, i.e. the adsorption on the surface and the further absorption from surface to bulk could be the limiting process for the growth of $c_\mathrm{surface}$ during such exposure.
The evolution of $c_\mathrm{surface}$ to the exposure condition for that range of energy (and therefore the inventory) would then be different.
The advantage of the presented method is that taking into account such process is realtively easy as no expensive simulations are needed.
One would only need to modify the model giving $c_\mathrm{surface}$
as a function of $(E_\mathrm{inc},\varphi_\mathrm{inc})$ to include the different surface processes.
To this end, one can use kinetic surface models \sidecite{hodille_retention_2017, zaloznik_deuterium_2017, pecovnik_influence_2019, guterl_effects_2019}.

% traps
Trap properties have a great impact on the inventory.
In this study, a homogeneous trap distribution is assumed for simplification purposes.
A more thorough study could investigate the influence on trap distribution, energy and density.
Trap properties might also vary along the divertor based on exposure conditions.
Moreover the impact of neutrons must be assessed as neutron-induced traps have a high detrapping energy.


% Helium

Finally, helium implantation in the materials and bubble formation could modify the hydrogen transport in monoblocks.

% Yann doesn't like Summaries, we'll see what the others say
% \subsection{Summary}
% ITER-like monoblocks have been studied using a novel method in order to estimate the hydrogen content as a function of exposure conditions such as the implanted particle flux, the ion energy, the heat load and the monoblock surface temperature.
% Several hundred data points have been simulated with FESTIM and analysed to estimate the hydrogen inventory in monoblocks for any input conditions using a Gaussian regression process, a machine learning algorithm which calculates the confidence interval for each point.
% Thanks to this relation, one can easily estimate hydrogen content in the whole divertor without having to run all the simulations.
% An application has been made based on the output from a SOLPS calculation of exposure conditions distribution on the ITER divertor and shows that for these conditions the inventory could reach \SI{e20}{H} per monoblock near strike points after a \SI{e7}{s} exposure.
% The total hydrogen content in ITER divertor is estimated to be \SI{8}{g} which is well below the inner-vessel safety limit of \SI{1}{kg}.

% This behaviour law will be used in \refch{Divertor inventory estimation} to estimate the hydrogen inventory of WEST and ITER divertors.

\section{Edge effects} \label{3D edge effects}
So far, only 2D monoblocks simulations were run, assuming an infinite thickness (or assuming no desorption from the poloidal gaps).
The goal of this section is to assess the influence of 3D edge effects on the monoblocks simulation results.
It will be shown that the error induced by 2D assumption decreases for thick monoblocks.

\subsection{Methodology}

The DEMO monoblock geometry differs slightly from the ITER geometry (see Figure \ref{fig: geometry DEMO monoblock}) but the general concept is the same, meaning the observations made in this Section are valid for the ITER geometry.


\begin{figure}
    \centering
        \begin{overpic}[width=\linewidth]{Figures/Chapter3/monoblocks/3D_monoblocks/sketch.pdf}
            \put(21, 15){\SI{23}{mm}}
            \put(51, 50){\SI{25}{mm}}
            % \put(21, 15){\SI{13.5}{mm}}
            \put(21, 56){ \diameter \SI{12}{mm}}
            \put(21, 63){ \diameter \SI{15}{mm}}
            \put(21, 68){ \diameter \SI{17}{mm}}
            \put(10, 70){\large$\Gamma_\mathrm{top}$}
            \put(0, 60){\large$\Gamma_\mathrm{lateral}$}
            \put(43, 60){\large$\Gamma_\mathrm{lateral}$}
            \put(21, 45){\large$\Gamma_\mathrm{coolant}$}
            \put(66, 71){\SI{4}{mm}}
            \put(66, 50){\SI{5}{mm}}
        \end{overpic}
    \caption{3D geometry of the DEMO monoblock used for the simulations showing W armour \cruleme[grey]{0.3cm}{0.3cm}, Cu interlayer \cruleme[orange]{0.3cm}{0.3cm}, CuCrZr alloy cooling pipe  \cruleme[yellow]{0.3cm}{0.3cm}.}
    \label{fig: geometry DEMO monoblock}
\end{figure}

The boundary conditions for the steady state heat transfer problem are the same as for the 2D case (see Equation \ref{eq: bc thermal DEMO monoblock}).
The boundary conditions for the transient H transport problem are similar (see Equations \ref{eq: bc H transport DEMO monoblock}).
A non-homogeneous mobile concentration is assumed at the plasma exposed surface to simulate an implanted source of particles (see Section \ref{triangle model}).
Depending on the simulation case (with or without desorption on the gaps), the other external surfaces (except the cooling surfaces) will either be insulated or an instantaneous recombination will be assumed.


\begin{subequations}
    \begin{align}
    -\lambda \vec{\nabla} T \cdot \vec{n} &=\varphi_\mathrm{heat} \quad  &\text { on } \Gamma_\mathrm{top}\\
    -\lambda \vec{\nabla} T\cdot \vec{n} &= -h \cdot \left(T_\mathrm{coolant} - T\right)\quad &\text { on } \Gamma_\mathrm{coolant}
    \end{align}
    \label{eq: bc thermal DEMO monoblock}
\end{subequations}
where $\varphi_\mathrm{heat}=\SI{10}{MW}$, $h=\SI{7e4}{W.m^{-2}.K^{-1}}$ is the heat exchange coefficient and $T_\mathrm{coolant} = \SI{323}{K}$ is the coolant temperature.

\begin{subequations}
    \begin{align}
    c_\mathrm{m} &=  \frac{\varphi_\mathrm{imp} R_p}{D} \quad &\text { on } \Gamma_\mathrm{top}\\
    -D \vec{\nabla} c_\mathrm{m} \cdot \vec{n} &= K_\mathrm{CuCrZr} \cdot c_\mathrm{m}^{2} \quad &\text { on } \Gamma_\mathrm{coolant} \\
    c_\mathrm{m} &=  0 \quad \text{or} \quad -D \vec{\nabla} c_\mathrm{m} \cdot \vec{n} = 0 &\text { on } \Gamma_\mathrm{lateral} \text{  and  } \Gamma_\mathrm{pipe}
    \end{align}
    \label{eq: bc H transport DEMO monoblock}
\end{subequations}
where $\varphi_\mathrm{imp} = \SI{1.6e22}{H.m^{-2}.s^{-1}}$ is the implanted particle flux, $R_p = \SI{1e-9}{m}$ is the particle implantation depth, $K_\mathrm{CuCrZr}=2.9\times 10^{-14}\exp{-1.92/(k_B T)}$ is the H recombination coefficient in CuCrZr expressed in \si{m^4.s^{-1}}.

Two intrinsic traps were set in W, one trap in the Cu interlayer and two traps in the CuCrZr cooling pipe (see Table \ref{tab:traps monoblock DEMO}).

\begin{table*}
    \centering
    \begin{tabular}{L{1.5cm} L{1.5cm} R{1.6cm} R{1.1cm} R{1.6cm} R{1.1cm} R{2cm}}
         & Material & $k_0 (\si{m^3.s^{-1}})$ &  $E_k (\si{eV})$ & $p_0 (\si{s^{-1}})$ & $E_p (\si{eV})$ & $n_i (\si{at.fr.})$ \\
        \hline
        \\
        Trap 1 & W & $9.0 \times 10^{-17}$ & 0.39 & $1 \times 10^{13}$& 0.78 & $1.0 \times 10^{-3}$ \\
        \\
       Trap 2 & W & $9.0 \times 10^{-17}$ & 0.39 & $1 \times 10^{13}$& 1.00 & $4.0 \times 10^{-4}$ \\
        \\
        Trap 3 & Cu & $6.0 \times 10^{-17}$ & 0.39 & $8.0 \times 10^{13}$ & 0.50 &$5.0 \times 10^{-5}$\\
        \\
        Trap 4 & CuCrZr & $1.2\times 10^{-16}$ & 0.42 & $8.0 \times 10^{13}$ & 0.50 &$5.0 \times 10^{-5}$\\
        \\
        Trap 5 & CuCrZr & $1.2\times 10^{-16}$ & 0.42 & $8.0 \times 10^{13}$ & 0.83 &$4.0 \times 10^{-2}$\\
        \\
    \end{tabular}
    \caption{Traps properties used in the 3D DEMO monoblocks simulations.}
    \label{tab:traps monoblock DEMO}
\end{table*}

Transient simulations up to \SI{e7}{s} were run.

\subsection{Standard case}

FESTIM simulations were run with and without desorption on the gaps.

% Temperature field

The temperature field obtained was very similar to the 2D case (see Figure \ref{fig: T field 3D monoblock}) with a top surface temperature of approximately \SI{1200}{K}.

\begin{figure} [h]
    \centering
    \includegraphics[width=\linewidth]{Figures/Chapter3/monoblocks/3D_monoblocks/temperature_3D_monoblock.png}
    \caption{Temperature field of the 3D DEMO monoblock.}
    \label{fig: T field 3D monoblock}
\end{figure}


% Retention fields
As expected, a higher retention was observed in the case without desorption (see Figure \ref{fig:retention fields 3D monoblocks}).
This is explained by the surface losses.
The maximum retention for the case with desorption is three orders of magnitude lower than that of the case without desorption.
In both cases, the higher retention was found to be in the CuCrZr cooling pipe.
This is consistent with the obsverations made previously.

\begin{figure} [h]
    \centering
    \begin{subfigure}{\linewidth}
        \centering
        \includegraphics[width=\linewidth]{Figures/Chapter3/monoblocks/3D_monoblocks/MB 3D desorption.png}
        \caption{Instantaneous recombination on the gaps.}
    \end{subfigure}
    \begin{subfigure}{\linewidth}
        \centering
        \includegraphics[width=\linewidth]{Figures/Chapter3/monoblocks/3D_monoblocks/MB 3D no desorption.png}
        \caption{No desorption on the gaps.}
    \end{subfigure}
    \caption{Retention fields of the DEMO monoblock after \SI{e7}{s} of continuous exposure with or without recombination on the gaps showing the isometric view (left), a central slice (top right) and a central clip showing the cooling pipe (bottom right). Note that the colour bars are different.}
    \label{fig:retention fields 3D monoblocks}
\end{figure}

% Inventory
The total H inventory in the monoblock was also between one and three orders of magnitude lower in the case with desorption (see Figure \ref{fig: inventory vs time DEMO monoblock}).
This difference increased with the exposure time.
Moreover, the steady state was reached way earlier for the case with desorption whereas the inventory kept increasing after \SI{e7}{s} for the insulated case.
This means that not taking desorption from the gaps into account in 2D simulations is a conservative assumption in terms of H inventory.
The simulations performed in Section \ref{influence of exposure conditions} then overestimate the monoblock H inventory.

\begin{figure} [h]
    \centering
    \includegraphics[width=\linewidth]{Figures/Chapter3/monoblocks/3D_monoblocks/inventory.pdf}
    \caption{Temporal evolution of the monoblock inventory.}
    \label{fig: inventory vs time DEMO monoblock}
\end{figure}

% fluxes
However, the outgassing fluxes from the monoblock gaps cannot be estimated with 2D (or 1D) simulations.
These fluxes were found to be five orders of magnitude higher than the permeation flux to the coolant: the particle flux towards the vacuum vessel was approximately \SI{e12}{H.s^{-1}} whereas the flux towards the cooling channel was below \SI{e8}{H.s^{-1}} (see Figure \ref{fig: fluxes DEMO monoblock}).
These fluxes can be compared to the retrodesorbed flux (\textit{ie} the flux of implanted particles that diffuse back to the exposed surface).
According to Equation \ref{eq:flux balance}, the value of this flux is equal to $\varphi_\mathrm{imp} \times \SI{23}{mm} \times \SI{4}{mm} = \SI{1.5e18}{H.s^{-1}}$.
The values of the outgassing fluxes from both the gaps and the cooling surface are therefore orders of magnitude lower than that of the retrodesorbed flux.
This means 3D edge effects will not affect previous results regarding the outgassing to the vessel.
They will however impact the value of the contamination flux towards the coolant as assuming an instantaneous recombination on the gaps will lead to way less particles reaching the cooling surface and therefore a lower flux.

\begin{figure} [h]
    \centering
    \includegraphics[width=\linewidth]{Figures/Chapter3/monoblocks/3D_monoblocks/fluxes.pdf}
    \caption{Temporal evolution of outgassing fluxes for the case with desorption from the gaps.}
    \label{fig: fluxes DEMO monoblock}
\end{figure}

\subsection{Influence of the monoblock thickness}

The same simulations were run with different monoblock thicknesses (3, 4 and \SI{14}{mm}).
In order to investigate the influence of the recombination on the gaps, the ratio $\mathrm{inv}_\mathrm{desorption} / \mathrm{inv}_\mathrm{no desorption}$ (monoblock inventories with or without desorption) was computed for each simulation case.
The closer the ratio is to unity, the less significant the desorption.

Increasing the monoblock thickness lead to an increase of the ratio (see Figure \ref{fig: ratio 3D thickness monoblock}).
For the smallest thickness (\SI{3}{mm}), the ratio $\mathrm{inv}_\mathrm{desorption} / \mathrm{inv}_\mathrm{no desorption}$ was approximately \SI{10}{\%} at \SI{10000}{s}.
It decreased down to $3\times 10^{-4}$ after \SI{1e7}{s} of exposure.
The same behaviour was observed for all thicknesses though the decrease was less significant for higher thicknesses.

\begin{figure} [h]
    \centering
    \includegraphics[width=\linewidth]{Figures/Chapter3/monoblocks/3D_monoblocks/influence_of_thickness.pdf}
    \caption{Temporal evolution of the ratio $\mathrm{inv}_\mathrm{desorption} / \mathrm{inv}_\mathrm{no desorption}$ for several thicknesses.}
    \label{fig: ratio 3D thickness monoblock}
\end{figure}

\subsection{Summary}
The influence of H desorption on poloidal and toroidal gaps between monoblocks was investigated.
The DEMO baseline geometry (based on an ITER-like concept) was simulated with FESTIM.
It was shown that taking into account recombination on these gaps could drastically decrease the monoblock H inventory (by several orders of magnitude).
Said otherwise, this makes the 2D assumption for monoblocks less valid for 1) thin monoblocks 2) long exposure times.
Indeed, a 2D assumption implies assuming an infinite monoblock thickness (which is totally unrealistic for the ITER and DEMO designs).
However, this assumption is conservative since it will only overestimate the H inventory.
Moreover, performing 3D simulations is more computationaly expensive than 2D simulations.

Regarding the outgassing flux assessment, taking into acount


\section{Summary}
H transport in ITER-like monoblocks was simulated with FESTIM.
Several aspects of the simulations were studied.

It was shown that the choice of interface conditions (continuity of chemical potential or continuity of mobile concentration) had low impact on the monoblock inventory.
Depending on the boundary condition used at the cooling surface of the monoblock, it could however have an impact on the permeation flux.

The effect of loading cycles was also investigated.
Between plasma pulses, a zone with higher retention appears near the plasma exposed surface due to the temperature variation.
These modifications of the retention fields vanish as soon as the next cycle starts again and this zone is heated up again.
This means that cycling has no effect on the global retention field and that cycles can safely be concatenated (continuous exposure) to simulate H transport.

A parametric study was then performed in order to assess the influence of exposure conditions (surface concentration and surface temperature).
A 2D behaviour law was obtained correlating exposure conditions to the monoblock inventory.
This law will be extremely useful to estimate H retention in divertors since not all monoblocks will be exposed to the same exposure.

Finally, 3D simulations of DEMO ITER-like monoblocks have been run.
The desorption on the poloidal and toroidal gaps drastically reduced the inventory while increasing the outgassing flux to the vacuum chamber.
The permeation flux to the coolant decreased.
This means that 2D simulations are conservative in terms of H inventory.
This however undersestimates outgassing fluxes.
Future work should involve refining the behaviour law obtained in Section \ref{influence of exposure conditions} by running new 3D simulations and compute not only the H inventory but also the total outgassing.

\setchapterimage{west_div_roux_cropped}
\setchapterpreamble[u]{\margintoc}
\chapter{Divertor inventory estimation}\label{Chapter4}\labch{Chapter4}
\labch{Divertor inventory estimation}

% \section{Introduction}

This Chapter focusses on the estimation of the \gls{H} \gls{inventory} in the \glspl{divertor} of \acrshort{west} and \gls{iter}.
This estimation relies on the \gls{monoblock} behaviour law computed in \refch{Chapter3}.
This behaviour law allows rapid evaluations of the \glspl{monoblock} \gls{H} \gls{inventory} for any exposure condition.
Inputs are taken from \gls{soledge}-EIRENE \cite{bufferand_three-dimensional_2019} and \gls{solps} \cite{kaveeva_solps-iter_2020} plasma simulations.
The influence of several control parameters is investigated: the input power (i.e.\ how much heating power is injected in the plasma), the gas puffing rate, and the \gls{divertor} pressure of neutral particles in \gls{iter}.
Gas puffing is used in most \glspl{tokamak} to locally increase the \gls{plasma} density \cite{zweben_effect_2014}.
One of the advantages of gas puffing is a better coupling of the \gls{icrf} heating with the plasma \cite{zhang_scrape-off_2019}.

\section{Methodology}
To make use of the \gls{monoblock} \gls{inventory} behaviour law, distribution of surface concentrations and surface temperatures along the \glspl{divertor} will be required.
They will be converted from plasma simulations outputs.

\subsection{Plasma simulations}

\begin{figure}[h!]
    \centering
    \includegraphics[width=0.95\linewidth]{Figures/Chapter4/coordinates.pdf}
    \caption{Poloidal cross section of \gls{west} and \gls{iter} showing the \glspl{divertor} in red.}
    \labfig{reactors}
\end{figure}
This Section describes the parameters of the plasma simulations.
These simulations were run with \gls{soledge}-EIRENE for \gls{west} and \gls{solps} for \gls{iter}.
In a nutshell, these codes solve, for each species (ions and electrons), the particle density, velocity and temperature.
The equations at stake are comparable to the Navier-Stokes equations coupled to the heat equation and interactions with the electromagnetic fields in the \gls{plasma} \sidecite{bufferand_numerical_2015}.
Subsequently, the incident particle fluxes, heat fluxes and particle energy can be calculated along the \gls{tokamak} wall.
For the \gls{soledge}-EIRENE runs, the puffing rate and the input power were used as control parameters.
For \gls{solps} calculation, the \gls{divertor} neutral pressure is the control parameter.

\subsubsection{\gls{soledge}-EIRENE runs}
The Lower-Single-Null magnetic configuration (i.e.\ a signle \gls{x-point} in the lower part of the vacuum vessel) used for the 2D simulations in \gls{soledge}-EIRENE transport code (v588.165) are based on the experimental \gls{west} \gls{plasma} discharge \#54903 at $T_\mathrm{flat-top} = \SI{8}{s}$ (see \reffig{reactors}).
In order to get as many \gls{divertor} conditions as possible, the puffing rate was varied from \SI{4.5e20}{molecule.s^{-1}} to \SI{4.72e21}{molecule.s^{-1}} and the input power from \SI{0.449}{MW} to \SI{2.5}{MW}.
The setup parameters of the simulation are listed in \reftab{soledge parameters}.
$R_\mathrm{wall}$ is the recycling coefficient of main chamber wall, $R_\mathrm{pump}$ is the recycling coefficient of the pump, $D$ is the cross-field mass diffusivity perpendicular to the flux surface, $\nu$ is the momentum diffusivity, $\chi_e$ and $\chi_i$ are the energy diffusivity for electrons and ions, respectively.
While the value of these coefficients is required for the sake of reproducibility, their detailed description \sidecite{ciraolo_first_2019} is outside the scope of this research.
The gas puff position is set inside the \gls{private flux region} (i.e.\ the region between the two \glspl{strike point}) and the pump position is set under the baffle.

\begin{table}[!ht]
    \centering
    \caption{Setup parameters used in the \gls{soledge} simulations.}
    \begin{tabular}{L{0.4\linewidth}  R{0.4\linewidth}}
    \hline \\
    Plasma composition & Deuterium, no impurity \\
    \\
    Recycling coefficients &  $R_\mathrm{wall} = 0.99$ \\
     & $R_\mathrm{pump} = 0.95$ \\
    \\
    SOL input power & from \SI{0.449}{MW} to \SI{2.5}{MW} \\
    \\
    Gas puffing rate & from \SI{4.5e20}{molecule.s^{-1}} to \SI{4.72e21}{molecule.s^{-1}} \\
    \\
    Drifts & - \\
    \\
    Transport coefficients & $D = \SI{0.3}{m^2.s^{-1}}$ \\
     & $\nu = \SI{0.3}{m^2.s^{-1}}$ \\
     & $\chi_e = \chi_i = \SI{1.0}{m^2.s^{-1}}$ \\
    \end{tabular}
    \labtab{soledge parameters}
\end{table}


\subsubsection{\gls{solps} runs}
Several \gls{iter} cases were taken from \sidecite{pitts_physics_2019} with \gls{divertor} neutral pressures varying from \SI{1.8}{Pa} to \SI{11.2}{Pa}.
These \gls{solps} \sidecite{kaveeva_solps-iter_2020} scenarios can be found in the \gls{iter} \gls{imas} database \sidecite{imbeaux_design_2015, park_assessment_2020}.
The nine simulations used in this work are labelled 122396, 122397, 122398, 122399, 122400, 122401, 122402, 122403 and 122404.
These have been run in baseline burning plasma conditions ($Q=10$ with \SI{50}{MW} of input power).


\begin{figure}[h!]
    \centering
    \begin{overpic}[width=\linewidth]{Figures/Chapter4/example.pdf}
        % \linethickness{2pt}
        \thicklines
        \put(55,36){\color{black}\vector(-1, 0){30}}
        \put(25,75){\color{black}\vector(0, -1){30}}
        \put(35,45){\color{black}\vector(1, 1){30}}
    \end{overpic}

    \caption{Method of WEST \gls{divertor} \gls{H} \gls{inventory} estimation based on the surface concentration, the surface temperature and the behaviour law obtained in \refch{Chapter3}.}
    \labfig{behaviour law example}
\end{figure}

\subsection{Estimation of exposure conditions}

\begin{figure*}[h]
    \centering
    \begin{subfigure}{0.5\linewidth}
        \includegraphics[width=\linewidth]{Figures/Chapter4/implantation_range.pdf}
        \caption{Implantation range $R_p$.}
        \labfig{implantation range vs energy}
    \end{subfigure}%
    \begin{subfigure}{0.5\linewidth}                          
        \includegraphics[width=\linewidth]{Figures/Chapter4/reflection_coeff.pdf}
        \caption{Reflection coefficient $r$.}
        \labfig{reflection coeff vs energy}
    \end{subfigure}
    \caption{Evolution of the implantation range and the reflection coefficient as a function of incident energy $E$ and angle of incidence.}
\end{figure*}

According to the behaviour law obtained in \refch{Chapter3}, the temporal evolution of the \gls{H} \gls{inventory} along the \glspl{divertor} can be estimated from the surface concentration of mobile hydrogen and surface temperature (see \reffig{behaviour law example}).
% This inventory distribution can then be projected onto the whole divertor geometry for better visualisation (see \reffig{top view}).

The distribution of the exposure conditions (angles of incidence, particles energies, particles fluxes and heat flux) are produced by \gls{soledge}/\gls{solps} along the \glspl{divertor} of \gls{west} and \gls{iter} (see \reffig{reactors} and \reffig{behaviour law example}).
These exposure conditions are converted into distributions of surface temperature $T_\mathrm{surface}$ and surface hydrogen concentration $c_\mathrm{surface}$ by \refeq{thermal behaviour law} and \refeq{c_surface}.

Note: see \refsec{monoblock thermal behaviour} for more details on the \gls{monoblock} thermal behaviour.
\begin{equation}
    c_\mathrm{surface} = c_\mathrm{surface, \, ions} + c_\mathrm{surface, \, atoms}
    \labeq{c_surface}
\end{equation}
$c_\mathrm{surface, \, ions}$ and $c_\mathrm{surface, \, atoms}$ are the contributions of the ions and atoms to the surface hydrogen concentration.
They can be expressed as:
\begin{equation}
    c_\mathrm{surface, \, i} = \frac{R_{p, \mathrm{i}} \ \varphi_\mathrm{imp, \,i}}{D(T_\mathrm{surface})}
\end{equation}
where $R_{p, i}$ is the implantation depth in \si{m}, $\varphi_{\mathrm{imp}, \,i}$ is the implanted particles flux in \si{m^{-2}.s^{-1}} and $D$ is the \gls{H} diffusion coefficient in \si{m^{2}.s^{-1}} (see \refsec{triangle model}).

Finally, the implanted flux can be expressed as:
\begin{equation}
    \varphi_{\mathrm{imp}, \,i} = (1 - r_\mathrm{i}) \, \varphi_{\mathrm{incident} \, i}
\end{equation}
where $r_i$ is the reflection coefficient and, $\varphi_{\mathrm{incident} \, i}$ is the incident particle flux expressed in \si{m^{-2}.s^{-1}}.

The implantation range $R_p$ and the reflection coefficient $r$ depend on the incident energy and angle of incidence of particles.
These relations can be obtained from \gls{srim} \sidecite{ziegler_srim_2010} simulations (see \reffig{implantation range vs energy}).
It was found that the angle of incidence had low influence on the implantation range.
$R_p$ can therefore be expressed as a function of the incident energy only:
\begin{equation}
    R_p = 1.9\times 10^{-10} E ^{0.59}
    \labeq{implantation range}
\end{equation}
where $E$ is the incident energy in \si{eV}.

The evolution of the reflection coefficient $r$ can also be estimated with \gls{srim}.
The reflection coefficient varies from around 0.5 at \SI{0}{^\circ} to 0.8 at \SI{80}{^\circ} (see \reffig{reflection coeff vs energy}).
According to \sidecite{park_assessment_2020}, the incident angles for ions and atoms were assumed to be \SI{60}{^\circ} and \SI{45}{^\circ}, respectively.
It should be noted that since \gls{srim} is based on the binary collision approximation, values around \SI{10}{eV} might not be fully valid.

All of these steps have been automated and packaged into a tool called divHretention.
divHretention can directly interpret \gls{solps}/\gls{soledge} data and produce a distribution of \gls{monoblock} \gls{inventory} as in \reffig{behaviour law example}.
\marginnote{
The source-code of the tool is under version control and openly available via GitHub under a MIT licence \cite{delaporte-mathurin_irfmdivhretention_2021}.
The divHretention python package is distributed via PyPi \cite{delaporte-mathurin_divhretention_nodate}.
Moreover, all the results obtained in this Chapter can be reproduced with the scripts available at \url{https://github.com/RemDelaporteMathurin/divHretention-Nucl.Fusion-2021}.
}

\section{ITER results}

\begin{figure*}[h!]
    \captionsetup[subfigure]{format=plain,singlelinecheck=true}  % needed to center the subcaptions
    \centering
    \begin{subfigure}{0.42\linewidth}
        \includegraphics[width=\linewidth]{Figures/Chapter4/ITER/inventory_along_inner_divertor.pdf}
        \caption{Inner Vertical Target.}
    \end{subfigure}%
    \begin{subfigure}{0.58\linewidth}
        \includegraphics[width=\linewidth]{Figures/Chapter4/ITER/inventory_along_outer_divertor.pdf}
        \caption{Outer Vertical Target.}
        \labfig{distrib outer target}
    \end{subfigure}
    \caption{Surface temperature, surface concentration and \gls{inventory} per unit thickness along the \gls{iter} \gls{divertor} with neutral pressures varying from \SI{2}{Pa} to \SI{11}{Pa}. The area corresponds to the 95\% confidence interval.}
\end{figure*}


\begin{figure}[h]
    \centering
    \includegraphics[width=\linewidth]{Figures/Chapter4/ITER/inventory_vs_divertor_pressure.pdf}
    \caption{Hydrogen \gls{inventory} in the \gls{iter} \gls{divertor} as a function of neutral pressure after \SI{e7}{s} of exposure (approximately 25 000 discharges).}
    \labfig{inventory vs neutral pressure}
\end{figure}


\begin{figure}[h!]
    \centering
    \begin{subfigure}{\linewidth}
        \includegraphics[width=\linewidth]{Figures/Chapter4/ITER/inventory_at_strike_points.pdf}
        \caption{\Gls{inventory} per unit thickness after \SI{e7}{s} of exposure (approximately 25 000 discharges). Area corresponds to the 95\% confidence interval.}
        \labfig{local inventory neutral pressure}
    \end{subfigure}
    \begin{subfigure}{\linewidth}
        \includegraphics[width=\linewidth]{Figures/Chapter4/ITER/ratio_ions_atoms.pdf}
        \caption{Contribution of ions to the surface concentration of H.}
        \labfig{ion contribution neutral pressure}
    \end{subfigure}%
    \caption{H retention at the \glspl{strike point} (defined as maximum temperature) as a function of the \gls{divertor} neutral pressure.}
\end{figure}

\begin{figure}[h!]
    \centering
    \includegraphics[width=\linewidth]{Figures/Chapter4/ITER/inventory_vs_time.pdf}
    \caption{Evolution of the \gls{H} \gls{inventory} of the \gls{iter} \gls{divertor} with the number of \SI{400}{s} discharges.}
    \labfig{iter vs time}
\end{figure}


% peak temperature
Peak temperatures at \glspl{strike point} increased when decreasing the \gls{divertor} neutral pressure (see \reffig{distrib outer target}).
The peak temperature at the outer strike point reached \SI{2000}{K} at \SI{2}{Pa} and more than \SI{1000}{K} at the inner strike point, which is in accordance with the results obtained by Pitts et al.\ \sidecite{pitts_physics_2019}.

% global inventory
The \gls{inventory} in the whole \gls{divertor} is computed as follows:
\begin{equation}
    \mathrm{inv_{divertor}} = N_\mathrm{cassettes} \cdot (\mathrm{inv_{IVT}} + \mathrm{inv_{OVT}})
\end{equation}
with $N_\mathrm{cassettes}=54$ the number of cassettes, $\mathrm{inv_{IVT}}$ and $\mathrm{inv_{OVT}}$ the total \gls{inventory} in the \gls{ivt} and \gls{ovt} respectively (in one cassette).

\begin{align}
    \mathrm{inv_{IVT}} &= N_\mathrm{PFU-IVT} \cdot \int_\mathrm{IVT} \mathrm{inv_{MB}}(x)\: dx \\
    \mathrm{inv_{OVT}} &= N_\mathrm{PFU-OVT} \cdot \int_\mathrm{OVT} \mathrm{inv_{MB}}(x)\: dx
\end{align}
$N_\mathrm{PFU-IVT}=16$ and $N_\mathrm{PFU-OVT}=22$ the number of plasma facing units per cassette in the inner and outer targets respectively (see \refsec{divertor section}), $\mathrm{inv_{MB}}$ is the \gls{monoblock} \gls{inventory} per unit thickness and $x$ the distance along the targets.
Here, $\int_\mathrm{OVT} \mathrm{inv_{MB}}(x) dx$ corresponds to the area of the profile shown on \reffig{distrib outer target}.

The \gls{inventory} in the outer target was found to be nearly twice that of the inner target.
This is largely explained by the larger number of plasma facing units in the outer target and therefore a greater exposed surface.
The global \gls{inventory} increased with the \gls{divertor} neutral pressure and a roll-over is observed above \SI{7}{Pa} (see \reffig{inventory vs neutral pressure}).
This roll-over is consistent with the results obtained in \sidecite{pitts_physics_2019}.
The \gls{inventory} increase was found to be more important in the outer vertical target.
This was explained by the fact that the plasma is more detached at the inner target.
Therefore the surface temperature reduction is more significant in the outer vertical target and the surface concentration is increased (see \reffig{distrib outer target}).

The maximum \gls{inventory} was found at around \SI{7}{Pa} and was approximately \SI{14}{g} of H, which is well below the \gls{iter} in-vessel safety limit of tritium (\SI{1}{kg}), especially considering only half of this quantity will be tritium.
This is especially true considering that this was for a very long exposure time of \SI{e7}{s}, which corresponds to 25 000 pulses of \SI{400}{s}.


% local inventories
The inventory at the inner \gls{strike point} is constant from \SI{4}{Pa} whereas the inventory at the outer \gls{strike point} globally increases with the \gls{divertor} neutral pressure (see \reffig{local inventory neutral pressure}).
The contribution of ions to the surface concentration at the inner strike point is around 50 \% and tends to decrease with increasing neutral pressure (see \reffig{ion contribution neutral pressure}).
At low \gls{divertor} neutral pressure, the contribution of ions at the outer strike point is around 90 \% and tends to decrease with increasing neutral pressure.
This can be explained by the fact that in both inner and outer targets, the integrated flux of ions decreases with increasing neutral pressure whereas the integrated flux of atoms increases, leading to a greater proportion of neutral particles.

% temporal evolution
For all \gls{divertor} neutral pressures, the temporal evolution of the \gls{divertor} \gls{inventory} is approximately the same (see \reffig{iter vs time}).
The \gls{inventory} is plotted as a function of the number of \gls{iter} discharges (see \reffig{plasma cycle}).
The additional \gls{inventory} per \SI{400}{s} discharge was found to decrease with time.
Past 300 discharges, the additional \gls{inventory} per discharge decreases with the number of discharges.
The maximum is around \SI{5}{mg/discharge} between 30 and 100 discharges.

\section{WEST results}

All the computations have been made for very long exposure times (\SI{e7}{s}) in order to better visualise trends.
Even though cycling can have an effect on \gls{H} outgassing at the \gls{monoblock} plasma facing surface, it was shown in \refsec{influence of cycling} that the evolution of the \gls{monoblock} \gls{inventory} with the fluence was not affected.
Moreover, it can be shown that the \glspl{divertor} inventories evolve with a power law dependence of time.

\subsection{Influence of the input power}

The input power was varied between \SI{0.49}{MW} and \SI{2.0}{MW}.
Two puffing rate values were used: \SI{2.5e21}{molecule.s^{-1}} and \SI{4.4e21}{molecule.s^{-1}}.

\begin{figure}[h]
    \centering
    \includegraphics[width=\linewidth]{Figures/Chapter4/WEST/inventory_along_divertor_input_power.pdf}
    \caption{Distribution of surface temperature $T_\mathrm{surface}$, surface concentration $c_\mathrm{surface}$ and \gls{inventory} per unit thickness along the \gls{west} \gls{divertor} with input powers varying from \SI{0.49}{MW} to \SI{2.0}{MW} with a puffing rate of \SI{2.5e21}{molecule.s^{-1}}.}
    \labfig{divertor distr power scan}
\end{figure}

\begin{figure}[h]
    \centering
    \begin{subfigure}{\linewidth}
        \includegraphics[width=\linewidth]{Figures/Chapter4/WEST/inventory_at_sps_and_private_zone_vs_input_power.pdf}
        \caption{\Gls{inventory} per unit thickness after \SI{e7}{s} of exposure. The area corresponds to the 95\% confidence interval.}
        \labfig{local retention vs input power}
    \end{subfigure}
    \begin{subfigure}{\linewidth}                          
        \includegraphics[width=\linewidth]{Figures/Chapter4/WEST/ions_ratio_vs_input_power.pdf}
        \caption{Contribution of ions to the surface concentration of H.}
        \labfig{ion ration vs input power}
    \end{subfigure}%
    \caption{H \gls{inventory} at the inner and outer \glspl{strike point} (\gls{isp} and \gls{osp}) and in the \gls{private flux region} as a function of the input power with a puffing rate of \SI{2.5e21}{molecule.s^{-1}}.}
\end{figure}

\begin{figure}
    \centering
    \includegraphics[width=\linewidth]{Figures/Chapter4/WEST/inventory_vs_input_power.pdf}
    \caption{Evolution of the \gls{west} \gls{divertor} \gls{inventory} as a function of input power for several puffing rates.}
    \labfig{inventory vs input power}
\end{figure}

\begin{figure}[h]
    \centering
    \includegraphics[width=\linewidth]{Figures/Chapter4/WEST/inventory_vs_time_west.pdf}
    \caption{Temporal evolution of \gls{pfuLabel} inventories for different values of puffing rate (left) and input power (right).}
    \labfig{temporal evolution west}
\end{figure}

% local inventories
The maximum \gls{retention} was found to be located at the \glspl{strike point} (see \reffig{divertor distr power scan}).
The \gls{retention} at the outer strike point was higher than at the inner strike point.
The \gls{retention} at the \glspl{strike point} was found to increase with the input power whereas it slightly decreased in the \gls{private flux region} (see \reffig{local retention vs input power}).
This was explained by an attachment of the \gls{plasma} decreasing the particle flux in the \gls{private flux region}.
Since the surface temperature is constant, this leads to a decrease in the surface concentration of hydrogen as seen on \reffig{divertor distr power scan}.
On the other hand, the increasing temperature at the \glspl{strike point} only enhanced the \gls{diffusion} process while remaining low enough so that hydrogen could get trapped.

The total \gls{inventory} in the \gls{west} \gls{divertor} is computed as follows:
\begin{equation}
    \mathrm{inv}_\mathrm{divertor} = N_\mathrm{PFU} \cdot \int \mathrm{inv}_\mathrm{MB}(x)\: dx
    \labeq{inventory WEST}
\end{equation}
where $N_\mathrm{PFU} = 480$ is the number of \gls{pfuLabel} in \gls{west}, $\mathrm{inv}_\mathrm{MB}$ is the \gls{inventory} per unit thickness in \si{H.m^{-1}} (see \reffig{divertor distr power scan}) and $x$ the distance along the target in \si{m}.

The \gls{divertor} \gls{inventory} increased with the input power (see \reffig{inventory vs input power}) and evolved as the power 0.3 of the input power.
The maximum \gls{divertor} \gls{inventory} was \SI{8.8e23}{H} at \SI{2.0}{MW} of input power.
This value of input power is still relatively low.
Increasing the puffing rate lead to an increase in the \gls{inventory}.
This will be explained more thoroughly in \refsec{density scan}.

At the \glspl{strike point}, the \gls{retention} is dominated by the ion flux whereas neutrals are dominant in the \gls{private flux region} (see \reffig{ion ration vs input power}).
The contribution of ions at the \glspl{strike point} increased with the input power but remained approximately constant in the \gls{private flux region}.

The \gls{divertor} \gls{inventory} was found to increase as a power law of time (see \reffig{temporal evolution west}).


\subsection{Influence of the puffing rate} \labsec{density scan}

A parametric study on the puffing rate was performed.
The puffing rate was varied between \SI{4.4e20}{molecule.s^{-1}} and \SI{4.7e21}{molecule.s^{-1}}.
The input power was fixed to \SI{0.45}{MW}.

\begin{figure}[h]
    \centering
    \includegraphics[width=\linewidth]{Figures/Chapter4/WEST/inventory_along_divertor.pdf}
    \caption{Distribution of surface temperature $T_\mathrm{surface}$, surface concentration $c_\mathrm{surface}$ and \gls{inventory} per unit thickness along the \gls{west} \gls{divertor} with a puffing rate varying from \SI{4.4e20}{s^{-1}} to \SI{4.7e21}{s^{-1}} with \SI{0.45}{MW} of input power.}
    \labfig{divertor distr density scan}
\end{figure}

\begin{figure}[h]
    \centering
    \includegraphics[width=\linewidth]{Figures/Chapter4/WEST/inventory_vs_puffing_rate.pdf}
    \caption{Evolution of the \gls{west} \gls{divertor} \gls{inventory} as a function of puffing rate.}
    \labfig{inventory vs puffing rate}
\end{figure}

\begin{figure}[h!]
    \centering
    \begin{subfigure}{\linewidth}
        \includegraphics[width=\linewidth]{Figures/Chapter4/WEST/inventory_at_sp_and_private_zone.pdf}
        \caption{\Gls{inventory} per unit thickness after \SI{e7}{s} of exposure. The area corresponds to the 95\% confidence interval.}
        \labfig{local retention vs puffing rate}
    \end{subfigure}
    \begin{subfigure}{\linewidth}
        \includegraphics[width=\linewidth]{Figures/Chapter4/WEST/ion_ratio_at_sp_and_private_zone.pdf}
        \caption{Contribution of ions to the surface concentration of H. \gls{isp} and \gls{osp} stand for Inner Strike Point and Outer Strike Point respectively.}
        \labfig{ion contribution vs puffing rate}
    \end{subfigure}%
    \caption{H \gls{retention} at the \glspl{strike point} and in the \gls{private flux region} as a function of puffing rate with \SI{0.45}{MW} of input power.}
\end{figure}

The maximum \gls{retention} was again located at the \glspl{strike point} for all puffing rates values (see \reffig{divertor distr density scan}).
The \gls{inventory} at the outer strike point was higher than at the inner strike point.
The \gls{inventory} in the \gls{private flux region} was found to increase with the puffing rate whereas it was almost constant at the \glspl{strike point} (see \reffig{local retention vs puffing rate}).
As for the power scan, the ions' contribution to the \gls{inventory} is rather low in the \gls{private flux region} (see \reffig{ion contribution vs puffing rate}).
Moreover, the contribution of ions decreases rapidly at the \glspl{strike point} and represents only half of the surface concentration at \SI{4e21}{molecule.s^{-1}}.

The \gls{inventory} in the whole \gls{west} \gls{divertor} is computed from \refeq{inventory WEST}.
As for the power scan, the \gls{divertor} \gls{inventory} increased as the power 0.2 of the puffing rate (see \reffig{inventory vs puffing rate}).
The maximum \gls{inventory} was found to be \SI{5e23}{H} at \SI{4.7e21}{molecule.s^{-1}}.

The \gls{divertor} \gls{inventory} was found to increase as a power law of time.

\section{Summary}


The \gls{monoblock} behaviour law proposed in \refch{Chapter3} was used to estimate fuel \gls{retention} in the \glspl{divertor} of \gls{west} and ITER.
The impact of key control parameters on the \gls{divertor} \gls{inventory} was studied (the input power, the puffing rate and the \gls{divertor} neutral pressure).

It was shown that the \gls{inventory} in \gls{west} increases as the power $0.3$ of the input power and as the power $0.2$ of the puffing rate.
The \gls{inventory} in the \gls{iter} \gls{divertor} was found to first increase with the neutral pressure up to \SI{7}{Pa} then decrease, though the variation was smoother.
The \gls{inventory} in the outer vertical target of the \gls{iter} \gls{divertor} is twice that of the inner vertical target.
These results were in good agreement with the observations made in \sidecite{pitts_physics_2019}.

However, it should be noted that both machines do not operate in the same regime.
While \gls{west} operates at low input power, \gls{iter} operates at high input power with a high recycling \gls{divertor} .
These differences in the operation regime can explain different trends.

The maximum hydrogen \gls{inventory} in the \gls{iter} \gls{divertor} was approximately \SI{14}{g} after \SI{e7}{s} of continuous plasma exposure (25 000 \gls{iter} discharges), which is well below the in-vessel safety limit (\SI{1}{kg} or \SI{700}{g} excluding the cryopumps).
Note that the total number of discharges in \gls{iter} will be approximately 23 300 \cite{pitts_physics_2019}.
Moreover, since the behaviour law is based on 2D \gls{monoblock} simulations, this value is an upper estimate (see \refsec{influence of dimensionality}).
2D simulations are indeed conservative in terms of \gls{inventory} (see \refsec{influence of dimensionality}).

The underlying \gls{monoblock} model has also a few limitations, as detailed in \refch{Chapter3}.
First, the set of trapping parameters that was used may not be relevant for every region of the \gls{divertor} .
These properties can however be experimentally estimated.
The accuracy of the results could therefore be improved by running a new batch of \gls{festim} \gls{monoblock} simulations with different trapping parameters like neutron-induced traps.

Then, this model does not take into account \gls{retention} in \gls{Be} co-deposited layers (i.e.\ \gls{Be} particles eroded from the \gls{first wall} redeposited in other locations of the vessel, trapping hydrogen).
These are expected to be the main driver for \gls{H} \gls{retention} in \gls{iter} \sidecite{de_temmerman_data_2021, schmid_walldyn_2015}.
However, this work is still relevant for full-tungsten environments like \gls{west} or \gls{demo}.

\setchapterimage{bubbles_tem}
\setchapterpreamble[u]{\margintoc}
\chapter{He transport in PFCs}\labch{Chapter5}
\label{Chapter5} % For referencing the chapter elsewhere, use \ref{Chapter2}

The divertor of a tokamak will not be only exposed to hydrogen.
Components will also be bombarded by high energy helium ions.

This Section will focus on the effect of helium on hydrogen transport.
To this end, a helium bubble growth model has been developed.
Based on the results of this model, experiments investigating the effect of helium transport on hydrogen trapping will be reproduced.

\section{Model description}\label{helium model description}
This Section describes the He transport model and the grouped approach employed to simplify it.

\subsection{Helium clustering model}

This model describes the evolution of the concentrations of pure interstitial He clusters (He$_x$) and mixed He-vacancies clusters (He$_x$V$_y$) that are formed by \gls{trap mutation} events.
\begin{figure*}
    \centering
    \begin{overpic}[width=0.7\linewidth]{Figures/Chapter2/He clustering.pdf}
        \put(10, 60){He$_1$}
        \put(25, 60){He$_2$}
        \put(45, 60){He$_3$}
        \put(60, 60){He$_4$}
        \put(85, 60){V$_1$He$_7$}
        \put(78, 15){V$_1$He$_8$}
        \put(50, 15){V$_1$He$_9$}
        \put(22, 15){V$_2$He$_{10}$}
        
    \end{overpic}
    \caption{Representation of He clustering in solids. Dissociation is omitted for simplification purposes. The grey arrows thicknesses represent the magnitude of the reaction rate between mobile He$_1$ and other clusters at the same distance}.
    \label{fig:clustering sketch}
\end{figure*}


The spatio-temporal evolution of each species of size $i$ is defined by:
\begin{equation}
    \frac{\partial c_i}{\partial t} =  \nabla \cdot (D_i\nabla c_i) + \Gamma_i + R_i
    \label{eq:model}
\end{equation}
In Equation \ref{eq:model}, the first term of the right hand-side is the diffusion term where ${D=D_0 \cdot \exp\big(-E_\mathrm{diff}/ (k_B \cdot T )\big)}$ is the thermally activated diffusion coefficient expressed in \si{m^2.s^{-1}} with $E_\mathrm{diff}$ the diffusion activation energy in \si{eV}, $k_B$ the Boltzmann constant in \si{eV.K^{-1}} and $T$ the temperature in \si{K}.
If a species $i$ is assumed to be immobile, its diffusion coefficient $D_i$ is zero.
$\Gamma_i$ is the external production rate of species $i$.

The term $R_i$ is the coupling term due to reactions between species.
A simple reaction between two species can be described as:
\begin{equation}
    \ce{A + B <=>[k^+_\mathrm{A,B}][k^-_\mathrm{A,B}] AB}
\end{equation}

The forward rate constant $k^+_{A,B}$ is the clustering rate and is calculated using the theory of diffusion-limited reactions \sidecite{goldstein_diffusion_2007}:
\begin{equation}
    k^+_\mathrm{A,B} = 4 \pi (r_\mathrm{A} + r_\mathrm{B}) (D_\mathrm{A} + D_\mathrm{B})
\end{equation}
where $r_\mathrm{A}$ and $r_\mathrm{B}$ are the capture radii and $D_\mathrm{A}$ and $D_\mathrm{B}$ are the diffusion coefficients of species A and B respectively.
The backward rate constant $k^-_\mathrm{A,B}$ is the dissociation rate and is obtained using chemical equilibrium principles \cite{goldstein_diffusion_2007}:
\begin{equation}
    k^-_\mathrm{A,B} =\rho k^+_\mathrm{A,B}e^{\frac{-E_b}{k_B T}}
\end{equation}
where $\rho$ is the atomic density in $\si{m^{-3}}$ ($\rho = \SI{6.3e28}{m^{-3}}$ for W), $k_B$ is the Boltzmann constant in \si{eV.K^{-1}}, $T$ is the temperature in \si{K} and $E_b$ is the binding energy for the reaction \ce{AB -> A + B} in \si{eV}.

The reaction term $R_i$ is the coupling term between concentrations and is expressed as:

\begin{equation}
    R_i=  \sum_{m} k^+_{m,i-m} c_m c_{i-m}  - c_i \sum_m \left( k_{i, m}^+ c_{m} + k_{i+1}^- c_{i+1} -  k_i^- c_i \right)
    \label{eq:reaction term}
\end{equation}


In Equation \ref{eq:reaction term}, $c_i$ is the concentration of clusters of size $i$ in \si{m^{-3}}.
The first term corresponds to the reactions producing clusters of size $i$.
The second one corresponds to the ones reacting with clusters of size $i$.
The third term accounts for bigger clusters dissociating.
Finally, the last term corresponds to clusters of size $i$ dissociating.

\subsection{Grouped approach}
Extending this clustering model to clusters containing millions of helium extremely increases the computational cost.
A grouped approach proposed by Faney et al \sidecite{faney_spatially_2014} for reducing the number of equations will therefore be employed.
This technique consists in grouping the big clusters that have a similar behaviour in a single equation while explicitly accounting for smaller clusters.

The clustering equations can be written as follows:

\begin{subequations}
    \begin{align}
        \frac{\partial c_1}{\partial t} &= \nabla \cdot (D_1 \nabla c_1) + \Gamma + \sum\limits_{i=2}^N k_{i}^- c_i - 2k_{1, 1}^+ c_1^2 - \sum\limits_{i=2}^N k_{1,i}^+ c_1 c_i - \sum\limits_{i=N+1}^\infty k_{1,i}^+ c_1 c_i \\
        \frac{\partial c_2}{\partial t} &= \nabla \cdot (D_2 \nabla c_2) - k_{1, 2}^+ c_1 c_2 + k_{1, 1}^+ c_1^2 - k_{2}^- c_2 + k_{3}^- c_3\\
        \vdots \nonumber\\
        \frac{\partial c_i}{\partial t} &= - k_{1, i}^+ c_1 c_i + k_{1, i-1}^+ c_1 c_{i-1} - k_{i}^- c_i\\
        \frac{\partial c_{i+1}}{\partial t} &= - k_{1, i+1}^+ c_1 c_{i+1} + k_{1, i}^+ c_1 c_i\\
        \vdots \nonumber
    \end{align}
    \label{eq: temporal evolution no grouping}
\end{subequations}
where $N$ is some threshold required for the grouping technique.

In order to simplify this model, the following quantities are defined:

\begin{align}
    c_b &= \sum\limits_{i=N+1}^\infty c_i \quad \text{ : total concentration of clusters containing more than $N$ He} \\
    \langle i_b \rangle &= \frac{1}{c_b} \sum\limits_{i=N+1}^\infty i c_i \quad \text{ : average He content in $c_b$} \\
    \langle r_b \rangle &=  \frac{1}{c_b}\sum\limits_{i=N+1}^\infty r_i c_i \quad \text{ : average radius in $c_b$}\\
    \langle k_b^+ \rangle &=  \frac{1}{c_b}\sum\limits_{i=N+1}^\infty k_{1,i}^+ c_i = 4 \pi D_1 (r_1 + \langle r_b \rangle) \quad \text{ : average clustering rate in $c_b$}
\end{align}

Clusters with more than $N$ He ($c_b$) will be referred as "bubbles" in the following.

Equation \ref{eq: temporal evolution no grouping} therefore reads:

\begin{subequations}
    \begin{align}
        \frac{\partial c_1}{\partial t} &= \nabla \cdot (D_1 \nabla c_1) + \Gamma + \sum\limits_{i=2}^N k_{i}^- c_i- 2k_{1, 1}^+ c_1^2 - \sum\limits_{i=2}^N k_{1,i}^+ c_1 c_i - \langle k_b^+ \rangle c_1 c_b \\
        \frac{\partial c_2}{\partial t} &= \nabla \cdot (D_2 \nabla c_2) - k_{1, 2}^+ c_1 c_2 + k_{1, 1}^+ c_1^2 - k_{2}^- c_2 + k_{3}^- c_3\\
        \vdots \nonumber\\
        \frac{\partial c_N}{\partial t} &= - k_{1, N}^+ c_1 c_N + k_{1, N-1}^+ c_1 c_{N-1} - k_{N}^- c_N\\
        \frac{\partial c_b}{\partial t} &= k_ {1,N}^+ c_1 c_N \\
        \frac{\partial (\langle i_b \rangle c_b)}{\partial t} &= (N+1)k_ {1,N}^+ c_1 c_N  + \langle k_b^+ \rangle c_1 c_b
    \end{align}
    \label{eq: temporal evolution grouping}
\end{subequations}

The mean radius of pure He clusters\sidecite{faney_spatially_2015} is given by:
\begin{equation}
    r_{\mathrm{He}_x} = r_{\mathrm{He}_1} + \left(\frac{3}{4\pi} \frac{a_0^3}{10} x \right)^{1/3} - \left( \frac{3}{4\pi} \frac{a_0^3}{10} \right)^{1/3}
    \label{eq: radius pure He}
\end{equation}
with $r_{\mathrm{He}_1} = \SI{0.3}{nm}$.

Several assumptions are made:
\begin{itemize}
    \item The average radius is assumed to be a function of $\langle i_b \rangle$:
    \begin{equation}
        \begin{split}
            \langle r_b \rangle &= r(\mathrm{He}_{\langle i_b \rangle}\mathrm{V}_{\langle i_b \rangle/4}) \\
            &= r_{\mathrm{He}_0 \mathrm{V}_1} + \left(\frac{3}{4 \pi} \frac{a_0^3}{2} \frac{\langle i_b \rangle}{4} \right)^{1/3} - \left(\frac{3}{4 \pi} \frac{a_0^3}{2} \right)^{1/3}
        \end{split}
        \label{eq: radius average}
    \end{equation}
    with $a_0 = \SI{0.318}{nm}$ the lattice parameter and $r_{\mathrm{He}_0 \mathrm{V}_1} =  a_0 \sqrt{3}/4$.
    The average radius $\langle r_b \rangle$ is assumed to be only dependent on $\langle i_b \rangle$.
    The number of vacancies in bubbles is assumed to be $\langle i_b \rangle/4$.
    This assumption is motivated by \gls{md} computations showing that \gls{trap mutation} events occur for every four additional helium in large \gls{vacancy}-helium clusters.
    Moreover, theoretical models for He bubbles growth in metal suggest a similar trend \sidecite{hammond_theoretical_2020}.
    \item Dissociation of large clusters is neglected (\textit{ie} $k_i^- = 0$ for $i>N$).
    Indeed, the activation energy for \gls{trap mutation} events is lower than that of He or \gls{vacancy} emission\sidecite{boisse_modeling_2014}. Dissociation of large clusters by \gls{vacancy} or He emission is therefore negligible.
\end{itemize}

The current implementation further simplifies Faney's model \sidecite{faney_spatially_2014}:
\begin{itemize}
    \item Interactions with \gls{self-interstitial} atoms or pre-existing vacancies are not taken into account.
    In this work, the only dissociations are He emissions from small mobile clusters and \gls{trap mutation} for large clusters.
    It was showed that this assumption did not have an impact on the results (see Figure \ref{fig:tendril profiles}).
    \item The only clusters explicitly computed are $\mathrm{He}_{x \leq 6}$ (\textit{ie} $N=6$) whereas Faney's work explicitly accounted for clusters up to $\mathrm{V}_{50}\mathrm{He}_{250}$ and solved a bigger system of equations.
    The influence of this threshold $N$ above which clusters are integrated in the quantity $c_b$ is discussed in Section \ref{impact of N}.
    \item Clusters containing more than six He atoms are assumed to be immobile (\textit{ie} $D_i = 0$ for $i>6$) due to \gls{trap mutation} events.
    This assumption is motivated by \gls{dft} and \gls{md} results suggesting that the \gls{self-trapping} energy is below the binding energy of one He atom in a pure He cluster for clusters containing more than five He atoms \sidecite{boisse_modeling_2014}.

    For smaller clusters ($\mathrm{He}_1$, $\mathrm{He}_2$, ..., $\mathrm{He}_6$) the diffusion coefficient and the dissociation by He emission energy vary with the number of He atoms in the cluster (see Table \ref{tab:clusters properties}).
\end{itemize}

\begin{table}
    \centering
    \begin{tabular}{L{1cm} R{2cm} R{1.6cm} R{1.1cm} R{1.6cm} R{1.1cm} R{2cm}}
        Cluster & $D_0 (\si{m^2 s^{-1}})$  & $E_\mathrm{diff} (\si{eV})$ &  $E_b (\si{eV})$   \\
        \hline
        \\
        He$_1$ & $2.95\times 10^{-8}$ & $0.13$ & - \\
        He$_2$ & $3.24\times 10^{-8}$ & $0.20$ & 1.0\\
        He$_3$ & $2.26\times 10^{-8}$ & $0.25$ & 1.5\\
        He$_4$ & $1.68\times 10^{-8}$ & $0.20$ & 1.5\\
        He$_5$ & $5.20\times 10^{-9}$ & $0.12$ & 1.6\\
        He$_6$ & $1.20\times 10^{-9}$ & $0.30$ & 2.0\\
    \end{tabular}
    \caption{Pure He clusters properties in W. Diffusion properties are taken from Faney \textit{et al} \cite{faney_spatially_2015} and binding energies are taken from Becquart \textit{et al} \cite{becquart_microstructural_2010}.}
    \label{tab:clusters properties}
\end{table}

% for some reason, uncommenting this activates the top table
% \begin{table}
%  \begin{tabular}{ c c c c }
%  \toprule
%  col1 & col2 & col3 & col 4 \\
%  \midrule
%  \multirow{3}{4em}{Multiple row} & cell2 & cell3 & cell4\\ &
%  cell5 & cell6 & cell7 \\ &
%  cell8 & cell9 & cell10 \\
%  \multirow{3}{4em}{Multiple row} & cell2 & cell3 & cell4 \\ &
%  cell5 & cell6 & cell7 \\ &
% cell8 & cell9 & cell10 \\
%  \bottomrule
%  \end{tabular}
% \end{table}


This He transport model was implemented in Python and solved using the finite element solving platform \gls{fenics} \sidecite{alnaes_fenics_2015}.

\section{Direct implantation of He}
In this Section, the current implementation is first compared with the one from Faney \sidecite{faney_spatially_2015} to ensure the additional assumptions do not produce different results.
A standard half-slab case is then described and a parametric study is performed by varying the exposure conditions.
Finally, the model is compared against experimental data.

\subsection{Half-slab case} \labsec{half slab}

\begin{figure}
    \centering
    \includegraphics[width=\linewidth]{Figures/Chapter4/half_slab/profiles_half_slab.pdf}
    \caption{Concentration profiles of He$_1$ (left) and bubbles (right) in W exposed to \SI{100}{eV} He at \SI{e22}{m^{-2}.s^{-1}} and \SI{1000}{K}.}
    \labfig{profiles half slab}
\end{figure}

\gls{He} transport was simulated in a 1D semi-infinite \gls{W} slab.
This case is the standard case describing the main quantities of interest of the parametric study performed in \refsec{parametric study}.

The domain size is \SI{0.1}{mm} which is much greater than the penetration depth of \gls{He} in the simulations.
\SI{100}{eV} \gls{He} were implanted in the first \SI{1.5}{nm} as in \refsec{tendril case}.
The implanted flux was \SI{1e22}{m^{-2} s^{-1}} and the temperature was \SI{1000}{K}.

At low \glspl{fluence}, \gls{He} diffused really quickly into the bulk (see \reffig{profiles half slab}) and the bubbles' concentration $c_b$ was found to be zero.
As the \gls{fluence} increased, bubbles started to appear and acted as strong sinks for mobile \gls{He}.
This lead to a great decrease in the mobile He concentration profile.

It is worth noticing the maximum of $c_b$ was not located at the maximum of $c_{\mathrm{He}_1}$ which is the implantation depth $R_p$.
This was explained by the \gls{diffusion} of small mobile clusters as shown by analytical models \sidecite{krasheninnikov_helium_2014}.
As \gls{He} clusters, small mobile clusters diffuse deeper into the bulk until \gls{trap mutation} occurs and bubbles nucleons (clusters with more than 6 \gls{He}) are created.
From that point, bubbles are formed relatively far from the surface.
Because \gls{He} is implanted in the first nanometres, $c_{\mathrm{He}_1}$ is maximum at $R_p = \SI{1.5}{nm}$ and interactions with bubbles are stronger in this region.
This tends to draw the maximum location of $c_b$ towards the surface.

The \gls{He} content in bubbles $\langle i_b \rangle$ and the radius $\langle r_b \rangle$ were computed.
After \SI{10}{s} of implantation, bubbles located in the near surface contained up to \SI{3e7}{He}.
The maximum of $\langle r_b \rangle$ was found to be very close to the surface at approximately \SI{2}{nm} (see \reffig{profiles rb half slab}).
This is explained by the high concentration of mobile \gls{He} in this near surface region.
Moreover, a bursting zone can be defined by the region where $\langle r_b \rangle$ is greater than the depth of the bubble.
In this region, bubble of this size would have likely burst.
% This result was in good agreement with the He bubbles observations performed by Ialovega et al.\ on W \sidecite{ialovega_hydrogen_2020}.

\begin{figure} [h]
    \centering
    \includegraphics[width=\linewidth]{Figures/Chapter4/half_slab/profile_rb.pdf}
    \caption{Profile of mean bubble radius $\langle r_b \rangle$ as a function of depth $x$ in \gls{W} exposed to \SI{100}{eV} \gls{He} at \SI{e22}{m^{-2}.s^{-1}} and \SI{1000}{K}.}
    \labfig{profiles rb half slab}
\end{figure}

\begin{figure*}
    \centering
    \includegraphics[width=0.75\linewidth]{Figures/Chapter4/half_slab/average_content_radius.pdf}
    \caption{Average helium content $\langle i \rangle$ and average radius $\langle r \rangle$ in all clusters (mobile and bubbles) in \gls{W} exposed to \SI{100}{eV} \gls{He} at \SI{e22}{m^{-2}.s^{-1}} and \SI{1000}{K}.}
    \labfig{content and radius}
\end{figure*}

From this average \gls{He} content in bubbles and from \refeq{radius average} and \refeq{radius pure He} expressing the clusters radii, the average radius $\langle r \rangle$ can be computed as:

\begin{equation}
        \langle r \rangle = \frac{\sum\limits_{i=1}^\infty c_i r_i}{\sum\limits_{i=1}^\infty c_i}
        = \frac{\sum\limits_{i=1}^N c_i r_i + c_b \langle r_b \rangle }{\sum\limits_{i=1}^N c_i + c_b}
\end{equation}

The average content of \gls{He} in all clusters $\langle i \rangle$ is computed similarly:
\begin{equation}
        \langle i \rangle = \frac{\sum\limits_{i=1}^\infty c_i i}{\sum\limits_{i=1}^\infty c_i}
        = \frac{\sum\limits_{i=1}^N c_i i + c_b \langle i_b \rangle }{\sum\limits_{i=1}^N c_i + c_b}
\end{equation}

The values of these two quantities are similar to the ones obtained by Faney et al.\ \sidecite{faney_spatially_2015}.
After \SI{100}{s} of exposure, the average radius \SI{50}{nm} below the surface was above \SI{10}{nm} (see \reffig{content and radius}).
Moreover, the location of the maximum of these quantities move towards the exposed surface.

The average radius $\langle r \rangle$ cannot be easily compared to experimental observations for it includes contributions from very small mobile He$_x$ clusters which are not visible experimentally (only bubbles with a radius greater than 1-\SI{3}{nm} are observable depending on the observation technique).

\subsection{Influence of exposure parameters on He bubble growth}
The impact of He flux and temperature $T$ was studied on the case described in \refsec{half slab} in order to identify trends.
Behaviour laws are identified and can be used to obtain information on He transport without needing to run any simulation.

\subsubsection{Parametric study} \labsec{parametric study}

\begin{figure*} [ht!]
    \centering
    \begin{subfigure}{0.5\linewidth}
        \centering
        \includegraphics[width=\linewidth]{Figures/Chapter4/parametric study/bubbles_total_T_phi.pdf}
        \caption{Bubbles inventory $I_b = \int c_b \; dx$.}
        \labfig{inventory bubbles T phi}
    \end{subfigure}%
    \begin{subfigure}{0.5\linewidth}
        \centering
        \includegraphics[width=\linewidth]{Figures/Chapter4/parametric study/inventory_T_phi.pdf}
        \caption{He inventory $I = \int \langle i_b \rangle c_b \; dx$.}
        \labfig{He inventory T phi}
    \end{subfigure}
    % \begin{subfigure}{0.5\linewidth}
    %     \centering
    %     \includegraphics[width=\linewidth]{Figures/Chapter4/parametric study/mean_ib_T_phi.pdf}
    %     \caption{mean $\langle i_b \rangle$}
    % \end{subfigure}%
    % \begin{subfigure}{0.5\linewidth}
    %     \centering
    %     \includegraphics[width=\linewidth]{Figures/Chapter4/parametric study/surface_flux_T_phi.pdf}
    %     \caption{Retrodesorbed flux}
    % \end{subfigure}
    \begin{subfigure}{0.5\linewidth}
        \centering
        \includegraphics[width=\linewidth]{Figures/Chapter4/parametric study/mean_ib_T_phi.pdf}
        \caption{Average \gls{He} content in bubbles $\Bar{\langle i_b \rangle} = I / I_b$.}
        \labfig{mean ib T phi}
    \end{subfigure}%
    \begin{subfigure}{0.5\linewidth}
        \centering
        \includegraphics[width=\linewidth]{Figures/Chapter4/parametric study/x_max_ib_T_phi.pdf}
        \caption{Location of max $\langle i_b \rangle$.}
        \labfig{x max ib T phi}
    \end{subfigure}
    \caption{Evolution of quantities as a function of the implanted flux and temperature after \SI{1}{h} of \SI{100}{eV} \gls{He} exposure. Grey crosses correspond to simulations points.}
    \labfig{T phi quantities}
\end{figure*}

A parametric study was performed by varying the implanted flux $\varphi_\mathrm{imp}$ between \SI{1e17}{m^{-2} s^{-1}} and \SI{5e21}{m^{-2} s^{-1}} and the sample temperature $T$ between \SI{100}{K} and \SI{1200}{K}.

Several quantities were computed.
First the bubbles inventory is defined as:
\begin{equation}
    I_\mathrm{bubbles}= \displaystyle \int c_b \; dx
\end{equation}
The total helium inventory is calculated by:
\begin{equation}
        I = \displaystyle \int \sum\limits_{i=1}^N i c_i + \langle i_b \rangle c_b \; dx
        \approx \displaystyle \int \langle i_b \rangle c_b \; dx
    \labeq{I}
\end{equation}
The spatial mean helium content in bubbles can be computed as:
\begin{equation}
        \Bar{\langle i_b \rangle} = \frac{\displaystyle \int \langle i_b \rangle c_b \; dx}{\displaystyle \int c_b \; dx}
        \approx \frac{I}{I_\mathrm{bubbles}}
    \labeq{mean ib}
\end{equation}
The approximation made in \refeq{I} and \refeq{mean ib} is valid as long as $\int \langle i_b \rangle c_b dx \gg  \int \sum\limits_{i=1}^N i c_i dx$ (i.e.\ the He inventory is dominated by that of the bubbles).
This is the case in these simulations because $N=6$ (the influence of this parameter is discussed in \refsec{impact of N}).

More than 160 simulations were performed simulating \SI{1}{h} of exposure.
For each simulation, the quantities of interest described above were computed.
A Gaussian regression process \sidecite{chris_bowman_c-bowmaninference-tools_2020} was used to interpolate the data based on Bayesian inference as done in \sidecite{delaporte-mathurin_parametric_2020} (see \reffig{T phi quantities}).
The temporal evolution of these quantities was also assessed (see \reffig{quantities time}).

% bubbles inventory
After \SI{1}{h} of exposure, the bubbles inventory $I_\mathrm{bubbles}$ shows a weak dependence on temperature at high temperature and a weak dependence on the implanted flux at low temperature (see \reffig{inventory bubbles T phi}).
$I_\mathrm{bubbles}$ varies from \SI{4e12}{\text{bubbles } m^{-2}} at high temperature and low flux to \SI{2e19}{\text{bubbles } m^{-2}} at low temperature and high flux.

% He inventory
The \gls{He} inventory $I$ varies from \SI{8e16}{m^{-2}} at high temperature and low flux to \SI{e25}{m^{-2}} at low temperature and high flux (see \reffig{He inventory T phi}).
For temperatures above \SI{600}{K}, the temperature dependence is rather weak compared to the flux dependence.

% mean ib
For temperatures above \SI{300}{K}, and after \SI{1}{h} of exposure, the sample temperature does not impact the value of $\Bar{\langle i_b \rangle}$ (see \reffig{mean ib T phi}).
The mean He content increases with the implanted flux as expected and varies between \SI{e3}{He} at low flux and \SI{5e8}{He} at high flux.

% x max ib
The position of the maximum of $\langle i_b \rangle$ tended to increase with temperature and decrease with implanted flux (see \reffig{x max ib T phi}).
After \SI{1}{h} of exposure, it was found to be really close to the surface down to \SI{0.1}{nm} at low temperatures and high fluxes.
The validity of the model in this region of the parameter space is questionable considering that the bubble radius is greater that the thickness of the ligament between the edge of the bubble and the surface.
Such a bubble would therefore have burst before reaching this size. 

\begin{figure*} [ht!]
    \centering
    \begin{subfigure}{0.5\linewidth}
        \centering
        \includegraphics[width=\linewidth]{Figures/Chapter4/parametric study/total_bubbles_time.pdf}
        \caption{Bubbles inventory $I_b = \int c_b \; dx$.}
        \labfig{inventory bubbles time}
    \end{subfigure}%
    \begin{subfigure}{0.5\linewidth}
        \centering
        \includegraphics[width=\linewidth]{Figures/Chapter4/parametric study/inventory_time.pdf}
        \caption{He inventory $I = \int \langle i_b \rangle c_b \; dx$.}
        \labfig{He inventory time}
    \end{subfigure}
    % \begin{subfigure}{0.5\linewidth}
    %     \centering
    %     \includegraphics[width=\linewidth]{Figures/Chapter4/parametric study/mean_ib_time.pdf}
    %     \caption{mean $\langle i_b \rangle$}
    % \end{subfigure}%
    % \begin{subfigure}{0.5\linewidth}
    %     \centering
    %     \includegraphics[width=\linewidth]{Figures/Chapter4/parametric study/surface_flux_T_phi.pdf}
    %     \caption{Retrodesorbed flux}
    % \end{subfigure}
    \begin{subfigure}{0.5\linewidth}
        \centering
        \includegraphics[width=\linewidth]{Figures/Chapter4/parametric study/mean_ib_time.pdf}
        \caption{Average He content in bubbles $\Bar{\langle i_b \rangle} = I / I_b$.}
        \labfig{mean ib time}
    \end{subfigure}%
    % \begin{subfigure}{0.5\linewidth}
    %     \centering
    %     \includegraphics[width=\linewidth]{Figures/Chapter4/parametric study/x_max_ib_time.pdf}
    %     \caption{x max $\langle i_b \rangle$}
    %     \labfig{x max ib time}
    % \end{subfigure}
    \begin{subfigure}{0.5\linewidth}
        \centering
        \includegraphics[width=\linewidth]{Figures/Chapter4/parametric study/points_with_parameter.pdf}
        \caption{Simulation points coloured according to $c_{\mathrm{He}_1, \mathrm{ideal}}$.}
        \labfig{simulations points with parameter}
    \end{subfigure}
    \caption{Temporal evolution of quantities in W exposed to \SI{100}{eV} He at \SI{e22}{m^{-2}.s^{-1}} and \SI{1000}{K} for temperatures varying from \SI{120}{K} to \SI{1200}{K} and implanted fluxes varying from \SI{e17}{m^{-2}s^{-1}} to \SI{e21}{m^{-2}s^{-1}}. Each line corresponds to a simulation point (grey crosses on \reffig{inventory bubbles T phi} and points on \reffig{simulations points with parameter}). The lines are coloured according to the parameter $c_{\mathrm{He}_1, \mathrm{ideal}} = \varphi_\mathrm{imp} \; R_p/D(T)$ with $R_p = \SI{1.5}{nm}$ and $D$ the diffusion coefficient of $\mathrm{He}_1$ in W.}
    \labfig{quantities time}
\end{figure*}

% time series
For each simulation point, the temporal evolution of the quantities described above has been computed.
To better identify the time series on the $\varphi_\mathrm{imp}, T$ plane, lines have been coloured according to the parameter $c_{\mathrm{He}_1, \mathrm{ideal}}$ which is a function of both the implanted flux and the temperature (see \refeq{c he1 ideal}) expressed in \si{m ^{-3}}.

\begin{equation}
    c_{\mathrm{He}_1, \mathrm{ideal}} = \frac{\varphi_\mathrm{imp} \; R_p}{D(T)}
    \labeq{c he1 ideal}
\end{equation}
where $\varphi_\mathrm{imp}$ is the implanted flux, $D$ is the diffusion coefficient of mobile $\mathrm{He}_1$ in W (see \reftab{clusters properties}), $R_p = \SI{1.5}{nm}$ is the implantation depth and $T$ is the temperature in \si{K}.

All these quantities showed a similar behaviour in time even though the kinetics were found to be different (see \reffig{quantities time}).
For instance, for each $(T, \varphi_\mathrm{imp})$ couple, $I_\mathrm{bubbles}$ first increased as a power law of time before reaching a maximum (see \reffig{inventory bubbles time}).
The total He inventory $I$ increased with time and for each simulation point, but the growth rate decreased at long exposure times (see \reffig{He inventory time}).
This phenomenon is explained in details in \refsec{nucleation growth phases}.
Similarly, $\Bar{\langle i_b \rangle}$ could be written as a power law of time described in Eq \refeq{ib evolution} (see \reffig{mean ib time}).
The depth of the maximum of $\langle i_b \rangle$ tended to decrease with time as it was observed in \refsec{half slab} (see \reffig{inventory bubbles time}).

\subsubsection{Inventory evolution regimes} \labsec{nucleation growth phases}

\begin{figure} [h]
    \centering
    \includegraphics[width=0.75\linewidth]{Figures/Chapter4/parametric study/inventory_bubbles_ib.pdf}
    \caption{Temporal evolution of $\Bar{\langle i_b \rangle}$, $I_\mathrm{bubbles}$ and $I$ in W exposed to \SI{100}{eV} He at \SI{1.59e18}{m^{-2}.s^{-1}} and \SI{1020}{K}. The dashed grey vertical line represents the transition from nucleation regime to bubble growth regime.}
    \labfig{two regimes}
\end{figure}

For every $(T, \varphi_\mathrm{imp})$ couple, $I_\mathrm{bubbles}$ increased rapidly at low \glspl{fluence} until reaching a maximum at high \glspl{fluence} (see \reffig{inventory bubbles time}).
On the other hand, the mean \gls{He} content $\Bar{\langle i_b \rangle}$ was constant at low \glspl{fluence} and increased as a power law of time at high \glspl{fluence} (see \reffig{mean ib time}).
The evolution of $\Bar{\langle i_b \rangle}$ can be described as:
\begin{equation}
    \Bar{\langle i_b \rangle} = N + 1 + a \; t^b
    \labeq{ib evolution}
\end{equation}
where $N=6$ in this model, $a$ and $b$ depend on $(T, \varphi_\mathrm{imp})$.
The choice of $N=6$ in this model is detailed in \refsec{impact of N}.
The total \gls{He} \gls{inventory} $I$ being the product of these two quantities, two different growth rates were observed (see \reffig{He inventory time} and \reffig{two regimes}).

This phenomenon can be attributed to two different regimes.
The first regime is the nucleation regime where new bubbles nucleons are created (i.e.\ $c_b$ and $I_\mathrm{bubbles}$ increase).
In the nucleation regime, the bubble concentration $c_b$ and the capture radius $\langle r_b \rangle$ are too low for the He content in bubbles $\langle i_b \rangle$ to increase significantly (i.e.\ $\Bar{\langle i_b \rangle}$ is constant).
The second regime is the bubble growth regime.
In this regime, $c_b$ is high enough for interactions between bubbles and mobile He to occur.
Implanted interstitial He atoms ($c_{\mathrm{He}_1}$) therefore interact preferably with bubbles rather than clustering with other interstitial He atoms.
This means that no additional bubbles nucleons are created (i.e.\ $c_b$ reaches a maximum).
Because interactions between bubbles and mobile He are strong, the term $\langle k_b^+ \rangle c_1 c_b$ in \refeq{temporal evolution grouping} becomes significant and the He content increases (i.e.\ $\Bar{\langle i_b \rangle}$ increases).
This is illustrated by the thickness of the arrows in \reffig{clustering sketch}.


\subsubsection{Influence of $N$} \labsec{impact of N}
In order to assess the impact of the parameter $N$ in \refeq{temporal evolution grouping}, the evolution of the He inventory $I$, the mean He content in immobile clusters (different from $\Bar{\langle i_b \rangle}$) and the bubbles inventory $I_\mathrm{bubbles}$ was computed with several values of $N$.

The flux of \SI{100}{eV} He in this test case was \SI{e20}{m^{-2} s^{-1}} and the temperature was \SI{1000}{K}. 

It was shown that varying $N$ had no impact on these quantities whatsoever (see \reffig{N variation}).
This highlights the very quick transition from nucleation regime to growth regime in this model.

The number of equations that need to be solved can therefore be minimised by setting the parameter $N$ to its minimum ($N=6$) without losing accuracy in the results.
This minimum value corresponds to the number of mobile clusters which have to be explicitly simulated in order to account for all the diffusion mechanisms.

\begin{figure} [h]
    \centering
    \includegraphics[width=\linewidth]{Figures/Chapter4/varying_N.pdf}
    \caption{Comparison of several quantities of interest for several values of $N$ in W exposed to \SI{100}{eV} He at \SI{e20}{m^{-2}.s^{-1}} and \SI{1000}{K}.}
    \labfig{N variation}
\end{figure}

\section{Indirect sources of He}
As detailed in \refsec{sources of helium}, helium can be produced in tungsten from neutron transmutation and from tritium decay.
This section will focus on comparing these two indirect sources with direct helium implantation in monoblocks.

\subsection{Neutron induced transmutation}

In combination with the Paramak code \sidecite{shimwell_paramak_2021} used for creating the geometry, a neutronics simulation was run to assess the total quantity of helium generation in a monoblock under neutron irradiation with the OpenMC code \sidecite{romano_openmc_2015}, a modern open-source Monte-Carlo neutron and photon transport code.
In this simulation, a neutron source was placed above the monoblock and the total helium production was tallied.
The neutron source corresponds to a \SI{500}{MW} DT source, which gives a neutron generation rate of \SI{1.8}{neutron.s^{-1}} (based on the energy produced by the DT fusion reaction).
50 batches of 1 million neutrons were simulated in order to reduce the stochastic error inherant to Monte-Carlo methods.

The production of helium was found to be more important close to the top surface and to the neutron source (see \reffig{transmutation helium in monoblock}).
It evolves as linearly with the distance from the top surface.
The maximum generation rate is $\approx \SI{7e18}{m^{-3}.s^{-1}}$, which is well below the generation rate from direct implantation.
\reffig{helium generation distribution} was obtained by averaging all the values by distance from the top surface.
The error bars were computed by averaging the standard deviation provided by OpenMC.

\begin{figure*}
    \centering
    \begin{subfigure}{0.5\linewidth}
        \includegraphics[width=\linewidth]{Figures/Chapter5/helium_transmutation_in_monoblock.pdf}
        \caption{2D distribution}
    \end{subfigure}%
    \begin{subfigure}{0.5\linewidth}
        \includegraphics[width=\linewidth]{Figures/Chapter5/he_generation_distribution.pdf}
        \caption{Distribution from the top surface. Errors bars correspond to the 95 \% confidence interval.}
        \labfig{helium generation distribution}
    \end{subfigure}
    \caption{Helium generation via transmutation in a monoblock (only tungsten is shown).}
    \labfig{transmutation helium in monoblock}
\end{figure*}

Note that this is a conservative case as the monoblock simulated is right below the neutron source.
Other monoblocks of the divertor will be tilted and shadowed by others and therefore will interact less with the neutrons.


\subsection{Tritium decay}

\begin{figure}
    \centering
    \includegraphics[width=\linewidth]{Figures/Chapter5/helium_generation.pdf}
    \caption{Comparison of the three sources of helium in a monoblock.}
    \labfig{comparison helium generation}
\end{figure}
\section{Influence on H transport}
This Section reproduces the experiment of Ialovega and co-workers \sidecite{ialovega_hydrogen_2020}.

\subsection{Experiment and FESTIM simulation description}

A \SI{100}{\micro\metre} thick tungsten sample was pre-damaged with \SI{75}{eV} He at \SI{1073}{K}.
The He flux was \SI{2.3e22}{m^{-2}.s^{-1}} and the exposure time was \SI{13}{s}.
An initial cleaning TDS was performed up to \SI{870}{K}.

Sequential deuterium loading and TDS were then repeated five times.
\SI{250}{eV} deuterium were implanted at room temperature with a flux of \SI{1.7e16}{m^{-2}.s^{-1}} and a fluence of \SI{4.5e19}{m^{-2}}.
The TDS phase ramps up to \SI{1350}{K} (\SI{1250}{K} for the first TDS) at a rate of \SI{1}{K.s^{-1}}.

Four traps are simulated: traps 1-3 are pre-existing defects and trap 4 represents the traps induced by helium bubbles.
The detrapping energies and trap densities are set as free parameters, including the trap density $n_b$ (see Table \ref{tab: trap properties}).

Considering deuterium is trapped on the surface of He bubbles, the bubble trap density $n_b$ is given by:

\begin{equation}
    n_b = f \cdot c_b \cdot \Ab(\langle r_b \rangle)
\end{equation}
where $f$ is a free parameter representing the number of trapping site per unit surface, $\Ab = 4\pi \langle r_b \rangle^2$ is the area in \si{m^2} of a spherical bubble of radius $\langle r_b \rangle$, and $c_b$ is the concentration of bubbles in $\si{m^{-3}}$.

The quantities $c_b$ and $\langle r_b \rangle$ have been computed from the model described in Section \ref{helium model description} (see Figure \ref{fig:trap bubbles distribution}).

\begin{figure}[h!]
    \centering
    \includegraphics[width=\linewidth]{Figures/Chapter5/trap_bubble_distribution.pdf}
    \caption{Spatial distribution of the bubbles density and the equivalent trap density (assuming $f=\SI{3e18}{m^{-2}}$).}
    \label{fig:trap bubbles distribution}
\end{figure}


\begin{table}[!h]
    \caption{Trap properties used to fit the TDS spectra. The density distribution $n_b$ as well as detrapping energies $E_p$ are assumed constant across TDS experiments.}
    \begin{tabular}{r l l l l l}
    \\
     & $k_0$ & $E_k$ & $p_0$ & $E_p$ & $n$ \\
     \ & [\si{m^{3}.s{-1}}] & [\si{eV}] & [\si{s^{-1}}] & [\si{eV}] & [\si{m^{-3}}] \\
    \\
    Trap 1 & \multirow{7}{*} { $9 \times 10 ^{-17}$ } & \multirow{7}{*} { 0.39 } & \multirow{7}{*} { $10^{13}$ } & free & free \\
    \\
    Trap 2 & & & & free & free \\
    \\
    Trap 3 & & & & free & free \\
    \\
    Trap bubbles & & & & free & $n_b$ \\
    \end{tabular}
    \label{tab: trap properties}
\end{table}

The diffusion coefficient of deuterium (\si{m^2.s^{-1}}) was set to $4.1\times 10 ^{-7} \exp{-0.39/k_B T}$ \sidecite{frauenfelder_solution_1969}
The \SI{250}{eV} deuterium implantation was represented by a Gaussian distrubiton with a mean implantation depth \SI{10}{nm} and a standard deviation of \SI{4.5}{nm} (calculated from SRIM \sidecite{ziegler_srim_2010}).
Finally, an instantaneous recombination was assumed on the surfaces.

Using the parametric optimisation method developed in \sidecite{delaporte-mathurin_parametric_2021}, the free parameters are identified.

\subsection{Results}

The properties obtained by the fitting procedure (see Table \ref{tab: trap properties results}) fitted well the three TDS spectra (see Figure \ref{fig: fitted TDS}).
As explained in \cite{ialovega_hydrogen_2020}, the last bump of desorption (around \SI{600}{K}) is due to a temperature control issue and was therefore ignored in the fitting procedure.
The detrapping energies of traps 1, 2 and 3 were found to be \SI{1.08}{eV}, \SI{1.20}{eV} and \SI{1.38}{eV} respectively, whereas the trap attributed to He bubbles has a detrapping energy of \SI{1.45}{eV}.


\begin{table}[!h]
    \caption{Results of the fitting procedure. Detrapping energies $E_p$ are given in \si{eV}, trap densities in \si{at.fr.} and $f$ in \si{m^{-2}}.}
    \begin{tabular}{r l l l l l l l l}
    \\
    &\multicolumn{2}{l}{Trap 1}  & \multicolumn{2}{l}{Trap 2} & \multicolumn{2}{l}{Trap 3} &\multicolumn{2}{l}{Trap bubbles} \\
     & $E_p$ & $n$ ($\times 10 ^{-3}$) & $E_p$ & $n$ ($\times 10 ^{-3}$) & $E_p$ & $n$ ($\times 10 ^{-3}$) & $E_p$ & $f$ ($\times 10 ^{18}$) \\
    \\
    1st TDS & - & 0.00 & - & 0.00 & - & 0.00 & 1.42 & $3.00$ \\
    \\
    2nd TDS & 1.08 & $2.20$ & 1.20 & $1.80$ & 1.37 & $2.00$ & 1.42 & $3.00$ \\
    \\
    5th TDS & 1.08 & $3.38$ & 1.20 & $3.10$ & 1.37 & $1.50$ & 1.42 & $3.00$ \\
    \end{tabular}
    \label{tab: trap properties results}
\end{table}

\begin{itemize}
    \item 1st He implantation: all pre-existing defects are saturated with He and bubbles are formed (proof from PAS that there are pre-existing defects before He implantation)
    \item 1st H implantation: H can only be trapped around bubbles since defects are saturated
    \item 1st TDS: H trapped around bubbles is desorbed (550K peak), He dissociates from HeV clusters (up to 1250K)
    \item 2nd H implantation: H is trapped around bubbles + in the non-saturated defects
    \item 2nd TDS: H trapped around bubbles is desorbed (550 K peak) + H trapped in non-saturated HeHV clusters (peaks 400K, 450K and 500K) + He trapped in deeper traps dissociate (TDS up to 1350K)
    
    \item 3rd to 5th implantation: H is trapped around bubbles + non-saturated defects
    \item 3rd to 5th TDS: H around bubbles is desorbed + more H dissociates from HeHV clusters
\end{itemize}




% After the first TDS, three additional peaks appear (traps 1, 2 and 3), which correlates with the fact that a small quantity of He left the sample after the first TDS.
% During all the following TDS cycles, the He desorption was much lower.
% This relates with the fact that the He bubbles traps density does not vary between the second and fifth TDS.
% The density of traps 1, 2 and 3 is increasing from the first TDS to the second.
% This suggests that some of the desorbed He was occupying defects such as vacancies.
% This attribution is also supported by the He TDS spectra shown in \cite{ialovega_hydrogen_2020}.
% These spectra showed that, during the first and second TDS, He desorbed at temperatures above \SI{1000}{K} which corresponds to $\mathrm{He}_m\mathrm{V}_1$ clusters \cite{faney_spatially_2014}.

This would mean that if the TDS were run up to temperature around \SI{1600}{K}, $\mathrm{He}_1\mathrm{V}_1$ clusters could dissociate resulting in additional free trapping sites for H and therefore different TDS spectra.

The H retention is not dominated by He-bubbles trapping but rather by secondary defects - potentially induced by the formation of bubbles.



\begin{figure}[h!]
    \centering
    \begin{subfigure}{\linewidth}
        \includegraphics[width=\linewidth]{Figures/Chapter5/ialovega_tds.pdf}
        \caption{Experimental TDS spectra fitted with FESTIM}
        \label{fig: 3 TDS}
    \end{subfigure}
    \begin{subfigure}{\linewidth}
        \includegraphics[width=\linewidth]{Figures/Chapter5/tds_trap_distribution.pdf}
        \caption{Traps contribution to the TDS spectra}
        \label{fig: trap contributions}
    \end{subfigure}
    \caption{Results of the TDS fitting procedure}
    \label{fig: fitted TDS}
\end{figure}

\begin{figure}
    \centering
    \includegraphics[width=\linewidth]{Figures/Chapter5/trap_densities.pdf}
    \caption{Caption}
    \label{fig:density evolution}
\end{figure}

The densities of traps 1 and 2 increased from the 2nd to the 5th TDS (see Figure \ref{fig:density evolution}).
However, the density of trap 3 decreased slightly (which is also visible on the TDS spectra shown in Figure \ref{fig: fitted TDS}).
This can be explained by either by 1) He leaving V-He complexes 2) annealing of some defects.
The processes at stake cannot yet be precisely described. 

\subsection*{Conclusion}

- H He implantation has been simulated with FESTIM
- One potential explanation of the effects observed in \cite{ialovega_hydrogen_2020}
- Though fitting a TDS spectrum is not sufficient to draw strong conclusions
- Suggests that 1) He doesn't leave bubbles in significant quantities at these temperatures 2) He occupies pre-existing defects avoiding H to get trapped 3) H retention is not dominated by He-induced bubbles

- Experimental suggestions: stop the TDS at \SI{750}{K} to limit He desorption. If the TDS spectra aren't affected this would confirm/infirm the He-induced trapping reduction effect.

\section{Summary}

FESTIM was used to simulate the influence of helium implantation on hydrogen transport.
The formation of helium bubble in tungsten was first studied.
It was shown that this phenomenon was limited to a small region near the exposed surface.

The experiment of Ialovega and co-workers described in \sidecite{ialovega_hydrogen_2020} was then reproduced and one possible explanation was given.
Although fitting a TDS spectrum is not sufficient to draw strong conclusions, this work suggests that:
\begin{itemize}
    \item He does not leave bubbles in significant quantities at these temperatures
    \item He occupies pre-existing defects avoiding H to get trapped
    \item Hydrogen retention is not dominated by He-induced bubbles but rather by the saturation of pre-existing defects by He
\end{itemize}

From these results, several experimental suggestions can be made.
Running the TDS up to \SI{750}{K} only would limit He desorption.
Indeed, it was shown in \cite{ialovega_hydrogen_2020} that there was no helium desorption below this temperature after the initial cleaning TDS.
If helium does not desorb and remains in the pre-existing defects, the deuterium TDS spectra should not be affected and only the desorption from bubbles should be observed.
This would confirm or infirm the interpretation of the results presented here.

Moreover, if this interpretation was confirmed, it could have implications for hydrogen retention.
Indeed, one could imagine reducing the tritium inventory of components by first exposing them to helium.
Helium would fill the existing defects, making it impossible for tritium to be trapped.
However, having a helium inventory in components can also have negative consequences.

\setchapterpreamble[u]{\margintoc}
\chapter{Conclusion}\labch{conclusion}

This Chapter will summarise the key findings of this research as well as the main contributions thereof.
It will also review the limitations of the study and propose recommandations for future work.

\section*{Key findings}

This study aimed to estimate the tritium inventory in the \gls{iter} tungsten \gls{divertor}.
One of the objectives was to evaluate the potential influence of helium exposure on the tritium retention.
The results indicate that, over the operation time of \gls{iter}, the \gls{divertor} tritium inventory should remains well below the safety limit.
Indeed, this component will not be the limiting factor and if the limit is hit, it would likely be due to other causes (retention in co-deposited layers, other in-vessel components...).
Moreover, the results suggest that the presence of helium could reduce the tritium inventory.

\section*{Contributions to the field}
The main contribution is the \gls{festim} code, a hydrogen transport code that has been developed to answer the main questions of the study.
The first article introducing the \gls{festim} code was published during the master preceding this PhD research \cite{delaporte-mathurin_finite_2019}.
At the time of writing, \gls{festim} is used by a handful of researchers, engineers and students and applied on other cases.
Making \gls{festim} available to other researchers (by making it open-source) would greatly benefit the broader community.

The parametric optimisation method used in \refch{Chapter2} now provides an efficient way of automatically fitting experimental data without manually tweaking parameters, saving precious time in the process.
This method was published in a proceedings article \cite{delaporte-mathurin_parametric_2021}.

The initial monoblock results shown in \refch{Chapter3} were published in several articles \cite{delaporte-mathurin_finite_2019, delaporte-mathurin_parametric_2021, delaporte-mathurin_influence_2021-1}.
The method of running parametric sub-component simulations and then extract a behaviour law was nothing new.
However, applying it to the estimation of the \gls{divertor} inventory saved a significant amount of time compared to using the ``brute force'' approach of modelling the whole \gls{divertor} at a time.
This method was published in 2020 \cite{delaporte-mathurin_parametric_2020}.
This method can be now be applied to other components like the \gls{breeding blanket}, where the tritium inventory is also an issue.

The model developed to simulate the growth of helium bubbles in tungsten gives a rapid way of implementing and testing new physical models.
This work was published in 2021 \cite{delaporte-mathurin_influence_2021}.

Finally, during this PhD research, several contributions were made to the open-source \gls{paramak} code and related neutronics tools \cite{shimwell_paramak_2021}.

\section*{Limitations}

Many of the technical limitations of this research lie with the development issues of \gls{festim}.
Since \gls{festim} was built from scratch, features were added gradually.
For instance, at the time the simulations in \refsec{influence of exposure conditions} were run, the surface concentration could not be inhomogeneous and directly dependent of the inhomogeneous surface temperature imposed by the imposed heat flux.
The choice was therefore made to impose an homogeneous surface temperature instead.
Even though the required feature was added a few months later, re-running all the \gls{festim} simulations was too much time-consuming given the time constraints.
Many of these development drawbacks could have been alleviated if \gls{festim} was open-sourced, as external community experts could have more easily contributed to its development or bug fixes.

This study also has physical limitations inherent to the assumptions that were made.
Some of these assumptions are conservative (i.e.\ represent a worst-case scenario) and do not jeopardise the key findings.
Assumptions were also made as a response to uncertainties.
For instance, no recombination was assumed on the monoblocks gaps.
The value of the recombination coefficient at the interface between the coolant and the monoblock also has a high uncertainty as it was measured in vacuum.


\section*{Recommandations for future work}

A more accurate behaviour law for the monoblock inventory can be obtained by redoing the parametric study.
Instead of varying the surface concentration and temperature, one should vary the incident heat flux as well as the product of the implantation range and implanted particle flux $\varphi_\mathrm{imp} \ R_p$.
Assuming the uncertainty regarding the recombination coefficient of tungsten is lifted, and recombination on gaps cannot be neglected, 3D simulations could also be run for the most exposed monoblocks.
Hydrogen recombination from CuCrZr in contact with water should also be studied for this can reduce even more the tritium inventory while increasing the permeation flux to the coolant.

To estimate the tritium inventory in the \gls{iter} vacuum vessel, studies should now focus on other components like the \gls{iter} \gls{first wall}, the tritium breeding system, etc.

Regarding the influence of helium on hydrogen transport, future work could involve experimental studies investigating the simultaneous exposure of hydrogen and helium in tungsten to confirm, or infirm the suggested results of this research.




\appendix % From here onwards, chapters are numbered with letters, as is the appendix convention

\pagelayout{wide} % No margins
\addpart{Appendix}
\pagelayout{margin} % Restore margins

\setchapterstyle{lines}
\chapter{FESTIM verification}\label{appendix verification}

\section{Conservation of chemical potential (MES)}
This verification case aims at checking the FESTIM code is correctly solving the governing \refeq{c/s conservation}, Equation \ref{eq:mobile} and \ref{eq: diffusion equation changed}.

\begin{figure*} [h]
    \centering
    \begin{subfigure}{0.5\linewidth}
        \centering
        \includegraphics[width=\linewidth]{Figures/Chapter3/monoblocks/interface_condition/out_MES_case1.pdf}
        \caption{Case 1: $\alpha = 2$, $\beta = 1.5$, $\gamma=0.6$, $\tilde{c}_L = 2$, $\tilde{f}=1$}
    \end{subfigure}%
    % \hfill
    \begin{subfigure}{0.5\linewidth}
        \centering
        \includegraphics[width=\linewidth]{Figures/Chapter3/monoblocks/interface_condition/out_MES_case2.pdf}
        \caption{Case 2:  $\alpha = 1.5$, $\beta = 0.5$, $\gamma=0.4$, $\tilde{c}_L = 0.25$, $\tilde{f}=2$}
    \end{subfigure}
    \caption{Concentration profiles simulated by FESTIM against analytical solutions.}
    \label{fig:comparison MES}
\end{figure*}

The uni-dimensional test case considered in this Section was made of two subdomains $\Omega_1$ and $\Omega_2$ and is described as follow:
\begin{subequations}
\begin{align}
    \Omega &= [0, L] = \Omega_1 \cup \Omega_2 \\
    \Omega_1 &= [0, x_\mathrm{int}] \\
    \Omega_2 &= [x_\mathrm{int}, L] \\
    D &= \begin{cases}
        D_1,& \text{ in } \Omega_1\\
        D_2,& \text{ in } \Omega_2
    \end{cases} \\
    S &= \begin{cases}
        S_1,& \text{ in } \Omega_1\\
        S_2,& \text{ in } \Omega_2
    \end{cases}
\end{align}
\end{subequations}

The following dimensionless quantities are introduced:
\begin{subequations}
    \begin{align}
        \tilde{c}_\mathrm{m} &= c_\mathrm{m} / c_0\\
        \tilde{x} &= x / L \\
        \tilde{f} &= f \frac{L^2}{D_\mathrm{eq} c_0} \\
        \alpha &= D_2/D_1 \\
        \beta &= S_2/S_1 \\
        \gamma &= x_\mathrm{int}/L\\
    \end{align} 
\end{subequations}
where $D_\mathrm{eq} = (D_1 D_2)^{1/2}$.

By integrating Equation \ref{eq:mobile} and assuming steady-state (\textit{i.e.} $\partial c/\partial t=0$), one can obtain the following dimensionless form:

\begin{equation}
        \tilde{c}_\mathrm{m}= 
\begin{cases}
    -\frac{1}{2}\alpha^{1/2}\tilde{f} \tilde{x}^2 + a_1 \tilde{x} + b_1,& \text{ in } \Omega_1\\
    -\frac{1}{2}\alpha^{-1/2}\tilde{f} \tilde{x}^2 + a_2 \tilde{x} + b_2,& \text{ in } \Omega_2
\end{cases}
\label{eq:MES c}
\end{equation}

where $a_1$, $b_1$, $a_2$, $b_2$ are the unknowns of the problem to be determined.
The boundary conditions and the equilibrium law at the interface are defined as:
\begin{subequations} \label{eq: bcs MES}
\begin{align} 
        \tilde{c}_\mathrm{m}(\tilde{x}=0) & = 1 \\
        \tilde{c}_\mathrm{m}(\tilde{x}=1) & =  \tilde{c}_L \\
        \tilde{c}_\mathrm{m}^-(\tilde{x}=\gamma) & =  \beta \; \tilde{c}_\mathrm{m}^+(\tilde{x}=\gamma)\\
        \nabla \tilde{c}_\mathrm{m}^-(\tilde{x}=\gamma) & =  \alpha \nabla \tilde{c}_\mathrm{m}^+(\tilde{x}=\gamma)
\end{align}
\end{subequations}


Equation \ref{eq:MES c} can be solved with these constraints and coefficients describing $\tilde{c_\mathrm{m}}$ therefore read:

\begin{align}
    \begin{split}
        a_1 &= a_0 \; \alpha^{1/2}  \\
        b_1 &= 1 \\
        a_2 &= a_0 \; \alpha^{-1/2}\\
        b_2 &= \tilde{c}_L + \frac{1}{2} \alpha^{-1/2} \tilde{f} - a_2 \\
        a_0 &= \frac{2 \alpha^{1/2}( \tilde{c}_L - \beta) + \tilde{f}  (\gamma^{2} \left(\alpha \beta - 1\right) + 1)}{1  - \gamma + \alpha \beta \gamma} \\
    \end{split}
    \label{eq: MES c coefficients}
\end{align}

It is worth noting that when $\beta=1$ (\textit{i.e.} $S_1 = S_2 = S$) the solution becomes independent of $S$ and {$c_\mathrm{m}^{-}(x_\mathrm{int}) = c_\mathrm{m}^{+}(x_\mathrm{int})$}.
Moreover, when $\alpha = 1$ (\textit{i.e.} $D_1 = D_2 = D$), then $a_1 = a_2 = a_0$ which is the solution for steady-state diffusion in a mono-material.

The solution computed by FESTIM was found to be in very good agreement with the analytical solution for several test cases (see Figure \ref{fig:comparison MES}).

 
However, this method does not exercise all the terms in the governing Equation \ref{eq: diffusion equation changed}.
For instance, this analytical solution is only uni-dimensional, steady state is assumed and material properties are constant within the materials.
Having an exact solution from an analytical resolution for a general problem (multidimensional, transient, heterogeneous material properties, etc...) is often complex.
In order to exercise all these terms, the Method of Manufactured Solutions (MMS) will therefore be employed for it offers a good alternative to unravel these complexities.

\section{Conservation of chemical potential (MMS)}

This verification case aims at checking the FESTIM code is correctly solving the governing \refeq{c/s conservation}, Equation \ref{eq:mobile} and \ref{eq: diffusion equation changed}.

The domain $\Omega$ for this test problem is a unit square composed of two subdomains $\Omega_1$ and $\Omega_2$ (see Equation \ref{eq: MMS domain}).
\begin{subequations} \label{eq: MMS domain}
\begin{align}
    \Omega &= [0, 1] \times [0, 1] \\
    \Omega_1 &= [0, x_\mathrm{int}] \times [0, 1] \\
    \Omega_2 &= [x_\mathrm{int}, 1] \times [0, 1] \\
\end{align}
\end{subequations}
In order to unravel the complexity of an analytical resolution of the direct problem, a manufactured solution $c_\mathrm{m}$ was constructed (see Equation \ref{eq: manufactured solution}) and the problem was solved backwards.

\begin{equation}
        c_M= 
\begin{cases}
    c_{M1},& \text{ on } \Omega_1\\
    \frac{S_2}{S_1} \cdot c_{M1},& \text{ on } \Omega_2
\end{cases}
\label{eq: manufactured solution}
\end{equation}
where $c_{M1} = 2 + \cos(2\pi x) \cdot \cos(2\pi y) + t$

It is worth noting that, when choosing a manufactured solution, one must ensure it satisfies all the governing equations (especially \refeq{c/s conservation}).
In our case, $c_M$ ensures the flux conservation at the interface and the continuity of the quantity $c_\mathrm{m}/S$.

Properties are assumed time and space dependent in order to test every portion of the code (see Equation \ref{eq: MMS properties}).
\begin{subequations}
    \begin{align}
        D_1(x, y, t) & =  D_{1_0} \exp(-E_{D_1}/(k_B \cdot T(x,y, t) )) \\
        D_2(x, y, t) & =  D_{2_0} \exp(-E_{D_2}/(k_B \cdot T(x,y, t) )) \\
        S_1(x, y, t) & =  S_{1_0} \exp(-E_{S_1}/(k_B \cdot T(x,y, t) )) \\
        S_2(x, y, t) & =  S_{2_0} \exp(-E_{S_2}/(k_B \cdot T(x,y, t) )) \\
        T(x, y, t) & = 500 + 30 \cos(2\pi x) \cos(2 \pi y) \cos(2 \pi t)
\end{align} \label{eq: MMS properties}
\end{subequations}
with ${k_B = \SI{8.617e-5}{eV.K^{-1}}}$ the Boltzmann constant, $D_{1_0} = 1$, $E_{D_1} = 0.1$, $D_{2_0} = 2$, $E_{D_2} = 0.2$, $S_{1_0} = 1$, $E_{S_1} = 0.1$, $S_{2_0} = 2$ and $E_{S_2} = 0.2$.
The temperature $T$ varies around \SI{500}{K} so that, given the activation energies, properties do not approach zero.

By injecting the manufactured solution $c_M$ into the governing Equation \ref{eq:mobile}, the source term can be expressed as:
\begin{align}
    f(x, y, t) &= \frac{\partial c_M}{\partial t} - \vec{\nabla} \cdot\left(D(x, y)
    \vec{\nabla}c_M\right) \nonumber \\
    &= \begin{cases}
        \frac{\partial c_{M_1}}{\partial t} - \vec{\nabla} \cdot\left(D_1(x, y)
    \vec{\nabla}c_{M_1}\right),& \text{ on } \Omega_1\\
    \frac{\partial c_{M_2}}{\partial t} - \vec{\nabla} \cdot\left(D_2(x, y)
    \vec{\nabla}c_{M_2}\right),& \text{ on } \Omega_2\\
    \end{cases}
\end{align}

The source term $f$ was then fed into FESTIM alongside with the initial and boundary conditions described below:
\begin{subequations}
    \begin{align}
        c_\mathrm{m}(x, y, t) &= c_M(x, y, t), \text{ on } \partial \Omega \\
        c_\mathrm{m}(x, y, t=0) &= c_M(x, y, t=0), \text{ on } \Omega
    \end{align}
\end{subequations}

The computed solution $c_\mathrm{comp}$ can then be compared with the manufactured solution $c_M$ in order to quantitatively measure the numerical error.
After running the MMS process, the computed solution and the manufactured solution were in very good agreement at several arbitrarily chosen times of simulation (see Figure \ref{fig: results MMS}).
% The stepsize was $\Delta t=0.01$.
The absolute difference between the manufactured solution and the computed one was found to be zero on the boundary and maximum at the interface between the two materials.
This is explained by the Dirichlet boundary conditions enforcing the computed solution on the boundary.
This difference decreases by increasing the mesh refinement and decreasing the stepsize.
Nonetheless, the error was found to remain orders of magnitude lower than the actual solution.

\begin{figure*}
    \centering
    \begin{subfigure}{0.33\linewidth}
        \centering
        \includegraphics[width=\linewidth]{Figures/Chapter3/monoblocks/interface_condition/u_computed_t0.01.pdf}
        \caption{Computed solution $c_\mathrm{comp}(t=0.01)$}
    \end{subfigure}%
    \begin{subfigure}{0.33\linewidth}
        \centering
        \includegraphics[width=\linewidth]{Figures/Chapter3/monoblocks/interface_condition/u_exact_t0.01.pdf}
        \caption{Exact solution $c_M(t=0.01)$}
    \end{subfigure}%
    \begin{subfigure}{0.33\linewidth}
        \centering
        \includegraphics[width=\linewidth]{Figures/Chapter3/monoblocks/interface_condition/diff_t0.01.pdf}
        \caption{Absolute difference}
    \end{subfigure}
    \begin{subfigure}{0.33\linewidth}
        \centering
        \includegraphics[width=\linewidth]{Figures/Chapter3/monoblocks/interface_condition/u_computed_t0.06.pdf}
        \caption{Computed solution $c_\mathrm{comp}(t=0.06)$}
    \end{subfigure}%
    \begin{subfigure}{0.33\linewidth}
        \centering
        \includegraphics[width=\linewidth]{Figures/Chapter3/monoblocks/interface_condition/u_exact_t0.06.pdf}
        \caption{Exact solution $c_M(t=0.06)$}
    \end{subfigure}%
    \begin{subfigure}{0.33\linewidth}
        \centering
        \includegraphics[width=\linewidth]{Figures/Chapter3/monoblocks/interface_condition/diff_t0.06.pdf}
        \caption{Absolute difference}
    \end{subfigure}
    \caption{Comparison of concentration fields simulated by FESTIM with manufactured solutions}
    \label{fig: results MMS}
\end{figure*}


\section{Heat transfer (MMS)}

% why not use a complex geometry? like an elbow

The heat transfer module in FESTIM can also be verified using the method of manufactured solutions.

Let us choose the following test problem on a elbow domain $\Omega = [0, 1] \times [0, 0.5] \cup [0, 0.5] \times [0.5, 1]$ with the manufactured solution $T_D(x, y) = \sin(\omega \pi x) \sin(\omega \pi y)$.

\begin{align}
    \nabla \cdot \lambda \nabla T &= -f \\
    T &= T_D \text{  on  } y \in [0, 1] \\
    -\lambda \nabla T \cdot \mathbf{n} &= -\lambda \nabla T_D \cdot \mathbf{n} \text{  on  other surfaces} \\
    \lambda(x, y) &= 2 + T_D^2 \\
\end{align}

The source term $f$ is therefore:
\begin{align}
    f &= 2 \pi^{2} \omega^{2} f_0 \sin{\left (\pi \omega x \right )} \sin{\left (\pi \omega y \right )} \\
    f_0 &= \left(3 f_1 f_2 - f_1 - f_2 + 2\right) \\
    f_1 &= \sin^{2}{\left (\pi \omega x \right )} \\
    f_2 &= \sin^{2}{\left (\pi \omega y \right )} \\
\end{align}

The computed FESTIM solution is extremely similar to the exact solution (see Figure \ref{fig: results MMS heat transfer}).
It is also possible to compute the L2-error for several number of cell divisions in the x and y directions $N$ to ensure the error decreases as a power law of $N$.
Moreover, the L2-error (the exponent of the power law) should vary as $h^{d+1}$ where $h=1/N$ and $d$ is the polynomial degree of the elements.
This was verified for P1 and P3 elements and a super-convergence rate was observed for the P2 elements (see Figure \ref{fig: convergence rates heat transfer}).

\begin{figure*}
    \centering
    \begin{subfigure}{0.5\linewidth}
        \centering
        \includegraphics[width=\linewidth]{Figures/Chapter2/T.pdf}
        \caption{Computed temperature $N=64$}
    \end{subfigure}%
    \begin{subfigure}{0.5\linewidth}
        \centering
        \includegraphics[width=\linewidth]{Figures/Chapter2/T_exact.pdf}
        \caption{Exact temperature}
    \end{subfigure}

    \caption{Verification of the heat transfer module in FESTIM}
    \label{fig: results MMS heat transfer}
\end{figure*}

\begin{figure}
    \centering
    \includegraphics[width=\linewidth]{Figures/Chapter2/convergence_rate_heat_transfer.pdf}
    \caption{Evolution of the L2 error on $T$ showing the convergence rates for the 2D heat transfer case.}
    \label{fig: convergence rates heat transfer}
\end{figure}

\setchapterstyle{lines}
\labpage{Interface transient model}
\chapter{Interface transient model}

\section{Model description}
A kinetic model of trapping/detrapping at the interface between two materials based on the idealised energy diagram shown in Figure~\ref{fig:diagram_E} is presented.
On this diagram, $E_\mathrm{diff,k}$ is the barrier for the diffusion of H from interstitial site to interstitial site, $E_{k\rightarrow \mathrm{i}}$ is the trapping energy from material $k$ to the interface, $E_{\mathrm{i}\rightarrow k}$ is the detrapping energy from the interface to the interface and $E_{S,k}$ is the solution energy of H in material $k$.
In this model, H is split into three populations: the concentrations of mobile H in materials 1 \& 2 ($c_1$ and $c_2$ respectively) expressed in \si{m^{-3}} and the concentration of H trapped at the interface $c_\mathrm{i}$ (in \si{m^{-2}}).
% \begin{itemize}{\itemsep=0pt}
%     \item[-] H in interstitial sites in material 1 \& 2. 
%     The concentration of these types of H are respectively $c_1$ and $c_2$ (\si{m^{-3}}).
%     \item[-] H trapped at the interface.
%     The concentration of this type of H is $c_\mathrm{i}$ (\si{m^{-2}}) and maximum concentration of site here is $n_\mathrm{i}$ (\si{m^{-2}}). 
% \end{itemize}
\begin{figure}[ht!]
    \centering
    \includegraphics[width=\linewidth]{Figures/appendix/interface_E.pdf}
    \caption{Idealised potential energy diagram describing the interactions of H at the interface between two materials. In this case, $E_\mathrm{S,1}>E_\mathrm{S,2}$ (consistent with a W/Cu interface \reftab{materials properties monoblock}).}
    \label{fig:diagram_E}
\end{figure}
The interface coverage is defined as: $\theta_\mathrm{i}=\frac{c_\mathrm{i}}{n_\mathrm{i}}$ where $n_\mathrm{i}$ is the sites concentration on the interface (in \si{m^{-2}}).
% In practice, there should also be maximum concentration of sites for interstitial H ($n_{k\in\lbrace 1,2 \rbrace}$ in \si{m^{-3}}) but the concentration of interstitial H is assumed to be low compared to this maximum value, $c_k \ll n_k$: there are always a free interstitial site to detrap from the interface.
The interface is considered as a 2D defect (like a surface), hence the unit of $c_\mathrm{i}$ and $n_\mathrm{i}$ is \si{m^{-2}}.\\
\indent In this model, 2 types of reactions are considered at the interface:
\begin{itemize}
    \item the trapping from material $k$ to the interface: $\mathrm{H}_k \rightarrow \mathrm{H_i}$.
    \item the detrapping from the interface to material $k$: $\mathrm{H_i} \rightarrow \mathrm{H}_k$.
\end{itemize}
The rate (\si{s^{-1}}) of each reaction $x\rightarrow y$ is written with an Arrhenius law: 
\begin{equation}
\nu_{x\rightarrow y}(T)=\nu_0^{x\rightarrow y}\exp\left(-\frac{E_{x\rightarrow y}}{k_\mathrm{B}T}\right)
\label{eq:TST}
\end{equation}
with $\nu_0^{x\rightarrow y}$ (\si{s^{-1}}) the pre-exponential factor, $E_{x\rightarrow y}$ (\si{eV}) the energy barrier for the reaction $x\rightarrow y$, $k_\mathrm{B}$ (\si{eV.K^{-1}}) the Boltzmann constant and $T$ (\si{K}) the temperature.\\

% \indent The model outputs the the temporal evolution of the concentration of H at the interface and in the materials at the position $x_\mathrm{i}$ of the interface.
\indent The jump from interstitial site to interstitial site in material $k$ is assumed to be described by the Fick's law on diffusion characterised by the diffusion coefficient of H in material $k$ $D_{k\in\lbrace 1,2\rbrace}$ (\si{m^{2}.s^{-1}}).
Thus, the flux balance at the interface gives:
\begin{align}
\lambda_1 \left( \frac{\partial c_1}{\partial t} \right)_{x_\mathrm{i}}
                      &= \nu_{\mathrm{i}\rightarrow 1}c_\mathrm{i}
                        -\lambda_1 c_1(1-\theta_\mathrm{i})\nu_{1\rightarrow\mathrm{i}}  \nonumber
                        \\&-D_1\left(\frac{\partial c_1}{\partial x}\right)_{x_\mathrm{i}}
                        \label{eq:c1}
                        \\
\frac{d c_\mathrm{i}}{d t} 
                     &= \lambda_1 c_1 (1-\theta_\mathrm{i})\nu_{1\rightarrow\mathrm{i}}
                       - \nu_{i\rightarrow 1}c_\mathrm{i} \nonumber
                       \\ &+\lambda_2 c_2 (1-\theta_\mathrm{i})\nu_{2\rightarrow\mathrm{i}}
                       - \nu_{i\rightarrow 2}c_\mathrm{i}
                       \label{eq:ci}
                       \\
\lambda_2 \left( \frac{\partial c_2}{\partial t}\right)_{x_\mathrm{i}}
                    &= \nu_{\mathrm{i}\rightarrow 2}c_\mathrm{i}
                      -\lambda_2 c_2 (1-\theta_\mathrm{i})\nu_{2\rightarrow\mathrm{i}} \nonumber
                      \\ &+D_2 \left(\frac{\partial c_2}{\partial x}\right)_{x_\mathrm{i}}
                      \label{eq:c2}
\end{align}
where $\lambda_{k\in\lbrace 1,2 \rbrace}$ (\si{m}) is the distance between two interstitial sites in material $k$ (such that $\lambda_k c_k(x_\mathrm{i})$ represent the areal density of interstitial H at the depth $x_\mathrm{i}$ which interacts with the interface).
In Equations~\ref{eq:c1} and~\ref{eq:c2}, the first term on the right-hand side corresponds to the detrapping from the interface to material $k$, the second term correspond to the trapping to the interface and the last term corresponds to the diffusion that carries away particles from the interface region (it is important to note that the signs of these fluxes are different).\\
\indent At steady-state, when the time derivatives are null and the diffusive flux can be neglected (i.e. when the diffusion depth is long compared to $\lambda_k$), one gets:
\begin{equation}
     \frac{c_2}{c_1}=\frac{\lambda_1}{\lambda_2}\frac{\nu_\mathrm{i\rightarrow2}(T)}{\nu_\mathrm{2\rightarrow i}(T)}\frac{\nu_\mathrm{1\rightarrow i}(T)}{\nu_\mathrm{i\rightarrow 1}(T)} = \frac{S_2(T)}{S_1(T)}
\end{equation}
% Replacing each $\nu_{x\rightarrow y}(T)$ by the Arrhenius law (equation~\ref{eq:TST}), it leads to:
%\begin{equation}
%    \frac{c_2}{S_2}=\frac{c_1}{S_1}
%\end{equation}
with $S_k(T)$ (\si{m^{-3}.Pa^{-0.5}}) the solubility of H in material $k$.
The condition to have such steady-state is:
\begin{equation}
    E_{S,1}-E_{S,2} = E_\mathrm{i\rightarrow 1}-E_\mathrm{i\rightarrow 2}-E_\mathrm{1\rightarrow i} +E_\mathrm{2\rightarrow i}
    \label{eq:steady}
\end{equation}
% This is equivalent to Equation~\ref{eq: c/s conservation} used in this study.

\section{Test case}
\indent The simple kinetic presented here allows us to see how fast the equilibrium given by Equation~\ref{eq: c/s conservation} is reached.
% If it is much faster than the characteristic time of the simulation presented in section~\ref{iter case}, one can consider that the use of such kinetic model is of now use for as it would not significantly affect the dynamic of permeation for the considered thickness.\\
For the test case, we further simplify the model by considering that the concentration of interstitial H in material 1 $c_1$ is constant (\textit{ie.} $\partial c_1/\partial t=0$).
The diffusive fluxes are neglected in Equation~\ref{eq:c2} in order to solve a 0D problem.
The system of equation is solved with the SciPy package~\cite{virtanen_scipy_2020} which uses the odepack library~\cite{hindmarch_odepack_1982}.\\
% \indent As for the trapping around defects, we consider that the activation energy for the trapping reaction to the interface from material $k$ is the energy barrier for the diffusion $E_{k\rightarrow\mathrm{i}}=E_{\mathrm{diff},k}$.
% In addition, we consider that all the pre-exponential factor are equal to 10$^{13}$ \si{s^{-1}}.
% It means that, in order to have the steady-state equation (equation~\ref{eq:steady}) equivalent to Equation~\ref{eq: c/s conservation}, one needs $\frac{S_{0,1}}{S_{0,2}}=\frac{\lambda_2}{\lambda_1}$.
% Fixing all these parameters, the only free parameters are the detrapping energy from the interface to the materials. 

% Again, to satisfy the steady-state condition, one needs:
% \begin{equation}
%     E_{S,1}-E_{S,2} = E_\mathrm{i\rightarrow 1}-E_\mathrm{i\rightarrow 2}-(E_\mathrm{1\rightarrow i}-E_\mathrm{2\rightarrow i})
% \end{equation}


% Since, we already fixed the value of $E_{\mathrm{i\rightarrow},k}$ to the diffusion barrier in material $k$ and since $E_{S,1}-E_{S,2}$ is fixed for a set of materials, there is actually only one free parameter.\\
\indent A W/Cu interface is simulated at \SI{475}{K}.
% We fix the concentration of H in tungsten to $c_1=10^{18}$ \si{m^{-3}}.
% The trapping energy to the interface are 0.39 eV for both materials (table~\ref{tab:materials properties}).
% For W, we consider $\lambda_1=110$ pm.
% Thus, as discussed in the previous paragraph, it fixes $\lambda_2=65$ pm.
All parameters in the model have been constrained so that it corresponds to the steady state condition in equation~\ref{eq:steady} in the case of a W/Cu interface considering for both W and Cu, $E_{k\rightarrow i}=E_{\mathrm{diff},k}$ with $k$=Cu or W.
The values of $\lambda$ for W and Cu are \SI{110}{pm} and \SI{65}{pm} respectively.
The concentration of H in W is set to $c_1=\SI{e18}{m^{-3}}$.
The concentration of trapping site at the interface is set to $n_\mathrm{i}=10^{19}$ \si{m^{-2}}.
All pre-exponential factors are set to \SI{e13}{s^{-1}}.
% No value for the detrapping energy from the W/Cu are known so far, so we consider $0.53 \ \mathrm{eV} \le E_\mathrm{i \rightarrow 2} \le  1.13 \ \mathrm{eV}$ which means that $1.00 \ \mathrm{eV} \le E_\mathrm{i \rightarrow 1} \le 1.60 \ \mathrm{eV}$.
The only free parameter left is $E_{\mathrm{i\rightarrow},2}$ and a parametric study is performed (see Figure~\ref{fig:kinetic_interface}).
% for the various values of $E_\mathrm{i\rightarrow 2}$: (a) the evolution of the concentration in material 2 with the normalised time $t/\tau_2$, (b) the evolution of interface coverage with the normalised time $t/\tau_\mathrm{i}$, (c) the evolution of $\tau_\mathrm{i}$ and $\tau_2$ and (d) of the interface coverage at steady-state with the detrapping energy from the interface to the material 2.

The ratio $c_2/c_1$ steady state value is approximately $10^5$ (see Figure~\ref{fig:kinetic_interface}(a)).
The ratio $\theta_\mathrm{i}/\theta_\mathrm{i}^\mathrm{eq}$ does not depend on the value of the detrapping energy (see Figure~\ref{fig:kinetic_interface}(b)).
$\tau_\mathrm{i}$ and $\tau_2$ are defined as the time at which $c_\mathrm{i}$ and $c_2$ have reached 95\% of their equilibrium value.
In this case, both $\tau_\mathrm{i}$ and $\tau_2$ remain below \SI{e3}{s} (see Figure \ref{fig:kinetic_interface}(c)).
The interface is saturated with H for $E_{\mathrm{i\rightarrow},2}/(k_B T) > 25$ (see Figure \ref{fig:kinetic_interface}(d)).
% However, the equilibrium concentration strongly depends on the value of $E_\mathrm{i\rightarrow 2}$: at 475 K, the interface is saturated with H above 1 eV while the coverage is below $10^{-4}$ for 0.53 eV (figure~\ref{fig:kinetic_interface}(d)).
% The evolution of $c_2/c_1$ does not depend on the value of $E_\mathrm{i\rightarrow2}$ only in the range 0.53 - 0.88 eV (figure~\ref{fig:kinetic_interface}(a)): at 475 K, the kinetic processes are fast with such energies and the value of $\tau_\mathrm{i}$ and $\tau_2$ are identical (figure~\ref{fig:kinetic_interface}(c)).
% However, above 0.88 eV, $\tau_\mathrm{i}<\tau_2$ and one can see a delay of the growth of $c_2/c_1$: as $\theta_\mathrm{i}^\mathrm{eq}$ is much higher, one needs to wait a longer time to first fill the interface and then transfer H from the interface to material 2 efficiently.
% Regarding the value of the characteristic time $\tau_\mathrm{i}$ and $\tau_2$, they are all below 1000 s for the considered range of detrapping energies: compared to the simulation of ~10$^7$ seconds, the equilibrium at the interface can be considered to be reached much faster than the growth of the inventory and the permeation flux (see section~\ref{bi-material} and section~\ref{iter case}).
% The only difference, is that the total tritium inventory lacks the amount retained in the interface which can be significant for the highest detrapping energies.
% No knowledge of detrapping energies are know so far for the W/Cu interface.
% However, the case $E_\mathrm{i\rightarrow2}=1.13$ eV corresponds to $E_\mathrm{i\rightarrow1}=1.60$ eV.
% In tungsten, a trap with a detrapping energy of 1.6 eV is a quite a deep trap such as monovacancy~\cite{fernandez_hydrogen_2015} or small vacancy cluster~\cite{pecovnik_new_2020, hou_predictive_2019}.
% If one considers that traps at the interface are as deep as trap in the materials (0.57 eV in table~\ref{tab:materials properties}), the characteristic time are very small (0.1 s) and the retention at the interface is negligible ($10^{15}$ m$^{-2}$) making equation~\ref{eq: c/s conservation} a good approximation for the estimation of very macroscopic retention and permeation properties of the ITER monobloc.

\begin{figure} [h!]
    \centering
    \includegraphics[width=0.8\linewidth]{Figures/appendix/kinetic_interface.pdf}
    \caption{(a) Evolution of the ratio $c_2/c_1$ with normalised time $t/\tau_\mathrm{i}$ for different values of $E_\mathrm{i \rightarrow 2}/k_\mathrm{B}T$.
    (b) Evolution of the ratio $\theta_\mathrm{i}/\theta_\mathrm{i}^\mathrm{eq}$ with normalised time $t/\tau_\mathrm{i}$ (independent of $E_\mathrm{i\rightarrow2}/k_\mathrm{B}T$).
    (c) Evolution of the time to reach 95\% of the steady-state concentration, $\tau_\mathrm{i}$ and $\tau_\mathrm{2}$ as a function of $E_\mathrm{i \rightarrow 2}/k_\mathrm{B}T$.
    (d) Evolution of $\theta_\mathrm{i}^\mathrm{eq}$ as a function of $E_\mathrm{i\rightarrow2}/k_\mathrm{B}T$.
    The temperature in the simulation is 475 K, the concentration in the material 1 (W) is $c_1=10^{18}$ \si{m^{-3}} and the concentration of trapping site at the interface is $n_\mathrm{i}=10^{19}$ \si{m^{-2}}.}
    \label{fig:kinetic_interface}
\end{figure}


\input{chapters/demo_monoblock.tex}
% \setchapterstyle{lines}
\labpage{Molten Salt}
\chapter{Molten Salt}
% \blinddocument

% \setchapterstyle{lines}
\labpage{Breeding blankets}
\chapter{Breeding blankets}
\section{Intro}
\section{Summary}
% \setchapterstyle{lines}
\labpage{Co-deposition model}
\chapter{Co-deposition model}

%----------------------------------------------------------------------------------------

\backmatter % Denotes the end of the main document content
\setchapterstyle{plain} % Output plain chapters from this point onwards

%----------------------------------------------------------------------------------------
%	BIBLIOGRAPHY
%----------------------------------------------------------------------------------------

% The bibliography needs to be compiled with biber using your LaTeX editor, or on the command line with 'biber main' from the template directory

% Redefine the citation style to Author (year) and using bracket like [i-ii]

\defbibnote{bibnote}{Here are the references in citation order.\par\bigskip} % Prepend this text to the bibliography
\printbibliography[heading=bibintoc, title=Bibliography, prenote=bibnote] % Add the bibliography heading to the ToC, set the title of the bibliography and output the bibliography note

% %----------------------------------------------------------------------------------------
% %	NOMENCLATURE
% %----------------------------------------------------------------------------------------

% % The nomenclature needs to be compiled on the command line with 'makeindex main.nlo -s nomencl.ist -o main.nls' from the template directory

% \nomenclature{$c$}{Speed of light in a vacuum inertial frame}
% \nomenclature{$h$}{Planck constant}

% \renewcommand{\nomname}{Notation} % Rename the default 'Nomenclature'
% \renewcommand{\nompreamble}{The next list describes several symbols that will be later used within the body of the document.} % Prepend this text to the nomenclature

% \printnomenclature % Output the nomenclature

%----------------------------------------------------------------------------------------
%	GREEK ALPHABET
% 	Originally from https://gitlab.com/jim.hefferon/linear-algebra
%----------------------------------------------------------------------------------------

\vspace{1cm}

{\usekomafont{chapter}Greek Letters with Pronounciation} \\[2ex]
\begin{center}
	\newcommand{\pronounced}[1]{\hspace*{.2em}\small\textit{#1}}
	\begin{tabular}{l l @{\hspace*{3em}} l l}
		\toprule
		Character & Name & Character & Name \\ 
		\midrule
		$\alpha$ & alpha \pronounced{AL-fuh} & $\nu$ & nu \pronounced{NEW} \\
		$\beta$ & beta \pronounced{BAY-tuh} & $\xi$, $\Xi$ & xi \pronounced{KSIGH} \\ 
		$\gamma$, $\Gamma$ & gamma \pronounced{GAM-muh} & o & omicron \pronounced{OM-uh-CRON} \\
		$\delta$, $\Delta$ & delta \pronounced{DEL-tuh} & $\pi$, $\Pi$ & pi \pronounced{PIE} \\
		$\epsilon$ & epsilon \pronounced{EP-suh-lon} & $\rho$ & rho \pronounced{ROW} \\
		$\zeta$ & zeta \pronounced{ZAY-tuh} & $\sigma$, $\Sigma$ & sigma \pronounced{SIG-muh} \\
		$\eta$ & eta \pronounced{AY-tuh} & $\tau$ & tau \pronounced{TOW (as in cow)} \\
		$\theta$, $\Theta$ & theta \pronounced{THAY-tuh} & $\upsilon$, $\Upsilon$ & upsilon \pronounced{OOP-suh-LON} \\
		$\iota$ & iota \pronounced{eye-OH-tuh} & $\phi$, $\Phi$ & phi \pronounced{FEE, or FI (as in hi)} \\
		$\kappa$ & kappa \pronounced{KAP-uh} & $\chi$ & chi \pronounced{KI (as in hi)} \\
		$\lambda$, $\Lambda$ & lambda \pronounced{LAM-duh} & $\psi$, $\Psi$ & psi \pronounced{SIGH, or PSIGH} \\
		$\mu$ & mu \pronounced{MEW} & $\omega$, $\Omega$ & omega \pronounced{oh-MAY-guh} \\
		\bottomrule
	\end{tabular} \\[1.5ex]
	Capitals shown are the ones that differ from Roman capitals.
\end{center}

%----------------------------------------------------------------------------------------
%	GLOSSARY
%----------------------------------------------------------------------------------------

% The glossary needs to be compiled on the command line with 'makeglossaries main' from the template directory

\setglossarystyle{listgroup} % Set the style of the glossary (see https://en.wikibooks.org/wiki/LaTeX/Glossary for a reference)
\printglossary[title=Special Terms, toctitle=List of Terms] % Output the glossary, 'title' is the chapter heading for the glossary, toctitle is the table of contents heading

%----------------------------------------------------------------------------------------
%	INDEX
%----------------------------------------------------------------------------------------

% The index needs to be compiled on the command line with 'makeindex main' from the template directory

\printindex % Output the index

%----------------------------------------------------------------------------------------
%	BACK COVER
%----------------------------------------------------------------------------------------

% If you have a PDF/image file that you want to use as a back cover, uncomment the following lines

%\clearpage
%\thispagestyle{empty}
%\null%
%\clearpage
%\includepdf{cover-back.pdf}

%----------------------------------------------------------------------------------------

\end{document}
